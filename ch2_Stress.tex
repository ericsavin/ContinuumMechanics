% MG3 TD #2: Brazilian test
% V1.0 January 2021
% $Header: /cvsroot/latex-beamer/latex-beamer/solutions/generic-talks/generic-ornate-15min-45min.en.tex,v 1.5 2007/01/28 20:48:23 tantau Exp $
\def\webDOI{http://dx.doi.org}
\def\Folder{/Users/ericsavin/Documents/Cours/SG3-MMC/SLIDES_TD/}
\def\Year{\Folder/2020-2021}
\def\Sections{\Year/SECTIONS}
\def\figs{\Folder/FIGS}
%\def\figs{/Users/ericsavin/Documents//Cours/DynSto/Figs}
%\def\figs{/Users/ericsavin/Documents/Cours/SG3-MMC/SLIDES_TD/FIGS}
\def\figdynsto{/Users/ericsavin/Documents//Figures/DYNSTO}
\def\symb{/Users/ericsavin/Documents/Latex/SYMBOL}
\def\fonts{/Users/ericsavin/Documents/Latex/FONTS}
\def\logos{/Users/ericsavin/Documents/Latex/LOGOS}
\def\Onera{ONERA}
\def\ECP{CentraleSup\'elec}


\documentclass{beamer}

% This file is a solution template for:

% - Giving a talk on some subject.
% - The talk is between 15min and 45min long.
% - Style is ornate.



% Copyright 2004 by Till Tantau <tantau@users.sourceforge.net>.
%
% In principle, this file can be redistributed and/or modified under
% the terms of the GNU Public License, version 2.
%
% However, this file is supposed to be a template to be modified
% for your own needs. For this reason, if you use this file as a
% template and not specifically distribute it as part of a another
% package/program, I grant the extra permission to freely copy and
% modify this file as you see fit and even to delete this copyright
% notice. 


\mode<presentation>
{
  \usetheme{Berkeley}
  % or ...

  \setbeamercovered{transparent}
  % or whatever (possibly just delete it)
}


\usepackage[english]{babel}
% or whatever

\usepackage[latin1]{inputenc}
% or whatever

%\usepackage{mathtime}
\usefonttheme{serif}
%\usefonttheme{professionalfonts}
\usepackage{amsfonts}
\usepackage{amssymb}

\usepackage{amsmath}
\usepackage{multimedia}
\usepackage{mathrsfs}
\usepackage{mathabx}
\usepackage{color}
\usepackage{pstricks}
\usepackage{graphicx}
%\usepackage[pdftex, pdfborderstyle={/S/U/W 1}]{hyperref}
\usepackage{hyperref}
\usepackage{bbm}
\usepackage{cancel}
\usepackage[Symbol]{upgreek}
%\usepackage{mathbbol}
%\DeclareSymbolFontAlphabet{\amsmathbb}{AMSb}
%\usepackage[bbgreekl]{mathbbol}
%\usepackage[mtpbbi]{mtpro2}

%\usepackage[svgnames]{xcolor}

%\input{\fonts/math0}
%\input{\symb/structac} % Notations E. Savin
%\input{\symb/logos}

\newcommand{\ci}{\mathrm{i}}
\newcommand{\trace}{\operatorname{Tr}}
\newcommand{\Nset}{\mathbb{N}}
\newcommand{\Zset}{\mathbb{Z}}
\newcommand{\Rset}{\mathbb{R}}
\newcommand{\Cset}{\mathbb{C}}
\newcommand{\Sset}{\mathbb{S}}
\newcommand{\Mset}{\mathbb{M}}
\newcommand{\PhaseSpace}{\Omega}
\newcommand{\ContSet}{{\mathcal C}}
\newcommand{\id}{d}
\newcommand{\iD}{\mathrm{D}}
\newcommand{\iexp}{\mathrm{e}}
\newcommand{\demi}{\frac{1}{2}}
\newcommand{\imply}{\Rightarrow}

% Algebra
\newcommand{\itr}{{\sf T}}
\newcommand{\Id}{{\boldsymbol I}}
\newcommand{\IId}{\mathbb{I}}
\newcommand{\aj}{a}
\newcommand{\bj}{b}
\newcommand{\cj}{c}
\renewcommand{\dj}{d}
\newcommand{\av}{{\boldsymbol\aj}}
\newcommand{\bv}{{\boldsymbol\bj}}
\newcommand{\cv}{{\boldsymbol\cj}}
\newcommand{\dv}{{\boldsymbol\dj}}
\newcommand{\uj}{u}
\newcommand{\vj}{v}
\newcommand{\xj}{x}
\newcommand{\yj}{y}
\newcommand{\zj}{z}
\newcommand{\uv}{{\boldsymbol\uj}}
\newcommand{\vv}{{\boldsymbol\vj}}
\newcommand{\xv}{{\boldsymbol\xj}}
\newcommand{\yv}{{\boldsymbol\yj}}
\newcommand{\zv}{{\boldsymbol\zj}}
\newcommand{\Aj}{A}
\newcommand{\Bj}{B}
\newcommand{\Av}{{\boldsymbol\Aj}}
\newcommand{\Bv}{{\boldsymbol\Bj}}
\newcommand{\Zgv}{{\boldsymbol Z}}

% Analysis
\newcommand{\grad}{{\boldsymbol\nabla}}
\newcommand{\gradx}{{\grad_\xv}}
\newcommand{\Grad}{{\mathbb D}}
\newcommand{\Gradx}{{\Grad_\xv}}
\renewcommand{\div}{\mathrm{div}}
\newcommand{\divx}{{\div_\xv}}
\newcommand{\Div}{\mathbf{Div}}
\newcommand{\Divx}{{\Div_\xv}}

% Kinematics
\newcommand{\ej}{e}
\renewcommand{\ij}{i}
\newcommand{\pj}{p}
\newcommand{\ev}{{\boldsymbol\ej}}
\newcommand{\iv}{{\boldsymbol\ij}}
\newcommand{\pv}{{\boldsymbol\pj}}
\newcommand{\posij}{f}
\newcommand{\posiv}{{\boldsymbol\posij}}
\newcommand{\iposiv}{{\boldsymbol g}}
\newcommand{\Fp}{{\mathbb F}}
\newcommand{\GreenLj}{E}
\newcommand{\GreenL}{{\mathbb\GreenLj}}
\newcommand{\medium}{\Omega}
\newcommand{\strainj}{\varepsilon}
\newcommand*{\strain}{\mbox{$\hspace{0.2em}\rotatebox[x=0pt,y=0.2pt]{90}{\rule{0.02\linewidth}{0.4pt}}\hspace{-0.23em}\upvarepsilon$}} 

% Dynamics
\newcommand{\fj}{f}
\newcommand{\Fj}{F}
\newcommand{\nj}{n}
\newcommand{\Tj}{T}
\newcommand{\fv}{{\boldsymbol\fj}}
\newcommand{\Fv}{{\boldsymbol\Fj}}
\newcommand{\nv}{{\boldsymbol\nj}}
\newcommand{\Tv}{{\boldsymbol\Tj}}
\newcommand{\roi}{\varrho}
\newcommand*{\stressj}{\sigma}
\newcommand*{\stress}{\mbox{$\hspace{0.3em}\rotatebox[x=0pt,y=0.2pt]{90}{\rule{0.017\linewidth}{0.4pt}}\hspace{-0.25em}\upsigma$}}
\newcommand{\acj}{a}
\newcommand{\acv}{{\boldsymbol\acj}}

%\newcommand{\Zg}{{\bf\zgj}}
%\newcommand{\xigj}{\xi}
%\newcommand{\xig}{{\boldsymbol\xigj}}
\newcommand{\kgj}{k}
%\newcommand{\kgh}{\kgj_\ygj}
%\newcommand{\kg}{{\bf\kgj}}
\newcommand{\Kg}{{\bf K}}
%\newcommand{\qg}{{\boldsymbol q}}
%\newcommand{\pg}{{\boldsymbol p}}
\newcommand{\hkg}{{\hat \kg}}
\newcommand{\hpg}{\hat{\pg}}
%\newcommand{\vg}{{\boldsymbol v}}
\newcommand{\sg}{{\boldsymbol s}}
%\newcommand{\stress}{\mathbb{\sigma}}
\newcommand{\tenselas}{{\rm {\large C}}}
\newcommand{\speci}{{\mathrm w}}
\newcommand{\specij}{{\mathrm W}}
\newcommand{\speciv}{{\bf \specij}}
%\newcommand{\cjg}[1]{\overline{#1}}
\newcommand{\eigv}{{\bf b}}
\newcommand{\eigw}{{\bf c}}
\newcommand{\eigl}{\lambda}
\newcommand{\jeig}{\alpha}
\newcommand{\keig}{\beta}
\newcommand{\cel}{c}
\newcommand{\bcel}{{\bf\cel}}
\newcommand{\deng}{{\mathcal E}}
\newcommand{\flowj}{\pi}
\newcommand{\flow}{\boldsymbol\flowj}
\newcommand{\Flowj}{\Pi}
\newcommand{\Flow}{\boldsymbol\Flowj}
\newcommand{\fluxinj}{g}
\newcommand{\fluxin}{{\bf\fluxinj}}
\newcommand{\dscat}{\sigma}
\newcommand{\tdscat}{\Sigma}
\newcommand{\collop}{{\mathcal Q}}
\newcommand{\epsd}{\delta}
\newcommand{\rscat}{\rho}
\newcommand{\tscat}{\tau}
\newcommand{\Rscat}{\mathcal{R}}
\newcommand{\Tscat}{\mathcal{T}}
\newcommand{\lscat}{\ell}
%\newcommand{\floss}{\eta}
\newcommand{\mdiff}{{\bf D}}
%\newcommand{\demi}{\frac{1}{2}}
\newcommand{\domain}{{\mathcal O}}
\newcommand{\bdomain}{{\mathcal D}}
\newcommand{\interface}{\Gamma}
\newcommand{\sinterface}{\gamma_D}
%\newcommand{\normal}{\hat{\bf n}}
\newcommand{\bnabla}{\boldsymbol\nabla}
%\newcommand{\esp}[1]{\mathbb{E}\{\smash{#1}\}}
\newcommand{\mean}[1]{\underline{#1}}
%\newcommand{\BB}{\mathbb{B}}
%\newcommand{\II}{{\boldsymbol I}}
%\newcommand{\TA}{\boldsymbol{\Gamma}}
\newcommand{\Mdisp}{{\mathbf H}}
\newcommand{\Hamil}{{\mathcal H}}
\newcommand{\bzero}{{\bf 0}}

\newcommand{\mass}{M}
\newcommand{\damp}{D}
\newcommand{\stif}{K}
\newcommand{\dsp}{S}
\newcommand{\dof}{q}
\newcommand{\pof}{p}
%\newcommand{\MM}{{\boldsymbol\mass}}
\newcommand{\MD}{{\boldsymbol\damp}}
\newcommand{\MK}{{\boldsymbol\stif}}
\newcommand{\MS}{{\boldsymbol\dsp}}
\newcommand{\Cov}{{\boldsymbol C}}
\newcommand{\dofg}{{\boldsymbol\dof}}
\newcommand{\pofg}{{\boldsymbol\pof}}
\newcommand{\driftj}{b}
\newcommand{\drifts}{{\boldsymbol \driftj}}
\newcommand{\drift}{{\underline\drifts}}
\newcommand{\scatj}{a}
\newcommand{\scat}{{\boldsymbol\scatj}}
\newcommand{\diff}{{\boldsymbol\sigma}}
\newcommand{\load}{F}
\newcommand{\loadg}{{\boldsymbol\load}}
\newcommand{\pdf}{\pi}
\newcommand{\tpdf}{\pdf_t}
\newcommand{\fg}{{\boldsymbol f}}
%\newcommand{\Ugj}{U}
\newcommand{\Vgj}{V}
\newcommand{\Xgj}{X}
\newcommand{\Ygj}{Y}
%\newcommand{\Ug}{{\boldsymbol\Ugj}}
\newcommand{\Vg}{{\boldsymbol\Vgj}}
%\newcommand{\Qg}{{\boldsymbol Q}}
\newcommand{\Pg}{{\boldsymbol P}}
\newcommand{\Xg}{{\boldsymbol\Xgj}}
\newcommand{\Yg}{{\boldsymbol\Ygj}}
\newcommand{\flux}{{\boldsymbol J}}
\newcommand{\wiener}{W}
\newcommand{\whitenoise}{B}
\newcommand{\Wiener}{{\boldsymbol\wiener}}
\newcommand{\White}{{\boldsymbol\white}}
\newcommand{\paraj}{\nu}
\newcommand{\parag}{{\boldsymbol\paraj}}
\newcommand{\parae}{\hat{\parag}}
\newcommand{\erroj}{\epsilon}
\newcommand{\error}{{\boldsymbol\erroj}}
\newcommand{\biaj}{b}
\newcommand{\bias}{{\boldsymbol\biaj}}
%\newcommand{\disp}{{\boldsymbol V}}
\newcommand{\Fisher}{{\mathcal I}}
\newcommand{\likelihood}{{\mathcal L}}

\newcommand{\heps}{\varepsilon}
%\newcommand{\roi}{\varrho}
\newcommand{\jump}[1]{\llbracket{#1}\rrbracket}
\newcommand{\scal}[1]{\left\langle{#1}\right\rangle}
\newcommand{\norm}[1]{\left\|#1\right\|}
\newcommand{\abs}[1]{\left|#1\right|}
%\newcommand{\po}{\operatorname{o}}
\newcommand{\FFT}[1]{\widehat{#1}}
\newcommand{\indic}[1]{{\mathbf 1}_{#1}}
\newcommand{\impulse}{{\mathbbm h}}
\newcommand{\frf}{\FFT{\impulse}}

%\renewcommand{\Moy}[1]{{\boldsymbol\mu}_{#1}}
%\renewcommand{\Rcor}[1]{{\boldsymbol R}_{#1}}
%\renewcommand{\Mw}[1]{{\boldsymbol M}_{#1}}
%\renewcommand{\Sw}[1]{{\boldsymbol S}_{#1}}
%\renewcommand{\esp}[1]{{\mathbb E}\{#1\}}

\newcommand{\emphb}[1]{\textcolor{blue}{#1}}
\newcommand{\mycite}[1]{\textcolor{red}{#1}}
\newcommand{\mycitb}[1]{\textcolor{red}{[{\it #1}]}}

\newcommand{\PDFU}{{\mathcal U}}
\newcommand{\PDFN}{{\mathcal N}}
\newcommand{\TK}{{\boldsymbol\Pi}}
\newcommand{\TKij}{\pi}
\newcommand{\TKi}{{\boldsymbol\pi}}
\newcommand{\SMi}{\TKij^*}
\newcommand{\SM}{\TKi^*}
\newcommand{\lagmuli}{\lambda}
\newcommand{\lagmul}{{\boldsymbol\lagmuli}}
\newcommand{\constraint}{{\boldsymbol C}}
\newcommand{\mconstraint}{\mean{\constraint}}

\newcommand{\mybox}[1]{\fbox{\begin{minipage}{0.93\textwidth}{#1}\end{minipage}}}
\newcommand{\defcolor}[1]{\textcolor{blue}{#1}}

%\definecolor{rose}{LightPink}%{rgb}{251,204,231}

\newtheorem{mydef}{Definition}
\newtheorem{mythe}{Theorem}
\newtheorem{myprop}{Proposition}

% Or whatever. Note that the encoding and the font should match. If T1
% does not look nice, try deleting the line with the fontenc.

\title[1EL5000/S2]
{Brazilian test}

\subtitle{1EL5000--Continuum Mechanics -- Tutorial Class \#2} % (optional)

\author[\'E. Savin] % (optional, use only with lots of authors)
{\'E. Savin\inst{1,2}\\ \scriptsize{\texttt{eric.savin@\{centralesupelec,onera\}.fr}}}%\inst{1} }
% - Use the \inst{?} command only if the authors have different
%   affiliation.

\institute[Onera] % (optional, but mostly needed)
{\inst{1}{Information Processing and Systems Dept.\\\Onera, France}
\and
 \inst{2}{Mechanical and Civil Engineering Dept.\\\ECP, France}}%
%  Department of Theoretical Philosophy\\
%  University of Elsewhere}
% - Use the \inst command only if there are several affiliations.
% - Keep it simple, no one is interested in your street address.

%\date[Short Occasion] % (optional)
\date{\today}

\subject{Brazilian test}
% This is only inserted into the PDF information catalog. Can be left
% out. 



% If you have a file called "university-logo-filename.xxx", where xxx
% is a graphic format that can be processed by latex or pdflatex,
% resp., then you can add a logo as follows:

% \pgfdeclareimage[height=0.5cm]{university-logo}{university-logo-filename}
% \logo{\pgfuseimage{university-logo}}



% Delete this, if you do not want the table of contents to pop up at
% the beginning of each subsection:
\AtBeginSection[]
%\AtBeginSubsection[]
{
  \begin{frame}<beamer>{Outline}
    \tableofcontents[currentsection]%,currentsubsection]
  \end{frame}
}


% If you wish to uncover everything in a step-wise fashion, uncomment
% the following command: 

%\beamerdefaultoverlayspecification{<+->}


\begin{document}

\begin{frame}
  \titlepage
\end{frame}

\begin{frame}{Outline}
  \tableofcontents
  % You might wish to add the option [pausesections]
\end{frame}


% Since this a solution template for a generic talk, very little can
% be said about how it should be structured. However, the talk length
% of between 15min and 45min and the theme suggest that you stick to
% the following rules:  

% - Exactly two or three sections (other than the summary).
% - At *most* three subsections per section.
% - Talk about 30s to 2min per frame. So there should be between about
%   15 and 30 frames, all told.

\section{Some algebra}
\subsection{Vector \& tensor products}

\begin{frame}{Some algebra}{Vector \& tensor products}

\begin{itemize}
\item Scalar product:
\begin{displaymath}
\av,\bv\in\Rset^a\,,\quad\scal{\av,\bv}=\sum_{j=1}^a\aj_j\bj_j=\aj_j\bj_j\,,
\end{displaymath}
The last equality is \emphb{Einstein's summation convention}.
\item Tensors and tensor product (or outer product):
\begin{displaymath}
\Av\in\Rset^a\to\Rset^b\,,\quad\Av=\av\otimes\bv\,,\quad\av\in\Rset^a\,,\bv\in\Rset^b\,.
\end{displaymath}
\item Tensor application to vectors:
\begin{displaymath}
\Av=\av\otimes\bv\in\Rset^a\to\Rset^b\,,\cv\in\Rset^b\,,\quad\Av\cv=\scal{\bv,\cv}\av\,.
\end{displaymath}
\item Product of tensors $\equiv$ composition of linear maps:
\begin{displaymath}
\Av=\av\otimes\bv\,,\Bv=\cv\otimes\dv\,,\quad\Av\Bv=\scal{\bv,\cv}\av\otimes\dv\,.
\end{displaymath}
\end{itemize}

\end{frame}

\begin{frame}{Some algebra}{Vector \& tensor products}

%\onslide<2|handout:2>

\begin{itemize}
\item Scalar product of tensors:
\begin{displaymath}
\scal{\Av,\Bv}=\trace(\Av\Bv^\itr):=\Av:\Bv=\Aj_{jk}\Bj_{jk}\,.
\end{displaymath}
\item Let $\{\ev_j\}_{j=1}^d$ be a Cartesian basis in $\Rset^d$. Then:
\begin{displaymath}
\begin{split}
\aj_j &=\scal{\av,\ev_j}\,, \\
\Aj_{jk} &=\scal{\Av,\ev_j\otimes\ev_k}=\Av:\ev_j\otimes\ev_k \\
&=\scal{\Av\ev_k,\ev_j}\,,
\end{split}
\end{displaymath}
such that:
\begin{displaymath}
\begin{split}
\av &=\aj_j\ev_j\,, \\
\Av &=\Aj_{jk}\ev_j\otimes\ev_k\,.
\end{split}
\end{displaymath}
\item Example: the identity matrix
\begin{displaymath}
\Id=\ev_j\otimes\ev_j\,.
\end{displaymath}
\end{itemize}

%\end{overprint}

\end{frame}

\subsection{Vector \& tensor analysis}

\begin{frame}{Some analysis}{Vector \& tensor analysis}

\begin{itemize}
\item Gradient of a vector function $\av(\xv)$, $\xv\in\Rset^d$:
\begin{displaymath}
\Gradx\av=\frac{\partial\av}{\partial\xj_j}\otimes\ev_j\,.
\end{displaymath}
\item Divergence of a vector function $\av(\xv)$, $\xv\in\Rset^d$:
\begin{displaymath}
\divx\av=\scal{\gradx,\av}=\trace(\Gradx\av)=\frac{\partial\aj_j}{\partial\xj_j}\,.
\end{displaymath}
\item Divergence of a tensor function $\Av(\xv)$, $\xv\in\Rset^d$:
\begin{displaymath}
\Divx\Av=\frac{\partial(\Av\ev_j)}{\partial\xj_j}\,.
\end{displaymath}
\end{itemize}

\end{frame}

\begin{frame}{Some analysis}{Vector \& tensor analysis in cylindrical coordinates}

\begin{itemize}
\item Gradient of a vector function $\av(r,\theta,\zj)$:
\begin{displaymath}
\Gradx\av=\frac{\partial\av}{\partial r}\otimes\ev_r+\frac{\partial\av}{\partial\theta}\otimes\frac{\ev_\theta}{r}+\frac{\partial\av}{\partial\zj}\otimes\ev_\zj\,.
\end{displaymath}
\item Divergence of a vector function $\av(r,\theta,\zj)$:
\begin{displaymath}
\divx\av=\scal{\frac{\partial\av}{\partial r},\ev_r}+\scal{\frac{\partial\av}{\partial\theta},\frac{\ev_\theta}{r}}+\scal{\frac{\partial\av}{\partial\zj},\ev_\zj}\,.
\end{displaymath}
\item Divergence of a tensor function $\Av(r,\theta,\zj)$:
\begin{displaymath}
\Divx\Av=\frac{\partial\Av}{\partial r}\ev_r+\frac{\partial\Av}{\partial\theta}\frac{\ev_\theta}{r}+\frac{\partial\Av}{\partial\zj}\ev_\zj\,.
\end{displaymath}
\end{itemize}

\end{frame}



\section{Stresses}
\begin{frame}{Cauchy's stress tensor}{Modeling}

\begin{figure}
\centering\includegraphics[scale=.25]{\figs/ch2-Domains}
\end{figure}
\begin{itemize}
\item Modeling the actions of $V_t^*=\medium_t\setminus V_t$ on $V_t$:
\begin{itemize}
\item Local contact actions $\Tv$ exerted by proximal subdomains with short ranges;
\item Virtual cutting surface $\partial V_t$ with outward unit normal $\nv(\xv)$ defining the tangent plane at $\xv$.
\end{itemize}
\end{itemize}

\end{frame}

\begin{frame}{Cauchy's stress tensor}{Cauchy's postulates}

\begin{figure}
\centering\includegraphics[scale=.25]{\figs/ch2-Domains}
\end{figure}
\begin{itemize}
\item\emphb{Cauchy's postulate \#1}: surface traction (in N/m$^2$)
\begin{displaymath}
\Tv(\xv,t;\partial V_t)=\fv_{\partial V_t}(\xv,t)\,,\quad\forall\xv\in\partial V_t\,,\forall V_t\,.
\end{displaymath}
\item\emphb{Cauchy's postulate \#2}: the surface curvature has no influence on the traction
\begin{displaymath}
\Tv(\xv,t;\partial V_t)=\Tv(\xv,t;\nv(\xv))\,,\quad\forall\xv\in\partial V_t\,,\forall V_t\,.
\end{displaymath}
\end{itemize}

\end{frame}

\begin{frame}{Cauchy's stress tensor}{Cauchy's theorem}

\begin{figure}
\centering\includegraphics[scale=.25]{\figs/ch2-Cauchy2}
\end{figure}
\begin{itemize}
\item\emphb{Cauchy's theorem}: consider elementary $V_t$ (thin cylinder, tetrahedron), then $\forall\xv\in\partial V_t$
\begin{enumerate}
\item Action/reaction $\Tv(\xv,t;\nv(\xv))+\Tv(\xv,t;-\nv(\xv))=\bzero$;
\item Tetrahedron's lemma $\Tv(\xv,t;\nv(\xv))=\stress(\xv,t)\nv(\xv)$.
\end{enumerate}
\item Boundary conditions:
\begin{displaymath}
\stress(\xv,t)\nv(\xv)=\fv_S(\xv,t)\,,\quad\forall\xv\in\partial\medium_t\,.
\end{displaymath}
\end{itemize}

\end{frame}

\begin{frame}{Cauchy's stress tensor}{Cauchy!}

\begin{columns}[t]
\column{.4\textwidth}
\centering\includegraphics[scale=.15]{\figs/ch2-Cauchy-portrait}
\column{.6\textwidth}
\vskip-100pt
Augustin Louis Cauchy [1789-1857]
\begin{itemize}
\item X 1805
\item Ing\'enieur Ponts-et-Chauss\'ees
\item Acad\'emie des Sciences 1816
\item Prof. Analyse et M\'ecanique \`a l'X 1815-1830
\end{itemize}
\end{columns}
\href{https://mathshistory.st-andrews.ac.uk/Biographies/Cauchy/}{\tiny\texttt{https://mathshistory.st-andrews.ac.uk/Biographies/Cauchy/}}

\end{frame}

\begin{frame}{Equations of motion}{}

\begin{overprint}

\onslide<1|handout:1>
\begin{figure}
\centering\includegraphics[scale=.25]{\figs/ch2-Cauchy2}
\end{figure}
\begin{itemize}
\item $\forall V_t\subseteq\medium_t$:
\begin{displaymath}
\begin{split}
\int_{V_t}\roi\acv\id V &=\int_{V_t}\fv_v\id V + \int_{\partial V_t}\Tv\id S\,, \\
\int_{V_t}\xv\times\roi\acv\id V &=\int_{V_t}\xv\times\fv_v\id V + \int_{\partial V_t}\xv\times\Tv\id S\,.
\end{split}
\end{displaymath}
\end{itemize}

\onslide<2|handout:2>
\begin{itemize}
\item $\forall V_t\subseteq\medium_t$, $\forall\cv\in\Rset^3$:
\begin{displaymath}
\begin{split}
\int_{V_t}\scal{\roi\acv-\fv_v,\cv}\id V &=\int_{\partial V_t}\scal{\stress\nv,\cv}\id S \\
&=\int_{\partial V_t}\scal{\nv,\stress^\itr\cv}\id S \\
&=\int_{V_t}\div(\stress^\itr\cv)\id V\quad\;\;(\text{Stokes formula}) \\
&\overset{\text{def}}{=}\int_{V_t}\scal{\Div\stress,\cv}\id V \\
\int_{V_t}(\roi\acv-\fv_v)\id V &=\int_{V_t}\Div\stress\,\id V \\
\roi\acv-\fv_v &=\Div\stress\quad\quad\quad\quad(\text{localization lemma})
\end{split}
\end{displaymath}
\end{itemize}

\onslide<3|handout:3>
\begin{itemize}
\item The equation of motion for a continuous medium $\xv\in\medium_t$:
\begin{displaymath}
\boxed{\roi\acv=\Divx\stress+\fv_v}\,,
\end{displaymath}
where $\Divx\stress\approx\frac{1}{V}\int_{\partial V}\stress\nv\id S$ as $\abs{V}\to 0$.
\item The balance of momentum yields:
\begin{displaymath}
\stress^\itr=\stress\,.
\end{displaymath}
This is no longer the case if Cauchy's postulate \#1 does not hold (surface torques N.m/m$^2$).
\item Boundary conditions $\xv\in\partial \medium_t$:
\begin{displaymath}
\stress\nv=\fv_S\,.
\end{displaymath}

\end{itemize}

\end{overprint}

\end{frame}


\section{2.3 Brazilian test}

\begin{frame}{Brazilian test}{Setup}

\begin{columns}[t]
\column{.5\textwidth}
\centering\includegraphics[scale=.3]{\figs/ch2-BrazTestSetting}
\column{.5\textwidth}
\vskip-100pt
\centering\includegraphics[scale=.45]{\figs/ch2-BrazTestPhoto}
%\vskip-5pt{\hspace{3.2truecm}\mbox{\tiny{\copyright\ G. Puel}}}
\end{columns}
\begin{displaymath}
\begin{split}
\pv= & \; r\iv_r(\theta)+\zj\iv_\zj\,,\quad(r,\theta,z)\in\left]0,\frac{D}{2}\right[\times]0,2\pi[\times]0,H[\,, \\
\stress(\xv)= & \; k\frac{\cos\theta_1}{r_1}\iv_{r_1}(\theta_1)\otimes\iv_{r_1}(\theta_1)+k\frac{\cos\theta_2}{r_2}\iv_{r_2}(\theta_2)\otimes\iv_{r_2}(\theta_2) \\
&-\frac{k}{D}(\Id-\iv_z\otimes\iv_z)
\end{split}
\end{displaymath}

\end{frame}

\begin{frame}{Brazilian test}{Solution}

\begin{overprint}

\onslide<1|handout:1>
\vskip-20pt
\begin{exampleblock}{Question \#1: $\Div(\frac{\cos\theta}{r}\iv_r(\theta)\otimes\iv_r(\theta))$?}
\begin{itemize}
\item Remind that $\Div\Av=\frac{\partial\Av}{\partial r}\iv_r+\frac{\partial\Av}{\partial\theta}\frac{\iv_\theta}{r}+\frac{\partial\Av}{\partial\zj}\iv_\zj$.
\item In the present case:
{\scriptsize
\begin{displaymath}
\begin{split}
\frac{\partial}{\partial r}\left(\frac{\cos\theta}{r}\iv_r(\theta)\otimes\iv_r(\theta)\right) &=-\frac{\cos\theta}{r^2}\iv_r(\theta)\otimes\iv_r(\theta)\,, \\
\frac{\partial}{\partial\theta}\left(\frac{\cos\theta}{r}\iv_r(\theta)\otimes\iv_r(\theta)\right) &=-\frac{\sin\theta}{r}\iv_r(\theta)\otimes\iv_r(\theta)+2\frac{\cos\theta}{r}\iv_r(\theta)\otimes_s\iv_\theta(\theta)\,, \\
\frac{\partial}{\partial\zj}\left(\frac{\cos\theta}{r}\iv_r(\theta)\otimes\iv_r(\theta)\right) &=\bzero\,.
\end{split}
\end{displaymath}}
\end{itemize}
\end{exampleblock}

\onslide<2|handout:2>
\vskip-20pt
\begin{exampleblock}{Question \#1: $\Div(\frac{\cos\theta}{r}\iv_r(\theta)\otimes\iv_r(\theta))$?}
\begin{itemize}
\item Remind that $\Div\Av=\frac{\partial\Av}{\partial r}\iv_r+\frac{\partial\Av}{\partial\theta}\frac{\iv_\theta}{r}+\frac{\partial\Av}{\partial\zj}\iv_\zj$.
\item In the present case:
{\scriptsize
\begin{displaymath}
\begin{split}
\frac{\partial}{\partial r}\left(\frac{\cos\theta}{r}\iv_r(\theta)\otimes\iv_r(\theta)\right) &=-\frac{\cos\theta}{r^2}\iv_r(\theta)\otimes\iv_r(\theta)\,, \\
\frac{\partial}{\partial\theta}\left(\frac{\cos\theta}{r}\iv_r(\theta)\otimes\iv_r(\theta)\right) &=-\frac{\sin\theta}{r}\iv_r(\theta)\otimes\iv_r(\theta)+2\frac{\cos\theta}{r}\iv_r(\theta)\otimes_s\iv_\theta(\theta)\,, \\
\frac{\partial}{\partial\zj}\left(\frac{\cos\theta}{r}\iv_r(\theta)\otimes\iv_r(\theta)\right) &=\bzero\,.
\end{split}
\end{displaymath}}
\item Hence (remind that $(\av\otimes\bv)\cv=\scal{\bv,\cv}\av$):
{\scriptsize
\begin{displaymath}
\Div\left(\frac{\cos\theta}{r}\iv_r(\theta)\otimes\iv_r(\theta)\right)=-\frac{\cos\theta}{r^2}\iv_r(\theta)+\frac{\cos\theta}{r^2}\iv_r(\theta)=\bzero.
\end{displaymath}}
\item Besides $\Div(\Id-\iv_\zj\otimes\iv_\zj)=\bzero$, therefore $\Div\stress=\bzero$ QED.
\end{itemize}
\end{exampleblock}

\end{overprint}

\end{frame}

\begin{frame}{Brazilian test}{Solution}

\begin{overprint}

\onslide<1|handout:1>
\vskip-20pt
\begin{exampleblock}{Question \#2: $\stress\nv=\bzero$ on $\{r=\frac{D}{2}\}$?}
\begin{itemize}
\item On $\{r=\frac{D}{2}\}$ we have $\theta_1+\theta_2=\frac{\pi}{2}$ s.t. $\iv_{r_1}(\theta_1)\perp\iv_{r_2}(\theta_2)$, and $\nv=\iv_r(\theta)$.
\item Therefore:
\begin{displaymath}
\stress\nv=k\frac{\cos\theta_1}{r_1}\scal{\iv_r,\iv_{r_1}}\iv_{r_1}+k\frac{\cos\theta_2}{r_2}\scal{\iv_r,\iv_{r_2}}\iv_{r_2}-\frac{k}{D}\iv_r\,.
\end{displaymath}
\end{itemize}
\end{exampleblock}

\onslide<2|handout:2>
\vskip-20pt
\begin{exampleblock}{Question \#2: $\stress\nv=\bzero$ on $\{r=\frac{D}{2}\}$?}
\begin{itemize}
\item On $\{r=\frac{D}{2}\}$ we have $\theta_1+\theta_2=\frac{\pi}{2}$ s.t. $\iv_{r_1}(\theta_1)\perp\iv_{r_2}(\theta_2)$, and $\nv=\iv_r(\theta)$.
\item Therefore:
\begin{displaymath}
\stress\nv=k\frac{\cos\theta_1}{r_1}\scal{\iv_r,\iv_{r_1}}\iv_{r_1}+k\frac{\cos\theta_2}{r_2}\scal{\iv_r,\iv_{r_2}}\iv_{r_2}-\frac{k}{D}\iv_r\,.
\end{displaymath}
\item But $\cos\theta_1=\frac{r_1}{D}$ and $\cos\theta_2=\frac{r_2}{D}$, hence:
\begin{displaymath}
\begin{split}
\stress\nv &=\frac{k}{D}(\scal{\iv_r,\iv_{r_1}}\iv_{r_1}+\scal{\iv_r,\iv_{r_2}}\iv_{r_2}-\iv_r) \\
&=\frac{k}{D}(\iv_r-\iv_r)\quad\text{QED}\,.
\end{split}
\end{displaymath}
\end{itemize}
\end{exampleblock}

\end{overprint}

\end{frame}

\begin{frame}{Brazilian test}{Solution}

\begin{overprint}

\onslide<1|handout:1>
\vskip-20pt
\begin{exampleblock}{Question \#3: $k$?}
\begin{itemize}
\item $k$ should be homogeneous to N/m.
\item Isolating the lower half of the cylinder of which unit outward normal on the plane $\{\xj_2=0\}$ is $\nv=+\iv_2$, the traction exerted by the upper half cylinder is:
\begin{displaymath}
\stress\iv_2\mid_{\xj_2=0}=k\frac{\cos\theta_1}{r_1}\scal{\iv_{r_1},\iv_2}\iv_{r_1}+k\frac{\cos\theta_2}{r_2}\scal{\iv_{r_2},\iv_2}\iv_{r_2}-\frac{k}{D}\iv_2\,.
\end{displaymath}
\item But on $\{\xj_2=0\}$, $r_1=r_2=\rho$, $\theta_1=\theta_2=\Theta$, and:
\begin{displaymath}
\begin{split}
\scal{\iv_{r_2},\iv_2} &=-\scal{\iv_{r_1},\iv_2}=\cos\Theta=\frac{D}{2\rho}\,, \\
\iv_{r_2}-\iv_{r_1} &=2\cos\Theta\,\iv_2\,.
\end{split}
\end{displaymath}
\end{itemize}
\end{exampleblock}

\onslide<2|handout:2>
\vskip-20pt
\begin{exampleblock}{Question \#3: $k$?}
\begin{itemize}
\item Therefore:
\vskip-10pt
\begin{displaymath}
\begin{split}
\stress\iv_2\mid_{\xj_2=0} &=\frac{k}{D}\left[\frac{D^3}{4\rho^3}(\iv_{r_2}-\iv_{r_1})-\iv_2\right] \\
&=\frac{k}{D}\left(\frac{D^4}{4\rho^4}-1\right)\iv_2
\end{split}
\end{displaymath}
\end{itemize}
\vskip-15pt
\begin{figure}
\centering\includegraphics[scale=.3]{\figs/ch2-sigma22}\\
$\frac{D}{k}\stressj_{22}\mid_{\xj_2=0}$
\end{figure}
\end{exampleblock}

\onslide<3|handout:3>
\vskip-20pt
\begin{exampleblock}{Question \#3: $k$?}
\begin{itemize}
\item The force exerted by the upper half on the lower half is:
\begin{displaymath}
\begin{split}
\Fv^\text{upper$\to$lower} &=\int_0^H\id\zj\int_{-\frac{D}{2}}^{+\frac{D}{2}}\id\xj_1\stress\iv_2\mid_{\xj_2=0} \\
&=\frac{kH}{D}\left[\int_{-\frac{D}{2}}^{+\frac{D}{2}}\left(\frac{D^4}{4(\frac{D^2}{4}+\xj_1^2)^2}-1\right)\id\xj_1\right]\iv_2 \\
&=kH\left(2\int_{-1}^{+1}\frac{\id u}{(1+u^2)^2} - 1\right)\iv_2 \\
&=\frac{k\pi H}{2}\iv_2
\end{split}
\end{displaymath}
\end{itemize}
\end{exampleblock}

\onslide<4|handout:4>
\vskip-20pt
\begin{exampleblock}{Question \#3: $k$?}
\begin{itemize}
\item The force exerted by the upper half on the lower half is:
\begin{displaymath}
\Fv^\text{upper$\to$lower}=\frac{k\pi H}{2}\iv_2\,.
\end{displaymath}
\item The equilibrium of the lower half cylinder reads:
\begin{displaymath}
\Fv^\text{upper$\to$lower}+PH\iv_2=\bzero
\end{displaymath}
(because $P$ is in N/m), hence:
\begin{displaymath}
k=-\frac{2P}{\pi}\,,\quad\stressj_{22}\mid_{\xj_2=0}=\frac{2P}{\pi D}\left(1-\frac{D^4}{4\rho^4}\right)
\end{displaymath}
and $\stressj_{22}\mid_{\xj_2=0}<0$ which is a compression.
\end{itemize}
\end{exampleblock}

\end{overprint}

\end{frame}

\begin{frame}{Brazilian test}{Solution}

\begin{exampleblock}{Question \#4: $\stress\nv\mid_{\xj_1=0}$}
\begin{itemize}
\item On $\{\xj_1=0\}$ we have $\nv=+\iv_1$ (for the left half, say), $\theta_1=\theta_2=0$, and $\iv_{r_2}=-\iv_{r_1}=\iv_2$.
\item Therefore:
\begin{displaymath}
\begin{split}
\stress\iv_1\mid_{\xj_1=0} &= \frac{k}{r_1}\scal{\iv_1,\iv_{r_1}}\iv_{r_1}+\frac{k}{r_2}\scal{\iv_1,\iv_{r_2}}\iv_{r_2}-\frac{k}{D}\iv_1 \\
&=+\frac{2P}{\pi D}\iv_1
\end{split}
\end{displaymath}
and $\stressj_{11}\mid_{\xj_1=0}>0$ which is a traction!!!
\end{itemize}
\end{exampleblock}

\end{frame}

\begin{frame}{Brazilian test}

\begin{figure}
\centering\includegraphics[scale=.4]{\figs/ch2-sigma0}
\end{figure}
\vskip-20pt
\hfill\mbox{\tiny{\copyright\ D. Aubry}}
\hspace*{1.5truecm}$\stressj_{11}$\hspace*{3truecm}$\stressj_{22}$\hspace*{3truecm}$\stressj_{12}$
\end{frame}

\begin{frame}{Brazilian test}{Solution}

\begin{exampleblock}{Question \#5: $HP_\text{max}=20$ kN, $H=D=10$ cm}
\begin{itemize}
\item $\stressj_{11}\simeq1.3$ MPa. 
\item It is easy to obtain a traction state with the Brazilian test!
\end{itemize}
\begin{columns}[t]
\column{.5\textwidth}
\centering\includegraphics[scale=.3]{\figs/ch2-Photo1}
\column{.5\textwidth}
\centering\includegraphics[scale=.3]{\figs/ch2-Photo2}
\vskip-5pt{\hspace*{3truecm}\mbox{\tiny{\copyright\ MSS-Mat}}}
\end{columns}
\end{exampleblock}

\end{frame}

\end{document}

