% MG3 TD #4: Settlement of a soil layer
% V1.0 January 2021
% $Header: /cvsroot/latex-beamer/latex-beamer/solutions/generic-talks/generic-ornate-15min-45min.en.tex,v 1.5 2007/01/28 20:48:23 tantau Exp $
\def\webDOI{http://dx.doi.org}
\def\webDOI{http://dx.doi.org}
\def\Folder{/Users/ericsavin/Documents/Cours/SG3-MMC/SLIDES_TD/}
\def\Year{\Folder/2020-2021}
\def\Sections{\Year/SECTIONS}
\def\figs{\Folder/FIGS}
%\def\figs{/Users/ericsavin/Documents//Cours/DynSto/Figs}
%\def\figs{/Users/ericsavin/Documents/Cours/SG3-MMC/SLIDES_TD/FIGS}
\def\figdynsto{/Users/ericsavin/Documents//Figures/DYNSTO}
\def\symb{/Users/ericsavin/Documents/Latex/SYMBOL}
\def\fonts{/Users/ericsavin/Documents/Latex/FONTS}
\def\logos{/Users/ericsavin/Documents/Latex/LOGOS}
\def\Onera{ONERA}
\def\ECP{CentraleSup\'elec}


\documentclass{beamer}

% This file is a solution template for:

% - Giving a talk on some subject.
% - The talk is between 15min and 45min long.
% - Style is ornate.



% Copyright 2004 by Till Tantau <tantau@users.sourceforge.net>.
%
% In principle, this file can be redistributed and/or modified under
% the terms of the GNU Public License, version 2.
%
% However, this file is supposed to be a template to be modified
% for your own needs. For this reason, if you use this file as a
% template and not specifically distribute it as part of a another
% package/program, I grant the extra permission to freely copy and
% modify this file as you see fit and even to delete this copyright
% notice. 


\mode<presentation>
{
  \usetheme{Berkeley}
  % or ...

  \setbeamercovered{transparent}
  % or whatever (possibly just delete it)
}


\usepackage[english]{babel}
% or whatever

\usepackage[latin1]{inputenc}
% or whatever

%\usepackage{mathtime}
\usefonttheme{serif}
%\usefonttheme{professionalfonts}
\usepackage{amsfonts}
\usepackage{amssymb}

\usepackage{amsmath}
\usepackage{multimedia}
\usepackage{mathrsfs}
\usepackage{mathabx}
\usepackage{color}
\usepackage{pstricks}
\usepackage{graphicx}
%\usepackage[pdftex, pdfborderstyle={/S/U/W 1}]{hyperref}
\usepackage{hyperref}
\usepackage{bbm}
\usepackage{cancel}
\usepackage[Symbol]{upgreek}
%\usepackage{mathbbol}
%\DeclareSymbolFontAlphabet{\amsmathbb}{AMSb}
%\usepackage[bbgreekl]{mathbbol}
%\usepackage[mtpbbi]{mtpro2}

%\usepackage[svgnames]{xcolor}

%\input{\fonts/math0}
%\input{\symb/structac} % Notations E. Savin
%\input{\symb/logos}

\newcommand{\ci}{\mathrm{i}}
\newcommand{\trace}{\operatorname{Tr}}
\newcommand{\Nset}{\mathbb{N}}
\newcommand{\Zset}{\mathbb{Z}}
\newcommand{\Rset}{\mathbb{R}}
\newcommand{\Cset}{\mathbb{C}}
\newcommand{\Sset}{\mathbb{S}}
\newcommand{\Mset}{\mathbb{M}}
\newcommand{\PhaseSpace}{\Omega}
\newcommand{\ContSet}{{\mathcal C}}
\newcommand{\id}{d}
\newcommand{\iD}{\mathrm{D}}
\newcommand{\iexp}{\mathrm{e}}
\newcommand{\demi}{\frac{1}{2}}
\newcommand{\imply}{\Rightarrow}

% Algebra
\newcommand{\itr}{{\sf T}}
\newcommand{\Id}{{\boldsymbol I}}
\newcommand{\IId}{\mathbb{I}}
\newcommand{\aj}{a}
\newcommand{\bj}{b}
\newcommand{\cj}{c}
\renewcommand{\dj}{d}
\newcommand{\av}{{\boldsymbol\aj}}
\newcommand{\bv}{{\boldsymbol\bj}}
\newcommand{\cv}{{\boldsymbol\cj}}
\newcommand{\dv}{{\boldsymbol\dj}}
\newcommand{\uj}{u}
\newcommand{\vj}{v}
\newcommand{\xj}{x}
\newcommand{\yj}{y}
\newcommand{\zj}{z}
\newcommand{\uv}{{\boldsymbol\uj}}
\newcommand{\vv}{{\boldsymbol\vj}}
\newcommand{\xv}{{\boldsymbol\xj}}
\newcommand{\yv}{{\boldsymbol\yj}}
\newcommand{\zv}{{\boldsymbol\zj}}
\newcommand{\Aj}{A}
\newcommand{\Bj}{B}
\newcommand{\Av}{{\boldsymbol\Aj}}
\newcommand{\Bv}{{\boldsymbol\Bj}}
\newcommand{\Zgv}{{\boldsymbol Z}}

% Analysis
\newcommand{\grad}{{\boldsymbol\nabla}}
\newcommand{\gradx}{{\grad_\xv}}
\newcommand{\Grad}{{\mathbb D}}
\newcommand{\Gradx}{{\Grad_\xv}}
\renewcommand{\div}{\mathrm{div}}
\newcommand{\divx}{{\div_\xv}}
\newcommand{\Div}{\mathbf{Div}}
\newcommand{\Divx}{{\Div_\xv}}

% Kinematics
\newcommand{\ej}{e}
\renewcommand{\ij}{i}
\newcommand{\pj}{p}
\newcommand{\ev}{{\boldsymbol\ej}}
\newcommand{\iv}{{\boldsymbol\ij}}
\newcommand{\pv}{{\boldsymbol\pj}}
\newcommand{\posij}{f}
\newcommand{\posiv}{{\boldsymbol\posij}}
\newcommand{\iposiv}{{\boldsymbol g}}
\newcommand{\Fp}{{\mathbb F}}
\newcommand{\GreenLj}{E}
\newcommand{\GreenL}{{\mathbb\GreenLj}}
\newcommand{\medium}{\Omega}
\newcommand{\strainj}{\varepsilon}
\newcommand*{\strain}{\mbox{$\hspace{0.2em}\rotatebox[x=0pt,y=0.2pt]{90}{\rule{0.02\linewidth}{0.4pt}}\hspace{-0.23em}\upvarepsilon$}}
\newcommand*{\rotation}{{\boldsymbol R}}

% Dynamics
\newcommand{\fj}{f}
\newcommand{\Fj}{F}
\newcommand{\mj}{m}
\newcommand{\nj}{n}
\newcommand{\Tj}{T}
\newcommand{\fv}{{\boldsymbol\fj}}
\newcommand{\Fv}{{\boldsymbol\Fj}}
\newcommand{\mv}{{\boldsymbol\mj}}
\newcommand{\nv}{{\boldsymbol\nj}}
\newcommand{\Tv}{{\boldsymbol\Tj}}
\newcommand{\roi}{\varrho}
\newcommand*{\stressj}{\sigma}
\newcommand*{\tressj}{\tau}
\newcommand{\stress}{\mbox{$\hspace{0.3em}\rotatebox[x=0pt,y=0.2pt]{90}{\rule{0.017\linewidth}{0.4pt}}\hspace{-0.25em}\upsigma$}}
\newcommand*{\tress}{{\boldsymbol\tressj}}
\newcommand{\acj}{a}
\newcommand{\acv}{{\boldsymbol\acj}}

%\newcommand{\Zg}{{\bf\zgj}}
%\newcommand{\xigj}{\xi}
%\newcommand{\xig}{{\boldsymbol\xigj}}
\newcommand{\kgj}{k}
%\newcommand{\kgh}{\kgj_\ygj}
%\newcommand{\kg}{{\bf\kgj}}
\newcommand{\Kg}{{\bf K}}
%\newcommand{\qg}{{\boldsymbol q}}
%\newcommand{\pg}{{\boldsymbol p}}
\newcommand{\hkg}{{\hat \kg}}
\newcommand{\hpg}{\hat{\pg}}
%\newcommand{\vg}{{\boldsymbol v}}
\newcommand{\sg}{{\boldsymbol s}}
%\newcommand{\stress}{\mathbb{\sigma}}
\newcommand{\tenselasj}{{\Large C}}
\newcommand{\tenselas}{\boldsymbol{\mathsf{\tenselasj}}}
\newcommand{\tenscomp}{\boldsymbol{\mathsf{\Large S}}}
\newcommand{\speci}{{\mathrm w}}
\newcommand{\specij}{{\mathrm W}}
\newcommand{\speciv}{{\bf \specij}}
%\newcommand{\cjg}[1]{\overline{#1}}
\newcommand{\eigv}{{\bf b}}
\newcommand{\eigw}{{\bf c}}
\newcommand{\eigl}{\lambda}
\newcommand{\jeig}{\alpha}
\newcommand{\keig}{\beta}
\newcommand{\cel}{c}
\newcommand{\bcel}{{\bf\cel}}
\newcommand{\deng}{{\mathcal E}}
\newcommand{\flowj}{\pi}
\newcommand{\flow}{\boldsymbol\flowj}
\newcommand{\Flowj}{\Pi}
\newcommand{\Flow}{\boldsymbol\Flowj}
\newcommand{\fluxinj}{g}
\newcommand{\fluxin}{{\bf\fluxinj}}
\newcommand{\dscat}{\sigma}
\newcommand{\tdscat}{\Sigma}
\newcommand{\collop}{{\mathcal Q}}
\newcommand{\epsd}{\delta}
\newcommand{\rscat}{\rho}
\newcommand{\tscat}{\tau}
\newcommand{\Rscat}{\mathcal{R}}
\newcommand{\Tscat}{\mathcal{T}}
\newcommand{\lscat}{\ell}
%\newcommand{\floss}{\eta}
\newcommand{\mdiff}{{\bf D}}
%\newcommand{\demi}{\frac{1}{2}}
\newcommand{\domain}{{\mathcal O}}
\newcommand{\bdomain}{{\mathcal D}}
\newcommand{\interface}{\Gamma}
\newcommand{\sinterface}{\gamma_D}
%\newcommand{\normal}{\hat{\bf n}}
\newcommand{\bnabla}{\boldsymbol\nabla}
%\newcommand{\esp}[1]{\mathbb{E}\{\smash{#1}\}}
\newcommand{\mean}[1]{\underline{#1}}
%\newcommand{\BB}{\mathbb{B}}
%\newcommand{\II}{{\boldsymbol I}}
%\newcommand{\TA}{\boldsymbol{\Gamma}}
\newcommand{\Mdisp}{{\mathbf H}}
\newcommand{\Hamil}{{\mathcal H}}
\newcommand{\bzero}{{\bf 0}}

\newcommand{\mass}{M}
\newcommand{\damp}{D}
\newcommand{\stif}{K}
\newcommand{\dsp}{S}
\newcommand{\dof}{q}
\newcommand{\pof}{p}
%\newcommand{\MM}{{\boldsymbol\mass}}
\newcommand{\MD}{{\boldsymbol\damp}}
\newcommand{\MK}{{\boldsymbol\stif}}
\newcommand{\MS}{{\boldsymbol\dsp}}
\newcommand{\Cov}{{\boldsymbol C}}
\newcommand{\dofg}{{\boldsymbol\dof}}
\newcommand{\pofg}{{\boldsymbol\pof}}
\newcommand{\driftj}{b}
\newcommand{\drifts}{{\boldsymbol \driftj}}
\newcommand{\drift}{{\underline\drifts}}
\newcommand{\scatj}{a}
\newcommand{\scat}{{\boldsymbol\scatj}}
\newcommand{\diff}{{\boldsymbol\sigma}}
\newcommand{\load}{F}
\newcommand{\loadg}{{\boldsymbol\load}}
\newcommand{\pdf}{\pi}
\newcommand{\tpdf}{\pdf_t}
\newcommand{\fg}{{\boldsymbol f}}
%\newcommand{\Ugj}{U}
\newcommand{\Vgj}{V}
\newcommand{\Xgj}{X}
\newcommand{\Ygj}{Y}
%\newcommand{\Ug}{{\boldsymbol\Ugj}}
\newcommand{\Vg}{{\boldsymbol\Vgj}}
%\newcommand{\Qg}{{\boldsymbol Q}}
\newcommand{\Pg}{{\boldsymbol P}}
\newcommand{\Xg}{{\boldsymbol\Xgj}}
\newcommand{\Yg}{{\boldsymbol\Ygj}}
\newcommand{\flux}{{\boldsymbol J}}
\newcommand{\wiener}{W}
\newcommand{\whitenoise}{B}
\newcommand{\Wiener}{{\boldsymbol\wiener}}
\newcommand{\White}{{\boldsymbol\white}}
\newcommand{\paraj}{\nu}
\newcommand{\parag}{{\boldsymbol\paraj}}
\newcommand{\parae}{\hat{\parag}}
\newcommand{\erroj}{\epsilon}
\newcommand{\error}{{\boldsymbol\erroj}}
\newcommand{\biaj}{b}
\newcommand{\bias}{{\boldsymbol\biaj}}
%\newcommand{\disp}{{\boldsymbol V}}
\newcommand{\Fisher}{{\mathcal I}}
\newcommand{\likelihood}{{\mathcal L}}

\newcommand{\heps}{\varepsilon}
%\newcommand{\roi}{\varrho}
\newcommand{\jump}[1]{\llbracket{#1}\rrbracket}
\newcommand{\scal}[1]{\left\langle{#1}\right\rangle}
\newcommand{\norm}[1]{\left\|#1\right\|}
\newcommand{\abs}[1]{\left|#1\right|}
%\newcommand{\po}{\operatorname{o}}
\newcommand{\FFT}[1]{\widehat{#1}}
\newcommand{\indic}[1]{{\mathbf 1}_{#1}}
\newcommand{\impulse}{{\mathbbm h}}
\newcommand{\frf}{\FFT{\impulse}}

%\renewcommand{\Moy}[1]{{\boldsymbol\mu}_{#1}}
%\renewcommand{\Rcor}[1]{{\boldsymbol R}_{#1}}
%\renewcommand{\Mw}[1]{{\boldsymbol M}_{#1}}
%\renewcommand{\Sw}[1]{{\boldsymbol S}_{#1}}
%\renewcommand{\esp}[1]{{\mathbb E}\{#1\}}

\newcommand{\emphb}[1]{\textcolor{blue}{#1}}
\newcommand{\mycite}[1]{\textcolor{red}{#1}}
\newcommand{\mycitb}[1]{\textcolor{red}{[{\it #1}]}}

\newcommand{\PDFU}{{\mathcal U}}
\newcommand{\PDFN}{{\mathcal N}}
\newcommand{\TK}{{\boldsymbol\Pi}}
\newcommand{\TKij}{\pi}
\newcommand{\TKi}{{\boldsymbol\pi}}
\newcommand{\SMi}{\TKij^*}
\newcommand{\SM}{\TKi^*}
\newcommand{\lagmuli}{\lambda}
\newcommand{\lagmul}{{\boldsymbol\lagmuli}}
\newcommand{\constraint}{{\boldsymbol C}}
\newcommand{\mconstraint}{\mean{\constraint}}

\newcommand{\mybox}[1]{\fbox{\begin{minipage}{0.93\textwidth}{#1}\end{minipage}}}
\newcommand{\defcolor}[1]{\textcolor{blue}{#1}}

%\definecolor{rose}{LightPink}%{rgb}{251,204,231}

\newtheorem{mydef}{Definition}
\newtheorem{mythe}{Theorem}
\newtheorem{myprop}{Proposition}

% Or whatever. Note that the encoding and the font should match. If T1
% does not look nice, try deleting the line with the fontenc.

\title[1EL5000/S4]
{Settlement of a soil layer}

\subtitle{1EL5000--Continuum Mechanics -- Tutorial Class \#4} % (optional)

\author[\'E. Savin] % (optional, use only with lots of authors)
{\'E. Savin\inst{1,2}\\ \scriptsize{\texttt{eric.savin@\{centralesupelec,onera\}.fr}}}%\inst{1} }
% - Use the \inst{?} command only if the authors have different
%   affiliation.

\institute[Onera] % (optional, but mostly needed)
{\inst{1}{Information Processing and Systems Dept.\\\Onera, France}
\and
 \inst{2}{Mechanical and Civil Engineering Dept.\\\ECP, France}}%
%  Department of Theoretical Philosophy\\
%  University of Elsewhere}
% - Use the \inst command only if there are several affiliations.
% - Keep it simple, no one is interested in your street address.

%\date[Short Occasion] % (optional)
\date[]{\today}
\date[]{February 18, 2021}

\subject{Settlement of a soil layer}
% This is only inserted into the PDF information catalog. Can be left
% out. 



% If you have a file called "university-logo-filename.xxx", where xxx
% is a graphic format that can be processed by latex or pdflatex,
% resp., then you can add a logo as follows:

% \pgfdeclareimage[height=0.5cm]{university-logo}{university-logo-filename}
% \logo{\pgfuseimage{university-logo}}



% Delete this, if you do not want the table of contents to pop up at
% the beginning of each subsection:
\AtBeginSection[]
%\AtBeginSubsection[]
{
  \begin{frame}<beamer>{Outline}
    \tableofcontents[currentsection]%,currentsubsection]
  \end{frame}
}


% If you wish to uncover everything in a step-wise fashion, uncomment
% the following command: 

%\beamerdefaultoverlayspecification{<+->}


\begin{document}

\begin{frame}
  \titlepage
\end{frame}

\begin{frame}{Outline}
  \tableofcontents
  % You might wish to add the option [pausesections]
\end{frame}


% Since this a solution template for a generic talk, very little can
% be said about how it should be structured. However, the talk length
% of between 15min and 45min and the theme suggest that you stick to
% the following rules:  

% - Exactly two or three sections (other than the summary).
% - At *most* three subsections per section.
% - Talk about 30s to 2min per frame. So there should be between about
%   15 and 30 frames, all told.

\section{Some algebra}
\subsection{Vector \& tensor products}

\begin{frame}{Some algebra}{Vector \& tensor products}

\begin{itemize}
\item Scalar product:
\begin{displaymath}
\av,\bv\in\Rset^a\,,\quad\scal{\av,\bv}=\sum_{j=1}^a\aj_j\bj_j=\aj_j\bj_j\,,
\end{displaymath}
The last equality is \emphb{Einstein's summation convention}.
\item Tensors and tensor product (or outer product):
\begin{displaymath}
\Av\in\Rset^a\to\Rset^b\,,\quad\Av=\av\otimes\bv\,,\quad\av\in\Rset^a\,,\bv\in\Rset^b\,.
\end{displaymath}
\item Tensor application to vectors:
\begin{displaymath}
\Av=\av\otimes\bv\in\Rset^a\to\Rset^b\,,\cv\in\Rset^b\,,\quad\Av\cv=\scal{\bv,\cv}\av\,.
\end{displaymath}
\item Product of tensors $\equiv$ composition of linear maps:
\begin{displaymath}
\Av=\av\otimes\bv\,,\Bv=\cv\otimes\dv\,,\quad\Av\Bv=\scal{\bv,\cv}\av\otimes\dv\,.
\end{displaymath}
\end{itemize}

\end{frame}

\begin{frame}{Some algebra}{Vector \& tensor products}

%\onslide<2|handout:2>

\begin{itemize}
\item Scalar product of tensors:
\begin{displaymath}
\scal{\Av,\Bv}=\trace(\Av\Bv^\itr):=\Av:\Bv=\Aj_{jk}\Bj_{jk}\,.
\end{displaymath}
\item Let $\{\ev_j\}_{j=1}^d$ be a Cartesian basis in $\Rset^d$. Then:
\begin{displaymath}
\begin{split}
\aj_j &=\scal{\av,\ev_j}\,, \\
\Aj_{jk} &=\scal{\Av,\ev_j\otimes\ev_k}=\Av:\ev_j\otimes\ev_k \\
&=\scal{\Av\ev_k,\ev_j}\,,
\end{split}
\end{displaymath}
such that:
\begin{displaymath}
\begin{split}
\av &=\aj_j\ev_j\,, \\
\Av &=\Aj_{jk}\ev_j\otimes\ev_k\,.
\end{split}
\end{displaymath}
\item Example: the identity matrix
\begin{displaymath}
\Id=\ev_j\otimes\ev_j\,.
\end{displaymath}
\end{itemize}

%\end{overprint}

\end{frame}

\subsection{Vector \& tensor analysis}

\begin{frame}{Some analysis}{Vector \& tensor analysis}

\begin{itemize}
\item Gradient of a vector function $\av(\xv)$, $\xv\in\Rset^d$:
\begin{displaymath}
\Gradx\av=\frac{\partial\av}{\partial\xj_j}\otimes\ev_j\,.
\end{displaymath}
\item Divergence of a vector function $\av(\xv)$, $\xv\in\Rset^d$:
\begin{displaymath}
\divx\av=\scal{\gradx,\av}=\trace(\Gradx\av)=\frac{\partial\aj_j}{\partial\xj_j}\,.
\end{displaymath}
\item Divergence of a tensor function $\Av(\xv)$, $\xv\in\Rset^d$:
\begin{displaymath}
\Divx\Av=\frac{\partial(\Av\ev_j)}{\partial\xj_j}\,.
\end{displaymath}
\end{itemize}

\end{frame}

\begin{frame}{Some analysis}{Vector \& tensor analysis in cylindrical coordinates}

\begin{itemize}
\item Gradient of a vector function $\av(r,\theta,\zj)$:
\begin{displaymath}
\Gradx\av=\frac{\partial\av}{\partial r}\otimes\ev_r+\frac{\partial\av}{\partial\theta}\otimes\frac{\ev_\theta}{r}+\frac{\partial\av}{\partial\zj}\otimes\ev_\zj\,.
\end{displaymath}
\item Divergence of a vector function $\av(r,\theta,\zj)$:
\begin{displaymath}
\divx\av=\scal{\frac{\partial\av}{\partial r},\ev_r}+\scal{\frac{\partial\av}{\partial\theta},\frac{\ev_\theta}{r}}+\scal{\frac{\partial\av}{\partial\zj},\ev_\zj}\,.
\end{displaymath}
\item Divergence of a tensor function $\Av(r,\theta,\zj)$:
\begin{displaymath}
\Divx\Av=\frac{\partial\Av}{\partial r}\ev_r+\frac{\partial\Av}{\partial\theta}\frac{\ev_\theta}{r}+\frac{\partial\Av}{\partial\zj}\ev_\zj\,.
\end{displaymath}
\end{itemize}

\end{frame}



\section{Material behavior}
\begin{frame}{Recap}

\begin{itemize}
\item Local equilibrium equation:
\begin{displaymath}
\Divx\stress+\fv_v=\roi\ddot{\uv}\,.
\end{displaymath}
\item Small strains assumption:
\begin{displaymath}
\strain=\demi(\Gradx\uv+\Gradx\uv^\itr)\,.
\end{displaymath}
\item Closure is missing: $\stress=f(\strain)$ or $\strain=g(\stress)$, the \emphb{material constitutive equation}.
\end{itemize}

\end{frame}

\begin{frame}{Recap}{Traction test}

\begin{overprint}

\onslide<1|handout:1>
\begin{columns}[t]
\column{.6\textwidth}
\centering\includegraphics[scale=.3]{\figs/ch4-TractionTest}
\column{.4\textwidth}
\vskip-120pt
%\begin{center}
Measurements along $\ev$:
\begin{itemize}
\item $\stressj_{ee}=\frac{F}{S}$;
\item $\strainj_{ee}=\frac{\Delta L}{L}$.
\end{itemize}
%\end{center}
\end{columns}

\onslide<2|handout:2>
\begin{figure}
\centering\includegraphics[scale=.2]{\figs/ch4-StressStrain}
\end{figure}
\begin{itemize}
\item Ductile vs. fragile materials;
\item At small strains $F=K\Delta L$ or $\stressj_{ee}=k\strainj_{ee}$;
\item Generalization: $\stress=\tenselas\strain$ where $\tenselas$ is the fourth-order ($\tenselasj_{jklm}$) \emphb{elasticity tensor}.
\end{itemize}

\end{overprint}

\end{frame}

\begin{frame}{Elasticity tensor}{Symmetries}

\begin{itemize}
\item Fourth-order tensor: linear application between second-order tensors,
\begin{displaymath}
\begin{split}
\stress(\xv,t) &=\tenselas(\xv)\strain(\xv,t)\,,\\
\stressj_{jk}(\xv,t) &=\tenselasj_{jklm}(\xv)\strainj_{lm}(\xv,t)\,,\quad\forall\xv\in\medium\,,\forall t\in\Rset\,.
\end{split}
\end{displaymath}
\item Minor symmetries from the symmetry of $\strain$ and $\stress$:
\begin{displaymath}
\begin{split}
\tenselasj_{jklm}(\xv) &= \tenselasj_{kjlm}(\xv)\,, \\
\tenselasj_{jklm}(\xv) &= \tenselasj_{kjml}(\xv)\,.
\end{split}
\end{displaymath}
\item Major symmetry from thermodynamics first principle:
\begin{displaymath}
\tenselasj_{jklm}(\xv) = \tenselasj_{lmjk}(\xv)\,.
\end{displaymath}
\item \# coefficients: $81\xrightarrow[]{\text{minor symmetries}}36\xrightarrow[]{\text{major symmetry}}21$.
\end{itemize}

\end{frame}

\begin{frame}{Elasticity tensor}{Isotropy}

\begin{columns}[t]
\column{.33\textwidth}
\centering\includegraphics[scale=.25]{\figs/ch4-Isotropic}
\column{.33\textwidth}
\centering\includegraphics[scale=.25]{\figs/ch4-Anisotropic1}
\column{.33\textwidth}
\centering\includegraphics[scale=.25]{\figs/ch4-Anisotropic2}
\end{columns}
\begin{itemize}
\item Isotropy: the constitutive equation is the same whatever the orientation of the sample is,
\begin{displaymath}
\stress=\tenselas\strain\;(\text{sample \#1})\,,\quad\stress^*=\tenselas\strain^*\;(\text{sample \#2})\,.
\end{displaymath}
\item Letting $\rotation$ be an arbitrary rotation ($\rotation\rotation^\itr=\Id$):
\begin{displaymath}
\strain^*=\rotation\strain\rotation^\itr\,,\quad\stress^*=\rotation\stress\rotation^\itr\,,
\end{displaymath}
then:
\begin{displaymath}
\stress^*=\rotation\stress\rotation^\itr=\tenselas(\rotation\strain\rotation^\itr)\imply \rotation(\tenselas\strain)\rotation^\itr=\tenselas(\rotation\strain\rotation^\itr)\,.
\end{displaymath}
\end{itemize}

\end{frame}

\begin{frame}{Elasticity tensor}{Isotropy}

\begin{itemize}
\item \emphb{Rivlin-Ericksen theorem}:
\begin{displaymath}
\rotation(\tenselas\strain)\rotation^\itr=\tenselas(\rotation\strain\rotation^\itr)\imply\tenselas\strain=\alpha_0\Id+\alpha_1\strain+\alpha_2\strain^2\,,
\end{displaymath}
where $\alpha_m(\trace(\strain),\trace(\strain^2),\trace(\strain^3))$, $m=1,2,3$.
%\item To the leading order in $\text{O}(\strain)$ ($\lambda,\mu$ are Lam\'e's parameters):
%\begin{displaymath}
%\tenselas\strain=\lambda(\trace\strain)\Id+2\mu\strain\,.
%\end{displaymath}
\vskip20pt
\begin{columns}[t]
\column{.5\textwidth}
\centering\includegraphics[scale=.3]{\figs/ch4-RivlinR}\\
{\scriptsize Ronald Rivlin [1915--2005]}
\column{.5\textwidth}
%\centering\includegraphics[scale=.38]{\figs/ch4-Ericksen}\\
%\centering\includegraphics[scale=.3]{\figs/ch4-EricksenJ}\\
\centering\includegraphics[scale=.09]{\figs/ch4-SavinE}\\
%\centering\includegraphics[scale=.19]{\figs/ch4-AustinP}\\
{\scriptsize Jerald Ericksen [1925--]}
\end{columns}
\end{itemize}

\end{frame}

\begin{frame}{Elasticity tensor}{Isotropy}

\begin{overprint}

\onslide<1|handout:1>
\begin{itemize}
\item To the leading order in $\text{O}(\strain)$:
\begin{displaymath}
\boxed{\tenselas\strain=\lambda\trace(\strain)\Id+2\mu\strain}
\end{displaymath}
where $\lambda,\mu$ are \emphb{Lam\'e's modulii}.
\item This relationship can be inverted:
\begin{displaymath}
\begin{split}
\strain &=-\frac{\lambda}{2\mu(3\lambda+2\mu)}\trace(\stress)\Id+\frac{1}{2\mu}\stress \\
&=\alpha\trace(\stress)\Id+\beta\stress \\
\end{split}
\end{displaymath}
where $\alpha,\beta$ are obtained from measurements.
\item More generally $\strain=\tenscomp\stress$, where $\tenscomp$ is the compliance (fourth-order) tensor.
\end{itemize}

\onslide<2|handout:2>
\begin{itemize}
\item Traction test $\stress=\frac{F}{S}\ev\otimes\ev$:
\begin{columns}[t]
\column{.5\textwidth}
\hspace*{-0.3truecm}\centering\includegraphics[scale=.25]{\figs/ch4-TestEnu}
\column{.5\textwidth}
\vskip-140pt
\emphb{Young's modulus} $E$: 
\begin{displaymath}
E=\frac{\stressj_{ee}}{\strainj_{ee}}=\frac{\frac{F}{S}}{\frac{\Delta L}{L_0}}\,;
\end{displaymath}
\emphb{Poisson's coefficient} $\nu$:
\begin{displaymath}
\nu=-\frac{\strainj_{hh}}{\strainj_{ee}}=-\frac{\frac{\Delta L_H}{L_{H0}}}{\frac{\Delta L}{L_0}}\,.
\end{displaymath}
\end{columns}
\item $E$ is in Pa (GPa), and $\nu$ is dimensionless.
\end{itemize}

\onslide<3|handout:3>
\begin{itemize}
\item $\stress$ and $\strain$ have the same principal directions:
\begin{figure}
\centering\includegraphics[scale=.2]{\figs/ch4-Hooke}
\end{figure}
\item Along \emph{e.g.} the principal direction \#1:
\begin{displaymath}
\strainj_1=\frac{\stressj_1}{E}-\nu\frac{\stressj_2}{E}-\nu\frac{\stressj_3}{E}\,.
\end{displaymath}
\item In the basis $(\iv_1,\iv_2,\iv_3)$ of the principal directions:
\begin{displaymath}
\begin{split}
\strain &=\, \scriptstyle (\frac{\stressj_1}{E}-\nu\frac{\stressj_2+\stressj_3}{E}) \iv_1\otimes\iv_1 + (\frac{\stressj_2}{E}-\nu\frac{\stressj_1+\stressj_3}{E}) \iv_2\otimes\iv_2 + (\frac{\stressj_3}{E}-\nu\frac{\stressj_1+\stressj_2}{E}) \iv_3\otimes\iv_3 \\
&=\, \scriptstyle \frac{1+\nu}{E}\stress-\frac{\nu}{E}\trace(\stress)\Id\,.
\end{split}
\end{displaymath}
\end{itemize}

\onslide<4|handout:4>
\begin{itemize}
\item Recap: linear elastic isotropy has 2 coefficients,
\begin{displaymath}
\begin{split}
\stress &=\tenselas\strain=\lambda\trace(\strain)\Id+2\mu\strain\,, \\
\strain &=\tenscomp\stress=\frac{1+\nu}{E}\stress-\frac{\nu}{E}\trace(\stress)\Id\,.
\end{split}
\end{displaymath}
\item $\lambda(\xv),\mu(\xv)$ are Lam\'e's modulii, $E(\xv)$ is Young's modulus, $\nu(\xv)$ is Poisson's ratio:
\begin{displaymath}
\lambda=\frac{\nu E}{(1+\nu)(1-2\nu)}\,,\quad\mu=\frac{E}{2(1+\nu)}\,.
\end{displaymath}
\item $E>0$, $\mu>0$, $3\lambda+2\mu>0$, and $-1<\nu<0.5$.
\end{itemize}

\end{overprint}

\end{frame}

\begin{frame}{Thermoelasticity}{Isotropic case}

\begin{itemize}
\item Thermal strains in the isotropic case with temperature gradient $\Delta T$:
\begin{displaymath}
\strain_\text{th}=\alpha\Delta T\Id
\end{displaymath}
where $\alpha$ is the coefficient of linear thermal expansion.
\item Total strain tensor:
\begin{displaymath}
\begin{split}
\strain &=\strain_\text{elas}+\strain_\text{th} \\
&=\tenscomp\stress+\alpha\Delta T\Id\,.
\end{split}
\end{displaymath}
\item Linear thermoelastic constitutive equation:
\begin{displaymath}
\boxed{\stress=\tenselas(\strain-\alpha\Delta T\Id)}\,.
\end{displaymath}
\end{itemize}

\end{frame}


\section{4.1 Settlement of a soil layer}

\begin{frame}{Settlement of a soil layer}{Setup}

\begin{columns}[t]
\column{.5\textwidth}
\centering\includegraphics[scale=.4]{\figs/ch4-OneLayer}
\vskip-10pt{\hspace{2.9truecm}\mbox{\tiny{\copyright\ G. Puel}}}
\column{.5\textwidth}
\vskip-90pt
\centering\includegraphics[scale=.15]{\figs/ch4-SoilLayer-photo}
\end{columns}
\begin{displaymath}
\uv(\xv)=\uj_3(\xj_3)\iv_3\,,\quad\xv=(\xj_1,\xj_2,\xj_3)\in\medium=\Rset^2\times]0,H[\,.
\end{displaymath}

\end{frame}

\begin{frame}{Settlement of a soil layer}{Solution}

\begin{overprint}

\onslide<1|handout:1>
\vskip-20pt
\begin{exampleblock}{Question \#0: $\uj_3(\xj_3)\iv_3$?}
\begin{itemize}
\item Symmetry of geometry {\bf AND} loads.
\item Translational invariance in $(\xj_1,\xj_2)\in\Rset^2$:
{\footnotesize
\begin{displaymath}
\uv(\xj_1+\Delta_1,\xj_2+\Delta_2,\xj_3)=\uv(\xj_1,\xj_2,\xj_3)\,,\quad\forall(\Delta_1,\Delta_2)\in\Rset^2\,,\forall\xv\in\medium\,,
\end{displaymath}}
thus $\uv(\xv)=\uv(\xj_3)$.
\item Mirror symmetries:
{\footnotesize
\begin{displaymath}
\begin{split}
\scal{\uv(\xj_3),\iv_1} &=-\scal{\uv(\xj_3),\iv_1}=0\,,\quad\forall\xj_3\in]0,H[\,, \\
\scal{\uv(\xj_3),\iv_2} &=-\scal{\uv(\xj_3),\iv_2}=0\,,\quad\forall\xj_3\in]0,H[\,,
\end{split}
\end{displaymath}}
thus $\uv(\xj_3)=\uj_3(\xj_3)\iv_3$.
\end{itemize}
\end{exampleblock}

\onslide<2|handout:2>
\vskip-20pt
\begin{exampleblock}{Question \#0: $\uj_3(\xj_3)\iv_3$?}
\begin{figure}
\centering\includegraphics[scale=.2]{\figs/ch4-soil-creep}
\end{figure}
\end{exampleblock}

\end{overprint}

\end{frame}

\begin{frame}{Settlement of a soil layer}{Solution}

\begin{overprint}

\onslide<1|handout:1>
\vskip-20pt
\begin{exampleblock}{Question \#1: Differential equation satisfied by $\xj_3\mapsto\uj_3(\xj_3)$?}
\begin{itemize}
\item Local equilibrium equation $\Divx\stress+\fv_v=\roi\acv$, where $\acv=\bzero$ because "we can neglect the effects of inertia," and $\fv_v=-\roi g\iv_3$.
\end{itemize}
\end{exampleblock}

\onslide<2|handout:2>
\vskip-20pt
\begin{exampleblock}{Question \#1: Differential equation satisfied by $\xj_3\mapsto\uj_3(\xj_3)$?}
\begin{itemize}
\item  Local equilibrium equation $\Divx\stress-\roi g\iv_3=\bzero$;
\item Constitutive relation $\stress=\lambda(\trace\strain)\Id+2\mu\strain$, because soil's "behavior is linear elastic and isotropic" and "we can adopt the framework of the infinitesimal deformation hypothesis."
\end{itemize}
\end{exampleblock}

\onslide<3|handout:3>
\vskip-20pt
\begin{exampleblock}{Question \#1: Differential equation satisfied by $\xj_3\mapsto\uj_3(\xj_3)$?}
\begin{itemize}
\item  Local equilibrium equation $\Divx\stress-\roi g\iv_3=\bzero$;
\item Constitutive relation $\stress=\lambda(\trace\strain)\Id+2\mu\strain$;
\item Compute $\strain$:
\begin{displaymath}
\begin{split}
\strain &=\frac{\partial\uv}{\partial\xj_j}\otimes_s\iv_j \\
&=\uj_3'(\xj_3)\iv_3\otimes\iv_3\,,
\end{split}
\end{displaymath}
where $\uj_3'(\xj_3)=\frac{\id\uj_3}{\id\xj_3}$.
\end{itemize}
\end{exampleblock}

\onslide<4|handout:4>
\vskip-20pt
\begin{exampleblock}{Question \#1: Differential equation satisfied by $\xj_3\mapsto\uj_3(\xj_3)$?}
\begin{itemize}
\item Local equilibrium equation $\Divx\stress-\roi g\iv_3=\bzero$;
\item Constitutive relation $\stress=\lambda(\trace\strain)\Id+2\mu\strain$;
\item $\strain=\uj_3'(\xj_3)\iv_3\otimes\iv_3$;
\item Compute $\stress$: 
\begin{displaymath}
\begin{split}
\stress &=\lambda\uj_3'(\xj_3)\Id+2\mu\uj_3'(\xj_3)\iv_3\otimes\iv_3\\
&=\uj_3'(\xj_3)\left[\lambda(\iv_1\otimes\iv_1+\iv_2\otimes\iv_2)+(\lambda+2\mu)\iv_3\otimes\iv_3\right]\,.
\end{split}
\end{displaymath}
\end{itemize}
\end{exampleblock}

\onslide<5|handout:5>
\vskip-20pt
\begin{exampleblock}{Question \#1: Differential equation satisfied by $\xj_3\mapsto\uj_3(\xj_3)$?}
\begin{itemize}
\item Local equilibrium equation $\Divx\stress-\roi g\iv_3=\bzero$;
\item Constitutive relation $\stress=\lambda(\trace\strain)\Id+2\mu\strain$;
\item $\strain=\uj_3'(\xj_3)\iv_3\otimes\iv_3$;
\item $\stress=\uj_3'(\xj_3)\left[\lambda(\iv_1\otimes\iv_1+\iv_2\otimes\iv_2)+(\lambda+2\mu)\iv_3\otimes\iv_3\right]$;
\item Compute $\Divx\stress$ (remind that $(\av\otimes\bv)\cv=\scal{\bv,\cv}\av$): 
\begin{displaymath}
\begin{split}
\Divx\stress &=\frac{\partial\stress}{\partial\xj_j}\iv_j\\
&=(\lambda+2\mu)\uj_3''(\xj_3)\iv_3\,,
\end{split}
\end{displaymath}
where $\uj_3''(\xj_3)=\frac{\id^2\uj_3}{\id\xj_3^2}$.
\end{itemize}
\end{exampleblock}

\onslide<6|handout:6>
\vskip-20pt
\begin{exampleblock}{Question \#1: Differential equation satisfied by $\xj_3\mapsto\uj_3(\xj_3)$?}
\begin{itemize}
\item Local equilibrium equation $\Divx\stress-\roi g\iv_3=\bzero$;
\item Constitutive relation $\stress=\lambda(\trace\strain)\Id+2\mu\strain$;
\item $\strain=\uj_3'(\xj_3)\iv_3\otimes\iv_3$;
\item $\stress=\uj_3'(\xj_3)\left[\lambda(\iv_1\otimes\iv_1+\iv_2\otimes\iv_2)+(\lambda+2\mu)\iv_3\otimes\iv_3\right]$;
\item $\Divx\stress=(\lambda+2\mu)\uj_3''(\xj_3)\iv_3$;
\item From the local equilibrium equation: 
\begin{displaymath}
(\lambda+2\mu)\uj_3''(\xj_3)\iv_3-\roi g\iv_3=\bzero \iff \boxed{\uj_3''(\xj_3)=\frac{\roi g}{\lambda+2\mu}}\,.
\end{displaymath}
\end{itemize}
\end{exampleblock}

\end{overprint}

\end{frame}

\begin{frame}{Settlement of a soil layer}{Solution}

\begin{overprint}

\onslide<1|handout:1>
\vskip-20pt
\begin{exampleblock}{Question \#2: Boundary conditions?}
\begin{itemize}
\item $\{\xj_3\leq 0\}$: a rock mass "fixed and perfectly rigid," hence $\uv(\xj_3\leq 0)=\bzero$ yielding $\uj_3(0)=0$.
\end{itemize}
\end{exampleblock}

\onslide<2|handout:2>
\vskip-20pt
\begin{exampleblock}{Question \#2: Boundary conditions?}
\begin{itemize}
\item $\uj_3(0)=0$;
\item $\{\xj_3=H\}$: "free of forces," hence $\stress\nv\mid_{\xj_3=H}=\bzero$ that is:
\end{itemize}
\end{exampleblock}

\onslide<3|handout:3>
\vskip-20pt
\begin{exampleblock}{Question \#2: Boundary conditions?}
\begin{itemize}
\item $\uj_3(0)=0$;
\item $\{\xj_3=H\}$: "free of forces," hence $\stress\nv\mid_{\xj_3=H}=\bzero$ that is:
\begin{displaymath}
\begin{split}
\stress\nv\mid_{\xj_3=H} &= \stress\iv_3\mid_{\xj_3=H} \\
&= (\lambda+2\mu)\uj_3'(H)\iv_3 \\
(&= \,{\scriptstyle \uj_3'(H)\left[\lambda(\iv_1\otimes\iv_1+\iv_2\otimes\iv_2)+(\lambda+2\mu)\iv_3\otimes\iv_3\right]\iv_3}\,)
\end{split}
\end{displaymath}
Therefore $\uj_3'(H)=0$.
\end{itemize}
\end{exampleblock}

\onslide<4|handout:4>
\vskip-20pt
\begin{exampleblock}{Question \#2: $\xj_3\mapsto\uj_3(\xj_3)$?}
\begin{itemize}
\item $\uj_3(0)=0$;
\item $\uj_3'(H)=0$;
\item Therefore:
\begin{displaymath}
\begin{split}
\uj_3'(\xj_3) &=\frac{\roi g(\xj_3-H)}{\lambda+2\mu}\,, \\
\uj_3(\xj_3) &=\frac{\roi g\xj_3(\xj_3-2H)}{2(\lambda+2\mu)}\,.
\end{split}
\end{displaymath}
\end{itemize}
\end{exampleblock}

\end{overprint}

\end{frame}

\begin{frame}{Settlement of a soil layer}{Solution}

\begin{overprint}

\onslide<1|handout:1>
\vskip-20pt
\begin{exampleblock}{Question \#3: $\stressj_\text{eq}\leq\stressj_e$?}
\begin{itemize}
\item von Mises equivalent stress $\stressj_\text{eq}=\sqrt{3J_2(\stress^D)}$.
\end{itemize}
\end{exampleblock}

\onslide<2|handout:2>
\vskip-20pt
\begin{exampleblock}{Question \#3: $\stressj_\text{eq}\leq\stressj_e$?}
\begin{itemize}
\item von Mises equivalent stress $\stressj_\text{eq}=\sqrt{3J_2(\stress^D)}$;
\item Compute $\smash{\stress^D}=\smash{\stress-\frac{\trace\stress}{3}}\Id$:
\begin{displaymath}
\begin{split}
\stress &= \uj_3'(\xj_3)\left[\lambda(\iv_1\otimes\iv_1+\iv_2\otimes\iv_2)+(\lambda+2\mu)\iv_3\otimes\iv_3\right] \\
\trace\stress &=(3\lambda+2\mu)\uj_3'(\xj_3) \\
\stress^D &=\frac{2}{3}\mu\uj_3'(\xj_3)\left[-\iv_1\otimes\iv_1-\iv_2\otimes\iv_2+2\iv_3\otimes\iv_3\right]
\end{split}
\end{displaymath}
\end{itemize}
\end{exampleblock}

\onslide<3|handout:3>
\vskip-20pt
\begin{exampleblock}{Question \#3: $\stressj_\text{eq}\leq\stressj_e$?}
\begin{itemize}
\item von Mises equivalent stress $\stressj_\text{eq}=\sqrt{3J_2(\stress^D)}$;
\item $\smash{\stress^D}=\smash{\frac{2}{3}\mu\uj_3'(\xj_3)[-\iv_1\otimes\iv_1-\iv_2\otimes\iv_2+2\iv_3\otimes\iv_3]}$;
\item Compute $J_2(\stress^D)=\demi\trace(\stress^{D2})$:
\begin{displaymath}
\begin{split}
\stress^D &=\frac{2}{3}\mu\uj_3'(\xj_3)\left[-\iv_1\otimes\iv_1-\iv_2\otimes\iv_2+2\iv_3\otimes\iv_3\right] \\
(\stress^D)^2 &= \frac{4}{9}(\mu\uj_3'(\xj_3))^2\left[\iv_1\otimes\iv_1+\iv_2\otimes\iv_2+4\iv_3\otimes\iv_3\right] \\
J_2(\stress^D) &=  \frac{4}{3}(\mu\uj_3'(\xj_3))^2\,.
\end{split}
\end{displaymath}
Therefore $\stressj_\text{eq} = 2\mu\abs{\uj_3'(\xj_3)}$.
\end{itemize}
\end{exampleblock}

\onslide<4|handout:4>
\vskip-20pt
\begin{exampleblock}{Question \#3: $\stressj_\text{eq}\leq\stressj_e$?}
\begin{itemize}
\item von Mises equivalent stress $\stressj_\text{eq}=\sqrt{3J_2(\stress^D)}$;
\item $\smash{\stress^D}=\smash{\frac{2}{3}\mu\uj_3'(\xj_3)[-\iv_1\otimes\iv_1-\iv_2\otimes\iv_2+2\iv_3\otimes\iv_3]}$;
\item $\stressj_\text{eq} = 2\mu\abs{\uj_3'(\xj_3)}$;
\item $\stressj_\text{eq} \leq \stressj_e$ provided that:
\begin{displaymath}
\begin{split}
\frac{2\mu}{\lambda+2\mu}\roi g \abs{\xj_3-H} &\leq \stressj_e\,,\quad\forall\xj_3\in[0,H]\,.
\end{split}
\end{displaymath}
\end{itemize}
\end{exampleblock}

\onslide<5|handout:5>
\vskip-20pt
\begin{exampleblock}{Question \#3: $\stressj_\text{eq}\leq\stressj_e$?}
\begin{itemize}
\item $\stressj_\text{eq} \leq \stressj_e$ provided that:
\begin{displaymath}
\begin{split}
\frac{2\mu}{\lambda+2\mu}\roi g \abs{\xj_3-H} &\leq \stressj_e\,,\quad\forall\xj_3\in[0,H]\,.
\end{split}
\end{displaymath}
\item This criterion is first reached where $\abs{\xj_3-H}$ is maximum \emph{i.e.} $\xj_3=0$;
\item Conversely, the von Mises criterion is fulfilled provided that $H\leq H_\text{max}$ where:
\begin{displaymath}
\boxed{H_\text{max}=\frac{(\lambda+2\mu)\stressj_e}{2\mu \roi g}}\,.
\end{displaymath}
\end{itemize}
\end{exampleblock}

\onslide<6|handout:6>
\vskip-20pt
\begin{exampleblock}{Question \#3: $\stressj_\text{eq}\leq\stressj_e$?}
\begin{itemize}
\item $\stressj_\text{eq} \leq \stressj_e$ provided that $H\leq H_\text{max}$ where:
\begin{displaymath}
H_\text{max}=\frac{(\lambda+2\mu)\stressj_e}{2\mu \roi g}\,.
\end{displaymath}
\item {\bf NB}: $\mu=\frac{E}{2(1+\nu)}$, $\lambda=\frac{\nu E}{(1+\nu)(1-2\nu)}$, thus:
\begin{displaymath}
H_\text{max}=\left(\frac{1-\nu}{1-2\nu}\right)\frac{\stressj_e}{\roi g}
\end{displaymath}
independently of the Young's modulus $E$.
\end{itemize}
\end{exampleblock}

\end{overprint}

\end{frame}

\begin{frame}{Settlement of a soil layer}{Solution}

\begin{overprint}

\onslide<1|handout:1>
\vskip-20pt
\begin{exampleblock}{Question \#4: Two layers?}
\begin{columns}[t]
\column{.5\textwidth}
\centering\includegraphics[scale=.35]{\figs/ch4-TwoLayers}
\vskip-7pt{\hspace{2.7truecm}\mbox{\tiny{\copyright\ G. Puel}}}
\column{.5\textwidth}
\vskip-85pt
\centering\includegraphics[scale=.15]{\figs/ch4-SoilLayer-photo}
\end{columns}
\begin{itemize}
\item Local equilibrium equation: unchanged! $\Divx\stress=\roi g\iv_3$.
\end{itemize}
\end{exampleblock}

\onslide<2|handout:2>
\vskip-20pt
\begin{exampleblock}{Question \#4: Two layers?}
\begin{columns}[t]
\column{.5\textwidth}
\centering\includegraphics[scale=.35]{\figs/ch4-TwoLayers}
\vskip-7pt{\hspace{2.7truecm}\mbox{\tiny{\copyright\ G. Puel}}}
\column{.5\textwidth}
\vskip-85pt
\centering\includegraphics[scale=.15]{\figs/ch4-SoilLayer-photo}
\end{columns}
\begin{itemize}
\item Constitutive relation:
\begin{displaymath}
\stress=\left\{
\begin{split}
 &=\lambda_1(\trace\strain)\Id+2\mu_1\strain\,,\quad\xv\in\medium_1=\Rset^2\times\left]0,\frac{H}{2}\right[\,, \\
 &=\lambda_2(\trace\strain)\Id+2\mu_2\strain\,,\quad\xv\in\medium_2=\Rset^2\times\left]\frac{H}{2},H\right[\,. \\
\end{split}
\right.
\end{displaymath}
\end{itemize}
\end{exampleblock}

\onslide<3|handout:3>
\vskip-20pt
\begin{exampleblock}{Question \#4: Two layers?}
\begin{columns}[t]
\column{.5\textwidth}
\centering\includegraphics[scale=.35]{\figs/ch4-TwoLayers}
\vskip-7pt{\hspace{2.7truecm}\mbox{\tiny{\copyright\ G. Puel}}}
\column{.5\textwidth}
\vskip-85pt
\centering\includegraphics[scale=.15]{\figs/ch4-SoilLayer-photo}
\end{columns}
\begin{itemize}
\item Boundary conditions:
\begin{displaymath}
\begin{split}
\scriptstyle \uv(\xj_3)=\bzero & \scriptstyle \quad\text{on}\;\{\xj_3=0\} \,, \\
\scriptstyle \stress(\xj_3)\iv_3=\bzero & \scriptstyle \quad\text{on}\;\{\xj_3=H\} \,, \\
\scriptstyle \uv(\xj_3)\mid_{\medium_1}=\uv(\xj_3)\mid_{\medium_2}\,,\; \stress(\xj_3)\iv_3\mid_{\medium_1}+\stress(\xj_3)(-\iv_3)\mid_{\medium_2}=\bzero  &  \scriptstyle \quad\text{on}\;\left\{\xj_3=\frac{H}{2}\right\}\,.
\end{split}
\end{displaymath}
\end{itemize}
\end{exampleblock}

\onslide<4|handout:4>
\vskip-20pt
\begin{exampleblock}{Question \#4: Two layers?}
\begin{itemize}
\item From Question \#1:
\begin{displaymath}
\begin{split}
\scriptstyle \uj_3(\xj_3)\mid_{\medium_1} & \scriptstyle =\frac{\roi_1 g\xj_3^2}{2(\lambda_1+2\mu_1)}+A_1\xj_3+B_1 \,, \\
\scriptstyle \uj_3(\xj_3)\mid_{\medium_2} & \scriptstyle =\frac{\roi_2 g\xj_3^2}{2(\lambda_2+2\mu_2)}+A_2\xj_3+B_2 \,.
\end{split}
\end{displaymath}
\item From boundary conditions:
\begin{displaymath}
\begin{split}
\scriptstyle 0 \, & \scriptstyle = \, \frac{\roi_2 gH}{\lambda_2+2\mu_2}+A_2 \,, \\
\scriptstyle \roi_1 g\frac{H}{2}+(\lambda_1+2\mu_1)A_1 \, & \scriptstyle =  \, \roi_2 g\frac{H}{2}+(\lambda_2+2\mu_2)A_2  \,, \\
\scriptstyle \frac{\roi_1 g H^2}{8(\lambda_1+2\mu_1)}+A_1\frac{H}{2}+B_1 \, & \scriptstyle = \, \frac{\roi_2 g H^2}{8(\lambda_2+2\mu_2)}+A_2\frac{H}{2}+B_2 \,, \\
\scriptstyle B_1 \, & \scriptstyle = \, 0 \,.
\end{split}
\end{displaymath}
\end{itemize}
\end{exampleblock}

\onslide<5|handout:5>
\vskip-20pt
\begin{exampleblock}{Question \#4: Two layers?}
\begin{itemize}
\item From Question \#1:
\begin{displaymath}
\begin{split}
\scriptstyle \uj_3(\xj_3)\mid_{\medium_1} & \scriptstyle =\frac{\roi_1 g\xj_3^2}{2(\lambda_1+2\mu_1)}+A_1\xj_3+B_1 \,, \\
\scriptstyle \uj_3(\xj_3)\mid_{\medium_2} & \scriptstyle =\frac{\roi_2 g\xj_3^2}{2(\lambda_2+2\mu_2)}+A_2\xj_3+B_2 \,.
\end{split}
\end{displaymath}
\item From boundary conditions:
\begin{displaymath}
\begin{split}
\scriptstyle A_2 \, & \scriptstyle = \, -\frac{\roi_2 gH}{\lambda_2+2\mu_2} \,, \\
\scriptstyle A_1 \, & \scriptstyle = \, -\frac{(\roi_1+\roi_2) gH}{2(\lambda_1+2\mu_1)}  \,, \\
\scriptstyle B_2 \, & \scriptstyle = \, \frac{gH^2}{8}\left(\frac{3\roi_2}{\lambda_2+2\mu_2}-\frac{\roi_1+2\roi_2}{\lambda_1+2\mu_1}\right) \,, \\
\scriptstyle B_1 \, & \scriptstyle = \, 0 \,.
\end{split}
\end{displaymath}
\end{itemize}
\end{exampleblock}

\end{overprint}

\end{frame}

\end{document}

