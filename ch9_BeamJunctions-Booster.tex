% MG3 TD #9: 
% V1.0 March 2021
% $Header: /cvsroot/latex-beamer/latex-beamer/solutions/generic-talks/generic-ornate-15min-45min.en.tex,v 1.5 2007/01/28 20:48:23 tantau Exp $
\def\webDOI{http://dx.doi.org}
\def\Folder{/Users/ericsavin/Documents/Cours/SG3-MMC/SLIDES_TD/}
\def\Year{\Folder/2020-2021}
\def\Sections{\Year/SECTIONS}
\def\figs{\Folder/FIGS}
%\def\figs{/Users/ericsavin/Documents/Cours/SG3-MMC/SLIDES_TD/FIGS}
\def\figdynsto{/Users/ericsavin/Documents//Figures/DYNSTO}
\def\symb{/Users/ericsavin/Documents/Latex/SYMBOL}
\def\fonts{/Users/ericsavin/Documents/Latex/FONTS}
\def\logos{/Users/ericsavin/Documents/Latex/LOGOS}
\def\Onera{ONERA}
\def\ECP{CentraleSup\'elec}


\documentclass{beamer}

% This file is a solution template for:

% - Giving a talk on some subject.
% - The talk is between 15min and 45min long.
% - Style is ornate.



% Copyright 2004 by Till Tantau <tantau@users.sourceforge.net>.
%
% In principle, this file can be redistributed and/or modified under
% the terms of the GNU Public License, version 2.
%
% However, this file is supposed to be a template to be modified
% for your own needs. For this reason, if you use this file as a
% template and not specifically distribute it as part of a another
% package/program, I grant the extra permission to freely copy and
% modify this file as you see fit and even to delete this copyright
% notice. 


\mode<presentation>
{
  \usetheme{Berkeley}
  % or ...

  \setbeamercovered{transparent}
  % or whatever (possibly just delete it)
}


\usepackage[english]{babel}
% or whatever

\usepackage[latin1]{inputenc}
% or whatever

%\usepackage{mathtime}
\usefonttheme{serif}
%\usefonttheme{professionalfonts}
\usepackage{amsfonts}
\usepackage{amssymb}

\usepackage{amsmath}
\usepackage{multimedia}
\usepackage{mathrsfs}
\usepackage{mathabx}
\usepackage{color}
\usepackage{pstricks}
\usepackage{graphicx}
%\usepackage[pdftex, pdfborderstyle={/S/U/W 1}]{hyperref}
\usepackage{hyperref}
\usepackage{bbm}
\usepackage{cancel}
\usepackage[Symbol]{upgreek}
%\usepackage{mathbbol}
%\DeclareSymbolFontAlphabet{\amsmathbb}{AMSb}
%\usepackage[bbgreekl]{mathbbol}
%\usepackage[mtpbbi]{mtpro2}

%\usepackage[svgnames]{xcolor}

%\input{\fonts/math0}
%\input{\symb/structac} % Notations E. Savin
%\input{\symb/logos}

\newcommand{\ci}{\mathrm{i}}
\newcommand{\trace}{\operatorname{Tr}}
\newcommand{\Nset}{\mathbb{N}}
\newcommand{\Zset}{\mathbb{Z}}
\newcommand{\Rset}{\mathbb{R}}
\newcommand{\Cset}{\mathbb{C}}
\newcommand{\Sset}{\mathbb{S}}
\newcommand{\Mset}{\mathbb{M}}
\newcommand{\PhaseSpace}{\Omega}
\newcommand{\ContSet}{{\mathcal C}}
\newcommand{\id}{d}
\newcommand{\iD}{\mathrm{D}}
\newcommand{\iexp}{\mathrm{e}}
\newcommand{\demi}{\frac{1}{2}}
\newcommand{\imply}{\Rightarrow}

% Algebra
\newcommand{\itr}{{\sf T}}
\newcommand{\Id}{{\boldsymbol I}}
\newcommand{\IId}{\mathbb{I}}
\newcommand{\aj}{a}
\newcommand{\bj}{b}
\newcommand{\cj}{c}
\renewcommand{\dj}{d}
\newcommand{\av}{{\boldsymbol\aj}}
\newcommand{\bv}{{\boldsymbol\bj}}
\newcommand{\cv}{{\boldsymbol\cj}}
\newcommand{\dv}{{\boldsymbol\dj}}
\newcommand{\uj}{u}
\newcommand{\vj}{v}
\newcommand{\xj}{x}
\newcommand{\yj}{y}
\newcommand{\zj}{z}
\newcommand{\uv}{{\boldsymbol\uj}}
\newcommand{\vv}{{\boldsymbol\vj}}
\newcommand{\xv}{{\boldsymbol\xj}}
\newcommand{\yv}{{\boldsymbol\yj}}
\newcommand{\zv}{{\boldsymbol\zj}}
\newcommand{\Aj}{A}
\newcommand{\Bj}{B}
\newcommand{\Av}{{\boldsymbol\Aj}}
\newcommand{\Bv}{{\boldsymbol\Bj}}
\newcommand{\Zgv}{{\boldsymbol Z}}

% Analysis
\newcommand{\grad}{{\boldsymbol\nabla}}
\newcommand{\gradx}{{\grad_\xv}}
\newcommand{\Grad}{{\mathbb D}}
\newcommand{\Gradx}{{\Grad_\xv}}
\renewcommand{\div}{\mathrm{div}}
\newcommand{\divx}{{\div_\xv}}
\newcommand{\Div}{\mathbf{Div}}
\newcommand{\Divx}{{\Div_\xv}}

% Kinematics
\newcommand{\ej}{e}
\renewcommand{\ij}{i}
\newcommand{\pj}{p}
\newcommand{\ev}{{\boldsymbol\ej}}
\newcommand{\iv}{{\boldsymbol\ij}}
\newcommand{\pv}{{\boldsymbol\pj}}
\newcommand{\posij}{f}
\newcommand{\posiv}{{\boldsymbol\posij}}
\newcommand{\iposiv}{{\boldsymbol g}}
\newcommand{\Fp}{{\mathbb F}}
\newcommand{\GreenLj}{E}
\newcommand{\GreenL}{{\mathbb\GreenLj}}
\newcommand{\medium}{\Omega}
\newcommand{\strainj}{\varepsilon}
\newcommand*{\strain}{\mbox{$\hspace{0.2em}\rotatebox[x=0pt,y=0.2pt]{90}{\rule{0.02\linewidth}{0.4pt}}\hspace{-0.23em}\upvarepsilon$}}
\newcommand*{\rotation}{{\boldsymbol R}}
\newcommand*{\xiu}{\chi_1}
\newcommand*{\xid}{\chi_2}
\newcommand*{\Rot}{{\boldsymbol R}}
\newcommand*{\dRot}{{\boldsymbol\Theta}}
\newcommand*{\drotj}{\theta}
\newcommand*{\drot}{{\boldsymbol\drotj}}
\newcommand*{\Mstaticj}{J}
\newcommand*{\Mstatic}{{\mathbb\Mstaticj}}

% Dynamics
\newcommand{\fj}{f}
\newcommand{\Fj}{F}
\newcommand{\gj}{g}
\newcommand{\mj}{m}
\newcommand{\nj}{n}
\newcommand{\Rj}{R}
\newcommand{\Tj}{T}
%\newcommand{\cv}{{\boldsymbol c}}
\newcommand{\fv}{{\boldsymbol\fj}}
\newcommand{\Fv}{{\boldsymbol\Fj}}
\newcommand{\gv}{{\boldsymbol\gj}}
\newcommand{\mv}{{\boldsymbol\mj}}
\newcommand{\nv}{{\boldsymbol\nj}}
\newcommand{\Rv}{{\boldsymbol\Rj}}
\newcommand{\Tv}{{\boldsymbol\Tj}}
\newcommand{\roi}{\varrho}
\newcommand*{\stressj}{\sigma}
\newcommand*{\tressj}{\tau}
\newcommand{\stress}{\mbox{$\hspace{0.3em}\rotatebox[x=0pt,y=0.2pt]{90}{\rule{0.017\linewidth}{0.4pt}}\hspace{-0.25em}\upsigma$}}
\newcommand*{\tress}{{\boldsymbol\tressj}}
\newcommand{\acj}{a}
\newcommand{\acv}{{\boldsymbol\acj}}
\newcommand{\Fresj}{R}
\newcommand{\Mresj}{M}
\newcommand{\Fres}{{\boldsymbol\Fresj}}
\newcommand{\Mres}{{\boldsymbol\Mresj}}

%\newcommand{\Zg}{{\bf\zgj}}
%\newcommand{\xigj}{\xi}
%\newcommand{\xig}{{\boldsymbol\xigj}}
\newcommand{\kgj}{k}
%\newcommand{\kgh}{\kgj_\ygj}
%\newcommand{\kg}{{\bf\kgj}}
\newcommand{\Kg}{{\bf K}}
%\newcommand{\qg}{{\boldsymbol q}}
%\newcommand{\pg}{{\boldsymbol p}}
\newcommand{\hkg}{{\hat \kg}}
\newcommand{\hpg}{\hat{\pg}}
%\newcommand{\vg}{{\boldsymbol v}}
\newcommand{\sg}{{\boldsymbol s}}
%\newcommand{\stress}{\mathbb{\sigma}}
\newcommand{\tenselasj}{{\Large C}}
\newcommand{\tenselas}{\boldsymbol{\mathsf{\tenselasj}}}
\newcommand{\tenscomp}{\boldsymbol{\mathsf{\Large S}}}
\newcommand{\speci}{{\mathrm w}}
\newcommand{\specij}{{\mathrm W}}
\newcommand{\speciv}{{\bf \specij}}
%\newcommand{\cjg}[1]{\overline{#1}}
\newcommand{\eigv}{{\bf b}}
\newcommand{\eigw}{{\bf c}}
\newcommand{\eigl}{\lambda}
\newcommand{\jeig}{\alpha}
\newcommand{\keig}{\beta}
\newcommand{\cel}{c}
\newcommand{\bcel}{{\bf\cel}}
\newcommand{\deng}{{\mathcal E}}
\newcommand{\flowj}{\pi}
\newcommand{\flow}{\boldsymbol\flowj}
\newcommand{\Flowj}{\Pi}
\newcommand{\Flow}{\boldsymbol\Flowj}
\newcommand{\fluxinj}{g}
\newcommand{\fluxin}{{\bf\fluxinj}}
\newcommand{\dscat}{\sigma}
\newcommand{\tdscat}{\Sigma}
\newcommand{\collop}{{\mathcal Q}}
\newcommand{\epsd}{\delta}
\newcommand{\rscat}{\rho}
\newcommand{\tscat}{\tau}
\newcommand{\Rscat}{\mathcal{R}}
\newcommand{\Tscat}{\mathcal{T}}
\newcommand{\lscat}{\ell}
%\newcommand{\floss}{\eta}
\newcommand{\mdiff}{{\bf D}}
%\newcommand{\demi}{\frac{1}{2}}
\newcommand{\domain}{{\mathcal O}}
\newcommand{\bdomain}{{\mathcal D}}
\newcommand{\interface}{\Gamma}
\newcommand{\sinterface}{\gamma_D}
%\newcommand{\normal}{\hat{\bf n}}
\newcommand{\bnabla}{\boldsymbol\nabla}
%\newcommand{\esp}[1]{\mathbb{E}\{\smash{#1}\}}
\newcommand{\mean}[1]{\underline{#1}}
%\newcommand{\BB}{\mathbb{B}}
%\newcommand{\II}{{\boldsymbol I}}
%\newcommand{\TA}{\boldsymbol{\Gamma}}
\newcommand{\Mdisp}{{\mathbf H}}
\newcommand{\Hamil}{{\mathcal H}}
\newcommand{\bzero}{{\bf 0}}

\newcommand{\mass}{M}
\newcommand{\damp}{D}
\newcommand{\stif}{K}
\newcommand{\dsp}{S}
\newcommand{\dof}{q}
\newcommand{\pof}{p}
%\newcommand{\MM}{{\boldsymbol\mass}}
\newcommand{\MD}{{\boldsymbol\damp}}
\newcommand{\MK}{{\boldsymbol\stif}}
\newcommand{\MS}{{\boldsymbol\dsp}}
\newcommand{\Cov}{{\boldsymbol C}}
\newcommand{\dofg}{{\boldsymbol\dof}}
\newcommand{\pofg}{{\boldsymbol\pof}}
\newcommand{\driftj}{b}
\newcommand{\drifts}{{\boldsymbol \driftj}}
\newcommand{\drift}{{\underline\drifts}}
\newcommand{\scatj}{a}
\newcommand{\scat}{{\boldsymbol\scatj}}
\newcommand{\diff}{{\boldsymbol\sigma}}
\newcommand{\load}{F}
\newcommand{\loadg}{{\boldsymbol\load}}
\newcommand{\pdf}{\pi}
\newcommand{\tpdf}{\pdf_t}
\newcommand{\fg}{{\boldsymbol f}}
%\newcommand{\Ugj}{U}
\newcommand{\Vgj}{V}
\newcommand{\Xgj}{X}
\newcommand{\Ygj}{Y}
%\newcommand{\Ug}{{\boldsymbol\Ugj}}
\newcommand{\Vg}{{\boldsymbol\Vgj}}
%\newcommand{\Qg}{{\boldsymbol Q}}
\newcommand{\Pg}{{\boldsymbol P}}
\newcommand{\Xg}{{\boldsymbol\Xgj}}
\newcommand{\Yg}{{\boldsymbol\Ygj}}
\newcommand{\flux}{{\boldsymbol J}}
\newcommand{\wiener}{W}
\newcommand{\whitenoise}{B}
\newcommand{\Wiener}{{\boldsymbol\wiener}}
\newcommand{\White}{{\boldsymbol\white}}
\newcommand{\paraj}{\nu}
\newcommand{\parag}{{\boldsymbol\paraj}}
\newcommand{\parae}{\hat{\parag}}
\newcommand{\erroj}{\epsilon}
\newcommand{\error}{{\boldsymbol\erroj}}
\newcommand{\biaj}{b}
\newcommand{\bias}{{\boldsymbol\biaj}}
%\newcommand{\disp}{{\boldsymbol V}}
\newcommand{\Fisher}{{\mathcal I}}
\newcommand{\likelihood}{{\mathcal L}}

\newcommand{\heps}{\varepsilon}
%\newcommand{\roi}{\varrho}
\newcommand{\jump}[1]{\llbracket{#1}\rrbracket}
\newcommand{\scal}[1]{\left\langle{#1}\right\rangle}
\newcommand{\norm}[1]{\left\|#1\right\|}
\newcommand{\abs}[1]{\left|#1\right|}
%\newcommand{\po}{\operatorname{o}}
\newcommand{\FFT}[1]{\widehat{#1}}
\newcommand{\indic}[1]{{\mathbf 1}_{#1}}
\newcommand{\impulse}{{\mathbbm h}}
\newcommand{\frf}{\FFT{\impulse}}

%\renewcommand{\Moy}[1]{{\boldsymbol\mu}_{#1}}
%\renewcommand{\Rcor}[1]{{\boldsymbol R}_{#1}}
%\renewcommand{\Mw}[1]{{\boldsymbol M}_{#1}}
%\renewcommand{\Sw}[1]{{\boldsymbol S}_{#1}}
%\renewcommand{\esp}[1]{{\mathbb E}\{#1\}}

\newcommand{\emphb}[1]{\textcolor{blue}{#1}}
\newcommand{\mycite}[1]{\textcolor{red}{#1}}
\newcommand{\mycitb}[1]{\textcolor{red}{[{\it #1}]}}

\newcommand{\PDFU}{{\mathcal U}}
\newcommand{\PDFN}{{\mathcal N}}
\newcommand{\TK}{{\boldsymbol\Pi}}
\newcommand{\TKij}{\pi}
\newcommand{\TKi}{{\boldsymbol\pi}}
\newcommand{\SMi}{\TKij^*}
\newcommand{\SM}{\TKi^*}
\newcommand{\lagmuli}{\lambda}
\newcommand{\lagmul}{{\boldsymbol\lagmuli}}
\newcommand{\constraint}{{\boldsymbol C}}
\newcommand{\mconstraint}{\mean{\constraint}}

\newcommand{\mybox}[1]{\fbox{\begin{minipage}{0.93\textwidth}{#1}\end{minipage}}}
\newcommand{\defcolor}[1]{\textcolor{blue}{#1}}

%\definecolor{rose}{LightPink}%{rgb}{251,204,231}

\newtheorem{mydef}{Definition}
\newtheorem{mythe}{Theorem}
\newtheorem{myprop}{Proposition}

% Or whatever. Note that the encoding and the font should match. If T1
% does not look nice, try deleting the line with the fontenc.

\title[1EL5000/S9]
{Mechanical strength of a rocket booster}

\subtitle{1EL5000--Continuum Mechanics -- Tutorial Class \#9} % (optional)

\author[\'E. Savin] % (optional, use only with lots of authors)
{\'E. Savin\inst{1,2}\\ \scriptsize{\texttt{eric.savin@\{centralesupelec,onera\}.fr}}}%\inst{1} }
% - Use the \inst{?} command only if the authors have different
%   affiliation.

\institute[Onera] % (optional, but mostly needed)
{\inst{1}{Information Processing and Systems Dept.\\\Onera, France}
\and
 \inst{2}{Mechanical and Civil Engineering Dept.\\\ECP, France}}%
%  Department of Theoretical Philosophy\\
%  University of Elsewhere}
% - Use the \inst command only if there are several affiliations.
% - Keep it simple, no one is interested in your street address.

%\date[Short Occasion] % (optional)
\date{\today}

\subject{Rocket booster}
% This is only inserted into the PDF information catalog. Can be left
% out. 



% If you have a file called "university-logo-filename.xxx", where xxx
% is a graphic format that can be processed by latex or pdflatex,
% resp., then you can add a logo as follows:

% \pgfdeclareimage[height=0.5cm]{university-logo}{university-logo-filename}
% \logo{\pgfuseimage{university-logo}}



% Delete this, if you do not want the table of contents to pop up at
% the beginning of each subsection:
\AtBeginSection[]
%\AtBeginSubsection[]
{
  \begin{frame}<beamer>{Outline}
    \tableofcontents[currentsection]%,currentsubsection]
  \end{frame}
}


% If you wish to uncover everything in a step-wise fashion, uncomment
% the following command: 

%\beamerdefaultoverlayspecification{<+->}


\begin{document}

\begin{frame}
  \titlepage
\end{frame}

\begin{frame}{Outline}
  \tableofcontents
  % You might wish to add the option [pausesections]
\end{frame}


% Since this a solution template for a generic talk, very little can
% be said about how it should be structured. However, the talk length
% of between 15min and 45min and the theme suggest that you stick to
% the following rules:  

% - Exactly two or three sections (other than the summary).
% - At *most* three subsections per section.
% - Talk about 30s to 2min per frame. So there should be between about
%   15 and 30 frames, all told.

\section{Kinematics}
\begin{frame}{Beam kinematics}{Reference configuration}

\begin{figure}
\centering\includegraphics[scale=.25]{\figs/ch7-Kinematics-p}
\end{figure}

\begin{displaymath}
\begin{split}
\pv(s,\xiu,\xid) &=\pv_G(s)+\pv_\Sigma(\xiu,\xid) \\
&= s\ev+\chi_1\ev_{\chi_1}+\chi_2\ev_{\chi_2}
\end{split}
\end{displaymath}

\end{frame}

\begin{frame}{Beam kinematics}{Actual configuration}

\vskip-10pt
%\begin{figure}
%\centering\includegraphics[scale=.3]{\figs/ch7-Kinematics-x}
%\end{figure}
\begin{columns}
\column{.4\textwidth}
\centering\includegraphics[scale=.25]{\figs/ch7-Kinematics-x}
\column{.6\textwidth}
\centering\includegraphics[scale=.25]{\figs/ch7-Accordeon}
\end{columns}

%\vskip-10pt
\begin{displaymath}
\begin{split}
\xv &=\xv_G(s)+\xv_\Sigma \\
&= \xv_G(s)+\Rot(s)\pv_\Sigma
\end{split}
\end{displaymath}

\end{frame}

\begin{frame}{Beam kinematics}{Small perturbations -- Timoshenko}

\vskip-10pt
\begin{figure}
\centering\includegraphics[scale=.2]{\figs/ch7-Kinematics-Timoshenko}
\end{figure}
\vskip-10pt
\begin{itemize}
\item Small perturbations $\Rot(s)=\Id+\dRot(s)$, $\dRot(s)^\itr=-\dRot(s)$:
\begin{displaymath}
\begin{split}
\xv_\Sigma &= \Rot(s)\pv_\Sigma \\
&=(\Id+\dRot(s))\pv_\Sigma\,.
\end{split}
\end{displaymath}
\item Small displacement $\xv_\Sigma\simeq\pv_\Sigma$:
\begin{displaymath}
\begin{split}
\uv &= \xv-\pv \\
&=\uv_G(s)+\drot(s)\times\xv_\Sigma\,.
\end{split}
\end{displaymath}
\end{itemize}

\end{frame}

\begin{frame}{Beam kinematics}{Small perturbations -- Euler-Bernoulli}

\vskip-10pt
\begin{figure}
\centering\includegraphics[scale=.3]{\figs/ch7-Kinematics-EBernoulli}
\end{figure}
\vskip-10pt
\begin{itemize}
\item Cross-sections remain perpendicular to the neutral line:
\begin{displaymath}
\Rot(s)\ev=\frac{\xv_G'(s)}{\norm{\xv_G'(s)}}\,.
\end{displaymath}
\item Small perturbations $\Rot(s)\ev=(\Id+\dRot(s))\ev\simeq\ev+\uv_G'(s)$:
\begin{displaymath}
\boxed{\drot_\Sigma(s)=\ev\times\uv_{G\Sigma}'(s)}
\end{displaymath}
\end{itemize}

\end{frame}

\begin{frame}{Recap: 1.3 Large beam bending}

\vskip-20pt
\begin{figure}
\centering\includegraphics[scale=.2]{\figs/ch1-BeamStraight-c}
\end{figure}
\vskip-10pt
\begin{columns}
\column{.5\textwidth}
\centering\includegraphics[scale=.2]{\figs/ch1-BeamHalfCircle-c}\\
$\alpha=1$
\column{.5\textwidth}
\centering\includegraphics[scale=.2]{\figs/ch1-BeamHalfCircleAlpha-c}\\
$\alpha=1.1$
\end{columns}

\end{frame}

\begin{frame}{Beam kinematics}{Independent unknowns}

\begin{figure}
\centering\includegraphics[scale=.2]{\figs/ch7-KinUnknowns}
\end{figure}

\begin{itemize}
\item Elongation $\uj_{Ge}=\scal{\uv_G,\ev}$;
\item Deflection $\uv_{G\Sigma}=\uv_G-\uj_{Ge}\ev$;
\item Torsion rotation $\drotj_e=\scal{\drot,\ev}$.
\end{itemize}

\end{frame}


\section{Statics}
\definecolor{ultramarine}{rgb}{0.07, 0.04, 0.56} 
\definecolor{aqua}{rgb}{0, 255, 255}

\begin{frame}{Beam statics}{Resultant forces}

\begin{figure}
\centering\includegraphics[scale=.2]{\figs/ch7-Forces}
\end{figure}

\begin{itemize}
\item Resultant force:
\begin{displaymath}
\Fres(s)=\int_\Sigma\stress\ev\,\id S\,;
\end{displaymath}
\item Normal force $\Fresj_e(s)=\scal{\Fres(s),\ev}$;
\item Shear force $\Fres_\Sigma(s)=\Fres(s)-\Fresj_e(s)\ev$;
\item $\Fres(0)$ and $\Fres(L)$ given by the boundary conditions.
\end{itemize}

\end{frame}

\begin{frame}{Beam statics}{Resultant moments}

\begin{figure}
\centering\includegraphics[scale=.2]{\figs/ch7-Moments}
\end{figure}

\begin{itemize}
\item Resultant moment:
\begin{displaymath}
\Mres(s)=\int_\Sigma\xv_\Sigma\times\stress\ev\,\id S\,;
\end{displaymath}
\item Torsion moment $\Mresj_e(s)=\scal{\Mres(s),\ev}$;
\item Bending moment $\Mres_\Sigma(s)=\Mres(s)-\Mresj_e(s)\ev$;
\item $\Mres(0)$ and $\Mres(L)$ given by the boundary conditions.
\end{itemize}

\end{frame}

\begin{frame}{Beam statics}{Local balance of forces}

\begin{figure}
\centering\includegraphics[scale=.25]{\figs/ch7-EqForces}
\end{figure}

\begin{itemize}
\item Linear external forces:
\begin{displaymath}
\fv_l(s)=\int_\Sigma\fv_v\id S + \int_{\partial\Sigma}\fv_s\id\zeta\,;
\end{displaymath}
\item Local equilibrium of the cross-section:
\begin{displaymath}
\Fres'(s)+\fv_l(s)=\bzero\,.
\end{displaymath}
\end{itemize}

\end{frame}

\begin{frame}{Beam statics}{Local balance of moments}

\begin{figure}
\centering\includegraphics[scale=.25]{\figs/ch7-EqMoments}
\end{figure}

\begin{itemize}
\item Linear external torques:
\begin{displaymath}
\cv_l(s)=\int_\Sigma\xv_\Sigma\times\fv_v\id S + \int_{\partial\Sigma}\xv_\Sigma\times\fv_s\id\zeta\,;
\end{displaymath}
\item Local equilibrium of the cross-section:
\begin{displaymath}
\Mres'(s)+\ev\times\Fres(s)+\cv_l(s)=\bzero\,.
\end{displaymath}
\end{itemize}

\end{frame}

\begin{frame}{Beam statics}{Local balance of forces--alternative point of view}

\begin{itemize}
\item Local static equilibrium of a continuum medium:
\begin{displaymath}
\Div\stress+\fv_v=\bzero\,.
\end{displaymath}
\item Then integrate over the cross-section $\Sigma$:
\begin{displaymath}
\begin{split}
\bzero &=\int_\Sigma\Div\stress\,\id S+\int_\Sigma\fv_v\,\id S \\
&=\int_\Sigma\frac{\partial\stress}{\partial s}\ev\,\id S + \int_\Sigma\Div_\Sigma\stress\,\id S +\int_\Sigma\fv_v\,\id S \\
&=\frac{\partial}{\partial s}\left(\int_\Sigma\stress\ev\,\id S\right)+\int_{\partial\Sigma}\stress\nv\,\id\zeta +\int_\Sigma\fv_v\,\id S \\
&=\Fres'(s)+ \int_{\partial\Sigma}\fv_s\,\id\zeta +\int_\Sigma\fv_v\,\id S \\
&=\Fres'(s)+\fv_l(s)\,.
\end{split}
\end{displaymath}
\end{itemize}

\end{frame}

\begin{frame}{Beam statics}{Local balance of moments--alternative point of view}

\begin{itemize}
%\item Local static equilibrium of a continuum medium:
%\begin{displaymath}
%\Div\stress+\fv_v=\bzero\,.
%\end{displaymath}
\item Then integrate over the cross-section $\Sigma$:
\begin{displaymath}
\begin{split}
\!\!\!\!\!\!\!\!\!\!\!\!\!\! \bzero &=\int_\Sigma\xv_\Sigma\times\Div\stress\,\id S+\int_\Sigma\xv_\Sigma\times\fv_v\,\id S \\
\!\!\!\!\!\!\!\!\!\!\!\!\!\! &= \int_\Sigma\xv_\Sigma\times\frac{\partial\stress}{\partial s}\ev\,\id S + \int_\Sigma\xv_\Sigma\times\frac{\partial\stress\ev_\alpha}{\partial\xj_\alpha}\,\id S +\int_\Sigma\xv_\Sigma\times\fv_v\,\id S \\
\!\!\!\!\!\!\!\!\!\!\!\!\!\! &= \frac{\partial}{\partial s}\left(\int_\Sigma\xv_\Sigma\times\stress\ev\,\id S\right)+\int_\Sigma\frac{\partial(\xv_\Sigma\times\stress\ev_\alpha)}{\partial\xj_\alpha}\,\id S \\
\!\!\!\!\!\!\!\!\!\!\!\!\!\! &\quad\quad -\int_\Sigma\ev_\alpha\times\stress\ev_\alpha\,\id S +\int_\Sigma\xv_\Sigma\times\fv_v\,\id S \\
\!\!\!\!\!\!\!\!\!\!\!\!\!\! &=\Mres'(s)+ \int_{\partial\Sigma}\xv_\Sigma\times\stress\nv\,\id\zeta + \int_\Sigma\ev\times\stress\ev\,\id S + \int_\Sigma\xv_\Sigma\times\fv_v\,\id S \\
\!\!\!\!\!\!\!\!\!\!\!\!\!\! &=\Mres'(s)+\ev\times\Fres(s)+\cv_l(s)\,,
\end{split}
\end{displaymath}
since $\ev_\alpha\times\stress\ev_\alpha+\ev\times\stress\ev=\bzero$ from the symmetry of $\stress$.
\end{itemize}

\end{frame}

\begin{frame}{Beam statics}{Global balance of forces}

\begin{figure}
\centering\includegraphics[scale=.25]{\figs/ch7-Forces2}
\end{figure}

\begin{itemize}
\item Global equilibrium of the \textcolor{aqua}{left section} $s\in[0,s]$:
\begin{displaymath}
\Fres(s)-\Fres(0)+\int_0^s\fv_l\id \zeta=\bzero\,;
\end{displaymath}
\item  Global equilibrium of the \textcolor{yellow}{right section} $s\in[s,L]$:
\begin{displaymath}
\Fres(L)-\Fres(s)+\int_s^L\fv_l\id \zeta=\bzero\,.
\end{displaymath}
\end{itemize}

\end{frame}

\begin{frame}{Beam statics}{Global balance of moments}

\begin{figure}
\centering\includegraphics[scale=.25]{\figs/ch7-Moments2}
\end{figure}

\begin{itemize}
\item Global equilibrium of the \textcolor{aqua}{left section} $s\in[0,s]$:
\begin{displaymath}
\!\!\!\!\Mres(s)-\Mres(0)+s\ev\times\Fres(0)+\int_0^s(\cv_l+(\zeta-s)\ev\times\fv_l)\,\id \zeta=\bzero\,;
\end{displaymath}
\item  Global equilibrium of the \textcolor{yellow}{right section} $s\in[s,L]$:
\begin{displaymath}
\!\!\!\!\Mres(L)+(L-s)\ev\times\Fres(L)-\Mres(s)+\int_s^L(\cv_l+(\zeta-s)\ev\times\fv_l)\,\id \zeta=\bzero\,.
\end{displaymath}
\end{itemize}

\end{frame}



\section{Stresses}
\begin{frame}{Beam elastic law}{Traction vector}

\begin{itemize}
\item Recap: basic kinematic assumption (Timoshenko)
\begin{displaymath}
\uv=\uv_G(s)+\drot(s)\times\xv_\Sigma\,;
\end{displaymath}
\item Linearized strains $\xv_\Sigma\simeq\pv_\Sigma$:
\begin{displaymath}
\strain=(\uv_G'+\drot'\times\xv_\Sigma)\otimes_s\ev-(\drot_\Sigma\times\ev)\otimes_s\ev\,;
\end{displaymath}
\item Linear elastic, isotropic behavior:
\begin{displaymath}
\begin{split}
\stress &=\lambda\trace(\strain)\Id+2\mu\strain \\
&=\lambda\strainj_{ee}\Id+2\mu(\strainj_{ee}\ev+{\color{green}{{\boldsymbol\gamma}_\Sigma}})\otimes_s\ev+{\color{green}{\stress_\Sigma}}\,;
\end{split}
\end{displaymath}
\item Traction vector:
\begin{displaymath}
\stress\ev=E(\underbrace{\uj_{Ge}'\ev}_{\color{red}{//\ev}}+\underbrace{\drot_\Sigma'\times\xv_\Sigma}_{\color{red}{//\ev,\propto\xv_\Sigma}})+\mu(\underbrace{\uv_{G\Sigma}'-\drot_\Sigma\times\ev}_{\color{red}{\perp\ev}}+\underbrace{\drotj_e'\ev\times\xv_\Sigma}_{\color{red}{\perp\ev,\propto\xv_\Sigma}})\,.
\end{displaymath}
\end{itemize}

\end{frame}

\begin{frame}{Beam elastic law}{Resultant force with Timoshenko's kinematical assumption}

\begin{itemize}
\item Recap: resultant force
\begin{displaymath}
\Fres=\int_\Sigma\stress\ev\,\id S\,;
\end{displaymath}
\item Assuming $\int_\Sigma\xv_\Sigma\id S=\bzero$, $S=\int_\Sigma\id S$:
\begin{displaymath}
\begin{split}
\Fres &= \;\scriptstyle \int_\Sigma E(\uj_{Ge}'\ev+\cancel{\drot_\Sigma'\times\xv_\Sigma})\id S+\int_\Sigma\mu(\uv_{G\Sigma}'-\drot_\Sigma\times\ev+\cancel{\drotj_e'\ev\times\xv_\Sigma})\id S \\
&=ES\uj_{Ge}'\ev+\Fres_\Sigma\,;
\end{split}
\end{displaymath}
\item Shear force with Timoshenko's assumption:
\begin{displaymath}
\Fres_\Sigma=\mu S(\uv_{G\Sigma}'-\drot_\Sigma\times\ev)\,.
\end{displaymath}
\end{itemize}

\end{frame}

\begin{frame}{Beam elastic law}{Resultant force with Euler-Bernoulli's kinematical assumption}

\begin{itemize}
\item Recap: resultant force
\begin{displaymath}
\Fres=\int_\Sigma\stress\ev\,\id S\,;
\end{displaymath}
\item Assuming $\int_\Sigma\xv_\Sigma\id S=\bzero$, $S=\int_\Sigma\id S$:
\begin{displaymath}
\begin{split}
\Fres &= \;\scriptstyle \int_\Sigma E(\uj_{Ge}'\ev+\cancel{\drot_\Sigma'\times\xv_\Sigma})\id S+\int_\Sigma\mu(\underbrace{\cancel{\uv_{G\Sigma}'-\drot_\Sigma\times\ev}}_{=\bzero\,\text{by Euler-Bernoulli}}+\cancel{\drotj_e'\ev\times\xv_\Sigma})\id S \\
&=ES\uj_{Ge}'\ev+\Fres_\Sigma\,;
\end{split}
\end{displaymath}
\item Shear force with Euler-Bernoulli's assumption:
\begin{displaymath}
\begin{split}
\Mres'+\ev\times\Fres+\cv_l=\bzero \\
\imply\quad \Fres_\Sigma=\ev\times(\Mres_\Sigma'+\cv_l)\,.
\end{split}
\end{displaymath}
\end{itemize}

\end{frame}

\begin{frame}{Beam elastic law}{Resultant moment}

\begin{itemize}
\item Recap: resultant moment
\begin{displaymath}
\Mres=\int_\Sigma\xv_\Sigma\times\stress\ev\,\id S\,;
\end{displaymath}
\item Assuming $\int_\Sigma\xv_\Sigma\id S=\bzero$:
\begin{displaymath}
\begin{split}
\Mres &= \;\scriptstyle \int_\Sigma E(\cancel{\xv_\Sigma\times\uj_{Ge}'\ev}+\xv_\Sigma\times(\drot_\Sigma'\times\xv_\Sigma))\id S \\
&\quad\quad\scriptstyle+\int_\Sigma\mu(\cancel{\xv_\Sigma\times(\uv_{G\Sigma}'-\drot_\Sigma\times\ev)}+\xv_\Sigma\times(\drotj_e'\ev\times\xv_\Sigma))\id S \\
&=E\Mstatic(\drot_\Sigma')+\mu\Mstatic(\drotj_e'\ev)\,;
\end{split}
\end{displaymath}
\item (Symmetric) inertia tensor:
\begin{displaymath}
\Mstatic=\int_\Sigma(\norm{\xv_\Sigma}^2\Id-\xv_\Sigma\otimes\xv_\Sigma)\,\id S\,;
\end{displaymath}
\end{itemize}

\begin{overprint}

\onslide<1|handout:1>
\begin{itemize}
\item Bending moment with Timoshenko's assumption:
\begin{displaymath}
\Mres_\Sigma=E\Mstatic(\drot_\Sigma')\,.
\end{displaymath}
\end{itemize}

\onslide<2|handout:2>
\begin{itemize}
\item Bending moment with Euler-Bernoulli's assumption:
\begin{displaymath}
\Mres_\Sigma=E\Mstatic(\ev\times\uv_{G\Sigma}'')\,.
\end{displaymath}
\end{itemize}

\end{overprint}

\end{frame}

\begin{frame}{Summary}{}

\begin{columns}[t]
\column{.5\textwidth}
\centering{Constitutive equations}:
\begin{displaymath}
\begin{array}{c}
\Fres=ES\uj_{Ge}'\ev+\Fres_\Sigma \\
\Fres_\Sigma=\ev\times(\Mres_\Sigma'+\cv_l)
\end{array}
\end{displaymath}
\begin{displaymath}
\begin{array}{c}
\Mres_\Sigma=E\Mstatic(\ev\times\uv_{G\Sigma}'') \\
\Mresj_e=\mu\Mstaticj_e\drotj'_e
\end{array}
\end{displaymath}
\begin{displaymath}
\begin{split}
\Mstaticj_e &=\scal{\Mstatic\ev,\ev} \\
&=\int_\Sigma\norm{\xv_\Sigma}^2\id S
\end{split}
\end{displaymath}
\column{.5\textwidth}
\centering Static equilibrium:
\begin{displaymath}
\begin{array}{c}
\Fres'+\fv_l=\bzero
\end{array}
\end{displaymath}
\vskip-7pt
\begin{displaymath}
\begin{array}{c}
\Mres_\Sigma''-\ev\times\fv_l+\cv_{l\Sigma}'=\bzero \\
\Mresj_e'+\cj_{le}=0
\end{array}
\end{displaymath}
\end{columns}

\end{frame}

\begin{frame}{Differential equations}{}

\begin{itemize}
\item Elongation:
\begin{displaymath}
\boxed{ES\uj_{Ge}''(s)+\scal{\fv_l(s),\ev}=0}
\end{displaymath}
with either kinematical $\uj_{Ge}(0),\uj_{Ge}(L)$ or mechanical $\Fresj_e(0),\Fresj_e(L)$ boundary conditions.
\item Torsion:
\begin{displaymath}
\boxed{\mu\Mstaticj_e\drotj_e''(s)+\scal{\cv_l(s),\ev}=0}
\end{displaymath}
with either kinematical $\drotj_e(0),\drotj_e(L)$ or mechanical $\Mresj_e(0),\Mresj_e(L)$ boundary conditions.
\item Bending:
\begin{displaymath}
\boxed{E\Mstatic(\ev\times\uv_{G\Sigma}^{(IV)}(s))-\ev\times\fv_l(s)+\cv_{l\Sigma}'(s)=\bzero}
\end{displaymath}
with~either~kinematical~$\uv_{G\Sigma}(0),\uv_{G\Sigma}(L),\uv_{G\Sigma}'(0),\uv_{G\Sigma}'(L)$ or mechanical $\Mres_\Sigma(0),\Mres_\Sigma(L),\Fres_\Sigma(0),\Fres_\Sigma(L)$ boundary conditions.
\end{itemize}

\end{frame}

\begin{frame}{Back to stresses...}{}

\begin{itemize}
\item Recap: traction vector
\begin{displaymath}
\begin{split}
\stress\ev &=E(\underbrace{\uj_{Ge}'\ev}_{\color{red}{//\ev}}+\underbrace{\drot_\Sigma'\times\xv_\Sigma}_{\color{red}{//\ev,\propto\xv_\Sigma}})+\mu(\underbrace{\uv_{G\Sigma}'-\drot_\Sigma\times\ev}_{\color{red}{\perp\ev}}+\underbrace{\drotj_e'\ev\times\xv_\Sigma}_{\color{red}{\perp\ev,\propto\xv_\Sigma}}) \\
&=\stressj_{ee}\ev+\tress_\Sigma\,;
\end{split}
\end{displaymath}
\item Normal stress:
\begin{displaymath}
\begin{split}
\stressj_{ee} &=E(\uj_{Ge}'+\scal{\drot_\Sigma'\times\xv_\Sigma,\ev}) \\
&=\frac{\Fresj_e}{S}+\scal{\xv_\Sigma,\ev\times\Mstatic^{-1}(\Mres_\Sigma)}\,;
\end{split}
\end{displaymath}
\item Shear stress:
\begin{displaymath}
\begin{split}
\tress_\Sigma &= \mu(\uv_{G\Sigma}'-\drot_\Sigma\times\ev)+\mu\drotj_e'(\ev\times\xv_\Sigma)\\
&=\frac{\Fres_\Sigma}{S}+\frac{\Mresj_e}{\Mstaticj_e}(\ev\times\xv_\Sigma)\,.
\end{split}
\end{displaymath}
\end{itemize}

\end{frame}



\section{6.9 Mechanical strength of a rocket booster}

\begin{frame}{Mechanical strength of a rocket booster}{Setup}

\begin{overprint}

\onslide<1|handout:1>
%\vskip10pt
\centering\includegraphics[scale=.45]{\figs/ch9-Ariane5}

\onslide<2|handout:2>
\centering\includegraphics[scale=.35]{\figs/ch9-Booster}
\vskip-8pt{\hspace{6truecm}\mbox{\tiny{\copyright\ G. Puel}}}

\begin{itemize}
\item Connection at $s=h$ with $\Mres_h^I=\bzero$; 
\item Connection at $s=H$ with $\Mres_H^I=\bzero$;
\item Weight of the propellant.
\end{itemize}

\end{overprint}

\end{frame}

\begin{frame}{Mechanical strength of a rocket booster}{Solution}

\begin{overprint}

\onslide<1|handout:1>
\vskip-20pt
\begin{exampleblock}{Question \#1: Action of the propellant on the beam?}
\vskip7pt
\centering\includegraphics[scale=.38]{\figs/ch9-Booster2}
\end{exampleblock}

\onslide<2|handout:2>
\vskip-20pt
\begin{exampleblock}{Question \#1: Action of the propellant on the beam?}
\begin{itemize}
\item Linear external force exerted by the propellant:
\begin{displaymath}
\begin{split}
\fv_l(s) &=\int_{\Sigma_p}\roi_p\gv\id S \\
&=-\int_{R_i}^{R_e}\roi_p g\iv_\zj 2\pi r\id r \\
&=-\roi_p g\pi(R_e^2-R_i^2)\iv_\zj \\
&=-\frac{m_p g}{H}\iv_\zj\,;
\end{split}
\end{displaymath}
\end{itemize}
\end{exampleblock}

\onslide<3|handout:3>
\vskip-20pt
\begin{exampleblock}{Question \#1: Action of the propellant on the beam?}
\begin{itemize}
\item Linear external force exerted by the propellant:
\begin{displaymath}
\fv_l(s) =-\frac{m_p g}{H}\iv_\zj\,;
\end{displaymath}
\item Linear external torque exerted by the propellant:
\begin{displaymath}
\begin{split}
\cv_l(s) &=\int_{\Sigma_p}\xv_\Sigma\times\roi_p\gv\id S \\
&=-\int_{R_i}^{R_e}\int_0^{2\pi}r\iv_r(\theta)\times\roi_p g\iv_\zj  r\id r\id\theta \\
&=\roi_p g\int_{R_i}^{R_e}r^2\id r\cancel{\int_0^{2\pi}\iv_\theta(\theta) \id\theta} \\
&=\bzero\,.
\end{split}
\end{displaymath}
\end{itemize}
\end{exampleblock}

\end{overprint}

\end{frame}

\begin{frame}{Mechanical strength of a rocket booster}{Solution}

\begin{overprint}

\onslide<1|handout:1>
\vskip-20pt
\begin{exampleblock}{Question \#2: Global equilibrium of the horizontal beam?}
\begin{itemize}
\item Global equilibrium of forces:
\begin{displaymath}
\begin{split}
\Rv_h^I+\Rv_H^I+\int_0^H\fv_l(s)\id s &=\bzero \\
\imply\quad \Rj_h^I+\Rj_H^I-m_p g &=0\,,
\end{split}
\end{displaymath}
where $m_p=\roi_p\pi(R_e^2-R_i^2)H$;
\end{itemize}
\end{exampleblock}

\onslide<2|handout:2>
\vskip-20pt
\begin{exampleblock}{Question \#2: Global equilibrium of the horizontal beam?}
\begin{itemize}
\item Global equilibrium of forces:
\begin{displaymath}
\Rj_h^I+\Rj_H^I-m_p g=0\,;
\end{displaymath}
\item Global equilibrium of moments (at, say, $s=H$):
\begin{displaymath}
\begin{split}
\cancel{\Mres_h^I}+(h-H)\iv_\xj\times\Rv_h^I+\cancel{\Mres_H^I} & \\
+\int_0^H(\cancel{\cv_l}+(s-H)\iv_\xj\times\fv_l) \id s &=\bzero \\
\Rj_h^I(h-H)\iv_\yj +\frac{\roi_p g \pi(R_e^2-R_i^2)H^2}{2}\iv_\yj &=\bzero \\
\imply\quad \Rj_h^I(h-H)+\frac{m_p gH}{2} &=0\,;
\end{split}
\end{displaymath}
\end{itemize}
\end{exampleblock}

\onslide<3|handout:3>
\vskip-20pt
\begin{exampleblock}{Question \#2: Global equilibrium of the horizontal beam?}
\begin{itemize}
\item Recap:
\begin{displaymath}
\begin{split}
\Rj_h^I+\Rj_H^I-m_p g &=0\,, \\
\Rj_h^I(h-H)+\frac{m_p gH}{2} &=0\,;
\end{split}
\end{displaymath}
\item Therefore:
\begin{displaymath}
\begin{split}
\Rj_h^I =\frac{m_p gH}{2(H-h)} &=\frac{5}{8}m_p g\,, \\
\Rj_H^I =\frac{m_p g(H-2h)}{2(H-h)} &=\frac{3}{8}m_p g\,,
\end{split}
\end{displaymath}
owing to $H=5h$.
\end{itemize}
\end{exampleblock}

\end{overprint}

\end{frame}

\begin{frame}{Mechanical strength of a rocket booster}{Solution}

\begin{overprint}

\onslide<1|handout:1>
\vskip-20pt
\begin{exampleblock}{Question \#3: $\Mres(s)$ in $s\in(h,H]$?}
\begin{itemize}
\item Global balance of moments for the beam $(s,H]$ with $s\in(h,H)$ and where $\ev=\iv_\xj$:
\begin{multline*}
-\Mres(s)+\Mres_H^I+(H-s)\iv_\xj\times\Rv_H^I \\
+\int_s^H(\cv_l+(\zeta-s)\iv_\xj\times\fv_l) \id\zeta=\bzero\,;
\end{multline*}
\end{itemize}
\end{exampleblock}
\vskip-15pt
\begin{figure}
\centering\includegraphics[scale=.2]{\figs/ch7-Moments2}
\end{figure}

\onslide<2|handout:2>
\vskip-20pt
\begin{exampleblock}{Question \#3: $\Mres(s)$ in $s\in(h,H]$?}
\begin{itemize}
\item Global balance of moments for the beam $(s,H)$:
\begin{multline*}
-\Mres(s)+\cancel{\Mres_H^I}+(H-s)\iv_\xj\times\Rv_H^I \\
+\int_s^H(\cancel{\cv_l}+(\zeta-s)\iv_\xj\times\fv_l) \id\zeta=\bzero\,;
\end{multline*}
\item Thus:
\begin{displaymath}
\begin{split}
\Mres(s) &= -\Rj_H^I(H-s)\iv_\yj+\frac{m_p g}{2H}\left[(\zeta-s)^2\right]_{\zeta=s}^{\zeta=H}\iv_\yj \\
&=\frac{m_pg}{8H}(H-s)(H-4s)\iv_\yj\,,\quad s\in(h,H]\,;
\end{split}
\end{displaymath}
\end{itemize}
\end{exampleblock}

\onslide<3|handout:3>
\vskip-20pt
\begin{exampleblock}{Question \#3: $\Mres(s)$ in $s\in(0,h]$?}
\begin{itemize}
\item Global balance of moments for the beam $(s,H]$ with $s\in(0,h)$ and where $\ev=\iv_\xj$:
\begin{multline*}
-\Mres(s)+\Mres_H^I+(H-s)\iv_\xj\times\Rv_H^I +\Mres_h^I+(h-s)\iv_\xj\times\Rv_h^I \\
+\int_s^H(\cv_l+(\zeta-s)\iv_\xj\times\fv_l) \id\zeta=\bzero\,;
\end{multline*}
\end{itemize}
\end{exampleblock}
\vskip-15pt
\begin{figure}
\centering\includegraphics[scale=.2]{\figs/ch7-Moments2}
\end{figure}

\onslide<4|handout:4>
\vskip-20pt
\begin{exampleblock}{Question \#3: $\Mres(s)$ in $s\in(0,h]$?}
\begin{itemize}
\item Global balance of moments for the beam $(s,H]$:
\begin{multline*}
-\Mres(s)+\cancel{\Mres_H^I}+(H-s)\iv_\xj\times\Rv_H^I +\cancel{\Mres_h^I}+(h-s)\iv_\xj\times\Rv_h^I \\
+\int_s^H(\cancel{\cv_l}+(\zeta-s)\iv_\xj\times\fv_l) \id\zeta=\bzero\,;
\end{multline*}
\item Thus:
\begin{displaymath}
\begin{split}
\!\!\!\!\!\!\!\!\!\!\!\!\!\Mres(s) &= -\Rj_H^I(H-s)\iv_\yj-\Rj_h^I(h-s)\iv_\yj+\frac{m_p g}{2H}\left[(\zeta-s)^2\right]_{\zeta=s}^{\zeta=H}\iv_\yj \\
&=m_p g\left[\frac{3}{8}(s-H)+\frac{5}{8}(s-h)+\frac{(H-s)^2}{2H}\right]\iv_\yj \\
&=\frac{m_p g}{2H}s^2\iv_\yj\,,\quad s\in(0,h]\,;
\end{split}
\end{displaymath}
\end{itemize}
\end{exampleblock}

\onslide<5|handout:5>
\vskip-20pt
\begin{exampleblock}{Question \#3: $\Mres(s)$ in $s\in[0,h)$?}
\begin{itemize}
\item {\bf Alternatively}: global balance of moments for the beam $[0,s)$ with $s\in(0,h)$ and where $\ev=\iv_\xj$:
\begin{displaymath}
-\Mres(0)+s\iv_\xj\times\Fres(0)+\Mres(s)+\int_0^s(\cv_l+(\zeta-s)\iv_\xj\times\fv_l) \id\zeta=\bzero\,;
\end{displaymath}
\end{itemize}
\end{exampleblock}
\begin{figure}
\centering\includegraphics[scale=.2]{\figs/ch7-Moments2}
\end{figure}

\onslide<6|handout:6>
\vskip-20pt
\begin{exampleblock}{Question \#3: $\Mres(s)$ in $s\in[0,h)$?}
\begin{itemize}
\item {\bf Alternatively}: global balance of moments for the beam $[0,s)$,
\begin{displaymath}
-\cancel{\Mres(0)}+s\iv_\xj\times\cancel{\Fres(0)}+\Mres(s)+\int_0^s(\cancel{\cv_l}+(\zeta-s)\iv_\xj\times\fv_l) \id\zeta=\bzero\,;
\end{displaymath}
\item Thus:
\begin{displaymath}
\begin{split}
\Mres(s) &= -\frac{m_pg}{2H}\left[(\zeta-s)^2\right]_{\zeta=0}^{\zeta=s}\iv_\yj \\
&=\frac{m_p g}{2H}s^2\iv_\yj\,,\quad s\in[0,h)\,;
\end{split}
\end{displaymath}
\end{itemize}
\end{exampleblock}

\onslide<7|handout:7>
\vskip-20pt
\begin{exampleblock}{Question \#3: $\Mres(s)$ in $s\in[0,H]$?}
\begin{itemize}
\item Recap:
\begin{displaymath}
\begin{split}
\Mres(s) &=\frac{m_pg}{2H}s^2\iv_\yj\,,\quad s\in[0,h]\,,\\
&=\frac{m_pg}{8H}(H-s)(H-4s)\iv_\yj\,,\quad s\in[h,H]\,;
\end{split}
\end{displaymath}
\item The bending moment is continuous at $s=h$ as expected, since $\Mres_h^I=\bzero$, with:
\begin{displaymath}
\Mres(s)=\frac{m_p g h}{10}\iv_\yj\,.
\end{displaymath}
\begin{displaymath}
\end{displaymath}
\end{itemize}
\end{exampleblock}

\onslide<8|handout:8>
\vskip-20pt
\begin{exampleblock}{Question \#3: $\Mres(s)$ in $s\in[0,H]$?}
\begin{itemize}
\item {\bf Alternative solution}: local equilibrium of the beam $s\in(0,h)\cup(h,H)$:
\begin{displaymath}
\begin{split}
\Fres'(s)+\fv_l(s) &=\bzero\,, \\
\Mres'(s)+\ev\times\Fres(s)+\cv_l(s) &=\bzero\,;
\end{split}
\end{displaymath}
\end{itemize}
\end{exampleblock}

\onslide<9|handout:9>
\vskip-20pt
\begin{exampleblock}{Question \#3: $\Mres(s)$ in $s\in[0,H]$?}
\begin{itemize}
\item Local equilibrium of the beam $s\in(0,h)\cup(h,H)$:
\begin{displaymath}
\begin{split}
\Fres'(s)-\frac{m_p g}{H}\iv_\zj &=\bzero\,, \\
\Mres'(s)+\iv_\xj\times\Fres(s)+\cancel{\cv_l(s)} &=\bzero\,;
\end{split}
\end{displaymath}
\item Then for $s\in(h,H)$:
\begin{displaymath}
\begin{split}
\Fres(s) &=\frac{m_p g}{H}(s-H)\iv_\zj+\Rv_H^I  \\
&=m_p g\left[\frac{s}{H}-\frac{H}{2(H-h)}\right]\iv_\zj \\
&=\frac{m_p g}{8H}(5s-8H)\iv_\zj \\
\end{split}
\end{displaymath}
\end{itemize}
\end{exampleblock}

\onslide<10|handout:10>
\vskip-20pt
\begin{exampleblock}{Question \#3: $\Mres(s)$ in $s\in[0,H]$?}
\begin{itemize}
\item Local equilibrium of the beam $s\in(0,h)\cup(h,H)$:
\begin{displaymath}
\begin{split}
\Fres'(s)-\frac{m_p g}{H}\iv_\zj &=\bzero\,, \\
\Mres'(s)+\iv_\xj\times\Fres(s)+\cancel{\cv_l(s)} &=\bzero\,;
\end{split}
\end{displaymath}
\item Then for $s\in(h,H)$:
\begin{displaymath}
\begin{split}
\Mres'(s) &=-m_p g\left[\frac{s}{H}-\frac{H}{2(H-h)}\right]\iv_\xj\times\iv_\zj \\
\Mres(s) &=m_p g\left[\frac{s^2-H^2}{2H}-\frac{H(s-H)}{2(H-h)}\right]\iv_\yj +\cancel{\Mres_H^I} \\
&=\frac{m_p g}{2H}\left(s-\frac{hH}{H-h}\right)(s-H)\iv_\yj\,.
\end{split}
\end{displaymath}
\end{itemize}
\end{exampleblock}

\onslide<11|handout:11>
\vskip-20pt
\begin{exampleblock}{Question \#3: $\Mres(s)$ in $s\in[0,H]$?}
\begin{itemize}
\item Local equilibrium of the beam $s\in(0,h)\cup(h,H)$:
\begin{displaymath}
\begin{split}
\Fres'(s)-\frac{m_p g}{H}\iv_\zj &=\bzero\,, \\
\Mres'(s)+\iv_\xj\times\Fres(s)+\cancel{\cv_l(s)} &=\bzero\,;
\end{split}
\end{displaymath}
\item Then for $s\in(0,h)$:
\begin{displaymath}
\begin{split}
\Fres(s) &=m_p g\frac{s}{H}\iv_\zj+\cancel{\Fres(0)}  \\
&=m_p g\frac{s}{H}\iv_\zj\,;
\end{split}
\end{displaymath}
\end{itemize}
\end{exampleblock}

\onslide<12|handout:12>
\vskip-20pt
\begin{exampleblock}{Question \#3: $\Mres(s)$ in $s\in[0,H]$?}
\begin{itemize}
\item Local equilibrium of the beam $s\in(0,h)\cup(h,H)$:
\begin{displaymath}
\begin{split}
\Fres'(s)-\frac{m_p g}{H}\iv_\zj &=\bzero\,, \\
\Mres'(s)+\iv_\xj\times\Fres(s)+\cancel{\cv_l(s)} &=\bzero\,;
\end{split}
\end{displaymath}
\item Then for $s\in(0,h)$:
\begin{displaymath}
\begin{split}
\Mres'(s) &=-m_p g\frac{s}{H}\iv_\xj\times\iv_\zj \\
\Mres(s) &=\frac{m_p g}{2H}s^2\iv_\yj +\cancel{\Mres(0)} \\
&=\frac{m_p g}{2H}s^2\iv_\yj \,.
\end{split}
\end{displaymath}
\end{itemize}
\end{exampleblock}

\onslide<13|handout:13>
\vskip-20pt
\begin{exampleblock}{Question \#3: $\Mres(s)$ in $s\in[0,H]$?}
\begin{itemize}
\item Recap:
\begin{columns}[t]
\column{.5\textwidth}
\centering for $s\in[0,h)$
\begin{displaymath}
\begin{split}
\Fres(s) &= \scriptstyle m_p g\frac{s}{H}\iv_\zj \\
\Mres(s) &=\scriptstyle \frac{m_p g}{2H}s^2\iv_\yj
\end{split}
\end{displaymath}
\column{.5\textwidth}
\centering for $s\in(h,H]$
\begin{displaymath}
\begin{split}
\Fres(s) &=\scriptstyle m_p g\left[\frac{s}{H}-\frac{H}{2(H-h)}\right]\iv_\zj\\
\Mres(s) &=\scriptstyle \frac{m_p g}{2H}\left(s-\frac{hH}{H-h}\right)(s-H)\iv_\yj
\end{split}
\end{displaymath}
\end{columns}
\vskip10pt
\item The resultant force is discontinuous at $s=h$ while the resultant moment is continuous: 
\begin{displaymath}
\begin{split}
\Fres(h^-)-\Fres(h^+)=\frac{m_p g H}{2(H-h)}\iv_\zj &=\Rv_h^I\,, \\
\Mres(h^-)-\Mres(h^+)=\bzero &=\Mres_h^I\,.
\end{split}
\end{displaymath}
\end{itemize}
\end{exampleblock}

\end{overprint}

\end{frame}

\begin{frame}{Mechanical strength of a rocket booster}{Solution}

\begin{overprint}

\onslide<1|handout:1>
\vskip-20pt
\begin{exampleblock}{Question \#4: $\stressj_{ee}$?}
\begin{itemize}
\item Normal stress:
\begin{displaymath}
\stressj_{ee}=\frac{\scal{\Fres,\ev}}{S}+\scal{\xv_\Sigma,\ev\times\Mstatic^{-1}(\Mres_\Sigma)}\,;
\end{displaymath}
\end{itemize}
\end{exampleblock}

\onslide<2|handout:2>
\vskip-20pt
\begin{exampleblock}{Question \#4: $\stressj_{ee}$?}
\begin{itemize}
\item Normal stress:
\begin{displaymath}
\stressj_{ee}=\frac{\scal{\Fres,\ev}}{S}+\scal{\xv_\Sigma,\ev\times\Mstatic^{-1}(\Mres_\Sigma)}\,;
\end{displaymath}
\item In the present case $\scal{\Fres,\ev}=0$ and $\Mres_\Sigma=\Mres=\Mresj\iv_\yj$;
\end{itemize}
\end{exampleblock}

\onslide<3|handout:3>
\vskip-20pt
\begin{exampleblock}{Question \#4: $\stressj_{ee}$?}
\begin{itemize}
\item Normal stress:
\begin{displaymath}
\stressj_{ee}=\frac{\scal{\Fres,\ev}}{S}+\scal{\xv_\Sigma,\ev\times\Mstatic^{-1}(\Mres_\Sigma)}\,;
\end{displaymath}
\item In the present case $\scal{\Fres,\ev}=0$ and $\Mres_\Sigma=\Mres=\Mresj\iv_\yj$;
\item But $\Mstatic=I(\iv_\xj\otimes\iv_\xj+\Id)$ therefore:
\begin{displaymath}
\begin{split}
\stressj_{ee}(\xv) &= \scriptstyle\;\frac{\Mresj(s)}{I}\scal{\xv_\Sigma,\iv_\xj\times\left(\Id-\demi\iv_\xj\otimes\iv_\xj\right)\iv_\yj} \\
&=\scriptstyle\; \frac{\Mresj(s)\zj}{I} \\
&=\left\{\begin{array}{ll}
\scriptstyle \frac{m_p g}{2HI}s^2\zj & \scriptstyle\text{if $0\leq s < h$}\,,\\
\scriptstyle \frac{m_p g}{2HI}\left(s-\frac{hH}{H-h}\right)(s-H)\zj & \scriptstyle\text{if $h\leq s \leq H$}\,;
\end{array}\right.
\end{split}
\end{displaymath}
\end{itemize}
\end{exampleblock}

\onslide<4|handout:4>
\vskip-20pt
\begin{exampleblock}{Question \#4: $\stressj_{ee}$?}
\begin{itemize}
\item $m_p=\roi_p\pi(R_e^2-R_i^2)H$, $I=\frac{\pi}{4}[(R+e)^4-R^4]$, and:
\begin{displaymath}
\stressj_{ee}(\xv) =\frac{m_p g\zj}{2HI}\times\left\{\begin{array}{ll}
\scriptstyle s^2 & \scriptstyle\text{if $0\leq s < h$}\,,\\
\scriptstyle \left(s-\frac{hH}{H-h}\right)(s-H) & \scriptstyle\text{if $h\leq s \leq H$}\,;
\end{array}\right.
\end{displaymath}
\item It is maximum for $s=\frac{H^2}{2(H-h)}$, $\zj=-(R+e)$, and:
\begin{displaymath}
\stressj_{ee,\text{max}}=\frac{m_p g}{8I}\left(\frac{H-2h}{H-h}\right)^2H(R+e)\,;
\end{displaymath}
\item N.A.: $R=R_e=1.5$m, $R_i=0.5$m, $e=0.01$m, $\roi_p=1500$kg/m$^3$, $H=5h=25m$,
\begin{displaymath}
\stressj_{ee,\text{max}}=57.3\,\text{MPa}<\stressj_e=100\,\text{MPa}\,.
\end{displaymath}
\end{itemize}
\end{exampleblock}

\onslide<5|handout:5>
\begin{figure}
\centering\includegraphics[scale=.4]{\figs/ch9-sigma_ee}
\end{figure}
\vskip-20pt{\hspace{6.7truecm}\mbox{\tiny{\copyright\ G. Puel}}}
\begin{displaymath}
s\mapsto\frac{\Mresj(s)}{m_pg}
\end{displaymath}

\end{overprint}

\end{frame}

%\begin{frame}{Mechanical strength of a rocket booster}{Solution}
%
%\begin{center}
%\textcolor{red}{\LARGE {\bf TO BE CONTINUED!!!}}
%\end{center}
%
%\end{frame}

%\end{document}

\begin{frame}{Mechanical strength of a rocket booster}{Solution}

\begin{overprint}

\onslide<1|handout:1>
\vskip-20pt
\begin{exampleblock}{Question \#5: Global equilibrium of the vertical beam?}
\begin{itemize}
\item Global equilibrium of forces:
\begin{displaymath}
\begin{split}
\Rv_h^{II}+\Rv_H^{II}+\int_0^H\fv_l(s)\id s &=\bzero \\
\imply\quad\Rj_h^{II}+\Rj_H^{II}-m_pg &=0\,;
\end{split}
\end{displaymath}
\end{itemize}
\end{exampleblock}

\onslide<2|handout:2>
\vskip-20pt
\begin{exampleblock}{Question \#5: $\Fres(s)$ in $s\in(h,H]$?}
\begin{itemize}
\item Global balance of forces for the beam $(s,H]$ with $s=\zj\in(h,H)$ and where $\ev=\iv_\zj$:
\begin{displaymath}
-\Fres(s)+\Rv_H^{II}+\int_s^H\fv_l(\zeta)\id\zeta=\bzero\,;
\end{displaymath}
\end{itemize}
\end{exampleblock}
\begin{figure}
\centering\includegraphics[scale=.2]{\figs/ch7-Forces2}
\end{figure}

\onslide<3|handout:3>
\vskip-20pt
\begin{exampleblock}{Question \#5: $\Fres(s)$ in $s\in(h,H]$?}
\begin{itemize}
\item Global balance of forces for the beam $(s,H]$:
\begin{displaymath}
-\Fres(s)+\Rv_H^{II}+\int_s^H\fv_l(\zeta)\id\zeta=\bzero\,;
\end{displaymath}
\item Thus:
\begin{displaymath}
\begin{split}
\Fres(s) &=\Rv_H^{II}-m_pg\left(1-\frac{s}{H}\right)\iv_\zj \\
&=-\Rv_h^{II}+m_pg\frac{s}{H}\iv_\zj\,;
\end{split}
\end{displaymath}
\end{itemize}
\end{exampleblock}

\onslide<4|handout:4>
\vskip-20pt
\begin{exampleblock}{Question \#5: $\Fres(s)$ in $s\in[0,h)$?}
\begin{itemize}
\item Global balance of forces for the beam $[0,s)$ with $s=\zj\in(0,h)$ and where $\ev=\iv_\zj$:
\begin{displaymath}
-\Fres(0)+\Fres(s)+\int_0^s\fv_l(\zeta)\id\zeta=\bzero\,;
\end{displaymath}
\end{itemize}
\end{exampleblock}
\begin{figure}
\centering\includegraphics[scale=.2]{\figs/ch7-Forces2}
\end{figure}

\onslide<5|handout:5>
\vskip-20pt
\begin{exampleblock}{Question \#5: $\Fres(s)$ in $s\in[0,h)$?}
\begin{itemize}
\item Global balance of forces for the beam $[0,s)$:
\begin{displaymath}
-\cancel{\Fres(0)}+\Fres(s)+\int_0^s\fv_l(\zeta)\id\zeta=\bzero\,;
\end{displaymath}
\item Thus:
\begin{displaymath}
\Fres(s)=m_pg\frac{s}{H}\iv_\zj\,.
\end{displaymath}
\end{itemize}
\end{exampleblock}

\onslide<6|handout:6>
\vskip-20pt
\begin{exampleblock}{Question \#5: $\Fres(s)$ in $s\in[0,H]$?}
\begin{itemize}
\item Recap:
\begin{displaymath}
\begin{split}
\Fres(s)=\left\{\begin{array}{ll}
\displaystyle m_pg\frac{s}{H}\iv_\zj & \text{if $0\leq s< h$}\,, \\
\displaystyle -\Rv_h^{II}+m_pg\frac{s}{H}\iv_\zj & \text{if $h< s\leq H$}\,;
\end{array}\right.
\end{split}
\end{displaymath}
\item The resultant force is discontinuous at $s=h$ with:
\begin{displaymath}
\Fres(h^-)-\Fres(h^+)=\Rv_h^{II}\,,
\end{displaymath}
as expected.
\end{itemize}
\end{exampleblock}

\end{overprint}

\end{frame}

\begin{frame}{Mechanical strength of a rocket booster}{Solution}

\begin{overprint}

\onslide<1|handout:1>
\vskip-20pt
\begin{exampleblock}{Question \#6: $\uj_{G\zj}(\zj)$?}
\begin{itemize}
\item Constitutive equation for the longitudinal displacement:
\begin{displaymath}
\scal{\Fres,\ev}=ES\uj_{Ge}'(s)\,,
\end{displaymath}
where $S=\pi[(R+e)^2-R^2]$;
\end{itemize}
\end{exampleblock}

\onslide<2|handout:2>
\vskip-20pt
\begin{exampleblock}{Question \#6: $\uj_{G\zj}(\zj)$?}
\begin{itemize}
\item Constitutive equation for the longitudinal displacement:
\begin{displaymath}
\scal{\Fres,\ev}=ES\uj_{Ge}'(s)\,;
\end{displaymath}
\item Therefore for $s=\zj\in[0,h)$:
\begin{displaymath}
\begin{split}
\uj_{G\zj}'(s) &=\frac{m_pg}{ES}\frac{s}{H} \\
\uj_{G\zj}(s) &=\frac{m_pg}{ES}\frac{(s^2-h^2)}{2H}+\cancel{\uj_{G\zj}(h)} \\
&=\frac{m_pg}{ES}\frac{(s^2-h^2)}{2H}\,;
\end{split}
\end{displaymath}
\end{itemize}
\end{exampleblock}

\onslide<3|handout:3>
\vskip-20pt
\begin{exampleblock}{Question \#6: $\uj_{G\zj}(\zj)$?}
\begin{itemize}
\item Constitutive equation for the longitudinal displacement:
\begin{displaymath}
\scal{\Fres,\ev}=ES\uj_{Ge}'(s)\,;
\end{displaymath}
\item Therefore for $s=\zj\in(h,H]$:
\begin{displaymath}
\begin{split}
\uj_{G\zj}'(s) &=\frac{\scal{\Fres,\ev}}{ES} \\
\uj_{G\zj}(s) &= \frac{1}{ES}\left[-\Rj_h^{II}(s-H)+\frac{m_pg}{2H}(s^2-H^2)\right]+\cancel{\uj_{G\zj}(H)} \\
&=\frac{s-H}{ES}\left[-\Rj_h^{II}+\frac{m_pg}{2}\left(1+\frac{s}{H}\right)\right]\,;
\end{split}
\end{displaymath}
\end{itemize}
\end{exampleblock}

\onslide<4|handout:4>
\vskip-20pt
\begin{exampleblock}{Question \#6: $\uj_{G\zj}(\zj)$?}
\begin{itemize}
\item Constitutive equation for the longitudinal displacement:
\begin{displaymath}
\scal{\Fres,\ev}=ES\uj_{Ge}'(s)\,;
\end{displaymath}
\item Therefore for $s=\zj\in(h,H]$:
\begin{displaymath}
\uj_{G\zj}(s)=\frac{s-H}{ES}\left[-\Rj_h^{II}+\frac{m_pg}{2}\left(1+\frac{s}{H}\right)\right]\,;
\end{displaymath}
\item Since $\uj_{G\zj}(h)=0$ one has $\Rj_h^{II}=\frac{m_pg}{2}(1+\frac{h}{H})$ and therefore:
\begin{displaymath}
\uj_{G\zj}(s)=\frac{m_p g}{2EHS}(s-h)(s-H)\,.
\end{displaymath}
\end{itemize}
\end{exampleblock}

\end{overprint}

\end{frame}

\begin{frame}{Mechanical strength of a rocket booster}{Solution}

\begin{overprint}

\onslide<1|handout:1>
\vskip-20pt
\begin{exampleblock}{Question \#7: $\Rj_h^{II}$ and $\Rj_H^{II}$?}
\begin{itemize}
\item From Q\#6 and Q\#5 we have:
\begin{displaymath}
\begin{split}
\Rj_h^{II} &=\frac{m_pg}{2}\left(1+\frac{h}{H}\right) \\
\Rj_H^{II} &=-\Rj_h^{II}+m_p g \\
&=\frac{m_pg}{2}\left(1-\frac{h}{H}\right)\,;
\end{split}
\end{displaymath}
\end{itemize}
\end{exampleblock}

\onslide<2|handout:2>
\vskip-20pt
\begin{exampleblock}{Question \#7: $\Fres(s)$?}
\begin{itemize}
\item From Q\#6 and Q\#5 we have:
\begin{displaymath}
\begin{split}
\Rj_h^{II} &=\frac{m_pg}{2}\left(1+\frac{h}{H}\right)\,, \\
\Rj_H^{II} &=\frac{m_pg}{2}\left(1-\frac{h}{H}\right)\,;
\end{split}
\end{displaymath}
\item Therefore:
\begin{displaymath}
\Fres(s)=\left\{\begin{array}{ll}
\displaystyle m_pg\frac{s}{H}\iv_\zj & \text{if $0\leq s< h$}\,, \\
\displaystyle \frac{m_pg}{2}\left(\frac{2s-h}{H}-1\right)\iv_\zj & \text{if $h< s\leq H$}\,;
\end{array}\right.
\end{displaymath}
\end{itemize}
\end{exampleblock}

\onslide<3|handout:3>
\vskip-20pt
\begin{exampleblock}{Question \#7: $\stressj_{ee}$?}
\begin{itemize}
\item Normal stress:
\begin{displaymath}
\stressj_{ee}=\frac{\scal{\Fres,\ev}}{S}+\scal{\xv_\Sigma,\ev\times\Mstatic^{-1}(\Mres_\Sigma)}\,;
\end{displaymath}
\item In the present case $\Mres_\Sigma=\bzero$ and therefore:
\begin{displaymath}
\stressj_{ee}(s)=\frac{m_pg}{S}\times\left\{\begin{array}{ll}
\displaystyle \frac{s}{H} & \text{if $0\leq s< h$}\,, \\
\displaystyle \frac{2s-h-H}{2H} & \text{if $h< s\leq H$}\,;
\end{array}\right.
\end{displaymath}
\item It is maximum for $s=H$ for which:
\begin{displaymath}
\stressj_{ee,\text{max}}= \frac{m_pg}{2S}\left(1-\frac{h}{H}\right)=9.8\,\text{MPa}<\stressj_e\,.
\end{displaymath}
\end{itemize}
\end{exampleblock}

\end{overprint}

\end{frame}

\end{document}