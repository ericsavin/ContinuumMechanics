% MG3 TD #1: Large beam bending
% V1.0 January 2021
% $Header: /cvsroot/latex-beamer/latex-beamer/solutions/generic-talks/generic-ornate-15min-45min.en.tex,v 1.5 2007/01/28 20:48:23 tantau Exp $
\def\webDOI{http://dx.doi.org}
\def\Folder{/Users/ericsavin/Documents/Cours/SG3-MMC/SLIDES_TD/}
\def\Year{\Folder/2020-2021}
\def\Sections{\Year/SECTIONS}
\def\figs{\Folder/FIGS}
%\def\figs{/Users/ericsavin/Documents//Cours/DynSto/Figs}
%\def\figs{/Users/ericsavin/Documents/Cours/SG3-MMC/SLIDES_TD/FIGS}
\def\figdynsto{/Users/ericsavin/Documents//Figures/DYNSTO}
\def\symb{/Users/ericsavin/Documents/Latex/SYMBOL}
\def\fonts{/Users/ericsavin/Documents/Latex/FONTS}
\def\logos{/Users/ericsavin/Documents/Latex/LOGOS}
\def\Onera{ONERA}
\def\ECP{CentraleSup\'elec}


\documentclass{beamer}

% This file is a solution template for:

% - Giving a talk on some subject.
% - The talk is between 15min and 45min long.
% - Style is ornate.



% Copyright 2004 by Till Tantau <tantau@users.sourceforge.net>.
%
% In principle, this file can be redistributed and/or modified under
% the terms of the GNU Public License, version 2.
%
% However, this file is supposed to be a template to be modified
% for your own needs. For this reason, if you use this file as a
% template and not specifically distribute it as part of a another
% package/program, I grant the extra permission to freely copy and
% modify this file as you see fit and even to delete this copyright
% notice. 


\mode<presentation>
{
  \usetheme{Berkeley}
  % or ...

  \setbeamercovered{transparent}
  % or whatever (possibly just delete it)
}


\usepackage[english]{babel}
% or whatever

\usepackage[latin1]{inputenc}
% or whatever

%\usepackage{mathtime}
\usefonttheme{serif}
%\usefonttheme{professionalfonts}
\usepackage{amsfonts}
\usepackage{amssymb}

\usepackage{amsmath}
\usepackage{multimedia}
\usepackage{mathrsfs}
\usepackage{mathabx}
\usepackage{color}
\usepackage{pstricks}
\usepackage{graphicx}
%\usepackage[pdftex, pdfborderstyle={/S/U/W 1}]{hyperref}
\usepackage{hyperref}
\usepackage{bbm}
\usepackage{cancel}
\usepackage[Symbol]{upgreek}
%\usepackage{mathbbol}
%\DeclareSymbolFontAlphabet{\amsmathbb}{AMSb}
%\usepackage[bbgreekl]{mathbbol}
%\usepackage[mtpbbi]{mtpro2}

%\usepackage[svgnames]{xcolor}

%\input{\fonts/math0}
%\input{\symb/structac} % Notations E. Savin
%\input{\symb/logos}

\newcommand{\ci}{\mathrm{i}}
\newcommand{\trace}{\operatorname{Tr}}
\newcommand{\Nset}{\mathbb{N}}
\newcommand{\Zset}{\mathbb{Z}}
\newcommand{\Rset}{\mathbb{R}}
\newcommand{\Cset}{\mathbb{C}}
\newcommand{\Sset}{\mathbb{S}}
\newcommand{\Mset}{\mathbb{M}}
\newcommand{\PhaseSpace}{\Omega}
\newcommand{\ContSet}{{\mathcal C}}
\newcommand{\id}{d}
\newcommand{\iD}{\mathrm{D}}
\newcommand{\iexp}{\mathrm{e}}
\newcommand{\demi}{\frac{1}{2}}
\newcommand{\imply}{\Rightarrow}

% Algebra
\newcommand{\itr}{{\sf T}}
\newcommand{\Id}{{\boldsymbol I}}
\newcommand{\IId}{\mathbb{I}}
\newcommand{\aj}{a}
\newcommand{\bj}{b}
\newcommand{\cj}{c}
\renewcommand{\dj}{d}
\newcommand{\av}{{\boldsymbol\aj}}
\newcommand{\bv}{{\boldsymbol\bj}}
\newcommand{\cv}{{\boldsymbol\cj}}
\newcommand{\dv}{{\boldsymbol\dj}}
\newcommand{\uj}{u}
\newcommand{\vj}{v}
\newcommand{\xj}{x}
\newcommand{\yj}{y}
\newcommand{\zj}{z}
\newcommand{\uv}{{\boldsymbol\uj}}
\newcommand{\vv}{{\boldsymbol\vj}}
\newcommand{\xv}{{\boldsymbol\xj}}
\newcommand{\yv}{{\boldsymbol\yj}}
\newcommand{\zv}{{\boldsymbol\zj}}
\newcommand{\Aj}{A}
\newcommand{\Bj}{B}
\newcommand{\Av}{{\boldsymbol\Aj}}
\newcommand{\Bv}{{\boldsymbol\Bj}}
\newcommand{\Zgv}{{\boldsymbol Z}}

% Analysis
\newcommand{\grad}{{\boldsymbol\nabla}}
\newcommand{\gradx}{{\grad_\xv}}
\newcommand{\Grad}{{\mathbb D}}
\newcommand{\Gradx}{{\Grad_\xv}}
\renewcommand{\div}{\mathrm{div}}
\newcommand{\divx}{{\div_\xv}}
\newcommand{\Div}{\mathbf{Div}}
\newcommand{\Divx}{{\Div_\xv}}

% Kinematics
\newcommand{\ej}{e}
\renewcommand{\ij}{i}
\newcommand{\pj}{p}
\newcommand{\ev}{{\boldsymbol\ej}}
\newcommand{\iv}{{\boldsymbol\ij}}
\newcommand{\pv}{{\boldsymbol\pj}}
\newcommand{\posij}{f}
\newcommand{\posiv}{{\boldsymbol\posij}}
\newcommand{\iposiv}{{\boldsymbol g}}
\newcommand{\Fp}{{\mathbb F}}
\newcommand{\GreenLj}{E}
\newcommand{\GreenL}{{\mathbb\GreenLj}}
\newcommand{\medium}{\Omega}
\newcommand{\strainj}{\varepsilon}
\newcommand*{\strain}{\mbox{$\hspace{0.2em}\rotatebox[x=0pt,y=0.2pt]{90}{\rule{0.02\linewidth}{0.4pt}}\hspace{-0.23em}\upvarepsilon$}} 

% Dynamics
\newcommand*{\stress}{\mbox{$\hspace{0.3em}\rotatebox[x=0pt,y=0.2pt]{90}{\rule{0.017\linewidth}{0.4pt}}\hspace{-0.25em}\upsigma$}} 

%\newcommand{\Zg}{{\bf\zgj}}
%\newcommand{\xigj}{\xi}
%\newcommand{\xig}{{\boldsymbol\xigj}}
\newcommand{\kgj}{k}
%\newcommand{\kgh}{\kgj_\ygj}
%\newcommand{\kg}{{\bf\kgj}}
\newcommand{\Kg}{{\bf K}}
%\newcommand{\qg}{{\boldsymbol q}}
%\newcommand{\pg}{{\boldsymbol p}}
\newcommand{\hkg}{{\hat \kg}}
\newcommand{\hpg}{\hat{\pg}}
%\newcommand{\vg}{{\boldsymbol v}}
\newcommand{\sg}{{\boldsymbol s}}
%\newcommand{\stress}{\mathbb{\sigma}}
\newcommand{\tenselas}{{\rm {\large C}}}
\newcommand{\speci}{{\mathrm w}}
\newcommand{\specij}{{\mathrm W}}
\newcommand{\speciv}{{\bf \specij}}
%\newcommand{\cjg}[1]{\overline{#1}}
\newcommand{\eigv}{{\bf b}}
\newcommand{\eigw}{{\bf c}}
\newcommand{\eigl}{\lambda}
\newcommand{\jeig}{\alpha}
\newcommand{\keig}{\beta}
\newcommand{\cel}{c}
\newcommand{\bcel}{{\bf\cel}}
\newcommand{\deng}{{\mathcal E}}
\newcommand{\flowj}{\pi}
\newcommand{\flow}{\boldsymbol\flowj}
\newcommand{\Flowj}{\Pi}
\newcommand{\Flow}{\boldsymbol\Flowj}
\newcommand{\fluxinj}{g}
\newcommand{\fluxin}{{\bf\fluxinj}}
\newcommand{\dscat}{\sigma}
\newcommand{\tdscat}{\Sigma}
\newcommand{\collop}{{\mathcal Q}}
\newcommand{\epsd}{\delta}
\newcommand{\rscat}{\rho}
\newcommand{\tscat}{\tau}
\newcommand{\Rscat}{\mathcal{R}}
\newcommand{\Tscat}{\mathcal{T}}
\newcommand{\lscat}{\ell}
%\newcommand{\floss}{\eta}
\newcommand{\mdiff}{{\bf D}}
%\newcommand{\demi}{\frac{1}{2}}
\newcommand{\domain}{{\mathcal O}}
\newcommand{\bdomain}{{\mathcal D}}
\newcommand{\interface}{\Gamma}
\newcommand{\sinterface}{\gamma_D}
%\newcommand{\normal}{\hat{\bf n}}
\newcommand{\bnabla}{\boldsymbol\nabla}
%\newcommand{\esp}[1]{\mathbb{E}\{\smash{#1}\}}
\newcommand{\mean}[1]{\underline{#1}}
%\newcommand{\BB}{\mathbb{B}}
%\newcommand{\II}{{\boldsymbol I}}
%\newcommand{\TA}{\boldsymbol{\Gamma}}
\newcommand{\Mdisp}{{\mathbf H}}
\newcommand{\Hamil}{{\mathcal H}}
\newcommand{\bzero}{{\bf 0}}

\newcommand{\mass}{M}
\newcommand{\damp}{D}
\newcommand{\stif}{K}
\newcommand{\dsp}{S}
\newcommand{\dof}{q}
\newcommand{\pof}{p}
%\newcommand{\MM}{{\boldsymbol\mass}}
\newcommand{\MD}{{\boldsymbol\damp}}
\newcommand{\MK}{{\boldsymbol\stif}}
\newcommand{\MS}{{\boldsymbol\dsp}}
\newcommand{\Cov}{{\boldsymbol C}}
\newcommand{\dofg}{{\boldsymbol\dof}}
\newcommand{\pofg}{{\boldsymbol\pof}}
\newcommand{\driftj}{b}
\newcommand{\drifts}{{\boldsymbol \driftj}}
\newcommand{\drift}{{\underline\drifts}}
\newcommand{\scatj}{a}
\newcommand{\scat}{{\boldsymbol\scatj}}
\newcommand{\diff}{{\boldsymbol\sigma}}
\newcommand{\load}{F}
\newcommand{\loadg}{{\boldsymbol\load}}
\newcommand{\pdf}{\pi}
\newcommand{\tpdf}{\pdf_t}
\newcommand{\fg}{{\boldsymbol f}}
%\newcommand{\Ugj}{U}
\newcommand{\Vgj}{V}
\newcommand{\Xgj}{X}
\newcommand{\Ygj}{Y}
%\newcommand{\Ug}{{\boldsymbol\Ugj}}
\newcommand{\Vg}{{\boldsymbol\Vgj}}
%\newcommand{\Qg}{{\boldsymbol Q}}
\newcommand{\Pg}{{\boldsymbol P}}
\newcommand{\Xg}{{\boldsymbol\Xgj}}
\newcommand{\Yg}{{\boldsymbol\Ygj}}
\newcommand{\flux}{{\boldsymbol J}}
\newcommand{\wiener}{W}
\newcommand{\whitenoise}{B}
\newcommand{\Wiener}{{\boldsymbol\wiener}}
\newcommand{\White}{{\boldsymbol\white}}
\newcommand{\paraj}{\nu}
\newcommand{\parag}{{\boldsymbol\paraj}}
\newcommand{\parae}{\hat{\parag}}
\newcommand{\erroj}{\epsilon}
\newcommand{\error}{{\boldsymbol\erroj}}
\newcommand{\biaj}{b}
\newcommand{\bias}{{\boldsymbol\biaj}}
%\newcommand{\disp}{{\boldsymbol V}}
\newcommand{\Fisher}{{\mathcal I}}
\newcommand{\likelihood}{{\mathcal L}}

\newcommand{\heps}{\varepsilon}
%\newcommand{\roi}{\varrho}
\newcommand{\jump}[1]{\llbracket{#1}\rrbracket}
\newcommand{\scal}[1]{\left\langle{#1}\right\rangle}
\newcommand{\norm}[1]{\left\|#1\right\|}
%\newcommand{\po}{\operatorname{o}}
\newcommand{\FFT}[1]{\widehat{#1}}
\newcommand{\indic}[1]{{\mathbf 1}_{#1}}
\newcommand{\impulse}{{\mathbbm h}}
\newcommand{\frf}{\FFT{\impulse}}

%\renewcommand{\Moy}[1]{{\boldsymbol\mu}_{#1}}
%\renewcommand{\Rcor}[1]{{\boldsymbol R}_{#1}}
%\renewcommand{\Mw}[1]{{\boldsymbol M}_{#1}}
%\renewcommand{\Sw}[1]{{\boldsymbol S}_{#1}}
%\renewcommand{\esp}[1]{{\mathbb E}\{#1\}}

\newcommand{\emphb}[1]{\textcolor{blue}{#1}}
\newcommand{\mycite}[1]{\textcolor{red}{#1}}
\newcommand{\mycitb}[1]{\textcolor{red}{[{\it #1}]}}

\newcommand{\PDFU}{{\mathcal U}}
\newcommand{\PDFN}{{\mathcal N}}
\newcommand{\TK}{{\boldsymbol\Pi}}
\newcommand{\TKij}{\pi}
\newcommand{\TKi}{{\boldsymbol\pi}}
\newcommand{\SMi}{\TKij^*}
\newcommand{\SM}{\TKi^*}
\newcommand{\lagmuli}{\lambda}
\newcommand{\lagmul}{{\boldsymbol\lagmuli}}
\newcommand{\constraint}{{\boldsymbol C}}
\newcommand{\mconstraint}{\mean{\constraint}}

\newcommand{\mybox}[1]{\fbox{\begin{minipage}{0.93\textwidth}{#1}\end{minipage}}}
\newcommand{\defcolor}[1]{\textcolor{blue}{#1}}

%\definecolor{rose}{LightPink}%{rgb}{251,204,231}

\newtheorem{mydef}{Definition}
\newtheorem{mythe}{Theorem}
\newtheorem{myprop}{Proposition}

% Or whatever. Note that the encoding and the font should match. If T1
% does not look nice, try deleting the line with the fontenc.

\title[1EL5000/S1]
{Large beam bending}

\subtitle{1EL5000--Continuum Mechanics -- Tutorial Class \#1} % (optional)

\author[\'E. Savin] % (optional, use only with lots of authors)
{\'E. Savin\inst{1,2}\\ \scriptsize{\texttt{eric.savin@\{centralesupelec,onera\}.fr}}}%\inst{1} }
% - Use the \inst{?} command only if the authors have different
%   affiliation.

\institute[Onera] % (optional, but mostly needed)
{\inst{1}{Information Processing and Systems Dept.\\\Onera, France}
\and
 \inst{2}{Mechanical and Civil Engineering Dept.\\\ECP, France}}%
%  Department of Theoretical Philosophy\\
%  University of Elsewhere}
% - Use the \inst command only if there are several affiliations.
% - Keep it simple, no one is interested in your street address.

%\date[Short Occasion] % (optional)
\date{\today}

\subject{Large beam bending}
% This is only inserted into the PDF information catalog. Can be left
% out. 



% If you have a file called "university-logo-filename.xxx", where xxx
% is a graphic format that can be processed by latex or pdflatex,
% resp., then you can add a logo as follows:

% \pgfdeclareimage[height=0.5cm]{university-logo}{university-logo-filename}
% \logo{\pgfuseimage{university-logo}}



% Delete this, if you do not want the table of contents to pop up at
% the beginning of each subsection:
\AtBeginSection[]
%\AtBeginSubsection[]
{
  \begin{frame}<beamer>{Outline}
    \tableofcontents[currentsection]%,currentsubsection]
  \end{frame}
}


% If you wish to uncover everything in a step-wise fashion, uncomment
% the following command: 

%\beamerdefaultoverlayspecification{<+->}


\begin{document}

\begin{frame}
  \titlepage
\end{frame}

\begin{frame}{Outline}
  \tableofcontents
  % You might wish to add the option [pausesections]
\end{frame}


% Since this a solution template for a generic talk, very little can
% be said about how it should be structured. However, the talk length
% of between 15min and 45min and the theme suggest that you stick to
% the following rules:  

% - Exactly two or three sections (other than the summary).
% - At *most* three subsections per section.
% - Talk about 30s to 2min per frame. So there should be between about
%   15 and 30 frames, all told.

\section{Some algebra}
\subsection{Vector \& tensor products}

\begin{frame}{Some algebra}{Vector \& tensor products}

\begin{itemize}
\item Scalar product:
\begin{displaymath}
\av,\bv\in\Rset^a\,,\quad\scal{\av,\bv}=\sum_{j=1}^a\aj_j\bj_j=\aj_j\bj_j\,,
\end{displaymath}
The last equality is \emphb{Einstein's summation convention}.
\item Tensors and tensor product (or outer product):
\begin{displaymath}
\Av\in\Rset^a\to\Rset^b\,,\quad\Av=\av\otimes\bv\,,\quad\av\in\Rset^a\,,\bv\in\Rset^b\,.
\end{displaymath}
\item Tensor application to vectors:
\begin{displaymath}
\Av=\av\otimes\bv\in\Rset^a\to\Rset^b\,,\cv\in\Rset^b\,,\quad\Av\cv=\scal{\bv,\cv}\av\,.
\end{displaymath}
\item Product of tensors $\equiv$ composition of linear maps:
\begin{displaymath}
\Av=\av\otimes\bv\,,\Bv=\cv\otimes\dv\,,\quad\Av\Bv=\scal{\bv,\cv}\av\otimes\dv\,.
\end{displaymath}
\end{itemize}

\end{frame}

\begin{frame}{Some algebra}{Vector \& tensor products}

%\onslide<2|handout:2>

\begin{itemize}
\item Scalar product of tensors:
\begin{displaymath}
\scal{\Av,\Bv}=\trace(\Av\Bv^\itr):=\Av:\Bv=\Aj_{jk}\Bj_{jk}\,.
\end{displaymath}
\item Let $\{\ev_j\}_{j=1}^d$ be a Cartesian basis in $\Rset^d$. Then:
\begin{displaymath}
\begin{split}
\aj_j &=\scal{\av,\ev_j}\,, \\
\Aj_{jk} &=\scal{\Av,\ev_j\otimes\ev_k}=\Av:\ev_j\otimes\ev_k \\
&=\scal{\Av\ev_k,\ev_j}\,,
\end{split}
\end{displaymath}
such that:
\begin{displaymath}
\begin{split}
\av &=\aj_j\ev_j\,, \\
\Av &=\Aj_{jk}\ev_j\otimes\ev_k\,.
\end{split}
\end{displaymath}
\item Example: the identity matrix
\begin{displaymath}
\Id=\ev_j\otimes\ev_j\,.
\end{displaymath}
\end{itemize}

%\end{overprint}

\end{frame}

\subsection{Vector \& tensor analysis}

\begin{frame}{Some analysis}{Vector \& tensor analysis}

\begin{itemize}
\item Gradient of a vector function $\av(\xv)$, $\xv\in\Rset^d$:
\begin{displaymath}
\Gradx\av=\frac{\partial\av}{\partial\xj_j}\otimes\ev_j\,.
\end{displaymath}
\item Divergence of a vector function $\av(\xv)$, $\xv\in\Rset^d$:
\begin{displaymath}
\divx\av=\scal{\gradx,\av}=\trace(\Gradx\av)=\frac{\partial\aj_j}{\partial\xj_j}\,.
\end{displaymath}
\item Divergence of a tensor function $\Av(\xv)$, $\xv\in\Rset^d$:
\begin{displaymath}
\Divx\Av=\frac{\partial(\Av\ev_j)}{\partial\xj_j}\,.
\end{displaymath}
\end{itemize}

\end{frame}

\begin{frame}{Some analysis}{Vector \& tensor analysis in cylindrical coordinates}

\begin{itemize}
\item Gradient of a vector function $\av(r,\theta,\zj)$:
\begin{displaymath}
\Gradx\av=\frac{\partial\av}{\partial r}\otimes\ev_r+\frac{\partial\av}{\partial\theta}\otimes\frac{\ev_\theta}{r}+\frac{\partial\av}{\partial\zj}\otimes\ev_\zj\,.
\end{displaymath}
\item Divergence of a vector function $\av(r,\theta,\zj)$:
\begin{displaymath}
\divx\av=\scal{\frac{\partial\av}{\partial r},\ev_r}+\scal{\frac{\partial\av}{\partial\theta},\frac{\ev_\theta}{r}}+\scal{\frac{\partial\av}{\partial\zj},\ev_\zj}\,.
\end{displaymath}
\item Divergence of a tensor function $\Av(r,\theta,\zj)$:
\begin{displaymath}
\Divx\Av=\frac{\partial\Av}{\partial r}\ev_r+\frac{\partial\Av}{\partial\theta}\frac{\ev_\theta}{r}+\frac{\partial\Av}{\partial\zj}\ev_\zj\,.
\end{displaymath}
\end{itemize}

\end{frame}



\section{Kinematics}
\begin{frame}{Kinematics}

\begin{figure}
\centering\includegraphics[scale=.25]{\figs/ch1-Configs0t}
\end{figure}
\begin{displaymath}
\boxed{\xv=\posiv(\pv,t)\,,\quad\uv=\posiv(\pv,t)-\pv}
\end{displaymath}
%\vskip-5pt
{\footnotesize
\begin{itemize}
\item $\pv\mapsto\posiv(\pv,t):\medium_0\to\medium_t$ is injective: two material points cannot occupy the same volume at the same time ($\pv_1\neq\pv_2\imply\xv_1\neq\xv_2$);
\item $\posiv$ is piecewise $\ContSet^1$ in space $\pv\in\medium_0$ and time $t\in\Rset$;
\item Some inverse transform $\xv\mapsto\iposiv(\xv,t):\medium_t\to\medium_0$ exists;
\item Matter cannot vanish:
\begin{displaymath}
\det(\ev_1,\ev_2,\ev_3)>0\iff\det(\iv_1,\iv_2,\iv_3)>0\quad\text{for}\;\ev_j=\posiv(\iv_j,t)\,.
\end{displaymath}
\end{itemize}}

\end{frame}

\section{Strains}
\begin{frame}{Space gradient tensor}{...or the homogeneous tangent transformation}

\begin{figure}
\centering\includegraphics[scale=.25]{\figs/ch1-Strains}
\end{figure}
\begin{displaymath}
\id\xj_j=\frac{\partial\posij_j}{\partial\pj_k}\id\pj_k
\end{displaymath}
\begin{itemize}
\item Space gradient tensor $\Fp=\Grad_\pv\xv$:
\begin{displaymath}
\id\xv=\Fp\id\pv\,,\quad\id\uv=(\Fp-\Id)\id\pv
\end{displaymath}
\item In a Cartesian frame/curvilinear frame:
\begin{displaymath}
\Fp=\frac{\partial\xv}{\partial\pj_j}\otimes\iv_j=\frac{\partial\xv}{\partial\xi_j}\otimes\grad_\pv\xi_j
\end{displaymath}
\end{itemize}

\end{frame}

\begin{frame}{Green-Lagrange tensor}{Relative length variation}

\begin{overprint}

\onslide<1|handout:1>

\begin{figure}
\centering\includegraphics[scale=.25]{\figs/ch1-Strains}
\end{figure}
\begin{displaymath}
\begin{split}
\id\xv &=\Fp\id\pv \\
\norm{\id\xv}^2 &=\scal{\Fp\id\pv,\Fp\id\pv} \\
&=\scal{\id\pv,\Fp^\itr\Fp\id\pv} \\
\frac{\norm{\id\xv}^2-\norm{\id\pv}^2}{\norm{\id\pv}^2} &=\scal{(\Fp^\itr\Fp-\Id)\frac{\id\pv}{\norm{\id\pv}},\frac{\id\pv}{\norm{\id\pv}}}
\end{split}
\end{displaymath}

\onslide<2|handout:2>

\begin{itemize}
\item The Green-Lagrange tensor:
\begin{displaymath}
\boxed{\GreenL=\demi\left(\Fp^\itr\Fp-\Id\right)}\,.
\end{displaymath}
\item The dilatation tensor, or right Cauchy-Green tensor $\Cset=\Fp^\itr\Fp$.
\item Measuring stretching of a fiber $\id\pv=\id\pj\iv_j$:
\begin{displaymath}
\frac{\norm{\id\xv}^2-\norm{\id\pv}^2}{\norm{\id\pv}^2}=2\scal{\GreenL\iv_j,\iv_j}
\end{displaymath}
\begin{displaymath}
\left(\GreenLj_{jj}=\demi\left(\frac{\ell^2-\ell_0^2}{\ell_0^2}\right)=\frac{\Delta\ell}{\ell_0}+\demi\left(\frac{\Delta\ell}{\ell_0}\right)^2\right)
\end{displaymath}
\end{itemize}

\end{overprint}

\end{frame}

\begin{frame}{Green-Lagrange tensor}{Relative angle variation}

\begin{overprint}

\onslide<1|handout:1>

\begin{figure}
\centering\includegraphics[scale=.25]{\figs/ch1-RotStretch}
\end{figure}
\begin{displaymath}
\id\xv_1=\Fp\id\pv_1\,,\quad\id\xv_2=\Fp\id\pv_2 \\
\end{displaymath}
\begin{displaymath}
\begin{split}
\scal{\id\xv_1,\id\xv_2} &=\scal{\Fp\id\pv_1,\Fp\id\pv_2} \\
&=\scal{\Fp^\itr\Fp\id\pv_1,\id\pv_2} \\
\norm{\id\xv_1}\norm{\id\xv_2}\cos\theta_{t12} &=\scal{(2\GreenL+\Id)\id\pv_1,\id\pv_2}
\end{split}
\end{displaymath}

\onslide<2|handout:2>

\begin{itemize}
\item The Green-Lagrange tensor:
\begin{displaymath}
\boxed{\GreenL=\demi\left(\Fp^\itr\Fp-\Id\right)}\,.
\end{displaymath}
\item Measuring angle between $\id\pv_1=\id\pj_1\iv_1$ and $\id\pv_2=\id\pj_2\iv_2$:
\begin{displaymath}
\begin{split}
\norm{\id\xv_1}\norm{\id\xv_2}\cos\theta_{t12} &=\scal{(2\GreenL+\Id)\id\pj_1\iv_1,\id\pj_2\iv_2} \\
&=2\id\pj_1\id\pj_2\scal{\GreenL\iv_1,\iv_2}
\end{split}
\end{displaymath}
\begin{displaymath}
\left(\GreenLj_{12}=\demi\left(\frac{\ell_1\ell_2}{\ell_{01}\ell_{02}}\right)\cos\theta_{t12}=\demi\frac{\ell_1}{\ell_{01}}\frac{\ell_2}{\ell_{02}}\sin\left(\frac{\pi}{2}-\theta_{t12}\right)\right)
\end{displaymath}
\end{itemize}
\vskip-10pt
\begin{figure}
\centering\includegraphics[scale=.15]{\figs/ch1-Angle}
\end{figure}

\end{overprint}

\end{frame}

\begin{frame}{Green-Lagrange tensor}{Units and scales}

\begin{itemize}
\item Strains are dimensionless "m/m";
\item Steel: $\norm{\GreenL}\approx10^{-3}$;
\item Soil: $\norm{\GreenL}\approx10^{-2}$;
\item Rubber: $\norm{\GreenL}\approx 2$;
\item Molecules: $L\approx 1\,\text{\AA}=10^{-10}\,\text{m}$, $T\approx1\,\text{ps}=10^{-12}\,\text{s}$;
\item Continuum mechanics: $L\gtrsim 10\,\text{nm}=10^{-8}\,\text{m}$, $T\gtrsim 1\,\text{ns}=10^{-9}\,\text{s}$.
\end{itemize}

%\begin{table}
%\centering
%\begin{tabular}{| c | c |}
%\hline
%Material & $\norm{\GreenL} \\ \hline
%Steel & $10^{-3}$ \\ \hline
%Soil & $10^{-2}$  \\  \hline
%Rubber & $2$ \\ \hline
%\end{tabular}
%\caption{Strains are dimensionless "$m/m$"}
%\end{table}

\end{frame}

\begin{frame}{Small strains}{Linearization}

\begin{figure}
\centering\includegraphics[scale=.25]{\figs/ch1-SmallStrains}
\end{figure}
\begin{displaymath}
\medium_t\simeq\medium_0\quad(\xv\simeq\pv)
\end{displaymath}
\begin{displaymath}
\max_{\pv,t}\norm{\uv}\ll L\,,\quad\max_{\pv,t}\norm{\Grad_\pv\uv}^2=\max_{\pv,t}\trace(\Grad_\pv\uv^\itr\Grad_\pv\uv)\ll 1
\end{displaymath}
\begin{displaymath}
\begin{split}
\xv &=\pv+\uv(\pv,t) \\
\Fp &=\Id+\Grad_\pv\uv \\
\GreenL &=\demi\left(\Fp^\itr\Fp-\Id\right) \\
&=\demi\left(\Grad_\pv\uv+\Grad_\pv\uv^\itr+\cancel{\Grad_\pv\uv^\itr\Grad_\pv\uv}\right)
\end{split}
\end{displaymath}

\end{frame}

\begin{frame}{Small strains}{Linearized strain tensor}

\begin{itemize}
\item Linearized (small) strain tensor with $\xv\simeq\pv$:
\begin{displaymath}
\GreenL\simeq\strain=\demi\left(\Gradx\uv+\Gradx\uv^\itr\right)=\frac{\partial\uv}{\partial\xj_j}\otimes_s\ev_j\,,
\end{displaymath}
where $\av\otimes_s\bv=\demi(\av\otimes\bv+\bv\otimes\av)$.
\item Small stretching and small relative rotations:
\begin{displaymath}
\strainj_{jj}=\scal{\strain\ev_j,\ev_j}=\frac{\Delta\ell}{\ell_0}\,,\quad\strainj_{jk}=\scal{\strain\ev_k,\ev_j}=\frac{\theta_{t12}^*}{2}\,.
\end{displaymath}
\item Local volume change:
\begin{displaymath}
\frac{\Delta V}{V_0}=\frac{\ell_{01}+\Delta\ell_1}{\ell_{01}}\frac{\ell_{02}+\Delta\ell_2}{\ell_{02}}\frac{\ell_{03}+\Delta\ell_3}{\ell_{03}}-1\approx\trace\strain\,.
\end{displaymath}
\end{itemize}

\end{frame}

\begin{frame}{Green-Lagrange tensor}{Example}

\begin{overprint}

\onslide<1|handout:1>
\vskip-20pt
\begin{block}{Torsion of a cylinder}
\begin{figure}
\centering\includegraphics[scale=.25]{\figs/ch1-torsion}
\end{figure}
\begin{itemize}
\item $\pv=r\ev_r(\theta)+z\ev_z$;
\item $\xv=r\iv_r(\theta+tr)+z\ev_z$;
\item $\GreenL$?
\end{itemize}
\end{block}

\onslide<2|handout:2>
\vskip-20pt
\begin{block}{Torsion of a cylinder}
\begin{itemize}
\item $\pv=r\ev_r(\theta)+z\ev_z$, $\xv=r\ev_r(\theta+tr)+z\ev_z$; $\GreenL$?
\item First compute $\Fp$:
{\scriptsize
\begin{displaymath}
\begin{split}
\Fp &=\frac{\partial\xv}{\partial r}\otimes\ev_r+\frac{\partial\xv}{\partial\theta}\otimes\frac{\ev_\theta}{r}+\frac{\partial\xv}{\partial\zj}\otimes\ev_\zj \\
&=(\ev_r(\theta+tr)+tr\ev_\theta(\theta+tr))\otimes\ev_r(\theta)+\ev_\theta(\theta+tr)\otimes\ev_\theta(\theta)+\ev_z\otimes\ev_z
\end{split}
\end{displaymath}}
\end{itemize}
\end{block}

\onslide<3|handout:3>
\vskip-20pt
\begin{block}{Torsion of a cylinder}
\begin{itemize}
\item $\pv=r\ev_r(\theta)+z\ev_z$, $\xv=r\ev_r(\theta+tr)+z\ev_z$; $\GreenL$?
\item First compute $\Fp$:
{\scriptsize
\begin{displaymath}
\begin{split}
\Fp &=\frac{\partial\xv}{\partial r}\otimes\ev_r+\frac{\partial\xv}{\partial\theta}\otimes\frac{\ev_\theta}{r}+\frac{\partial\xv}{\partial\zj}\otimes\ev_\zj \\
&=(\ev_r(\theta+tr)+tr\ev_\theta(\theta+tr))\otimes\ev_r(\theta)+\ev_\theta(\theta+tr)\otimes\ev_\theta(\theta)+\ev_z\otimes\ev_z \\
\Fp^\itr &=\ev_r(\theta)\otimes(\ev_r(\theta+tr)+tr\ev_\theta(\theta+tr))+\ev_\theta(\theta)\otimes\ev_\theta(\theta+tr)+\ev_z\otimes\ev_z
\end{split}
\end{displaymath}}
\end{itemize}
\end{block}

\onslide<4|handout:4>
\vskip-20pt
\begin{block}{Torsion of a cylinder}
\begin{itemize}
\item $\pv=r\ev_r(\theta)+z\ev_z$, $\xv=r\ev_r(\theta+tr)+z\ev_z$; $\GreenL$?
\item First compute $\Fp$:
{\scriptsize
\begin{displaymath}
\begin{split}
\Fp&=(\ev_r(\theta+tr)+tr\ev_\theta(\theta+tr))\otimes\ev_r(\theta)+\ev_\theta(\theta+tr)\otimes\ev_\theta(\theta)+\ev_z\otimes\ev_z \\
\Fp^\itr &=\ev_r(\theta)\otimes(\ev_r(\theta+tr)+tr\ev_\theta(\theta+tr))+\ev_\theta(\theta)\otimes\ev_\theta(\theta+tr)+\ev_z\otimes\ev_z
\end{split}
\end{displaymath}}
\item Then $\Fp^\itr\Fp$ (remind that $(\av\otimes\bv)(\cv\otimes\dv)=\scal{\bv,\cv}\av\otimes\dv$):
{\scriptsize
\begin{displaymath}
\begin{split}
\Fp^\itr\Fp= &\,(1+(tr)^2)\ev_r(\theta)\otimes\ev_r(\theta)+tr(\ev_r(\theta)\otimes\ev_\theta(\theta)+\ev_\theta(\theta)\otimes\ev_r(\theta)) \\
&\,+\ev_\theta(\theta)\otimes\ev_\theta(\theta)+\ev_z\otimes\ev_z
\end{split}
\end{displaymath}}
\end{itemize}
\end{block}

\onslide<5|handout:5>
\vskip-20pt
\begin{block}{Torsion of a cylinder}
\begin{itemize}
\item $\pv=r\ev_r(\theta)+z\ev_z$, $\xv=r\ev_r(\theta+tr)+z\ev_z$; $\GreenL$?
\item First compute $\Fp$:
{\scriptsize
\begin{displaymath}
\begin{split}
\Fp&=(\ev_r(\theta+tr)+tr\ev_\theta(\theta+tr))\otimes\ev_r(\theta)+\ev_\theta(\theta+tr)\otimes\ev_\theta(\theta)+\ev_z\otimes\ev_z \\
\Fp^\itr &=\ev_r(\theta)\otimes(\ev_r(\theta+tr)+tr\ev_\theta(\theta+tr))+\ev_\theta(\theta)\otimes\ev_\theta(\theta+tr)+\ev_z\otimes\ev_z
\end{split}
\end{displaymath}}
\item Then $\Fp^\itr\Fp$ (remind that $(\av\otimes\bv)(\cv\otimes\dv)=\scal{\bv,\cv}\av\otimes\dv$):
{\scriptsize
\begin{displaymath}
\begin{split}
\Fp^\itr\Fp= &\,(1+(tr)^2)\ev_r(\theta)\otimes\ev_r(\theta)+tr(\ev_r(\theta)\otimes\ev_\theta(\theta)+\ev_\theta(\theta)\otimes\ev_r(\theta)) \\
&\,+\ev_\theta(\theta)\otimes\ev_\theta(\theta)+\ev_z\otimes\ev_z
\end{split}
\end{displaymath}}
\item Then $\GreenL=\demi(\Fp^\itr\Fp-\Id)$ ($\Id=\ev_r\otimes\ev_r+\ev_\theta\otimes\ev_\theta+\ev_z\otimes\ev_z$):
{\scriptsize
\begin{displaymath}
\GreenL =\demi(tr)^2\ev_r(\theta)\otimes\ev_r(\theta)+tr\ev_r(\theta)\otimes_s\ev_\theta(\theta)\,.
\end{displaymath}}
\end{itemize}
\end{block}

\end{overprint}

\end{frame}


\section{1.3 Large beam bending}

\begin{frame}{Large beam bending}{Setup}

\begin{columns}[t]
\column{.5\textwidth}
\centering\includegraphics[scale=.25]{\figs/ch1-LargeDefBeam1}
\column{.5\textwidth}
\vskip-145pt
\centering\includegraphics[scale=.45]{\figs/ch1-LargeDefBeam-ex}
\vskip-5pt{\hspace{3.2truecm}\mbox{\tiny{\copyright\ G. Puel}}}
\end{columns}
\begin{displaymath}
\begin{split}
\pv &=\pj_1\iv_1+\pj_2\iv_2\,,\quad(\pj_1,\pj_2)\in]-h,h[\times]0,L[ \\
\xv &=\posiv(\pv,t) \\
&=\xv_G(\pj_2,t)+\pj_1\iv_r(\theta(\pj_2,t))
\end{split}
\end{displaymath}

\end{frame}

\begin{frame}{Large beam bending}{Solution}

\begin{exampleblock}{Question \#1: Physical interpretation}
\begin{itemize}
\item $\pj_1=0\imply\xv(0,\pj_2,t)=\xv_G(\pj_2,t)$: position of the middle-line;
\item $\pj_1\iv_r(\theta(\pj_2,t))$: rotation of the cross section when $\pj_2\mapsto\theta(\pj_2,t)$ increases.
\end{itemize}
\vskip-10pt
\begin{figure}
\centering\includegraphics[scale=.25]{\figs/ch1-BeamBending1}
\vskip-5pt{\hspace{3truecm}\mbox{\tiny{\copyright\ F. Gatti}}}
\end{figure}
\vskip-10pt
\begin{displaymath}
\xv(\pv,t)=\xv_G(\pj_2,t)+\pj_1\iv_r(\theta(\pj_2,t))
\end{displaymath}
\end{exampleblock}

\end{frame}

\begin{frame}{Large beam bending}{Solution}

\begin{overprint}

\onslide<1|handout:1>
\begin{exampleblock}{Question \#2: Green-Lagrange tensor}
\begin{itemize}
\item $\GreenL=\demi(\Fp^\itr\Fp-\Id)$ with $\Fp=\Grad_\pv\xv$, or:
\begin{displaymath}
\Fp=\frac{\partial\xv}{\partial\pj_j}\otimes\iv_j=\frac{\partial\xv}{\partial\pj_1}\otimes\iv_1+\frac{\partial\xv}{\partial\pj_2}\otimes\iv_2\,.
\end{displaymath}
\end{itemize}
\end{exampleblock}

\onslide<2|handout:2>
\vskip-20pt
\begin{exampleblock}{Question \#2: Green-Lagrange tensor}
\begin{itemize}
\item $\GreenL=\demi(\Fp^\itr\Fp-\Id)$ with $\Fp=\Grad_\pv\xv$, or:
{\footnotesize
\begin{displaymath}
\Fp=\frac{\partial\xv}{\partial\pj_j}\otimes\iv_j=\frac{\partial\xv}{\partial\pj_1}\otimes\iv_1+\frac{\partial\xv}{\partial\pj_2}\otimes\iv_2\,.
\end{displaymath}}
\item By direct computation ($\xv=\xv_G(\pj_2,t)+\pj_1\iv_r(\theta(\pj_2,t))$):
{\footnotesize
\begin{displaymath}
\begin{split}
\frac{\partial\xv}{\partial\pj_1} &=\iv_r(\theta(\pj_2,t)) \\
\frac{\partial\xv}{\partial\pj_2} &=\frac{\partial\xv_G}{\partial\pj_2}(\pj_2,t)+\pj_1\frac{\partial\theta}{\partial\pj_2}(\pj_2,t)\iv_\theta(\theta(\pj_2,t))
\end{split}
\end{displaymath}}
\item Therefore:
{\footnotesize
\begin{displaymath}
\Fp=\iv_r(\theta(\pj_2,t))\otimes\iv_1+\frac{\partial\xv_G}{\partial\pj_2}(\pj_2,t)\otimes\iv_2+\pj_1\frac{\partial\theta}{\partial\pj_2}(\pj_2,t)\iv_\theta(\theta(\pj_2,t))\otimes\iv_2\,.
\end{displaymath}}
\end{itemize}
\end{exampleblock}

\onslide<3|handout:3>
\vskip-20pt
\begin{exampleblock}{Question \#2: Green-Lagrange tensor}
\begin{itemize}
\item $\GreenL=\demi(\Fp^\itr\Fp-\Id)$ with:
\vskip-10pt
{\scriptsize
\begin{displaymath}
\begin{split}
\!\!\!\!\!\!\!\!\Fp &=\iv_r(\theta(\pj_2,t))\otimes\iv_1+\frac{\partial\xv_G}{\partial\pj_2}(\pj_2,t)\otimes\iv_2+\pj_1\frac{\partial\theta}{\partial\pj_2}(\pj_2,t)\iv_\theta(\theta(\pj_2,t))\otimes\iv_2\,, \\
\!\!\!\!\!\!\!\!\Fp^\itr &=\iv_1\otimes\iv_r(\theta(\pj_2,t))+\iv_2\otimes\frac{\partial\xv_G}{\partial\pj_2}(\pj_2,t)+\pj_1\frac{\partial\theta}{\partial\pj_2}(\pj_2,t)\iv_2\otimes\iv_\theta(\theta(\pj_2,t))\,.
\end{split}
\end{displaymath}}
\item Reminding that $(\av\otimes\bv)(\cv\otimes\dv)=\scal{\bv,\cv}\av\otimes\dv$:
{\scriptsize
\begin{multline*}
\Fp^\itr\Fp=\scal{\iv_r,\iv_r}\iv_1\otimes\iv_1+\scal{\frac{\partial\xv_G}{\partial\pj_2},\iv_r}\iv_1\otimes\iv_2+\pj_1\frac{\partial\theta}{\partial\pj_2}\scal{\iv_r,\iv_\theta}\iv_1\otimes\iv_2 \\
+\scal{\frac{\partial\xv_G}{\partial\pj_2},\iv_r}\iv_2\otimes\iv_1+\scal{\frac{\partial\xv_G}{\partial\pj_2},\frac{\partial\xv_G}{\partial\pj_2}}\iv_2\otimes\iv_2+\pj_1\frac{\partial\theta}{\partial\pj_2}\scal{\frac{\partial\xv_G}{\partial\pj_2},\iv_\theta}\iv_2\otimes\iv_2 \\
+\pj_1\frac{\partial\theta}{\partial\pj_2}\scal{\iv_\theta,\iv_r}\iv_2\otimes\iv_1+\pj_1\frac{\partial\theta}{\partial\pj_2}\scal{\frac{\partial\xv_G}{\partial\pj_2},\iv_\theta}\iv_2\otimes\iv_2+\left(\pj_1\frac{\partial\theta}{\partial\pj_2}\right)^2\scal{\iv_\theta,\iv_\theta}\iv_2\otimes\iv_2
\end{multline*}}
\end{itemize}
\end{exampleblock}

\onslide<4|handout:4>
\vskip-20pt
\begin{exampleblock}{Question \#2: Green-Lagrange tensor}
\begin{itemize}
\item $\GreenL=\demi(\Fp^\itr\Fp-\Id)$ with:
\vskip-10pt
{\scriptsize
\begin{displaymath}
\begin{split}
\!\!\!\!\!\!\!\!\Fp &=\iv_r(\theta(\pj_2,t))\otimes\iv_1+\frac{\partial\xv_G}{\partial\pj_2}(\pj_2,t)\otimes\iv_2+\pj_1\frac{\partial\theta}{\partial\pj_2}(\pj_2,t)\iv_\theta(\theta(\pj_2,t))\otimes\iv_2\,, \\
\!\!\!\!\!\!\!\!\Fp^\itr &=\iv_1\otimes\iv_r(\theta(\pj_2,t))+\iv_2\otimes\frac{\partial\xv_G}{\partial\pj_2}(\pj_2,t)+\pj_1\frac{\partial\theta}{\partial\pj_2}(\pj_2,t)\iv_2\otimes\iv_\theta(\theta(\pj_2,t))\,.
\end{split}
\end{displaymath}}
\item Reminding that $(\av\otimes\bv)(\cv\otimes\dv)=\scal{\bv,\cv}\av\otimes\dv$:
{\scriptsize
\begin{displaymath}
\begin{split}
\Fp^\itr\Fp= &\iv_1\otimes\iv_1+2\scal{\frac{\partial\xv_G}{\partial\pj_2},\iv_r}\iv_1\otimes_s\iv_2+2\pj_1\frac{\partial\theta}{\partial\pj_2}\scal{\frac{\partial\xv_G}{\partial\pj_2},\iv_\theta}\iv_2\otimes\iv_2 \\
&+\norm{\frac{\partial\xv_G}{\partial\pj_2}}^2\iv_2\otimes\iv_2+\left(\pj_1\frac{\partial\theta}{\partial\pj_2}\right)^2\iv_2\otimes\iv_2
\end{split}
\end{displaymath}}
\end{itemize}
\end{exampleblock}

\onslide<5|handout:5>
\vskip-20pt
\begin{exampleblock}{Question \#2: Green-Lagrange tensor}
\begin{itemize}
\item $\GreenL=\demi(\Fp^\itr\Fp-\Id)$ with:
\vskip-10pt
{\scriptsize
\begin{displaymath}
\begin{split}
\Fp^\itr\Fp= &\iv_1\otimes\iv_1+2\scal{\frac{\partial\xv_G}{\partial\pj_2},\iv_r}\iv_1\otimes_s\iv_2+2\pj_1\frac{\partial\theta}{\partial\pj_2}\scal{\frac{\partial\xv_G}{\partial\pj_2},\iv_\theta}\iv_2\otimes\iv_2 \\
&+\norm{\frac{\partial\xv_G}{\partial\pj_2}}^2\iv_2\otimes\iv_2+\left(\pj_1\frac{\partial\theta}{\partial\pj_2}\right)^2\iv_2\otimes\iv_2
\end{split}
\end{displaymath}}
\item But $\Id=\iv_1\otimes\iv_1+\iv_2\otimes\iv_2$ and therefore:
\vskip-10pt
{\scriptsize
\begin{displaymath}
\boxed{\GreenL=\scal{\frac{\partial\xv_G}{\partial\pj_2},\iv_r}\iv_1\otimes_s\iv_2+\demi\left[\left(\frac{\partial\xv_G}{\partial\pj_2}+\pj_1\frac{\partial\theta}{\partial\pj_2}\iv_\theta\right)^2-1\right]\iv_2\otimes\iv_2}\,.
\end{displaymath}}
\end{itemize}
\end{exampleblock}

\end{overprint}

\end{frame}

\begin{frame}{Large beam bending}{Solution}

\begin{exampleblock}{Question \#3: $\scal{\GreenL\iv_1,\iv_1}$}
\begin{itemize}
\item $\scal{\GreenL\iv_1,\iv_1}=\GreenLj_{11}$ is the $\iv_1\otimes\iv_1$ term of $\GreenL$: $\GreenLj_{11}=0$!!!
\item There is no stretching along the $\iv_1$ axis/$\pv_1$ coordinate.
\item However:
\begin{displaymath}
\scal{\GreenL\iv_2,\iv_2}=\GreenLj_{22}=\demi\left[\left(\frac{\partial\xv_G}{\partial\pj_2}+\pj_1\frac{\partial\theta}{\partial\pj_2}\iv_\theta\right)^2-1\right]\,.
\end{displaymath}
\end{itemize}
\end{exampleblock}

\end{frame}

\begin{frame}{Large beam bending}{Solution}

\begin{exampleblock}{Question \#4: $\xv_G(\pj_2)=\frac{L}{\pi}(-\iv_1+\iv_r(\theta(\pj_2)))$, $\theta(\pj_2)=\frac{\pi\pj_2}{L}$}
\begin{itemize}
\item $\theta(0)=0$ and $\theta(L)=\pi$ so that $\pj_2\mapsto\xv_G(\pj_2)$ is an half-circle with $\xv_G(0)=\bzero$ and $\xv_G(L)=-\frac{2L}{\pi}\iv_1$. 
\end{itemize}
\centering\includegraphics[scale=.3]{\figs/ch1-BeamHalfCircle-c}
%\vskip-5pt{\hspace{3truecm}\mbox{\tiny{\copyright\ F. Gatti}}}
\end{exampleblock}

\end{frame}

\begin{frame}{Large beam bending}{Solution}

\begin{exampleblock}{Question \#5: $\GreenLj_{22}$}
\begin{itemize}
\item $\frac{\partial\xv_G}{\partial\pj_2}=\frac{L}{\pi}\theta'(\pj_2)\iv_\theta(\theta(\pj_2))$ and $\theta'(\pj_2)=\frac{\pi}{L}$, therefore:
\begin{columns}
\column{.45\textwidth}
{\scriptsize
\begin{displaymath}
\begin{split}
\;\GreenLj_{22} &=\scal{\GreenL\iv_2,\iv_2} \\
&=\demi\left[\left(\frac{\partial\xv_G}{\partial\pj_2}+\pj_1\frac{\partial\theta}{\partial\pj_2}\iv_\theta\right)^2-1\right] \\
&=\demi\left[\left(1+\pi\frac{\pj_1}{L}\right)^2-1\right] \\
&=\pi\frac{\pj_1}{L}\left(1+\frac{\pi}{2}\frac{\pj_1}{L}\right)
\end{split}
\end{displaymath}}
\column{.55\textwidth}
\centering\includegraphics[scale=.2]{\figs/ch1-E22}
\vskip-5pt{\hspace{4.3truecm}\mbox{\tiny{\copyright\ G. Puel}}}
\end{columns}
\item The maximum is obtained for $\pj_1=+h$ and the minimum for $\pj_1=-h$.
\end{itemize}
\end{exampleblock}

\end{frame}

\begin{frame}{Large beam bending}{Solution}

\begin{exampleblock}{Question \#6: $\GreenLj_{12}$}
\begin{itemize}
\item $\frac{\partial\xv_G}{\partial\pj_2}=\frac{L}{\pi}\theta'(\pj_2)\iv_\theta(\theta(\pj_2))$ and $\theta'(\pj_2)=\frac{\pi}{L}$, therefore:
\begin{displaymath}
\GreenLj_{12}=\scal{\iv_1,\GreenL\iv_2}=\scal{\GreenL\iv_1,\iv_2}=\demi\scal{\frac{\partial\xv_G}{\partial\pj_2},\iv_r}=0\;\text{!!}
\end{displaymath}
\item The cross-sections remain $\perp$ to the middle-line.
\end{itemize}
\begin{columns}
\column{.3\textwidth}
\centering\includegraphics[scale=.2]{\figs/ch1-BeamStraight-c}
\column{.7\textwidth}
\centering\includegraphics[scale=.2]{\figs/ch1-BeamHalfCircle-c}
\end{columns}
\end{exampleblock}

\end{frame}

\begin{frame}{Large beam bending}{Solution}

\begin{exampleblock}{Question \#7: $\xv(\pv)=\xv_G(\pj_2)+\pj_1\iv_r(1.1\theta(\pv_2))$}
\begin{itemize}
\item After tedious computations ($\alpha=1.1$):
{\scriptsize
\begin{displaymath}
\GreenL=\scal{\frac{\partial\xv_G}{\partial\pj_2},\iv_r(\alpha\theta)}\iv_1\otimes_s\iv_2+\demi\left[\left(\frac{\partial\xv_G}{\partial\pj_2}+\alpha\pj_1\frac{\partial\theta}{\partial\pj_2}\iv_\theta(\alpha\theta)\right)^2-1\right]\iv_2\otimes\iv_2
\end{displaymath}
\begin{displaymath}
\GreenLj_{12}=\demi\scal{\frac{\partial\xv_G}{\partial\pj_2},\iv_r(\alpha\theta)}=\demi\scal{\iv_\theta(\theta),\iv_r(\alpha\theta)}=\demi\sin\left(\pi(\alpha-1)\frac{\pj_2}{L}\right)\,.
\end{displaymath}}
\item The cross-sections are no longer $\perp$ to the middle-line.
\end{itemize}
\begin{columns}
\column{.5\textwidth}
\centering\includegraphics[scale=.2]{\figs/ch1-BeamHalfCircle-c}\\
$\alpha=1$
\column{.5\textwidth}
\centering\includegraphics[scale=.2]{\figs/ch1-BeamHalfCircleAlpha-c}\\
$\alpha=1.1$
\end{columns}
\end{exampleblock}

\end{frame}

\begin{frame}{Large beam bending}{Solution}

\begin{exampleblock}{Question \#8: Small strain tensor}
\begin{itemize}
\item $\uv(\pv)=\xv(\pv)-\pv=\uv_G(\pj_2)+\pj_1(\iv_r(\theta(\pj_2))-\iv_1)$ with $\uv_G(\pj_2)=\xv_G(\pj_2)-\pj_2\iv_2$, therefore ($\strain=\frac{\partial\uv}{\partial\pj_j}\otimes_s\iv_j$):
\begin{displaymath}
\begin{split}
\strain =&(\iv_r(\theta(\pj_2))-\iv_1)\otimes_s\iv_1+\uv_G'(\pj_2)\otimes_s\iv_2 \\
&+\pj_1\theta'(\pj_2)\iv_\theta(\theta(\pj_2))\otimes_s\iv_2
\end{split}
\end{displaymath}
\item $\norm{\uv}\ll h\imply\uv\approx\uv_G\ll h<L$ and $\theta\ll 1$ such that $\iv_r\approx\iv_1+\theta\iv_2$;
\item $\norm{\Grad_\pv\uv}\ll 1\imply\norm{\uv_G'(\pj_2)}\ll1$ and $\theta'(\pj_2)\ll 1$;
\item This yields $\strainj_{11}\approx 0$ (no "breathing" of the beam) and:
{\footnotesize
\begin{displaymath}
\begin{split}
\strainj_{12} &\approx\demi\left(\scal{\uv_G'(\pj_2),\iv_1}+\theta(\pj_2)\right)\,, \\
\strainj_{22} &=\scal{\uv_G'(\pj_2),\iv_2}+\pj_1\theta'(\pj_2)\,.
\end{split}
\end{displaymath}}
\end{itemize}
\end{exampleblock}

\end{frame}

\end{document}

