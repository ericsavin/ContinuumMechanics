% MG3 TD #7: 
% V1.0 February 2021
% $Header: /cvsroot/latex-beamer/latex-beamer/solutions/generic-talks/generic-ornate-15min-45min.en.tex,v 1.5 2007/01/28 20:48:23 tantau Exp $
\def\webDOI{http://dx.doi.org}
\def\Folder{/Users/ericsavin/Documents/Cours/SG3-MMC/SLIDES_TD/}
\def\Year{\Folder/2020-2021}
\def\Sections{\Year/SECTIONS}
\def\figs{\Folder/FIGS}
%\def\figs{/Users/ericsavin/Documents/Cours/SG3-MMC/SLIDES_TD/FIGS}
\def\figdynsto{/Users/ericsavin/Documents//Figures/DYNSTO}
\def\symb{/Users/ericsavin/Documents/Latex/SYMBOL}
\def\fonts{/Users/ericsavin/Documents/Latex/FONTS}
\def\logos{/Users/ericsavin/Documents/Latex/LOGOS}
\def\Onera{ONERA}
\def\ECP{CentraleSup\'elec}


\documentclass{beamer}

% This file is a solution template for:

% - Giving a talk on some subject.
% - The talk is between 15min and 45min long.
% - Style is ornate.



% Copyright 2004 by Till Tantau <tantau@users.sourceforge.net>.
%
% In principle, this file can be redistributed and/or modified under
% the terms of the GNU Public License, version 2.
%
% However, this file is supposed to be a template to be modified
% for your own needs. For this reason, if you use this file as a
% template and not specifically distribute it as part of a another
% package/program, I grant the extra permission to freely copy and
% modify this file as you see fit and even to delete this copyright
% notice. 


\mode<presentation>
{
  \usetheme{Berkeley}
  % or ...

  \setbeamercovered{transparent}
  % or whatever (possibly just delete it)
}


\usepackage[english]{babel}
% or whatever

\usepackage[latin1]{inputenc}
% or whatever

%\usepackage{mathtime}
\usefonttheme{serif}
%\usefonttheme{professionalfonts}
\usepackage{amsfonts}
\usepackage{amssymb}

\usepackage{amsmath}
\usepackage{multimedia}
\usepackage{mathrsfs}
\usepackage{mathabx}
\usepackage{color}
\usepackage{pstricks}
\usepackage{graphicx}
%\usepackage[pdftex, pdfborderstyle={/S/U/W 1}]{hyperref}
\usepackage{hyperref}
\usepackage{bbm}
\usepackage{cancel}
\usepackage[Symbol]{upgreek}
%\usepackage{mathbbol}
%\DeclareSymbolFontAlphabet{\amsmathbb}{AMSb}
%\usepackage[bbgreekl]{mathbbol}
%\usepackage[mtpbbi]{mtpro2}

%\usepackage[svgnames]{xcolor}

%\input{\fonts/math0}
%\input{\symb/structac} % Notations E. Savin
%\input{\symb/logos}

\newcommand{\ci}{\mathrm{i}}
\newcommand{\trace}{\operatorname{Tr}}
\newcommand{\Nset}{\mathbb{N}}
\newcommand{\Zset}{\mathbb{Z}}
\newcommand{\Rset}{\mathbb{R}}
\newcommand{\Cset}{\mathbb{C}}
\newcommand{\Sset}{\mathbb{S}}
\newcommand{\Mset}{\mathbb{M}}
\newcommand{\PhaseSpace}{\Omega}
\newcommand{\ContSet}{{\mathcal C}}
\newcommand{\id}{d}
\newcommand{\iD}{\mathrm{D}}
\newcommand{\iexp}{\mathrm{e}}
\newcommand{\demi}{\frac{1}{2}}
\newcommand{\imply}{\Rightarrow}

% Algebra
\newcommand{\itr}{{\sf T}}
\newcommand{\Id}{{\boldsymbol I}}
\newcommand{\IId}{\mathbb{I}}
\newcommand{\aj}{a}
\newcommand{\bj}{b}
\newcommand{\cj}{c}
\renewcommand{\dj}{d}
\newcommand{\av}{{\boldsymbol\aj}}
\newcommand{\bv}{{\boldsymbol\bj}}
\newcommand{\cv}{{\boldsymbol\cj}}
\newcommand{\dv}{{\boldsymbol\dj}}
\newcommand{\uj}{u}
\newcommand{\vj}{v}
\newcommand{\xj}{x}
\newcommand{\yj}{y}
\newcommand{\zj}{z}
\newcommand{\uv}{{\boldsymbol\uj}}
\newcommand{\vv}{{\boldsymbol\vj}}
\newcommand{\xv}{{\boldsymbol\xj}}
\newcommand{\yv}{{\boldsymbol\yj}}
\newcommand{\zv}{{\boldsymbol\zj}}
\newcommand{\Aj}{A}
\newcommand{\Bj}{B}
\newcommand{\Av}{{\boldsymbol\Aj}}
\newcommand{\Bv}{{\boldsymbol\Bj}}
\newcommand{\Zgv}{{\boldsymbol Z}}

% Analysis
\newcommand{\grad}{{\boldsymbol\nabla}}
\newcommand{\gradx}{{\grad_\xv}}
\newcommand{\Grad}{{\mathbb D}}
\newcommand{\Gradx}{{\Grad_\xv}}
\renewcommand{\div}{\mathrm{div}}
\newcommand{\divx}{{\div_\xv}}
\newcommand{\Div}{\mathbf{Div}}
\newcommand{\Divx}{{\Div_\xv}}

% Kinematics
\newcommand{\ej}{e}
\renewcommand{\ij}{i}
\newcommand{\pj}{p}
\newcommand{\ev}{{\boldsymbol\ej}}
\newcommand{\iv}{{\boldsymbol\ij}}
\newcommand{\pv}{{\boldsymbol\pj}}
\newcommand{\posij}{f}
\newcommand{\posiv}{{\boldsymbol\posij}}
\newcommand{\iposiv}{{\boldsymbol g}}
\newcommand{\Fp}{{\mathbb F}}
\newcommand{\GreenLj}{E}
\newcommand{\GreenL}{{\mathbb\GreenLj}}
\newcommand{\medium}{\Omega}
\newcommand{\strainj}{\varepsilon}
\newcommand*{\strain}{\mbox{$\hspace{0.2em}\rotatebox[x=0pt,y=0.2pt]{90}{\rule{0.02\linewidth}{0.4pt}}\hspace{-0.23em}\upvarepsilon$}}
\newcommand*{\rotation}{{\boldsymbol R}}
\newcommand*{\xiu}{\chi_1}
\newcommand*{\xid}{\chi_2}
\newcommand*{\Rot}{{\boldsymbol R}}
\newcommand*{\dRot}{{\boldsymbol\Theta}}
\newcommand*{\drotj}{\theta}
\newcommand*{\drot}{{\boldsymbol\drotj}}
\newcommand*{\Mstaticj}{J}
\newcommand*{\Mstatic}{{\mathbb\Mstaticj}}

% Dynamics
\newcommand{\fj}{f}
\newcommand{\Fj}{F}
\newcommand{\gj}{g}
\newcommand{\mj}{m}
\newcommand{\nj}{n}
\newcommand{\Tj}{T}
%\newcommand{\cv}{{\boldsymbol c}}
\newcommand{\fv}{{\boldsymbol\fj}}
\newcommand{\Fv}{{\boldsymbol\Fj}}
\newcommand{\gv}{{\boldsymbol\gj}}
\newcommand{\mv}{{\boldsymbol\mj}}
\newcommand{\nv}{{\boldsymbol\nj}}
\newcommand{\Tv}{{\boldsymbol\Tj}}
\newcommand{\roi}{\varrho}
\newcommand*{\stressj}{\sigma}
\newcommand*{\tressj}{\tau}
\newcommand{\stress}{\mbox{$\hspace{0.3em}\rotatebox[x=0pt,y=0.2pt]{90}{\rule{0.017\linewidth}{0.4pt}}\hspace{-0.25em}\upsigma$}}
\newcommand*{\tress}{{\boldsymbol\tressj}}
\newcommand{\acj}{a}
\newcommand{\acv}{{\boldsymbol\acj}}
\newcommand{\Fresj}{R}
\newcommand{\Mresj}{M}
\newcommand{\Fres}{{\boldsymbol\Fresj}}
\newcommand{\Mres}{{\boldsymbol\Mresj}}

%\newcommand{\Zg}{{\bf\zgj}}
%\newcommand{\xigj}{\xi}
%\newcommand{\xig}{{\boldsymbol\xigj}}
\newcommand{\kgj}{k}
%\newcommand{\kgh}{\kgj_\ygj}
%\newcommand{\kg}{{\bf\kgj}}
\newcommand{\Kg}{{\bf K}}
%\newcommand{\qg}{{\boldsymbol q}}
%\newcommand{\pg}{{\boldsymbol p}}
\newcommand{\hkg}{{\hat \kg}}
\newcommand{\hpg}{\hat{\pg}}
%\newcommand{\vg}{{\boldsymbol v}}
\newcommand{\sg}{{\boldsymbol s}}
%\newcommand{\stress}{\mathbb{\sigma}}
\newcommand{\tenselasj}{{\Large C}}
\newcommand{\tenselas}{\boldsymbol{\mathsf{\tenselasj}}}
\newcommand{\tenscomp}{\boldsymbol{\mathsf{\Large S}}}
\newcommand{\speci}{{\mathrm w}}
\newcommand{\specij}{{\mathrm W}}
\newcommand{\speciv}{{\bf \specij}}
%\newcommand{\cjg}[1]{\overline{#1}}
\newcommand{\eigv}{{\bf b}}
\newcommand{\eigw}{{\bf c}}
\newcommand{\eigl}{\lambda}
\newcommand{\jeig}{\alpha}
\newcommand{\keig}{\beta}
\newcommand{\cel}{c}
\newcommand{\bcel}{{\bf\cel}}
\newcommand{\deng}{{\mathcal E}}
\newcommand{\flowj}{\pi}
\newcommand{\flow}{\boldsymbol\flowj}
\newcommand{\Flowj}{\Pi}
\newcommand{\Flow}{\boldsymbol\Flowj}
\newcommand{\fluxinj}{g}
\newcommand{\fluxin}{{\bf\fluxinj}}
\newcommand{\dscat}{\sigma}
\newcommand{\tdscat}{\Sigma}
\newcommand{\collop}{{\mathcal Q}}
\newcommand{\epsd}{\delta}
\newcommand{\rscat}{\rho}
\newcommand{\tscat}{\tau}
\newcommand{\Rscat}{\mathcal{R}}
\newcommand{\Tscat}{\mathcal{T}}
\newcommand{\lscat}{\ell}
%\newcommand{\floss}{\eta}
\newcommand{\mdiff}{{\bf D}}
%\newcommand{\demi}{\frac{1}{2}}
\newcommand{\domain}{{\mathcal O}}
\newcommand{\bdomain}{{\mathcal D}}
\newcommand{\interface}{\Gamma}
\newcommand{\sinterface}{\gamma_D}
%\newcommand{\normal}{\hat{\bf n}}
\newcommand{\bnabla}{\boldsymbol\nabla}
%\newcommand{\esp}[1]{\mathbb{E}\{\smash{#1}\}}
\newcommand{\mean}[1]{\underline{#1}}
%\newcommand{\BB}{\mathbb{B}}
%\newcommand{\II}{{\boldsymbol I}}
%\newcommand{\TA}{\boldsymbol{\Gamma}}
\newcommand{\Mdisp}{{\mathbf H}}
\newcommand{\Hamil}{{\mathcal H}}
\newcommand{\bzero}{{\bf 0}}

\newcommand{\mass}{M}
\newcommand{\damp}{D}
\newcommand{\stif}{K}
\newcommand{\dsp}{S}
\newcommand{\dof}{q}
\newcommand{\pof}{p}
%\newcommand{\MM}{{\boldsymbol\mass}}
\newcommand{\MD}{{\boldsymbol\damp}}
\newcommand{\MK}{{\boldsymbol\stif}}
\newcommand{\MS}{{\boldsymbol\dsp}}
\newcommand{\Cov}{{\boldsymbol C}}
\newcommand{\dofg}{{\boldsymbol\dof}}
\newcommand{\pofg}{{\boldsymbol\pof}}
\newcommand{\driftj}{b}
\newcommand{\drifts}{{\boldsymbol \driftj}}
\newcommand{\drift}{{\underline\drifts}}
\newcommand{\scatj}{a}
\newcommand{\scat}{{\boldsymbol\scatj}}
\newcommand{\diff}{{\boldsymbol\sigma}}
\newcommand{\load}{F}
\newcommand{\loadg}{{\boldsymbol\load}}
\newcommand{\pdf}{\pi}
\newcommand{\tpdf}{\pdf_t}
\newcommand{\fg}{{\boldsymbol f}}
%\newcommand{\Ugj}{U}
\newcommand{\Vgj}{V}
\newcommand{\Xgj}{X}
\newcommand{\Ygj}{Y}
%\newcommand{\Ug}{{\boldsymbol\Ugj}}
\newcommand{\Vg}{{\boldsymbol\Vgj}}
%\newcommand{\Qg}{{\boldsymbol Q}}
\newcommand{\Pg}{{\boldsymbol P}}
\newcommand{\Xg}{{\boldsymbol\Xgj}}
\newcommand{\Yg}{{\boldsymbol\Ygj}}
\newcommand{\flux}{{\boldsymbol J}}
\newcommand{\wiener}{W}
\newcommand{\whitenoise}{B}
\newcommand{\Wiener}{{\boldsymbol\wiener}}
\newcommand{\White}{{\boldsymbol\white}}
\newcommand{\paraj}{\nu}
\newcommand{\parag}{{\boldsymbol\paraj}}
\newcommand{\parae}{\hat{\parag}}
\newcommand{\erroj}{\epsilon}
\newcommand{\error}{{\boldsymbol\erroj}}
\newcommand{\biaj}{b}
\newcommand{\bias}{{\boldsymbol\biaj}}
%\newcommand{\disp}{{\boldsymbol V}}
\newcommand{\Fisher}{{\mathcal I}}
\newcommand{\likelihood}{{\mathcal L}}

\newcommand{\heps}{\varepsilon}
%\newcommand{\roi}{\varrho}
\newcommand{\jump}[1]{\llbracket{#1}\rrbracket}
\newcommand{\scal}[1]{\left\langle{#1}\right\rangle}
\newcommand{\norm}[1]{\left\|#1\right\|}
\newcommand{\abs}[1]{\left|#1\right|}
%\newcommand{\po}{\operatorname{o}}
\newcommand{\FFT}[1]{\widehat{#1}}
\newcommand{\indic}[1]{{\mathbf 1}_{#1}}
\newcommand{\impulse}{{\mathbbm h}}
\newcommand{\frf}{\FFT{\impulse}}

%\renewcommand{\Moy}[1]{{\boldsymbol\mu}_{#1}}
%\renewcommand{\Rcor}[1]{{\boldsymbol R}_{#1}}
%\renewcommand{\Mw}[1]{{\boldsymbol M}_{#1}}
%\renewcommand{\Sw}[1]{{\boldsymbol S}_{#1}}
%\renewcommand{\esp}[1]{{\mathbb E}\{#1\}}

\newcommand{\emphb}[1]{\textcolor{blue}{#1}}
\newcommand{\mycite}[1]{\textcolor{red}{#1}}
\newcommand{\mycitb}[1]{\textcolor{red}{[{\it #1}]}}

\newcommand{\PDFU}{{\mathcal U}}
\newcommand{\PDFN}{{\mathcal N}}
\newcommand{\TK}{{\boldsymbol\Pi}}
\newcommand{\TKij}{\pi}
\newcommand{\TKi}{{\boldsymbol\pi}}
\newcommand{\SMi}{\TKij^*}
\newcommand{\SM}{\TKi^*}
\newcommand{\lagmuli}{\lambda}
\newcommand{\lagmul}{{\boldsymbol\lagmuli}}
\newcommand{\constraint}{{\boldsymbol C}}
\newcommand{\mconstraint}{\mean{\constraint}}

\newcommand{\mybox}[1]{\fbox{\begin{minipage}{0.93\textwidth}{#1}\end{minipage}}}
\newcommand{\defcolor}[1]{\textcolor{blue}{#1}}

%\definecolor{rose}{LightPink}%{rgb}{251,204,231}

\newtheorem{mydef}{Definition}
\newtheorem{mythe}{Theorem}
\newtheorem{myprop}{Proposition}

% Or whatever. Note that the encoding and the font should match. If T1
% does not look nice, try deleting the line with the fontenc.

\title[1EL5000/S7]
{Move crane}

\subtitle{1EL5000--Continuum Mechanics -- Tutorial Class \#7} % (optional)

\author[\'E. Savin] % (optional, use only with lots of authors)
{\'E. Savin\inst{1,2}\\ \scriptsize{\texttt{eric.savin@\{centralesupelec,onera\}.fr}}}%\inst{1} }
% - Use the \inst{?} command only if the authors have different
%   affiliation.

\institute[Onera] % (optional, but mostly needed)
{\inst{1}{Information Processing and Systems Dept.\\\Onera, France}
\and
 \inst{2}{Mechanical and Civil Engineering Dept.\\\ECP, France}}%
%  Department of Theoretical Philosophy\\
%  University of Elsewhere}
% - Use the \inst command only if there are several affiliations.
% - Keep it simple, no one is interested in your street address.

%\date[Short Occasion] % (optional)
\date{\today}
\date{March 8, 2021}

\subject{Move crane}
% This is only inserted into the PDF information catalog. Can be left
% out. 



% If you have a file called "university-logo-filename.xxx", where xxx
% is a graphic format that can be processed by latex or pdflatex,
% resp., then you can add a logo as follows:

% \pgfdeclareimage[height=0.5cm]{university-logo}{university-logo-filename}
% \logo{\pgfuseimage{university-logo}}



% Delete this, if you do not want the table of contents to pop up at
% the beginning of each subsection:
\AtBeginSection[]
%\AtBeginSubsection[]
{
  \begin{frame}<beamer>{Outline}
    \tableofcontents[currentsection]%,currentsubsection]
  \end{frame}
}


% If you wish to uncover everything in a step-wise fashion, uncomment
% the following command: 

%\beamerdefaultoverlayspecification{<+->}


\begin{document}

\begin{frame}
  \titlepage
\end{frame}

\begin{frame}{Outline}
  \tableofcontents
  % You might wish to add the option [pausesections]
\end{frame}


% Since this a solution template for a generic talk, very little can
% be said about how it should be structured. However, the talk length
% of between 15min and 45min and the theme suggest that you stick to
% the following rules:  

% - Exactly two or three sections (other than the summary).
% - At *most* three subsections per section.
% - Talk about 30s to 2min per frame. So there should be between about
%   15 and 30 frames, all told.

\section{Kinematics}
\begin{frame}{Beam kinematics}{Reference configuration}

\begin{figure}
\centering\includegraphics[scale=.25]{\figs/ch7-Kinematics-p}
\end{figure}

\begin{displaymath}
\begin{split}
\pv(s,\xiu,\xid) &=\pv_G(s)+\pv_\Sigma(\xiu,\xid) \\
&= s\ev+\chi_1\ev_{\chi_1}+\chi_2\ev_{\chi_2}
\end{split}
\end{displaymath}

\end{frame}

\begin{frame}{Beam kinematics}{Actual configuration}

\vskip-10pt
%\begin{figure}
%\centering\includegraphics[scale=.3]{\figs/ch7-Kinematics-x}
%\end{figure}
\begin{columns}
\column{.4\textwidth}
\centering\includegraphics[scale=.25]{\figs/ch7-Kinematics-x}
\column{.6\textwidth}
\centering\includegraphics[scale=.25]{\figs/ch7-Accordeon}
\end{columns}

%\vskip-10pt
\begin{displaymath}
\begin{split}
\xv &=\xv_G(s)+\xv_\Sigma \\
&= \xv_G(s)+\Rot(s)\pv_\Sigma
\end{split}
\end{displaymath}

\end{frame}

\begin{frame}{Beam kinematics}{Small perturbations -- Timoshenko}

\vskip-10pt
\begin{figure}
\centering\includegraphics[scale=.2]{\figs/ch7-Kinematics-Timoshenko}
\end{figure}
\vskip-10pt
\begin{itemize}
\item Small perturbations $\Rot(s)=\Id+\dRot(s)$, $\dRot(s)^\itr=-\dRot(s)$:
\begin{displaymath}
\begin{split}
\xv_\Sigma &= \Rot(s)\pv_\Sigma \\
&=(\Id+\dRot(s))\pv_\Sigma\,.
\end{split}
\end{displaymath}
\item Small displacement $\xv_\Sigma\simeq\pv_\Sigma$:
\begin{displaymath}
\begin{split}
\uv &= \xv-\pv \\
&=\uv_G(s)+\drot(s)\times\xv_\Sigma\,.
\end{split}
\end{displaymath}
\end{itemize}

\end{frame}

\begin{frame}{Beam kinematics}{Small perturbations -- Euler-Bernoulli}

\vskip-10pt
\begin{figure}
\centering\includegraphics[scale=.3]{\figs/ch7-Kinematics-EBernoulli}
\end{figure}
\vskip-10pt
\begin{itemize}
\item Cross-sections remain perpendicular to the neutral line:
\begin{displaymath}
\Rot(s)\ev=\frac{\xv_G'(s)}{\norm{\xv_G'(s)}}\,.
\end{displaymath}
\item Small perturbations $\Rot(s)\ev=(\Id+\dRot(s))\ev\simeq\ev+\uv_G'(s)$:
\begin{displaymath}
\boxed{\drot_\Sigma(s)=\ev\times\uv_{G\Sigma}'(s)}
\end{displaymath}
\end{itemize}

\end{frame}

\begin{frame}{Recap: 1.3 Large beam bending}

\vskip-20pt
\begin{figure}
\centering\includegraphics[scale=.2]{\figs/ch1-BeamStraight-c}
\end{figure}
\vskip-10pt
\begin{columns}
\column{.5\textwidth}
\centering\includegraphics[scale=.2]{\figs/ch1-BeamHalfCircle-c}\\
$\alpha=1$
\column{.5\textwidth}
\centering\includegraphics[scale=.2]{\figs/ch1-BeamHalfCircleAlpha-c}\\
$\alpha=1.1$
\end{columns}

\end{frame}

\begin{frame}{Beam kinematics}{Independent unknowns}

\begin{figure}
\centering\includegraphics[scale=.2]{\figs/ch7-KinUnknowns}
\end{figure}

\begin{itemize}
\item Elongation $\uj_{Ge}=\scal{\uv_G,\ev}$;
\item Deflection $\uv_{G\Sigma}=\uv_G-\uj_{Ge}\ev$;
\item Torsion rotation $\drotj_e=\scal{\drot,\ev}$.
\end{itemize}

\end{frame}


\section{Statics}
\definecolor{ultramarine}{rgb}{0.07, 0.04, 0.56} 
\definecolor{aqua}{rgb}{0, 255, 255}

\begin{frame}{Beam statics}{Resultant forces}

\begin{figure}
\centering\includegraphics[scale=.2]{\figs/ch7-Forces}
\end{figure}

\begin{itemize}
\item Resultant force:
\begin{displaymath}
\Fres(s)=\int_\Sigma\stress\ev\,\id S\,;
\end{displaymath}
\item Normal force $\Fresj_e(s)=\scal{\Fres(s),\ev}$;
\item Shear force $\Fres_\Sigma(s)=\Fres(s)-\Fresj_e(s)\ev$;
\item $\Fres(0)$ and $\Fres(L)$ given by the boundary conditions.
\end{itemize}

\end{frame}

\begin{frame}{Beam statics}{Resultant moments}

\begin{figure}
\centering\includegraphics[scale=.2]{\figs/ch7-Moments}
\end{figure}

\begin{itemize}
\item Resultant moment:
\begin{displaymath}
\Mres(s)=\int_\Sigma\xv_\Sigma\times\stress\ev\,\id S\,;
\end{displaymath}
\item Torsion moment $\Mresj_e(s)=\scal{\Mres(s),\ev}$;
\item Bending moment $\Mres_\Sigma(s)=\Mres(s)-\Mresj_e(s)\ev$;
\item $\Mres(0)$ and $\Mres(L)$ given by the boundary conditions.
\end{itemize}

\end{frame}

\begin{frame}{Beam statics}{Local balance of forces}

\begin{figure}
\centering\includegraphics[scale=.25]{\figs/ch7-EqForces}
\end{figure}

\begin{itemize}
\item Linear external forces:
\begin{displaymath}
\fv_l(s)=\int_\Sigma\fv_v\id S + \int_{\partial\Sigma}\fv_s\id\zeta\,;
\end{displaymath}
\item Local equilibrium of the cross-section:
\begin{displaymath}
\Fres'(s)+\fv_l(s)=\bzero\,.
\end{displaymath}
\end{itemize}

\end{frame}

\begin{frame}{Beam statics}{Local balance of moments}

\begin{figure}
\centering\includegraphics[scale=.25]{\figs/ch7-EqMoments}
\end{figure}

\begin{itemize}
\item Linear external torques:
\begin{displaymath}
\cv_l(s)=\int_\Sigma\xv_\Sigma\times\fv_v\id S + \int_{\partial\Sigma}\xv_\Sigma\times\fv_s\id\zeta\,;
\end{displaymath}
\item Local equilibrium of the cross-section:
\begin{displaymath}
\Mres'(s)+\ev\times\Fres(s)+\cv_l(s)=\bzero\,.
\end{displaymath}
\end{itemize}

\end{frame}

\begin{frame}{Beam statics}{Local balance of forces--alternative point of view}

\begin{itemize}
\item Local static equilibrium of a continuum medium:
\begin{displaymath}
\Div\stress+\fv_v=\bzero\,.
\end{displaymath}
\item Then integrate over the cross-section $\Sigma$:
\begin{displaymath}
\begin{split}
\bzero &=\int_\Sigma\Div\stress\,\id S+\int_\Sigma\fv_v\,\id S \\
&=\int_\Sigma\frac{\partial\stress}{\partial s}\ev\,\id S + \int_\Sigma\Div_\Sigma\stress\,\id S +\int_\Sigma\fv_v\,\id S \\
&=\frac{\partial}{\partial s}\left(\int_\Sigma\stress\ev\,\id S\right)+\int_{\partial\Sigma}\stress\nv\,\id\zeta +\int_\Sigma\fv_v\,\id S \\
&=\Fres'(s)+ \int_{\partial\Sigma}\fv_s\,\id\zeta +\int_\Sigma\fv_v\,\id S \\
&=\Fres'(s)+\fv_l(s)\,.
\end{split}
\end{displaymath}
\end{itemize}

\end{frame}

\begin{frame}{Beam statics}{Local balance of moments--alternative point of view}

\begin{itemize}
%\item Local static equilibrium of a continuum medium:
%\begin{displaymath}
%\Div\stress+\fv_v=\bzero\,.
%\end{displaymath}
\item Then integrate over the cross-section $\Sigma$:
\begin{displaymath}
\begin{split}
\!\!\!\!\!\!\!\!\!\!\!\!\!\! \bzero &=\int_\Sigma\xv_\Sigma\times\Div\stress\,\id S+\int_\Sigma\xv_\Sigma\times\fv_v\,\id S \\
\!\!\!\!\!\!\!\!\!\!\!\!\!\! &= \int_\Sigma\xv_\Sigma\times\frac{\partial\stress}{\partial s}\ev\,\id S + \int_\Sigma\xv_\Sigma\times\frac{\partial\stress\ev_\alpha}{\partial\xj_\alpha}\,\id S +\int_\Sigma\xv_\Sigma\times\fv_v\,\id S \\
\!\!\!\!\!\!\!\!\!\!\!\!\!\! &= \frac{\partial}{\partial s}\left(\int_\Sigma\xv_\Sigma\times\stress\ev\,\id S\right)+\int_\Sigma\frac{\partial(\xv_\Sigma\times\stress\ev_\alpha)}{\partial\xj_\alpha}\,\id S \\
\!\!\!\!\!\!\!\!\!\!\!\!\!\! &\quad\quad -\int_\Sigma\ev_\alpha\times\stress\ev_\alpha\,\id S +\int_\Sigma\xv_\Sigma\times\fv_v\,\id S \\
\!\!\!\!\!\!\!\!\!\!\!\!\!\! &=\Mres'(s)+ \int_{\partial\Sigma}\xv_\Sigma\times\stress\nv\,\id\zeta + \int_\Sigma\ev\times\stress\ev\,\id S + \int_\Sigma\xv_\Sigma\times\fv_v\,\id S \\
\!\!\!\!\!\!\!\!\!\!\!\!\!\! &=\Mres'(s)+\ev\times\Fres(s)+\cv_l(s)\,,
\end{split}
\end{displaymath}
since $\ev_\alpha\times\stress\ev_\alpha+\ev\times\stress\ev=\bzero$ from the symmetry of $\stress$.
\end{itemize}

\end{frame}

\begin{frame}{Beam statics}{Global balance of forces}

\begin{figure}
\centering\includegraphics[scale=.25]{\figs/ch7-Forces2}
\end{figure}

\begin{itemize}
\item Global equilibrium of the \textcolor{aqua}{left section} $s\in[0,s]$:
\begin{displaymath}
\Fres(s)-\Fres(0)+\int_0^s\fv_l\id \zeta=\bzero\,;
\end{displaymath}
\item  Global equilibrium of the \textcolor{yellow}{right section} $s\in[s,L]$:
\begin{displaymath}
\Fres(L)-\Fres(s)+\int_s^L\fv_l\id \zeta=\bzero\,.
\end{displaymath}
\end{itemize}

\end{frame}

\begin{frame}{Beam statics}{Global balance of moments}

\begin{figure}
\centering\includegraphics[scale=.25]{\figs/ch7-Moments2}
\end{figure}

\begin{itemize}
\item Global equilibrium of the \textcolor{aqua}{left section} $s\in[0,s]$:
\begin{displaymath}
\!\!\!\!\Mres(s)-\Mres(0)+s\ev\times\Fres(0)+\int_0^s(\cv_l+(\zeta-s)\ev\times\fv_l)\,\id \zeta=\bzero\,;
\end{displaymath}
\item  Global equilibrium of the \textcolor{yellow}{right section} $s\in[s,L]$:
\begin{displaymath}
\!\!\!\!\Mres(L)+(L-s)\ev\times\Fres(L)-\Mres(s)+\int_s^L(\cv_l+(\zeta-s)\ev\times\fv_l)\,\id \zeta=\bzero\,.
\end{displaymath}
\end{itemize}

\end{frame}



\section{Example}

\begin{frame}{Example: branched beam}{Setup}

\begin{figure}
\centering\includegraphics[scale=.4]{\figs/ch7-BranchedBeam}
\end{figure}
\vskip-25pt{\hspace{7truecm}\mbox{\tiny{\copyright\ G. Puel}}}\\
\vskip20pt
Question: Resultant forces and moments within this beam?

\end{frame}

\begin{frame}{Example: branched beam}{Local approach}

\begin{overprint}

\onslide<1|handout:1>
\begin{itemize}
\item Local equations of equilibrium:
\begin{displaymath}
\begin{split}
\Fres'(s)+\fv_l(s) &=\bzero\,, \\
\Mres'(s)+\ev\times\Fres(s)+\cv_l(s) &=\bzero\,;
\end{split}
\end{displaymath}
\item For $\medium_3$ (beam part \#3):
\begin{displaymath} 
\ev=\iv_\yj\,,\quad\fv_l=\bzero\,,\quad\cv_l=\bzero\,;
\end{displaymath}
\item Thus for $s\equiv\yj\in(0,L]$:
\begin{displaymath}
\begin{split}
\Fres'(\yj) &=\bzero\,, \\
\Mres'(\yj)+\iv_\yj\times\Fres(\yj) &=\bzero\,;
\end{split}
\end{displaymath}
\end{itemize}

\onslide<2|handout:2>
\begin{itemize}
\item Local equations of equilibrium:
\begin{displaymath}
\begin{split}
\Fres'(s)+\fv_l(s) &=\bzero\,, \\
\Mres'(s)+\ev\times\Fres(s)+\cv_l(s) &=\bzero\,;
\end{split}
\end{displaymath}
\item For $\medium_2$ (beam part \#2):
\begin{displaymath} 
\ev=\iv_\xj\,,\quad\fv_l=\bzero\,,\quad\cv_l=\bzero\,;
\end{displaymath}
\item Thus for $s\equiv\xj\in(0,L]$:
\begin{displaymath}
\begin{split}
\Fres'(\xj) &=\bzero\,, \\
\Mres'(\xj)+\iv_\xj\times\Fres(\xj) &=\bzero\,;
\end{split}
\end{displaymath}
\end{itemize}

\onslide<3|handout:3>
\begin{itemize}
\item Local equations of equilibrium:
\begin{displaymath}
\begin{split}
\Fres'(s)+\fv_l(s) &=\bzero\,, \\
\Mres'(s)+\ev\times\Fres(s)+\cv_l(s) &=\bzero\,;
\end{split}
\end{displaymath}
\item For $\medium_1$ (beam part \#1):
\begin{displaymath} 
\ev=\iv_\zj\,,\quad\fv_l=\bzero\,,\quad\cv_l=\bzero\,;
\end{displaymath}
\item Thus for $s\equiv\zj\in(0,L]$:
\begin{displaymath}
\begin{split}
\Fres'(\zj) &=\bzero\,, \\
\Mres'(\zj)+\iv_\zj\times\Fres(\zj) &=\bzero\,;
\end{split}
\end{displaymath}
\end{itemize}

\end{overprint}

\end{frame}

\begin{frame}{Example: branched beam}{Local approach in $\medium_3$}

\begin{itemize}
\item Resultant force in $\medium_3$ for $\yj\in(0,L]$:
\begin{displaymath}
\begin{split}
\Fres'(\yj) &=\bzero\,, \\
\Fres(\yj) &=\Fres(\xv_D) \\
&=-F\iv_\zj\,;
\end{split}
\end{displaymath}
\item Resultant moment in $\medium_3$ for $\yj\in(0,L]$:
\begin{displaymath}
\begin{split}
\Mres'(\yj) &=\iv_\yj\times F\iv_\zj \\
\Mres(\yj) &=F(\yj-L)\iv_\xj+\cancel{\Mres(\xv_D)} \\
&=F(\yj-L)\iv_\xj\,.
\end{split}
\end{displaymath}
\end{itemize}

\end{frame}

\begin{frame}{Example: branched beam}{Local approach in $\medium_2$}

\begin{overprint}

\onslide<1|handout:1>
\begin{itemize}
\item Resultant force in $\medium_2$ for $\xj\in(0,L]$:
\begin{displaymath}
\begin{split}
\Fres'(\xj) &=\bzero\,, \\
\Fres(\xj) &=\Fres(\xv_B^-) \\
&=\Fres(\xv_B^+) \\
&=-F\iv_\zj\,,
\end{split}
\end{displaymath}
since from the equilibrium of $\Sigma(\xv_B)$:
\begin{displaymath}
-\Fres(\xv_B^-)+\Fres(\xv_B^+)=\bzero\,;
\end{displaymath}
\end{itemize}

\onslide<2|handout:2>
\begin{itemize}
\item Resultant moment in $\medium_2$ for $\xj\in(0,L]$:
\begin{displaymath}
\begin{split}
\Mres'(\xj) &=\iv_\xj\times F\iv_\zj \\
\Mres(\xj) &=-F(\xj-L)\iv_\yj+\Mres(\xv_B^-) \\
&=F(L-\xj)\iv_\yj+\Mres(\xv_B^+) \\
&=F(L-\xj)\iv_\yj-FL\iv_\xj\,,
\end{split}
\end{displaymath}
since from the equilibrium of $\Sigma(\xv_B)$:
\begin{displaymath}
-\Mres(\xv_B^-)+\Mres(\xv_B^+)=\bzero\,.
\end{displaymath}
\end{itemize}

\end{overprint}

\end{frame}

\begin{frame}{Example: branched beam}{Local approach in $\medium_1$}

\begin{overprint}

\onslide<1|handout:1>
\begin{itemize}
\item Resultant force in $\medium_1$ for $\zj\in(0,L]$:
\begin{displaymath}
\begin{split}
\Fres'(\zj) &=\bzero\,, \\
\Fres(\zj) &=\Fres(\xv_A^-) \\
&=\Fres(\xv_A^+) \\
&=-F\iv_\zj\,,
\end{split}
\end{displaymath}
since from the equilibrium of $\Sigma(\xv_A)$:
\begin{displaymath}
-\Fres(\xv_A^-)+\Fres(\xv_A^+)=\bzero\,;
\end{displaymath}
\item In particular $\Fres_0=-\Fres(\xv_O)=F\iv_\zj$: force of the ground on the beam.
\end{itemize}

\onslide<2|handout:2>
\begin{itemize}
\item Resultant moment in $\medium_1$ for $\zj\in(0,L]$:
\begin{displaymath}
\begin{split}
\Mres'(\zj) &=\iv_\zj\times F\iv_\zj \\
\Mres(\zj) &=\Mres(\xv_A^-) \\
&=\Mres(\xv_A^+) \\
&=FL(\iv_\yj-\iv_\xj)\,,
\end{split}
\end{displaymath}
since from the equilibrium of $\Sigma(\xv_A)$:
\begin{displaymath}
-\Mres(\xv_A^-)+\Mres(\xv_A^+)=\bzero\,.
\end{displaymath}
\item In particular $\Mres_0=-\Mres(\xv_O)=FL(\iv_\xj-\iv_\yj)$: moment of the ground on the beam.
\end{itemize}

\end{overprint}

\end{frame}

\begin{frame}{Example: branched beam}{Global approach}

\begin{columns}[t]
\column{.33\textwidth}
\centering\includegraphics[scale=.2]{\figs/ch7-BranchedBeam1}
\column{.33\textwidth}
\centering\includegraphics[scale=.2]{\figs/ch7-BranchedBeam2}
\column{.33\textwidth}
\centering\includegraphics[scale=.2]{\figs/ch7-BranchedBeam3}
\end{columns}
\vskip-7pt{\hspace{9truecm}\mbox{\tiny{\copyright\ G. Puel}}}

\begin{overprint}

\onslide<1|handout:1>
\begin{itemize}
\item Cutting the beam at any location $\xv_G=s\ev(s)$ in $\medium_1$ or $\medium_2$ or $\medium_3$ and considering the equilibrium of the \textcolor{red}{downstream part} we have:
\begin{displaymath}
-\Fres(\xv_G)+\Fres(\xv_D)+\int_{\xv_G}^{\xv_D}\cancel{\fv_l(\zeta)}\,\id\zeta=\bzero\,,
\end{displaymath}
or since $\fv_l=\bzero$:
\begin{displaymath}
\Fres(\xv_G)=-F\iv_\zj\,.
\end{displaymath}
\end{itemize}

\onslide<2|handout:2>
\begin{itemize}
\item Cutting the beam at any location $\xv_G=s\ev(s)$ in $\medium_1$ or $\medium_2$ or $\medium_3$ and considering the equilibrium of the \textcolor{red}{downstream part} we have:
\begin{displaymath}
\!\!\!\!\!\!\!\!\!\!\!\!\scriptstyle -\Mres(\xv_G)+\cancel{\Mres(\xv_D)}+(\xv_D-\xv_G)\times\Fres(\xv_D) +\int_{\xv_G}^{\xv_D}\cancel{\cv_l(\zeta)}+(\zeta\ev(\zeta)-s\ev(s))\times\cancel{\fv_l(\zeta)}\,\id\zeta=\bzero\,,
\end{displaymath}
or since $\Mres(\xv_D)=\bzero$, $\cv_l=\bzero$:
\begin{displaymath}
\Mres(\xv_G)=(\xv_D-\xv_G)\times F\iv_\zj\,.
\end{displaymath}
\end{itemize}

\end{overprint}

\end{frame}

\section{Stresses}
\begin{frame}{Beam elastic law}{Traction vector}

\begin{itemize}
\item Recap: basic kinematic assumption (Timoshenko)
\begin{displaymath}
\uv=\uv_G(s)+\drot(s)\times\xv_\Sigma\,;
\end{displaymath}
\item Linearized strains $\xv_\Sigma\simeq\pv_\Sigma$:
\begin{displaymath}
\strain=(\uv_G'+\drot'\times\xv_\Sigma)\otimes_s\ev-(\drot_\Sigma\times\ev)\otimes_s\ev\,;
\end{displaymath}
\item Linear elastic, isotropic behavior:
\begin{displaymath}
\begin{split}
\stress &=\lambda\trace(\strain)\Id+2\mu\strain \\
&=\lambda\strainj_{ee}\Id+2\mu(\strainj_{ee}\ev+{\color{green}{{\boldsymbol\gamma}_\Sigma}})\otimes_s\ev+{\color{green}{\stress_\Sigma}}\,;
\end{split}
\end{displaymath}
\item Traction vector:
\begin{displaymath}
\stress\ev=E(\underbrace{\uj_{Ge}'\ev}_{\color{red}{//\ev}}+\underbrace{\drot_\Sigma'\times\xv_\Sigma}_{\color{red}{//\ev,\propto\xv_\Sigma}})+\mu(\underbrace{\uv_{G\Sigma}'-\drot_\Sigma\times\ev}_{\color{red}{\perp\ev}}+\underbrace{\drotj_e'\ev\times\xv_\Sigma}_{\color{red}{\perp\ev,\propto\xv_\Sigma}})\,.
\end{displaymath}
\end{itemize}

\end{frame}

\begin{frame}{Beam elastic law}{Resultant force with Timoshenko's kinematical assumption}

\begin{itemize}
\item Recap: resultant force
\begin{displaymath}
\Fres=\int_\Sigma\stress\ev\,\id S\,;
\end{displaymath}
\item Assuming $\int_\Sigma\xv_\Sigma\id S=\bzero$, $S=\int_\Sigma\id S$:
\begin{displaymath}
\begin{split}
\Fres &= \;\scriptstyle \int_\Sigma E(\uj_{Ge}'\ev+\cancel{\drot_\Sigma'\times\xv_\Sigma})\id S+\int_\Sigma\mu(\uv_{G\Sigma}'-\drot_\Sigma\times\ev+\cancel{\drotj_e'\ev\times\xv_\Sigma})\id S \\
&=ES\uj_{Ge}'\ev+\Fres_\Sigma\,;
\end{split}
\end{displaymath}
\item Shear force with Timoshenko's assumption:
\begin{displaymath}
\Fres_\Sigma=\mu S(\uv_{G\Sigma}'-\drot_\Sigma\times\ev)\,.
\end{displaymath}
\end{itemize}

\end{frame}

\begin{frame}{Beam elastic law}{Resultant force with Euler-Bernoulli's kinematical assumption}

\begin{itemize}
\item Recap: resultant force
\begin{displaymath}
\Fres=\int_\Sigma\stress\ev\,\id S\,;
\end{displaymath}
\item Assuming $\int_\Sigma\xv_\Sigma\id S=\bzero$, $S=\int_\Sigma\id S$:
\begin{displaymath}
\begin{split}
\Fres &= \;\scriptstyle \int_\Sigma E(\uj_{Ge}'\ev+\cancel{\drot_\Sigma'\times\xv_\Sigma})\id S+\int_\Sigma\mu(\underbrace{\cancel{\uv_{G\Sigma}'-\drot_\Sigma\times\ev}}_{=\bzero\,\text{by Euler-Bernoulli}}+\cancel{\drotj_e'\ev\times\xv_\Sigma})\id S \\
&=ES\uj_{Ge}'\ev+\Fres_\Sigma\,;
\end{split}
\end{displaymath}
\item Shear force with Euler-Bernoulli's assumption:
\begin{displaymath}
\begin{split}
\Mres'+\ev\times\Fres+\cv_l=\bzero \\
\imply\quad \Fres_\Sigma=\ev\times(\Mres_\Sigma'+\cv_l)\,.
\end{split}
\end{displaymath}
\end{itemize}

\end{frame}

\begin{frame}{Beam elastic law}{Resultant moment}

\begin{itemize}
\item Recap: resultant moment
\begin{displaymath}
\Mres=\int_\Sigma\xv_\Sigma\times\stress\ev\,\id S\,;
\end{displaymath}
\item Assuming $\int_\Sigma\xv_\Sigma\id S=\bzero$:
\begin{displaymath}
\begin{split}
\Mres &= \;\scriptstyle \int_\Sigma E(\cancel{\xv_\Sigma\times\uj_{Ge}'\ev}+\xv_\Sigma\times(\drot_\Sigma'\times\xv_\Sigma))\id S \\
&\quad\quad\scriptstyle+\int_\Sigma\mu(\cancel{\xv_\Sigma\times(\uv_{G\Sigma}'-\drot_\Sigma\times\ev)}+\xv_\Sigma\times(\drotj_e'\ev\times\xv_\Sigma))\id S \\
&=E\Mstatic(\drot_\Sigma')+\mu\Mstatic(\drotj_e'\ev)\,;
\end{split}
\end{displaymath}
\item (Symmetric) inertia tensor:
\begin{displaymath}
\Mstatic=\int_\Sigma(\norm{\xv_\Sigma}^2\Id-\xv_\Sigma\otimes\xv_\Sigma)\,\id S\,;
\end{displaymath}
\end{itemize}

\begin{overprint}

\onslide<1|handout:1>
\begin{itemize}
\item Bending moment with Timoshenko's assumption:
\begin{displaymath}
\Mres_\Sigma=E\Mstatic(\drot_\Sigma')\,.
\end{displaymath}
\end{itemize}

\onslide<2|handout:2>
\begin{itemize}
\item Bending moment with Euler-Bernoulli's assumption:
\begin{displaymath}
\Mres_\Sigma=E\Mstatic(\ev\times\uv_{G\Sigma}'')\,.
\end{displaymath}
\end{itemize}

\end{overprint}

\end{frame}

\begin{frame}{Summary}{}

\begin{columns}[t]
\column{.5\textwidth}
\centering{Constitutive equations}:
\begin{displaymath}
\begin{array}{c}
\Fres=ES\uj_{Ge}'\ev+\Fres_\Sigma \\
\Fres_\Sigma=\ev\times(\Mres_\Sigma'+\cv_l)
\end{array}
\end{displaymath}
\begin{displaymath}
\begin{array}{c}
\Mres_\Sigma=E\Mstatic(\ev\times\uv_{G\Sigma}'') \\
\Mresj_e=\mu\Mstaticj_e\drotj'_e
\end{array}
\end{displaymath}
\begin{displaymath}
\begin{split}
\Mstaticj_e &=\scal{\Mstatic\ev,\ev} \\
&=\int_\Sigma\norm{\xv_\Sigma}^2\id S
\end{split}
\end{displaymath}
\column{.5\textwidth}
\centering Static equilibrium:
\begin{displaymath}
\begin{array}{c}
\Fres'+\fv_l=\bzero
\end{array}
\end{displaymath}
\vskip-7pt
\begin{displaymath}
\begin{array}{c}
\Mres_\Sigma''-\ev\times\fv_l+\cv_{l\Sigma}'=\bzero \\
\Mresj_e'+\cj_{le}=0
\end{array}
\end{displaymath}
\end{columns}

\end{frame}

\begin{frame}{Differential equations}{}

\begin{itemize}
\item Elongation:
\begin{displaymath}
\boxed{ES\uj_{Ge}''(s)+\scal{\fv_l(s),\ev}=0}
\end{displaymath}
with either kinematical $\uj_{Ge}(0),\uj_{Ge}(L)$ or mechanical $\Fresj_e(0),\Fresj_e(L)$ boundary conditions.
\item Torsion:
\begin{displaymath}
\boxed{\mu\Mstaticj_e\drotj_e''(s)+\scal{\cv_l(s),\ev}=0}
\end{displaymath}
with either kinematical $\drotj_e(0),\drotj_e(L)$ or mechanical $\Mresj_e(0),\Mresj_e(L)$ boundary conditions.
\item Bending:
\begin{displaymath}
\boxed{E\Mstatic(\ev\times\uv_{G\Sigma}^{(IV)}(s))-\ev\times\fv_l(s)+\cv_{l\Sigma}'(s)=\bzero}
\end{displaymath}
with~either~kinematical~$\uv_{G\Sigma}(0),\uv_{G\Sigma}(L),\uv_{G\Sigma}'(0),\uv_{G\Sigma}'(L)$ or mechanical $\Mres_\Sigma(0),\Mres_\Sigma(L),\Fres_\Sigma(0),\Fres_\Sigma(L)$ boundary conditions.
\end{itemize}

\end{frame}

\begin{frame}{Back to stresses...}{}

\begin{itemize}
\item Recap: traction vector
\begin{displaymath}
\begin{split}
\stress\ev &=E(\underbrace{\uj_{Ge}'\ev}_{\color{red}{//\ev}}+\underbrace{\drot_\Sigma'\times\xv_\Sigma}_{\color{red}{//\ev,\propto\xv_\Sigma}})+\mu(\underbrace{\uv_{G\Sigma}'-\drot_\Sigma\times\ev}_{\color{red}{\perp\ev}}+\underbrace{\drotj_e'\ev\times\xv_\Sigma}_{\color{red}{\perp\ev,\propto\xv_\Sigma}}) \\
&=\stressj_{ee}\ev+\tress_\Sigma\,;
\end{split}
\end{displaymath}
\item Normal stress:
\begin{displaymath}
\begin{split}
\stressj_{ee} &=E(\uj_{Ge}'+\scal{\drot_\Sigma'\times\xv_\Sigma,\ev}) \\
&=\frac{\Fresj_e}{S}+\scal{\xv_\Sigma,\ev\times\Mstatic^{-1}(\Mres_\Sigma)}\,;
\end{split}
\end{displaymath}
\item Shear stress:
\begin{displaymath}
\begin{split}
\tress_\Sigma &= \mu(\uv_{G\Sigma}'-\drot_\Sigma\times\ev)+\mu\drotj_e'(\ev\times\xv_\Sigma)\\
&=\frac{\Fres_\Sigma}{S}+\frac{\Mresj_e}{\Mstaticj_e}(\ev\times\xv_\Sigma)\,.
\end{split}
\end{displaymath}
\end{itemize}

\end{frame}



\section{6.2 Move crane}

\begin{frame}{Move crane}{Setup}

\begin{columns}[t]
\column{.45\textwidth}
\centering\includegraphics[scale=.2]{\figs/ch7-MoveCrane}
\vskip-10pt{\hspace{2.5truecm}\mbox{\tiny{\copyright\ G. Puel}}}
\column{.55\textwidth}
\vskip-110pt
%\centering\includegraphics[scale=.13]{\figs/ch7-MoveCrane-SpaceX-1}
\centering\includegraphics[scale=.08]{\figs/ch7-MoveCrane-SpaceX-2}
\end{columns}
\vskip20pt
\begin{itemize}
\item Action of the crane's support $\medium_0$ on $\medium$: $\Fres_0$, $\Mres_0$; 
\item Action of the plate $\medium_P$ on on $\medium$: $\Fres_L$, $\Mres_L$. 
\end{itemize}
\end{frame}

\begin{frame}{Move crane}{Solution}

\begin{exampleblock}{Question \#1: $\Fres_L$? $\Mres_L$?}
\begin{itemize}
\item Equilibrium of the plate:
\begin{displaymath}
\begin{split}
-\Fres_L+m\gv &=\bzero \,,\\
-\Mres_L+(\xv_{C_m}-\xv_P)\times m\gv &=\bzero\,;
\end{split}
\end{displaymath}
\item But $\gv=-g\iv_\zj$ hence:
\begin{displaymath}
\begin{split}
\Fres_L &=-mg\iv_\zj\,, \\
\Mres_L &=mg(l_\xj\iv_\yj-l_\yj\iv_\xj)\,.
\end{split}
\end{displaymath}
\end{itemize}
\end{exampleblock}

\end{frame}

\begin{frame}{Move crane}{Solution}

\begin{overprint}

\onslide<1|handout:1>
\vskip-20pt
\begin{exampleblock}{Question \#2: Resultant force $\Fres$?}
\begin{itemize}
\item Local balance of forces on the cross-section $\Sigma$ at $s\in]0,L[$:
\begin{displaymath}
\Fres'+\fv_l=\bzero\,,
\end{displaymath}
where $\fv_l=\bzero$ because "the only actions exerted on the beam $\medium$ are..."
\item Therefore:
\begin{displaymath}
\Fres(s)={\boldsymbol C}^\text{st}=\Fres(L)=\Fres_L=-mg\iv_\zj\,.
\end{displaymath}
\end{itemize}
\end{exampleblock}

\onslide<2|handout:2>
\vskip-20pt
\begin{exampleblock}{Question \#2: Resultant force $\Fres$?}
\begin{itemize}
\item {\bf Alternative solution}: global balance of forces of the beam $[s,L]$ for $s\in(0,L]$
\begin{displaymath}
-\Fres(s)+\Fres(L)+\int_s^L\fv_l\,\id\zeta=\bzero\,;
\end{displaymath}
\item But $\fv_l=\bzero$ and $\Fres(L)=\Fres_L$, therefore:
\begin{displaymath}
\Fres(s)=\Fres(L)=\Fres_L=-mg\iv_\zj\,.
\end{displaymath}
\end{itemize}
\end{exampleblock}

\onslide<3|handout:3>
\vskip-20pt
\begin{exampleblock}{Question \#2: Normal and shear forces?}
\begin{itemize}
\item Normal force:
\begin{displaymath}
\Fresj_e(s)=\scal{\Fres(s),\ev}=-mg\cos\alpha\,;
\end{displaymath}
\item Shear force:
\begin{displaymath}
\Fres_\Sigma(s)=\Fres(s)-\Fresj_e(s)\ev=-mg\sin\alpha\, \ev_{\xiu}\,;
\end{displaymath}
\item The normal force vanishes for $\alpha=\frac{\pi}{2}$ while the shear force vanishes for $\alpha=0$.
\end{itemize}
\end{exampleblock}

\end{overprint}

\end{frame}

\begin{frame}{Move crane}{Solution}

\begin{overprint}

\onslide<1|handout:1>
\vskip-20pt
\begin{exampleblock}{Question \#3: Resultant moment $\Mres$?}
\begin{itemize}
\item Local balance of moments on the cross-section $\Sigma$ at $s\in]0,L[$:
\begin{displaymath}
\Mres'+\ev\times\Fres+\cv_l=\bzero\,,
\end{displaymath}
where $\cv_l=\bzero$ because "the only actions exerted on the beam $\medium$ are..."
\item Therefore:
\begin{displaymath}
\begin{split}
\Mres'(s) &=-\ev\times\Fres(s) \\
&=\ev\times mg\iv_\zj\,;
\end{split}
\end{displaymath}
\end{itemize}
\end{exampleblock}

\onslide<2|handout:2>
\vskip-20pt
\begin{exampleblock}{Question \#3: Resultant moment $\Mres$?}
\begin{itemize}
\item Local balance of moments on the cross-section $\Sigma$ at $s\in]0,L[$:
\begin{displaymath}
\Mres'+\ev\times\Fres=\bzero\,;
\end{displaymath}
\item Therefore:
\begin{displaymath}
\begin{split}
\Mres'(s)=\ev\times mg\iv_\zj\,;
\end{split}
\end{displaymath}
\item Consequently:
\begin{displaymath}
\begin{split}
\Mres(s) &=mg(s-L)\ev\times\iv_\zj+\Mres(L) \\
&=mg(s-L)\sin\alpha\,\iv_\xj+mg(l_\xj\iv_\yj-l_\yj\iv_\xj) \\
&=mg\{[(s-L)\sin\alpha-l_\yj]\iv_\xj+l_\xj\iv_\yj\}
\end{split}
\end{displaymath}
\end{itemize}
\end{exampleblock}

\onslide<3|handout:3>
\vskip-20pt
\begin{exampleblock}{Question \#3: Resultant moment $\Mres$?}
\begin{itemize}
\item {\bf Alternative solution}: global balance of moments of the beam $[s,L]$ for $s\in(0,L]$
\begin{multline*}
-\Mres(s)+\Mres(L)+(L-s)\ev\times\Fres(L) \\
+\int_s^L(\cv_l+(\zeta-s)\times\fv_l)\,\id\zeta=\bzero\,;
\end{multline*}
\item But $\fv_l=\bzero$, $\cv_l=\bzero$, $\Fres(L)=\Fres_L$, and $\Mres(L)=\Mres_L$, therefore:
\begin{displaymath}
\begin{split}
\Mres(s) &=mg(l_\xj\iv_\yj-l_\yj\iv_\xj)-(L-s)\ev\times mg\iv_\zj \\
&=mg(l_\xj\iv_\yj-l_\yj\iv_\xj)-mg(L-s)\sin\alpha\,\iv_\xj\,.
\end{split}
\end{displaymath}
\end{itemize}
\end{exampleblock}

\onslide<4|handout:4>
\vskip-20pt
\begin{exampleblock}{Question \#3: Torsion and bending moments?}
\begin{itemize}
\item Torsion moment:
\begin{displaymath}
\Mresj_e(s)=\scal{\Mres(s),\ev}=mgl_\xj \sin\alpha\,;
\end{displaymath}
\item Bending moment:
\begin{displaymath}
\begin{split}
\Mres_\Sigma(s) &=\Mres(s)-\Mresj_e(s)\ev\,; \\
&=mg\{[(s-L)\sin\alpha-l_\yj]\iv_\xj-l_\xj\cos\alpha\,\ev_{\xiu}\}
\end{split}
\end{displaymath}
\item The torsion moment vanishes for $\alpha=0$ while the bending moment vanishes for $\alpha=0$ and $l_\xj=l_\yj=0$.
\end{itemize}
\end{exampleblock}

\end{overprint}

\end{frame}

%\begin{frame}{Move crane}{Solution}
%
%\begin{center}
%\textcolor{red}{\LARGE {\bf TO BE CONTINUED!!!}}
%\end{center}
%
%\end{frame}
%
%\end{document}

\begin{frame}{Move crane}{Solution}

\begin{overprint}

\onslide<1|handout:1>
\vskip-20pt
\begin{exampleblock}{Question \#4: Longitudinal displacement $\uj_{Ge}$?}
\begin{itemize}
\item Constitutive equation for the longitudinal displacement:
\begin{displaymath}
\begin{split}
\Fresj_e(s) &=ES\uj_{Ge}'(s) \\
\imply\quad\uj_{Ge}'(s) &=-\frac{mg\cos\alpha}{ES}\,;
\end{split}
\end{displaymath}
\end{itemize}
\end{exampleblock}

\onslide<2|handout:2>
\vskip-20pt
\begin{exampleblock}{Question \#4: Longitudinal displacement $\uj_{Ge}$?}
\begin{itemize}
\item Constitutive equation for the longitudinal displacement:
\begin{displaymath}
\begin{split}
\Fresj_e(s) &=ES\uj_{Ge}'(s) \\
\imply\quad\uj_{Ge}'(s) &=-\frac{mg\cos\alpha}{ES}\,;
\end{split}
\end{displaymath}
\item Consequently:
\begin{displaymath}
\begin{split}
\uj_{Ge}(s) &=-\frac{mg\cos\alpha}{ES}s+\uj_{Ge}(0) \\
&=-\frac{mg\cos\alpha}{ES}s\,, \\
\end{split}
\end{displaymath}
since $\uj_{Ge}(0)=0$ ("the beam is clamped at point $O$").
\end{itemize}
\end{exampleblock}

\end{overprint}

\end{frame}

\begin{frame}{Move crane}{Solution}

\begin{overprint}

\onslide<1|handout:1>
\vskip-20pt
\begin{exampleblock}{Question \#5: Twist angle $\drotj_e$?}
\begin{itemize}
\item Constitutive equation for the twist angle:
\begin{displaymath}
\begin{split}
\Mresj_e(s) &=\mu\Mstaticj_e\drotj_e'(s) \\
\imply\quad\drotj_e'(s) &=\frac{mgl_\xj\sin\alpha}{\mu J}\,;
\end{split}
\end{displaymath}
\end{itemize}
\end{exampleblock}

\onslide<2|handout:2>
\vskip-20pt
\begin{exampleblock}{Question \#5: Twist angle $\drotj_e$?}
\begin{itemize}
\item Constitutive equation for the twist angle:
\begin{displaymath}
\begin{split}
\Mresj_e(s) &=\mu\Mstaticj_e\drotj_e'(s) \\
\imply\quad\drotj_e'(s) &=\frac{mgl_\xj\sin\alpha}{\mu J}\,;
\end{split}
\end{displaymath}
\item Consequently:
\begin{displaymath}
\begin{split}
\drotj_e(s) &=\frac{mgl_\xj\sin\alpha}{\mu J}s+\drotj_e(0) \\
&=\frac{mgl_\xj\sin\alpha}{\mu J}s\,, \\
\end{split}
\end{displaymath}
since $\drotj_e(0)=0$ ("the beam is clamped at point $O$").
\end{itemize}
\end{exampleblock}

\end{overprint}

\end{frame}

\begin{frame}{Move crane}{Solution}

\begin{overprint}

\onslide<1|handout:1>
\vskip-20pt
\begin{exampleblock}{Question \#6: Transverse displacement $\uv_{G\Sigma}$?}
\begin{itemize}
\item Constitutive equation for the bending rotation:
\begin{displaymath}
\begin{split}
\Mres_\Sigma(s) &=E\Mstatic(\drot_\Sigma'(s)) \\
\imply\quad \drot_\Sigma'(s) &=\frac{mg}{EI}\{[(s-L)\sin\alpha-l_\yj]\iv_\xj-l_\xj\cos\alpha\,\ev_{\xiu}\}\,;
\end{split}
\end{displaymath}
\end{itemize}
\end{exampleblock}

\onslide<2|handout:2>
\vskip-20pt
\begin{exampleblock}{Question \#6: Transverse displacement $\uv_{G\Sigma}$?}
\begin{itemize}
\item Constitutive equation for the bending rotation:
\begin{displaymath}
\begin{split}
\Mres_\Sigma(s) &=E\Mstatic(\drot_\Sigma'(s)) \\
\imply\quad \drot_\Sigma'(s) &=\frac{mg}{EI}\{[(s-L)\sin\alpha-l_\yj]\iv_\xj-l_\xj\cos\alpha\,\ev_{\xiu}\}\,;
\end{split}
\end{displaymath}
\item Consequently:
\begin{displaymath}
\begin{split}
\drot_\Sigma(s) &= \;\scriptstyle\frac{mg}{EI}\left\{\left[\left(\frac{s}{2}-L\right)\sin\alpha-l_\yj\right]\iv_\xj-l_\xj\cos\alpha\,\ev_{\xiu}\right\}s+\drot_\Sigma(0) \\
&=\frac{mgs}{2EI}\left\{\left[\left(s-2L\right)\sin\alpha-2l_\yj\right]\iv_\xj-2l_\xj\cos\alpha\,\ev_{\xiu}\right\}\,, \\
\end{split}
\end{displaymath}
since $\drot_\Sigma(0)=0$ ("the beam is clamped at point $O$").
\end{itemize}
\end{exampleblock}


\onslide<3|handout:3>
\vskip-20pt
\begin{exampleblock}{Question \#6: Transverse displacement $\uv_{G\Sigma}$?}
\begin{itemize}
\item "The beam $\medium$ satisfies the Euler-Bernoulli hypothesis," hence:
\begin{displaymath}
\begin{split}
\uv_{G\Sigma}'(s) &=\drot_\Sigma(s)\times\ev \\
&=\frac{mgs}{EI}\left\{\left[\left(\frac{s}{2}-L\right)\sin\alpha-l_\yj\right]\ev_{\xiu}+l_\xj\cos\alpha\,\iv_\xj\right\}
\end{split}
\end{displaymath}
\end{itemize}
\end{exampleblock}

\onslide<4|handout:4>
\vskip-20pt
\begin{exampleblock}{Question \#6: Transverse displacement $\uv_{G\Sigma}$?}
\begin{itemize}
\item "The beam $\medium$ satisfies the Euler-Bernoulli hypothesis," hence:
\begin{displaymath}
\begin{split}
\uv_{G\Sigma}'(s) &=\drot_\Sigma(s)\times\ev \\
&=\frac{mgs}{EI}\left\{\left[\left(\frac{s}{2}-L\right)\sin\alpha-l_\yj\right]\ev_{\xiu}+l_\xj\cos\alpha\,\iv_\xj\right\}
\end{split}
\end{displaymath}
\item Consequently:
\begin{displaymath}
\begin{split}
\uv_{G\Sigma}(s) &= \;\scriptstyle \frac{mg}{2EI}\left\{\left[\left(\frac{s}{3}-L\right)\sin\alpha-l_\yj\right]\ev_{\xiu}+l_\xj\cos\alpha\,\iv_\xj\right\}s^2+\uv_{G\Sigma}(0) \\
&= \;\scriptstyle \frac{mgs^2}{6EI}\left\{\left[\left(s-3L\right)\sin\alpha-3l_\yj\right]\ev_{\xiu}+3l_\xj\cos\alpha\,\iv_\xj\right\}\,, \\
\end{split}
\end{displaymath}
since $\uv_{G\Sigma}(0)=0$ ("the beam is clamped at point $O$").
\end{itemize}
\end{exampleblock}

\end{overprint}

\end{frame}

\begin{frame}{Move crane}{Solution}

\begin{exampleblock}{Question \#7: Displacement of a point $M=(s,\xiu,\xid)$?}
\begin{itemize}
\item From beam's basic kinematical assumption:
\begin{displaymath}
\uv(s,\xiu,\xid)=\uv_G(s)+\drot(s)\times\xv_\Sigma\,,
\end{displaymath}
where:
\begin{displaymath}
\begin{split}
\uv_G(s) &=\uv_{G\Sigma}(s)+\uj_{Ge}(s)\ev \\
&=\;\scriptstyle \frac{mgs^2}{6EI}\left\{\left[\left(s-3L\right)\sin\alpha-3l_\yj\right]\ev_{\xiu}+3l_\xj\cos\alpha\,\iv_\xj\right\} - \frac{mgs\cos\alpha}{ES}\ev \,, \\
\drot(s) &=\drot_\Sigma(s)+\drotj_e(s)\ev \\
&=\;\scriptstyle \frac{mgs}{2EI}\left\{\left[\left(s-2L\right)\sin\alpha-2l_\yj\right]\iv_\xj-2l_\xj\cos\alpha\,\ev_{\xiu}\right\} + \frac{mgsl_\xj\sin\alpha}{\mu J}\ev\,, \\
\xv_\Sigma &=\xiu\ev_{\xiu}+\xid\iv_\xj\,;
\end{split}
\end{displaymath}
\end{itemize}
\end{exampleblock}

\end{frame}

\begin{frame}{Move crane}{Solution}

\begin{exampleblock}{Question \#8: Normal stress $\stressj_{ee}$?}
\begin{itemize}
\item Normal stress:
\begin{displaymath}
\begin{split}
\stressj_{ee} &=\frac{\Fresj_e}{S}+\scal{\xv_\Sigma,\ev\times\Mstatic^{-1}(\Mres_\Sigma)} \\
&=\;\scriptstyle -\frac{mg\cos\alpha}{S}-\scal{\xiu\ev_{\xiu}+\xid\iv_\xj,\frac{mg\{[(s-L)\sin\alpha-l_\yj]\ev_{\xiu}+l_\xj\cos\alpha\,\iv_\xj\}}{I}} \\
&=\;\scriptstyle -mg\left\{\frac{\cos\alpha}{S}+\frac{[(s-L)\sin\alpha-l_\yj]\xiu+l_\xj\xid\cos\alpha}{I}\right\}\,;
\end{split}
\end{displaymath}
\item The maximum normal stress is reached when the bracketed term is minimal (negative) which occurs at $(s,\xiu,\xid)=(0,\chi_{1,\text{max}},\chi_{2,\text{min}})$.
\end{itemize}
\end{exampleblock}

\end{frame}

\begin{frame}{Move crane}{Solution}

\begin{exampleblock}{Question \#9: Shear stress $\tress_\Sigma$?}
\begin{itemize}
\item Shear stress:
\begin{displaymath}
\begin{split}
\tress_\Sigma &=\underbrace{\frac{\Fres_\Sigma}{S}}_{\simeq\bzero}+\frac{\Mresj_e}{\Mstaticj_e}(\ev\times\xv_\Sigma)  \\
&=\frac{mgl_\xj \sin\alpha}{J}\ev\times(\xiu\ev_{\xiu}+\xid\iv_\xj)  \\
&=\frac{mgl_\xj \sin\alpha}{J}(\xiu\iv_\xj-\xid\ev_{\xiu})\,;
\end{split}
\end{displaymath}
\item The shear stress does not vary along the beam and varies linearly within the cross-section, where it vanishes at the neutral line.
\end{itemize}
\end{exampleblock}

\end{frame}

\begin{frame}{Move crane}{Solution}

\begin{exampleblock}{Question \#10: Maximum $l_\xj$ under von Mises criterion?}
\begin{itemize}
\item Stress tensor: $\stress=\stressj_{ee}\ev\otimes\ev+\tress_\Sigma\otimes\ev+\ev\otimes\tress_\Sigma$;
\item Deviatoric stress tensor:
\begin{displaymath}
\begin{split}
\stress^D &=\stress-\frac{\trace\stress}{3}\Id \\
&=\;\scriptstyle \frac{\stressj_{ee}}{3}(2\ev\otimes\ev-\iv_\xj\otimes\iv_\xj-\ev_{\xiu}\otimes\ev_{\xiu})+\tress_\Sigma\otimes\ev+\ev\otimes\tress_\Sigma\,;
\end{split}
\end{displaymath}
\item Von Mises stress:
\begin{displaymath}
\begin{split}
\stressj_\text{eq}=\sqrt{\frac{3}{2}\trace(\stress^{D2})}=\sqrt{\stressj_{ee}^2+3\norm{\tress_\Sigma}^2}\,;
\end{split}
\end{displaymath}
\item We can then find $l_\xj$ such that $\stressj_\text{eq}<\stressj_r$.
\end{itemize}
\end{exampleblock}

\end{frame}

\begin{frame}{Move crane}{Solution}

\begin{exampleblock}{Question \#11: $\Fres_0$, $\Mres_0$?}
\begin{itemize}
\item Resultant force exerted by the crane's support on $\medium$:
\begin{displaymath}
\Fres(0)=-mg\iv_\zj\,;
\end{displaymath}
therefore $\Fres_0=-\Fres(0)=mg\iv_\zj$;
\item Resultant force exerted by the crane's support on $\medium$:
\begin{displaymath}
\Mres(0) =mg(l_\xj\iv_\yj-l_\yj\iv_\xj)-mgL\sin\alpha\,\iv_\xj\,;
\end{displaymath}
therefore $\Mres_0=-\Mres(0)=mg[(l_\yj+L\sin\alpha)\iv_\xj-l_\xj\iv_\yj]$.
\end{itemize}
\end{exampleblock}

\end{frame}

\end{document}

