% MG3 TD #11: Interference fit
% V1.0 March 2021
% $Header: /cvsroot/latex-beamer/latex-beamer/solutions/generic-talks/generic-ornate-15min-45min.en.tex,v 1.5 2007/01/28 20:48:23 tantau Exp $
\def\webDOI{http://dx.doi.org}
\def\Folder{/Users/ericsavin/Documents/Cours/SG3-MMC/SLIDES_TD/}
\def\Year{\Folder/2020-2021}
\def\Sections{\Year/SECTIONS}
\def\figs{\Folder/FIGS}
%\def\figs{/Users/ericsavin/Documents//Cours/DynSto/Figs}
%\def\figs{/Users/ericsavin/Documents/Cours/SG3-MMC/SLIDES_TD/FIGS}
\def\figdynsto{/Users/ericsavin/Documents//Figures/DYNSTO}
\def\symb{/Users/ericsavin/Documents/Latex/SYMBOL}
\def\fonts{/Users/ericsavin/Documents/Latex/FONTS}
\def\logos{/Users/ericsavin/Documents/Latex/LOGOS}
\def\Onera{ONERA}
\def\ECP{CentraleSup\'elec}


\documentclass{beamer}

% This file is a solution template for:

% - Giving a talk on some subject.
% - The talk is between 15min and 45min long.
% - Style is ornate.



% Copyright 2004 by Till Tantau <tantau@users.sourceforge.net>.
%
% In principle, this file can be redistributed and/or modified under
% the terms of the GNU Public License, version 2.
%
% However, this file is supposed to be a template to be modified
% for your own needs. For this reason, if you use this file as a
% template and not specifically distribute it as part of a another
% package/program, I grant the extra permission to freely copy and
% modify this file as you see fit and even to delete this copyright
% notice. 


\mode<presentation>
{
  \usetheme{Berkeley}
  % or ...

  \setbeamercovered{transparent}
  % or whatever (possibly just delete it)
}


\usepackage[english]{babel}
% or whatever

\usepackage[latin1]{inputenc}
% or whatever

%\usepackage{mathtime}
\usefonttheme{serif}
%\usefonttheme{professionalfonts}
\usepackage{amsfonts}
\usepackage{amssymb}

\usepackage{amsmath}
\usepackage{multimedia}
\usepackage{mathrsfs}
\usepackage{mathabx}
\usepackage{color}
\usepackage{pstricks}
\usepackage{graphicx}
%\usepackage[pdftex, pdfborderstyle={/S/U/W 1}]{hyperref}
\usepackage{hyperref}
\usepackage{bbm}
\usepackage{cancel}
\usepackage[Symbol]{upgreek}
%\usepackage{mathbbol}
%\DeclareSymbolFontAlphabet{\amsmathbb}{AMSb}
%\usepackage[bbgreekl]{mathbbol}
%\usepackage[mtpbbi]{mtpro2}

\usepackage{tikz}
\newcommand*\circled[1]{\tikz[baseline=(char.base)]{
            \node[shape=circle,draw,inner sep=2pt] (char) {#1};}}

%\usepackage[svgnames]{xcolor}

%\input{\fonts/math0}
%\input{\symb/structac} % Notations E. Savin
%\input{\symb/logos}

\newcommand{\ci}{\mathrm{i}}
\newcommand{\trace}{\operatorname{Tr}}
\newcommand{\Nset}{\mathbb{N}}
\newcommand{\Zset}{\mathbb{Z}}
\newcommand{\Rset}{\mathbb{R}}
\newcommand{\Cset}{\mathbb{C}}
\newcommand{\Sset}{\mathbb{S}}
\newcommand{\Mset}{\mathbb{M}}
\newcommand{\PhaseSpace}{\Omega}
\newcommand{\ContSet}{{\mathcal C}}
\newcommand{\id}{d}
\newcommand{\iD}{\mathrm{D}}
\newcommand{\iexp}{\mathrm{e}}
\newcommand{\demi}{\frac{1}{2}}
\newcommand{\imply}{\Rightarrow}

% Algebra
\newcommand{\itr}{{\sf T}}
\newcommand{\Id}{{\boldsymbol I}}
\newcommand{\IId}{\mathbb{I}}
\newcommand{\aj}{a}
\newcommand{\bj}{b}
\newcommand{\cj}{c}
\renewcommand{\dj}{d}
\newcommand{\av}{{\boldsymbol\aj}}
\newcommand{\bv}{{\boldsymbol\bj}}
\newcommand{\cv}{{\boldsymbol\cj}}
\newcommand{\dv}{{\boldsymbol\dj}}
\newcommand{\uj}{u}
\newcommand{\vj}{v}
\newcommand{\xj}{x}
\newcommand{\yj}{y}
\newcommand{\zj}{z}
\newcommand{\uv}{{\boldsymbol\uj}}
\newcommand{\vv}{{\boldsymbol\vj}}
\newcommand{\xv}{{\boldsymbol\xj}}
\newcommand{\yv}{{\boldsymbol\yj}}
\newcommand{\zv}{{\boldsymbol\zj}}
\newcommand{\Aj}{A}
\newcommand{\Bj}{B}
\newcommand{\Av}{{\boldsymbol\Aj}}
\newcommand{\Bv}{{\boldsymbol\Bj}}
\newcommand{\Zgv}{{\boldsymbol Z}}

% Analysis
\newcommand{\grad}{{\boldsymbol\nabla}}
\newcommand{\gradx}{{\grad_\xv}}
\newcommand{\Grad}{{\mathbb D}}
\newcommand{\Gradx}{{\Grad_\xv}}
\renewcommand{\div}{\mathrm{div}}
\newcommand{\divx}{{\div_\xv}}
\newcommand{\Div}{\mathbf{Div}}
\newcommand{\Divx}{{\Div_\xv}}

% Kinematics
\newcommand{\ej}{e}
\renewcommand{\ij}{i}
\newcommand{\pj}{p}
\newcommand{\ev}{{\boldsymbol\ej}}
\newcommand{\iv}{{\boldsymbol\ij}}
\newcommand{\pv}{{\boldsymbol\pj}}
\newcommand{\posij}{f}
\newcommand{\posiv}{{\boldsymbol\posij}}
\newcommand{\iposiv}{{\boldsymbol g}}
\newcommand{\Fp}{{\mathbb F}}
\newcommand{\GreenLj}{E}
\newcommand{\GreenL}{{\mathbb\GreenLj}}
\newcommand{\medium}{\Omega}
\newcommand{\strainj}{\varepsilon}
\newcommand*{\strain}{\mbox{$\hspace{0.2em}\rotatebox[x=0pt,y=0.2pt]{90}{\rule{0.02\linewidth}{0.4pt}}\hspace{-0.23em}\upvarepsilon$}}
\newcommand*{\rotation}{{\boldsymbol R}}

% Dynamics
\newcommand{\fj}{f}
\newcommand{\Fj}{F}
\newcommand{\mj}{m}
\newcommand{\nj}{n}
\newcommand{\Tj}{T}
\newcommand{\fv}{{\boldsymbol\fj}}
\newcommand{\Fv}{{\boldsymbol\Fj}}
\newcommand{\mv}{{\boldsymbol\mj}}
\newcommand{\nv}{{\boldsymbol\nj}}
\newcommand{\Tv}{{\boldsymbol\Tj}}
\newcommand{\roi}{\varrho}
\newcommand*{\stressj}{\sigma}
\newcommand*{\tressj}{\tau}
\newcommand{\stress}{\mbox{$\hspace{0.3em}\rotatebox[x=0pt,y=0.2pt]{90}{\rule{0.017\linewidth}{0.4pt}}\hspace{-0.25em}\upsigma$}}
\newcommand*{\tress}{{\boldsymbol\tressj}}
\newcommand{\acj}{a}
\newcommand{\acv}{{\boldsymbol\acj}}

%\newcommand{\Zg}{{\bf\zgj}}
%\newcommand{\xigj}{\xi}
%\newcommand{\xig}{{\boldsymbol\xigj}}
\newcommand{\kgj}{k}
%\newcommand{\kgh}{\kgj_\ygj}
%\newcommand{\kg}{{\bf\kgj}}
\newcommand{\Kg}{{\bf K}}
%\newcommand{\qg}{{\boldsymbol q}}
%\newcommand{\pg}{{\boldsymbol p}}
\newcommand{\hkg}{{\hat \kg}}
\newcommand{\hpg}{\hat{\pg}}
%\newcommand{\vg}{{\boldsymbol v}}
\newcommand{\sg}{{\boldsymbol s}}
%\newcommand{\stress}{\mathbb{\sigma}}
\newcommand{\tenselasj}{{\Large C}}
\newcommand{\tenselas}{\boldsymbol{\mathsf{\tenselasj}}}
\newcommand{\tenscomp}{\boldsymbol{\mathsf{\Large S}}}
\newcommand{\speci}{{\mathrm w}}
\newcommand{\specij}{{\mathrm W}}
\newcommand{\speciv}{{\bf \specij}}
%\newcommand{\cjg}[1]{\overline{#1}}
\newcommand{\eigv}{{\bf b}}
\newcommand{\eigw}{{\bf c}}
\newcommand{\eigl}{\lambda}
\newcommand{\jeig}{\alpha}
\newcommand{\keig}{\beta}
\newcommand{\cel}{c}
\newcommand{\bcel}{{\bf\cel}}
\newcommand{\deng}{{\mathcal E}}
\newcommand{\flowj}{\pi}
\newcommand{\flow}{\boldsymbol\flowj}
\newcommand{\Flowj}{\Pi}
\newcommand{\Flow}{\boldsymbol\Flowj}
\newcommand{\fluxinj}{g}
\newcommand{\fluxin}{{\bf\fluxinj}}
\newcommand{\dscat}{\sigma}
\newcommand{\tdscat}{\Sigma}
\newcommand{\collop}{{\mathcal Q}}
\newcommand{\epsd}{\delta}
\newcommand{\rscat}{\rho}
\newcommand{\tscat}{\tau}
\newcommand{\Rscat}{\mathcal{R}}
\newcommand{\Tscat}{\mathcal{T}}
\newcommand{\lscat}{\ell}
%\newcommand{\floss}{\eta}
\newcommand{\mdiff}{{\bf D}}
%\newcommand{\demi}{\frac{1}{2}}
\newcommand{\domain}{{\mathcal O}}
\newcommand{\bdomain}{{\mathcal D}}
\newcommand{\interface}{\Gamma}
\newcommand{\sinterface}{\gamma_D}
%\newcommand{\normal}{\hat{\bf n}}
\newcommand{\bnabla}{\boldsymbol\nabla}
%\newcommand{\esp}[1]{\mathbb{E}\{\smash{#1}\}}
\newcommand{\mean}[1]{\underline{#1}}
%\newcommand{\BB}{\mathbb{B}}
%\newcommand{\II}{{\boldsymbol I}}
%\newcommand{\TA}{\boldsymbol{\Gamma}}
\newcommand{\Mdisp}{{\mathbf H}}
\newcommand{\Hamil}{{\mathcal H}}
\newcommand{\bzero}{{\bf 0}}

\newcommand{\mass}{M}
\newcommand{\damp}{D}
\newcommand{\stif}{K}
\newcommand{\dsp}{S}
\newcommand{\dof}{q}
\newcommand{\pof}{p}
%\newcommand{\MM}{{\boldsymbol\mass}}
\newcommand{\MD}{{\boldsymbol\damp}}
\newcommand{\MK}{{\boldsymbol\stif}}
\newcommand{\MS}{{\boldsymbol\dsp}}
\newcommand{\Cov}{{\boldsymbol C}}
\newcommand{\dofg}{{\boldsymbol\dof}}
\newcommand{\pofg}{{\boldsymbol\pof}}
\newcommand{\driftj}{b}
\newcommand{\drifts}{{\boldsymbol \driftj}}
\newcommand{\drift}{{\underline\drifts}}
\newcommand{\scatj}{a}
\newcommand{\scat}{{\boldsymbol\scatj}}
\newcommand{\diff}{{\boldsymbol\sigma}}
\newcommand{\load}{F}
\newcommand{\loadg}{{\boldsymbol\load}}
\newcommand{\pdf}{\pi}
\newcommand{\tpdf}{\pdf_t}
\newcommand{\fg}{{\boldsymbol f}}
%\newcommand{\Ugj}{U}
\newcommand{\Vgj}{V}
\newcommand{\Xgj}{X}
\newcommand{\Ygj}{Y}
%\newcommand{\Ug}{{\boldsymbol\Ugj}}
\newcommand{\Vg}{{\boldsymbol\Vgj}}
%\newcommand{\Qg}{{\boldsymbol Q}}
\newcommand{\Pg}{{\boldsymbol P}}
\newcommand{\Xg}{{\boldsymbol\Xgj}}
\newcommand{\Yg}{{\boldsymbol\Ygj}}
\newcommand{\flux}{{\boldsymbol J}}
\newcommand{\wiener}{W}
\newcommand{\whitenoise}{B}
\newcommand{\Wiener}{{\boldsymbol\wiener}}
\newcommand{\White}{{\boldsymbol\white}}
\newcommand{\paraj}{\nu}
\newcommand{\parag}{{\boldsymbol\paraj}}
\newcommand{\parae}{\hat{\parag}}
\newcommand{\erroj}{\epsilon}
\newcommand{\error}{{\boldsymbol\erroj}}
\newcommand{\biaj}{b}
\newcommand{\bias}{{\boldsymbol\biaj}}
%\newcommand{\disp}{{\boldsymbol V}}
\newcommand{\Fisher}{{\mathcal I}}
\newcommand{\likelihood}{{\mathcal L}}

\newcommand{\heps}{\varepsilon}
%\newcommand{\roi}{\varrho}
\newcommand{\jump}[1]{\llbracket{#1}\rrbracket}
\newcommand{\scal}[1]{\left\langle{#1}\right\rangle}
\newcommand{\norm}[1]{\left\|#1\right\|}
\newcommand{\abs}[1]{\left|#1\right|}
%\newcommand{\po}{\operatorname{o}}
\newcommand{\FFT}[1]{\widehat{#1}}
\newcommand{\indic}[1]{{\mathbf 1}_{#1}}
\newcommand{\impulse}{{\mathbbm h}}
\newcommand{\frf}{\FFT{\impulse}}

%\renewcommand{\Moy}[1]{{\boldsymbol\mu}_{#1}}
%\renewcommand{\Rcor}[1]{{\boldsymbol R}_{#1}}
%\renewcommand{\Mw}[1]{{\boldsymbol M}_{#1}}
%\renewcommand{\Sw}[1]{{\boldsymbol S}_{#1}}
%\renewcommand{\esp}[1]{{\mathbb E}\{#1\}}

\newcommand{\emphb}[1]{\textcolor{blue}{#1}}
\newcommand{\mycite}[1]{\textcolor{red}{#1}}
\newcommand{\mycitb}[1]{\textcolor{red}{[{\it #1}]}}

\newcommand{\PDFU}{{\mathcal U}}
\newcommand{\PDFN}{{\mathcal N}}
\newcommand{\TK}{{\boldsymbol\Pi}}
\newcommand{\TKij}{\pi}
\newcommand{\TKi}{{\boldsymbol\pi}}
\newcommand{\SMi}{\TKij^*}
\newcommand{\SM}{\TKi^*}
\newcommand{\lagmuli}{\lambda}
\newcommand{\lagmul}{{\boldsymbol\lagmuli}}
\newcommand{\constraint}{{\boldsymbol C}}
\newcommand{\mconstraint}{\mean{\constraint}}

\newcommand{\mybox}[1]{\fbox{\begin{minipage}{0.93\textwidth}{#1}\end{minipage}}}
\newcommand{\defcolor}[1]{\textcolor{blue}{#1}}

%\definecolor{rose}{LightPink}%{rgb}{251,204,231}

\newtheorem{mydef}{Definition}
\newtheorem{mythe}{Theorem}
\newtheorem{myprop}{Proposition}

% Or whatever. Note that the encoding and the font should match. If T1
% does not look nice, try deleting the line with the fontenc.

\title[1EL5000/S11]
{Interference fit}

\subtitle{1EL5000--Continuum Mechanics -- Tutorial Class \#11} % (optional)

\author[\'E. Savin] % (optional, use only with lots of authors)
{\'E. Savin\inst{1,2}\\ \scriptsize{\texttt{eric.savin@\{centralesupelec,onera\}.fr}}}%\inst{1} }
% - Use the \inst{?} command only if the authors have different
%   affiliation.

\institute[\Onera] % (optional, but mostly needed)
{\inst{1}{Information Processing and Systems Dept.\\\Onera, France}
\and
 \inst{2}{Mechanical and Civil Engineering Dept.\\\ECP, France}}%
%  Department of Theoretical Philosophy\\
%  University of Elsewhere}
% - Use the \inst command only if there are several affiliations.
% - Keep it simple, no one is interested in your street address.

%\date[Short Occasion] % (optional)
\date{\today}

\subject{Interference fit}
% This is only inserted into the PDF information catalog. Can be left
% out. 



% If you have a file called "university-logo-filename.xxx", where xxx
% is a graphic format that can be processed by latex or pdflatex,
% resp., then you can add a logo as follows:

% \pgfdeclareimage[height=0.5cm]{university-logo}{university-logo-filename}
% \logo{\pgfuseimage{university-logo}}



% Delete this, if you do not want the table of contents to pop up at
% the beginning of each subsection:
\AtBeginSection[]
%\AtBeginSubsection[]
{
  \begin{frame}<beamer>{Outline}
    \tableofcontents[currentsection]%,currentsubsection]
  \end{frame}
}


% If you wish to uncover everything in a step-wise fashion, uncomment
% the following command: 

%\beamerdefaultoverlayspecification{<+->}


\begin{document}

\begin{frame}
  \titlepage
\end{frame}

\begin{frame}{Outline}
  \tableofcontents
  % You might wish to add the option [pausesections]
\end{frame}


% Since this a solution template for a generic talk, very little can
% be said about how it should be structured. However, the talk length
% of between 15min and 45min and the theme suggest that you stick to
% the following rules:  

% - Exactly two or three sections (other than the summary).
% - At *most* three subsections per section.
% - Talk about 30s to 2min per frame. So there should be between about
%   15 and 30 frames, all told.

\section{Some algebra}
\subsection{Vector \& tensor products}

\begin{frame}{Some algebra}{Vector \& tensor products}

\begin{itemize}
\item Scalar product:
\begin{displaymath}
\av,\bv\in\Rset^a\,,\quad\scal{\av,\bv}=\sum_{j=1}^a\aj_j\bj_j=\aj_j\bj_j\,,
\end{displaymath}
The last equality is \emphb{Einstein's summation convention}.
\item Tensors and tensor product (or outer product):
\begin{displaymath}
\Av\in\Rset^a\to\Rset^b\,,\quad\Av=\av\otimes\bv\,,\quad\av\in\Rset^a\,,\bv\in\Rset^b\,.
\end{displaymath}
\item Tensor application to vectors:
\begin{displaymath}
\Av=\av\otimes\bv\in\Rset^a\to\Rset^b\,,\cv\in\Rset^b\,,\quad\Av\cv=\scal{\bv,\cv}\av\,.
\end{displaymath}
\item Product of tensors $\equiv$ composition of linear maps:
\begin{displaymath}
\Av=\av\otimes\bv\,,\Bv=\cv\otimes\dv\,,\quad\Av\Bv=\scal{\bv,\cv}\av\otimes\dv\,.
\end{displaymath}
\end{itemize}

\end{frame}

\begin{frame}{Some algebra}{Vector \& tensor products}

%\onslide<2|handout:2>

\begin{itemize}
\item Scalar product of tensors:
\begin{displaymath}
\scal{\Av,\Bv}=\trace(\Av\Bv^\itr):=\Av:\Bv=\Aj_{jk}\Bj_{jk}\,.
\end{displaymath}
\item Let $\{\ev_j\}_{j=1}^d$ be a Cartesian basis in $\Rset^d$. Then:
\begin{displaymath}
\begin{split}
\aj_j &=\scal{\av,\ev_j}\,, \\
\Aj_{jk} &=\scal{\Av,\ev_j\otimes\ev_k}=\Av:\ev_j\otimes\ev_k \\
&=\scal{\Av\ev_k,\ev_j}\,,
\end{split}
\end{displaymath}
such that:
\begin{displaymath}
\begin{split}
\av &=\aj_j\ev_j\,, \\
\Av &=\Aj_{jk}\ev_j\otimes\ev_k\,.
\end{split}
\end{displaymath}
\item Example: the identity matrix
\begin{displaymath}
\Id=\ev_j\otimes\ev_j\,.
\end{displaymath}
\end{itemize}

%\end{overprint}

\end{frame}

\subsection{Vector \& tensor analysis}

\begin{frame}{Some analysis}{Vector \& tensor analysis}

\begin{itemize}
\item Gradient of a vector function $\av(\xv)$, $\xv\in\Rset^d$:
\begin{displaymath}
\Gradx\av=\frac{\partial\av}{\partial\xj_j}\otimes\ev_j\,.
\end{displaymath}
\item Divergence of a vector function $\av(\xv)$, $\xv\in\Rset^d$:
\begin{displaymath}
\divx\av=\scal{\gradx,\av}=\trace(\Gradx\av)=\frac{\partial\aj_j}{\partial\xj_j}\,.
\end{displaymath}
\item Divergence of a tensor function $\Av(\xv)$, $\xv\in\Rset^d$:
\begin{displaymath}
\Divx\Av=\frac{\partial(\Av\ev_j)}{\partial\xj_j}\,.
\end{displaymath}
\end{itemize}

\end{frame}

\begin{frame}{Some analysis}{Vector \& tensor analysis in cylindrical coordinates}

\begin{itemize}
\item Gradient of a vector function $\av(r,\theta,\zj)$:
\begin{displaymath}
\Gradx\av=\frac{\partial\av}{\partial r}\otimes\ev_r+\frac{\partial\av}{\partial\theta}\otimes\frac{\ev_\theta}{r}+\frac{\partial\av}{\partial\zj}\otimes\ev_\zj\,.
\end{displaymath}
\item Divergence of a vector function $\av(r,\theta,\zj)$:
\begin{displaymath}
\divx\av=\scal{\frac{\partial\av}{\partial r},\ev_r}+\scal{\frac{\partial\av}{\partial\theta},\frac{\ev_\theta}{r}}+\scal{\frac{\partial\av}{\partial\zj},\ev_\zj}\,.
\end{displaymath}
\item Divergence of a tensor function $\Av(r,\theta,\zj)$:
\begin{displaymath}
\Divx\Av=\frac{\partial\Av}{\partial r}\ev_r+\frac{\partial\Av}{\partial\theta}\frac{\ev_\theta}{r}+\frac{\partial\Av}{\partial\zj}\ev_\zj\,.
\end{displaymath}
\end{itemize}

\end{frame}



%\section{Linear elasticity}
%\input{\Sections/LineraElasticity}


\section{5.4 Interference fit}
\newcommand{\ishaft}{{\it a}}
\newcommand{\ihub}{{\it m}}
\newcommand{\iany}{{\it s}}
\newcommand{\cshaft}{{\small\textcircled{\ishaft}}}
\newcommand{\chub}{{\small\textcircled{\ihub}}}
\newcommand{\cany}{{\small\textcircled{\iany}}}

\begin{frame}{Interference fit}{Setup}

\begin{figure}[t]
\centering\includegraphics[scale=.5]{\figs/ch11-Fit}
\end{figure}
%\vskip-10pt{\hspace{2.5truecm}\mbox{\tiny{\copyright\ G. Puel}}}
\vskip20pt
\begin{itemize}
\item Hub:
\begin{displaymath}
\!\!\!\!\!\!\!\!\medium_\ihub=\{\pv=(r,\theta,\zj);\,r\in(R_i,R_e),\theta\in[0,2\pi),\zj\in(0,H_\ihub)\}\,;
\end{displaymath}
\item Shaft:
\begin{displaymath}
\!\!\!\!\!\!\!\!\medium_\ishaft=\left\{\pv=(r,\theta,\zj);\,r\in\left(0,R_i+\frac{s}{2}\right),\theta\in[0,2\pi),\zj\in(0,H_\ishaft)\right\}\,.
\end{displaymath}
\end{itemize}

\end{frame}

\begin{frame}{Interference fit}{Solution}

\begin{center}
\textcolor{blue}{\LARGE {\bf Part 1: Practical realization}}
\end{center}

\end{frame}

\begin{frame}{Interference fit}{Solution}

\begin{overprint}

\onslide<1|handout:1>
\vskip-20pt
\begin{exampleblock}{Question \#1: Equations for the heated hub $\medium_\ihub$?}
\begin{itemize}
\item Equilibrium equation $\Div\stress+\fv_v=\roi\ddot{\uv}$ within the "infinitesimal deformation hypothesis;"
\end{itemize}
\end{exampleblock}

\onslide<2|handout:2>
\vskip-20pt
\begin{exampleblock}{Question \#1: Equations for the heated hub $\medium_\ihub$?}
\begin{itemize}
\item Equilibrium equation $\Div\stress+\fv_v=\roi\ddot{\uv}$ within the "infinitesimal deformation hypothesis;"
\item "The effects of inertia and the action of gravity can be neglected," hence $\roi\ddot{\uv}=\bzero$, $\fv_v=\bzero$, and $\Div\stress=\bzero$;
\end{itemize}
\end{exampleblock}

\onslide<3|handout:3>
\vskip-20pt
\begin{exampleblock}{Question \#1: Equations for the heated hub $\medium_\ihub$?}
\begin{itemize}
\item Equilibrium equation $\Div\stress=\bzero$;
\item Constitutive equation for a "linear thermoelastic, isotropic and homogeneous" material (in compliance):
\begin{displaymath}
\begin{split}
\strain &=\strain_\text{elas}+\strain_\text{th} \\
&=\frac{1+\nu}{E}\stress-\frac{\nu}{E}(\trace\stress)\Id+\alpha\Delta T\Id\,,
\end{split}
\end{displaymath}
or (in stiffness):
\begin{displaymath}
\begin{split}
\stress &=\tenselas(\strain-\strain_\text{th}) \\
&=\lambda(\trace\strain)\Id+2\mu\strain-(3\lambda+2\mu)\alpha\Delta T \Id\,;
\end{split}
\end{displaymath}
\end{itemize}
\end{exampleblock}

\onslide<4|handout:4>
\vskip-20pt
\begin{exampleblock}{Question \#1: Equations for the heated hub $\medium_\ihub$?}
\begin{itemize}
\item Equilibrium equation $\Div\stress=\bzero$;
\item Constitutive equation:
\begin{displaymath}
\begin{split}
\strain &=\frac{1+\nu}{E}\stress-\frac{\nu}{E}(\trace\stress)\Id+\alpha\Delta T\Id\,, \\
\stress &=\lambda(\trace\strain)\Id+2\mu\strain-(3\lambda+2\mu)\alpha\Delta T \Id\,;
\end{split}
\end{displaymath}
\item Initial conditions: useless for statics; 
\end{itemize}
\end{exampleblock}

\onslide<5|handout:5>
\vskip-20pt
\begin{exampleblock}{Question \#1: Equations for the heated hub $\medium_\ihub$?}
\begin{itemize}
\item Equilibrium equation $\Div\stress=\bzero$;
\item Constitutive equation:
\begin{displaymath}
\begin{split}
\strain &=\frac{1+\nu}{E}\stress-\frac{\nu}{E}(\trace\stress)\Id+\alpha\Delta T\Id\,, \\
\stress &=\lambda(\trace\strain)\Id+2\mu\strain-(3\lambda+2\mu)\alpha\Delta T \Id\,;
\end{split}
\end{displaymath}
\item Initial conditions: useless for statics; 
\item Boundary conditions: "the hub is simply placed at $\zj=0$ on a fixed and perfectly rigid support on which it can slide with no friction,"
\begin{displaymath}
\begin{split}
\scal{\uv,\iv_\zj}|_{\{\zj=0\}} &=0\,, \\
\scal{\stress(-\iv_\zj),\iv_\xj}|_{\{\zj=0\}}=\scal{\stress(-\iv_\zj),\iv_\yj}|_{\{\zj=0\}} &=0\,;
\end{split}
\end{displaymath}
\end{itemize}
\end{exampleblock}

\onslide<6|handout:6>
\vskip-20pt
\begin{exampleblock}{Question \#1: Equations for the heated hub $\medium_\ihub$?}
\begin{itemize}
\item Equilibrium equation $\Div\stress=\bzero$;
\item Constitutive equation:
\begin{displaymath}
\begin{split}
\strain &=\frac{1+\nu}{E}\stress-\frac{\nu}{E}(\trace\stress)\Id+\alpha\Delta T\Id\,, \\
\stress &=\lambda(\trace\strain)\Id+2\mu\strain-(3\lambda+2\mu)\alpha\Delta T \Id\,;
\end{split}
\end{displaymath}
\item Initial conditions: useless for statics; 
\item Boundary conditions: free surface at $\{r=R_i\}$, at $\{r=R_e\}$, and at $\{\zj=H_\ihub\}$,
\begin{displaymath}
\begin{split}
\stress\iv_r(\theta)|_{\{r=R_e\}} &=\bzero\,, \\
\stress(-\iv_r(\theta))|_{\{r=R_i\}} &=\bzero\,, \\
\stress\iv_\zj|_{\{\zj=H_\ihub\}} &=\bzero\,.
\end{split}
\end{displaymath}
\end{itemize}
\end{exampleblock}

\onslide<7|handout:7>
\vskip-20pt
\begin{exampleblock}{Question \#1: Equations for the heated hub $\medium_\ihub$?}
\begin{itemize}
\item Recap:
\begin{displaymath}
\Div\stress=\bzero \,,
\end{displaymath}
\begin{displaymath}
\begin{split}
\strain &=\frac{1+\nu}{E}\stress-\frac{\nu}{E}(\trace\stress)\Id+\alpha\Delta T\Id\,, \\
\stress &=\lambda(\trace\strain)\Id+2\mu\strain-(3\lambda+2\mu)\alpha\Delta T \Id\,, \\
\end{split}
\end{displaymath}
\begin{displaymath}
\begin{split}
\scal{\uv,\iv_\zj}|_{\{\zj=0\}} &=0\,, \\
\scal{\stress\iv_\zj,\iv_\xj}|_{\{\zj=0\}}=\scal{\stress\iv_\zj,\iv_\yj}|_{\{\zj=0\}} &=0\,, \\
\stress\iv_r(\theta)|_{\{r=R_e\}} &=\bzero\,, \\
\stress\iv_r(\theta)|_{\{r=R_i\}} &=\bzero\,, \\
\stress\iv_\zj|_{\{\zj=H_\ihub\}} &=\bzero\,.
\end{split}
\end{displaymath}
\end{itemize}
\end{exampleblock}

\end{overprint}

\end{frame}

\begin{frame}{Interference fit}{Solution}

\begin{overprint}

\onslide<1|handout:1>
\vskip-20pt
\begin{exampleblock}{Question \#2: Is $\stress=\bzero$ a solution? $\uv$?}
\begin{itemize}
\item $\stress=\bzero$ satisfies the equilibrium equation and boundary conditions; 
\end{itemize}
\end{exampleblock}

\onslide<2|handout:2>
\vskip-20pt
\begin{exampleblock}{Question \#2: Is $\stress=\bzero$ a solution? $\uv$?}
\begin{itemize}
\item $\stress=\bzero$ satisfies the equilibrium equation and boundary conditions; 
\item Then $\strain=\alpha\Delta T\Id$ is constant;
\end{itemize}
\end{exampleblock}

\onslide<3|handout:3>
\vskip-20pt
\begin{exampleblock}{Question \#2: Is $\stress=\bzero$ a solution? $\uv$?}
\begin{itemize}
\item $\stress=\bzero$ satisfies the equilibrium equation and boundary conditions; 
\item Then $\strain=\demi(\Gradx\uv+\Gradx\uv^\itr)=\alpha\Delta T\Id$ is constant;
\item Consequently the solution for $\uv$ is $\uv=\alpha\Delta T\xv$ up to a rigid body motion in the $(\xj,\yj)$ plane;
\item Indeed $\Gradx(\Av\xv)=\Av_j\otimes\ev_j$ where $\Av_j$ is the $j^\text{th}$ column of $\Av$, and $\strain_\xv(\Av\xv)=\Av_j\otimes_s\ev_j$ which must be proportional to the identitity.
\end{itemize}
\end{exampleblock}

\end{overprint}

\end{frame}

\begin{frame}{Interference fit}{Solution}

\begin{overprint}
\onslide<1|handout:1>
\vskip-20pt
\begin{exampleblock}{Question \#3: $\Delta T$?}
\begin{itemize}
\item The solution for $\uv$ is $\uv=\alpha\Delta T\xv$ up to a rigid body motion in the $(\xj,\yj)$ plane;
\item Thus $\uj_r(\xv)=\alpha\Delta T r$ and one must choose $\Delta T$ such that $\uj_r(R_i,\theta,\zj)=\frac{s}{2}$, or:
\begin{displaymath}
\Delta T=\frac{s}{2\alpha R_i}\,;
\end{displaymath}
\end{itemize}
\end{exampleblock}

\onslide<2|handout:2>
\vskip-20pt
\begin{exampleblock}{Question \#3: $\Delta T$?}
\begin{itemize}
\item The solution for $\uv$ is $\uv=\alpha\Delta T\xv$ up to a rigid body motion in the $(\xj,\yj)$ plane;
\item Thus $\uj_r(\xv)=\alpha\Delta T r$ and one must choose $\Delta T$ such that $\uj_r(R_i,\theta,\zj)=\frac{s}{2}$, or:
\begin{displaymath}
\Delta T=\frac{s}{2\alpha R_i}\,;
\end{displaymath}
\item N.A.: $s/2R_i=10^{-3}$, $\alpha=10^{-5}\,\text{K}^{-1}$ $\imply$ $\Delta T=100$ �K.
\end{itemize}
\end{exampleblock}

\end{overprint}

\end{frame}

\begin{frame}{Interference fit}{Solution}

\begin{center}
\textcolor{blue}{\LARGE {\bf Part 2: Solid cylinder under external pressure}}
\end{center}
\begin{figure}[t]
\centering\includegraphics[scale=.5]{\figs/ch11-Shaft}
\end{figure}

\end{frame}

\begin{frame}{Interference fit}{Solution}

\begin{overprint}

\onslide<1|handout:1>
\vskip-20pt
\begin{exampleblock}{Question \#4: Equations for the shaft $\medium_\ishaft$?}
\begin{itemize}
\item Equilibrium equation $\Div\stress+\fv_v=\roi\ddot{\uv}$ within the "infinitesimal deformation hypothesis;"
\end{itemize}
\end{exampleblock}

\onslide<2|handout:2>
\vskip-20pt
\begin{exampleblock}{Question \#4: Equations for the shaft $\medium_\ishaft$?}
\begin{itemize}
\item Equilibrium equation $\Div\stress+\fv_v=\roi\ddot{\uv}$ within the "infinitesimal deformation hypothesis;"
\item "The effects of inertia and the action of gravity can be neglected," hence $\roi\ddot{\uv}=\bzero$, $\fv_v=\bzero$, and $\Div\stress=\bzero$;
\end{itemize}
\end{exampleblock}

\onslide<3|handout:3>
\vskip-20pt
\begin{exampleblock}{Question \#4: Equations for the shaft $\medium_\ishaft$?}
\begin{itemize}
\item Equilibrium equation $\Div\stress=\bzero$;
\item Constitutive equation for a "linear elastic, isotropic, homogeneous" material:
\begin{displaymath}
\begin{split}
\strain &=\frac{1+\nu}{E}\stress-\frac{\nu}{E}(\trace\stress)\Id\,, \\
\stress &=\lambda(\trace\strain)\Id+2\mu\strain\,;
\end{split}
\end{displaymath}
\end{itemize}
\end{exampleblock}

\onslide<4|handout:4>
\vskip-20pt
\begin{exampleblock}{Question \#4: Equations for the shaft $\medium_\ishaft$?}
\begin{itemize}
\item Equilibrium equation $\Div\stress=\bzero$;
\item Constitutive equation:
\begin{displaymath}
\begin{split}
\strain &=\frac{1+\nu}{E}\stress-\frac{\nu}{E}(\trace\stress)\Id\,, \\
\stress &=\lambda(\trace\strain)\Id+2\mu\strain\,;
\end{split}
\end{displaymath}
\item Initial conditions: useless for statics; 
\end{itemize}
\end{exampleblock}

\onslide<5|handout:5>
\vskip-20pt
\begin{exampleblock}{Question \#4: Equations for the shaft $\medium_\ishaft$?}
\begin{itemize}
\item Equilibrium equation $\Div\stress=\bzero$;
\item Constitutive equation:
\begin{displaymath}
\begin{split}
\strain &=\frac{1+\nu}{E}\stress-\frac{\nu}{E}(\trace\stress)\Id\,, \\
\stress &=\lambda(\trace\strain)\Id+2\mu\strain\,;
\end{split}
\end{displaymath}
\item Initial conditions: useless for statics; 
\item Boundary conditions: "this contact pressure $P_0$ is uniform,"
\begin{displaymath}
\stress\iv_r(\theta)|_{\{r=R_e'\}}=-P_0\iv_r(\theta)\,;
\end{displaymath}
\end{itemize}
\end{exampleblock}

\onslide<6|handout:6>
\vskip-20pt
\begin{exampleblock}{Question \#4: Equations for the shaft $\medium_\ishaft$?}
\begin{itemize}
\item Equilibrium equation $\Div\stress=\bzero$;
\item Constitutive equation:
\begin{displaymath}
\begin{split}
\strain &=\frac{1+\nu}{E}\stress-\frac{\nu}{E}(\trace\stress)\Id\,, \\
\stress &=\lambda(\trace\strain)\Id+2\mu\strain\,;
\end{split}
\end{displaymath}
\item Initial conditions: useless for statics; 
\item Boundary conditions: "the faces at both ends $\zj=0$ and $\zj=H_\ishaft$ are considered to be free of forces,"
\begin{displaymath}
\begin{split}
\stress(-\iv_\zj)|_{\{\zj=0\}} &=\bzero\,, \\
\stress\iv_\zj|_{\{\zj=H_\ishaft\}} &=\bzero\,.
\end{split}
\end{displaymath}
\end{itemize}
\end{exampleblock}

\onslide<7|handout:7>
\vskip-20pt
\begin{exampleblock}{Question \#4: Equations for the shaft $\medium_\ishaft$?}
\begin{itemize}
\item Recap:
\begin{displaymath}
\Div\stress=\bzero \,,
\end{displaymath}
\begin{displaymath}
\begin{split}
\strain &=\frac{1+\nu}{E}\stress-\frac{\nu}{E}(\trace\stress)\Id\,, \\
\stress &=\lambda(\trace\strain)\Id+2\mu\strain\,, \\
\end{split}
\end{displaymath}
\begin{displaymath}
\begin{split}
\stress\iv_r(\theta)|_{\{r=R_e'\}} &=-P_0\iv_r(\theta)\,, \\
\stress\iv_\zj|_{\{\zj=0\}} &=\bzero\,, \\
\stress\iv_\zj|_{\{\zj=H_\ishaft\}} &=\bzero\,.
\end{split}
\end{displaymath}
\end{itemize}
\end{exampleblock}

\end{overprint}

\end{frame}

\begin{frame}{Interference fit}{Solution}

\begin{overprint}

\onslide<1|handout:1>
\vskip-20pt
\begin{exampleblock}{Question \#5: $\stressj_{rr}$,  $\stressj_{\theta\theta}$,  $\stressj_{\zj\zj}$?}
\begin{itemize}
\item We seek for a stress tensor of the form:
\begin{displaymath}
\stress_{fc}(\xv)=\stressj_{rr}\iv_r(\theta)\otimes\iv_r(\theta)+\stressj_{\theta\theta}\iv_\theta(\theta)\otimes\iv_\theta(\theta)+\stressj_{\zj\zj}\iv_\zj\otimes\iv_\zj\,;
\end{displaymath}
\item From the boundary conditions:
\begin{displaymath}
\begin{split}
\stress\iv_r(\theta)|_{\{r=R_e'\}} &=\stressj_{rr}\iv_r(\theta)=-P_0\iv_r(\theta)\,, \\
\stress(-\iv_\zj)|_{\{\zj=0\}} &=-\stressj_{\zj\zj}\iv_z =\bzero\,, \\
\stress\iv_\zj|_{\{\zj=H_\ishaft\}} &=\stressj_{\zj\zj}\iv_z=\bzero\,;
\end{split}
\end{displaymath}
\item Therefore $\stressj_{rr}=-P_0$ and $\stressj_{\zj\zj}=0$;
\end{itemize}
\end{exampleblock}

\onslide<2|handout:2>
\vskip-20pt
\begin{exampleblock}{Question \#5: $\stressj_{rr}$, $\stressj_{\theta\theta}$, $\stressj_{\zj\zj}$?}
\begin{itemize}
\item We seek for a stress tensor of the form:
\begin{displaymath}
\stress_{fc}(\xv)=-P_0\iv_r(\theta)\otimes\iv_r(\theta)+\stressj_{\theta\theta}\iv_\theta(\theta)\otimes\iv_\theta(\theta)\,;
\end{displaymath}
\item From the equilibrium equation:
\begin{displaymath}
\begin{split}
\!\!\!\!\!\!\!\!\!\!\!\!\Div\stress_{fc} &=\frac{\partial\stress_{fc}}{\partial r}\iv_r(\theta)+\frac{\partial\stress_{fc}}{\partial\theta}\frac{\iv_\theta(\theta)}{r}+\frac{\partial\stress_{fc}}{\partial\zj}\iv_\zj \\
&=\left(-P_0\frac{\partial(\iv_r(\theta)\otimes\iv_r(\theta))}{\partial\theta}+ \stressj_{\theta\theta}\frac{\partial(\iv_\theta(\theta)\otimes\iv_\theta(\theta))}{\partial\theta}\right)\frac{\iv_\theta(\theta)}{r} \\
&=-2\Big(P_0\iv_\theta(\theta)\otimes_s\iv_r(\theta)+\stressj_{\theta\theta}\iv_r(\theta)\otimes_s\iv_\theta(\theta)\Big)\frac{\iv_\theta(\theta)}{r} \\
&=-\left(\frac{P_0+\stressj_{\theta\theta}}{r}\right)\iv_r(\theta)\,;
\end{split}
\end{displaymath}
\end{itemize}
\end{exampleblock}

\onslide<3|handout:3>
\vskip-20pt
\begin{exampleblock}{Question \#5: $\stressj_{rr}$,  $\stressj_{\theta\theta}$,  $\stressj_{\zj\zj}$?}
\begin{itemize}
\item But $\Div\stress_{fc}=\bzero$ and therefore $\stressj_{\theta\theta}=-P_0$;
\item Finally:
\begin{displaymath}
\stress_{fc}(\xv)=-P_0\Big(\iv_r(\theta)\otimes\iv_r(\theta)+\iv_\theta(\theta)\otimes\iv_\theta(\theta)\Big)\,.
\end{displaymath}
satisfies all equations of the shaft.
\end{itemize}
\end{exampleblock}

\end{overprint}

\end{frame}

\begin{frame}{Interference fit}{Solution}

\begin{overprint}

\onslide<1|handout:1>
\vskip-20pt
\begin{exampleblock}{Question \#6: Displacement field $\uv_{fc}$ for $\stress_{fc}$?}
\begin{itemize}
\item From the constitutive equation:
\begin{displaymath}
\begin{split}
\mspace{-40mu}\strain_{fc} &=\frac{1+\nu}{E}\stress_{fc}-\frac{\nu}{E}(\trace\stress_{fc})\Id \\
\mspace{-40mu} &=-\frac{P_0}{E}\left[(1+\nu)\left(\iv_r(\theta)\otimes\iv_r(\theta)+\iv_\theta(\theta)\otimes\iv_\theta(\theta)\right)-2\nu\Id\right] \\
\mspace{-40mu} &=-\frac{P_0}{E}\left[(1-\nu)\left(\iv_r(\theta)\otimes\iv_r(\theta)+\iv_\theta(\theta)\otimes\iv_\theta(\theta)\right)-2\nu\iv_\zj\otimes\iv_\zj\right]\,;
\end{split}
\end{displaymath}
\end{itemize}
\end{exampleblock}

\onslide<2|handout:2>
\vskip-20pt
\begin{exampleblock}{Question \#6: Displacement field $\uv_{fc}$ for $\stress_{fc}$?}
\begin{itemize}
\item From the constitutive equation:
\begin{displaymath}
\begin{split}
\mspace{-40mu}\strain_{fc} &=\frac{1+\nu}{E}\stress_{fc}-\frac{\nu}{E}(\trace\stress_{fc})\Id \\
\mspace{-40mu} &=-\frac{P_0}{E}\left[(1+\nu)\left(\iv_r(\theta)\otimes\iv_r(\theta)+\iv_\theta(\theta)\otimes\iv_\theta(\theta)\right)-2\nu\Id\right] \\
\mspace{-40mu} &=-\frac{P_0}{E}\left[(1-\nu)\left(\iv_r(\theta)\otimes\iv_r(\theta)+\iv_\theta(\theta)\otimes\iv_\theta(\theta)\right)-2\nu\iv_\zj\otimes\iv_\zj\right]\,;
\end{split}
\end{displaymath}
\item Therefore $\strainj_{r\theta}=\strainj_{r\zj}=\strainj_{\theta\zj}=0$ and:
\begin{displaymath}
\begin{split}
\strainj_{rr}=\strainj_{\theta\theta} &=\frac{P_0}{E}(\nu-1)\,, \\
\strainj_{\zj\zj} &=\frac{2\nu P_0}{E}\,;
\end{split}
\end{displaymath}
\end{itemize}
\end{exampleblock}

\onslide<3|handout:3>
\vskip-20pt
\begin{exampleblock}{Question \#6: Displacement field $\uv_{fc}$ for $\stress_{fc}$?}
\begin{itemize}
\item From the symmetry of the {\bf geometry and loads} one can choose $\uv(r,\theta,\zj)=\uj_r(r)\iv_r(\theta)+\uj_\zj(\zj)\iv_\zj$ such that:
\begin{displaymath}
\begin{split}
\strainj_{rr} &=\frac{\id\uj_r}{\id r}=\frac{P_0}{E}(\nu-1)\,, \\
\strainj_{\theta\theta} &=\frac{\uj_r}{r}=\frac{P_0}{E}(\nu-1)\,, \\
\strainj_{\zj\zj} &=\frac{\id\uj_\zj}{\id\zj}=\frac{2\nu P_0}{E}\,;
\end{split}
\end{displaymath}
\item Consequently $\uj_r(r)=\frac{P_0}{E}(\nu-1)r$, $\uj_\zj(\zj)=\frac{2\nu P_0}{E}\zj$, and:
\begin{displaymath}
\uv(r,\theta,\zj)=\frac{P_0}{E}\left((\nu-1)r\iv_r(\theta)+2\nu\zj\iv_\zj\right)\,,
\end{displaymath}
up to a rigid-body motion.
\end{itemize}
\end{exampleblock}

\end{overprint}

\end{frame}

\begin{frame}{Interference fit}{Solution}

\begin{center}
\textcolor{blue}{\LARGE {\bf Part 3: Hollow cylinder under internal pressure}}
\end{center}
\begin{figure}[t]
\centering\includegraphics[scale=.5]{\figs/ch11-Hub}
\end{figure}

\end{frame}

\begin{frame}{Interference fit}{Solution}

\begin{overprint}

\onslide<1|handout:1>
\vskip-20pt
\begin{exampleblock}{Question \#7: Equations for the hub $\medium_\ihub$?}
\begin{itemize}
\item Equilibrium equation $\Div\stress+\fv_v=\roi\ddot{\uv}$ within the "infinitesimal deformation hypothesis;"
\end{itemize}
\end{exampleblock}

\onslide<2|handout:2>
\vskip-20pt
\begin{exampleblock}{Question \#7: Equations for the hub $\medium_\ihub$?}
\begin{itemize}
\item Equilibrium equation $\Div\stress+\fv_v=\roi\ddot{\uv}$ within the "infinitesimal deformation hypothesis;"
\item "The effects of inertia and the action of gravity can be neglected," hence $\roi\ddot{\uv}=\bzero$, $\fv_v=\bzero$, and $\Div\stress=\bzero$;
\end{itemize}
\end{exampleblock}

\onslide<3|handout:3>
\vskip-20pt
\begin{exampleblock}{Question \#7: Equations for the hub $\medium_\ihub$?}
\begin{itemize}
\item Equilibrium equation $\Div\stress=\bzero$;
\item Constitutive equation for a "linear elastic, isotropic, homogeneous" material:
\begin{displaymath}
\begin{split}
\strain &=\frac{1+\nu}{E}\stress-\frac{\nu}{E}(\trace\stress)\Id\,, \\
\stress &=\lambda(\trace\strain)\Id+2\mu\strain\,;
\end{split}
\end{displaymath}
\end{itemize}
\end{exampleblock}

\onslide<4|handout:4>
\vskip-20pt
\begin{exampleblock}{Question \#7: Equations for the hub $\medium_\ihub$?}
\begin{itemize}
\item Equilibrium equation $\Div\stress=\bzero$;
\item Constitutive equation:
\begin{displaymath}
\begin{split}
\strain &=\frac{1+\nu}{E}\stress-\frac{\nu}{E}(\trace\stress)\Id\,, \\
\stress &=\lambda(\trace\strain)\Id+2\mu\strain\,;
\end{split}
\end{displaymath}
\item Initial conditions: useless for statics; 
\end{itemize}
\end{exampleblock}

\onslide<5|handout:5>
\vskip-20pt
\begin{exampleblock}{Question \#7: Equations for the hub $\medium_\ihub$?}
\begin{itemize}
\item Equilibrium equation $\Div\stress=\bzero$;
\item Constitutive equation:
\begin{displaymath}
\begin{split}
\strain &=\frac{1+\nu}{E}\stress-\frac{\nu}{E}(\trace\stress)\Id\,, \\
\stress &=\lambda(\trace\strain)\Id+2\mu\strain\,;
\end{split}
\end{displaymath}
\item Initial conditions: useless for statics; 
\item Boundary conditions: "it is subjected on its inner boundary to the contact pressure $P_0$,"
\begin{displaymath}
\begin{split}
\stress(-\iv_r(\theta))|_{\{r=R_i\}} &=P_0\iv_r(\theta)\,, \\
\stress\iv_r(\theta)|_{\{r=R_e\}} &=\bzero\,; \\
\end{split}
\end{displaymath}
\end{itemize}
\end{exampleblock}

\onslide<6|handout:6>
\vskip-20pt
\begin{exampleblock}{Question \#7: Equations for the hub $\medium_\ihub$?}
\begin{itemize}
\item Equilibrium equation $\Div\stress=\bzero$;
\item Constitutive equation:
\begin{displaymath}
\begin{split}
\strain &=\frac{1+\nu}{E}\stress-\frac{\nu}{E}(\trace\stress)\Id\,, \\
\stress &=\lambda(\trace\strain)\Id+2\mu\strain\,;
\end{split}
\end{displaymath}
\item Initial conditions: useless for statics; 
\item Boundary conditions: "the two faces $\zj=0$ and $\zj=H_\ihub$ are free of forces,"
\begin{displaymath}
\begin{split}
\stress(-\iv_\zj)|_{\{\zj=0\}} &=\bzero\,, \\
\stress\iv_\zj|_{\{\zj=H_\ihub\}} &=\bzero\,.
\end{split}
\end{displaymath}
\end{itemize}
\end{exampleblock}

\onslide<7|handout:7>
\vskip-20pt
\begin{exampleblock}{Question \#7: Equations for the hub $\medium_\ihub$?}
\begin{itemize}
\item Recap:
\begin{displaymath}
\Div\stress=\bzero \,,
\end{displaymath}
\begin{displaymath}
\begin{split}
\strain &=\frac{1+\nu}{E}\stress-\frac{\nu}{E}(\trace\stress)\Id\,, \\
\stress &=\lambda(\trace\strain)\Id+2\mu\strain\,, \\
\end{split}
\end{displaymath}
\begin{displaymath}
\begin{split}
\stress(-\iv_r(\theta))|_{\{r=R_i\}} &=P_0\iv_r(\theta)\,, \\
\stress\iv_r(\theta)|_{\{r=R_e\}} &=\bzero\,, \\
\stress\iv_\zj|_{\{\zj=0\}} &=\bzero\,, \\
\stress\iv_\zj|_{\{\zj=H_\ihub\}} &=\bzero\,.
\end{split}
\end{displaymath}
\end{itemize}
\end{exampleblock}

\end{overprint}

\end{frame}

\begin{frame}{Interference fit}{Solution}

\begin{exampleblock}{Question \#8: Justify $\uv_{fd}$?}
\begin{itemize}
\item We seek for a displacement field of the form:
\begin{displaymath}
\uv_{fd}(\xv)=\uj_r(r)\iv_r(\theta)+\uj_\zj(\zj)\iv_\zj\,;
\end{displaymath}
\item Its is justified by the symmetry of the {\bf geometry and loads} of the problem:
\begin{itemize}
\item The problem is axisymmetric hence $\uj_\theta=0$ and $\uj_r,\uj_\theta$ are independent of $\theta$;
\item The radial displacement is independent of $\zj$ by translational invariance;
\item The vertical displacement is independent of $r$ for there is not warping of the hub.
\end{itemize}
\end{itemize}
\end{exampleblock}

\end{frame}

\begin{frame}{Interference fit}{Solution}

\begin{overprint}

\onslide<1|handout:1>
\vskip-20pt
\begin{exampleblock}{Question \#9: $\strain_{fd}$?}
\begin{itemize}
\item We seek for a displacement field of the form:
\begin{displaymath}
\uv_{fd}(\xv)=\uj_r(r)\iv_r(\theta)+\uj_\zj(\zj)\iv_\zj\,;
\end{displaymath}
\item Strain tensor in cylindrical coordinates:
\begin{displaymath}
\begin{split}
\mspace{-40mu}\strain_{fd}(\xv) &=\frac{\partial\uv_{fd}}{\partial r}\otimes_s\iv_r(\theta) + \frac{\partial\uv_{fd}}{\partial\theta}\otimes_s\frac{\iv_\theta(\theta)}{r}+\frac{\partial\uv_{fd}}{\partial\zj}\otimes_s\iv_\zj \\
\mspace{-40mu}&=\frac{\id\uj_r}{\id r}\iv_r(\theta)\otimes\iv_r(\theta) + \frac{\uj_r}{r}\iv_\theta(\theta)\otimes\iv_\theta(\theta) + \frac{\id\uj_\zj}{\id\zj}\iv_\zj\otimes\iv_\zj\,;
\end{split}
\end{displaymath}
\end{itemize}
\end{exampleblock}

\onslide<2|handout:2>
\vskip-20pt
\begin{exampleblock}{Question \#9: $\stress_{fd}$?}
\begin{itemize}
\item We seek for a displacement field of the form:
\begin{displaymath}
\uv_{fd}(\xv)=\uj_r(r)\iv_r(\theta)+\uj_\zj(\zj)\iv_\zj\,;
\end{displaymath}
\item From the constitutive equation:
\begin{displaymath}
\begin{split}
\mspace{-40mu}\stress_{fd} &=\lambda(\trace\strain_{fd})\Id+2\mu\strain_{fd} \\
\mspace{-40mu}&=\lambda\left(\frac{\id\uj_r}{\id r} + \frac{\uj_r}{r} + \frac{\id\uj_\zj}{\id\zj}\right)\Id \\
\mspace{-40mu}&\quad+2\mu\left[\frac{\id\uj_r}{\id r}\iv_r(\theta)\otimes\iv_r(\theta) + \frac{\uj_r}{r}\iv_\theta(\theta)\otimes\iv_\theta(\theta) + \frac{\id\uj_\zj}{\id\zj}\iv_\zj\otimes\iv_\zj\right]\,.
\end{split}
\end{displaymath}
\end{itemize}
\end{exampleblock}

\end{overprint}

\end{frame}

\begin{frame}{Interference fit}{Solution}

\begin{overprint}

\onslide<1|handout:1>
\vskip-20pt
\begin{exampleblock}{Question \#10: Differential equations for $\uj_r$ and $\uj_\zj$?}
\begin{itemize}
\item From the equilibrium equation $\Div\stress_{fd}=\bzero$;
\item In cylindrical coordinates:
\begin{displaymath}
\begin{split}
\Div\stress_{fd} &=\frac{\partial\stress_{fd}}{\partial r}\iv_r(\theta) + \frac{\partial\stress_{fd}}{\partial\theta}\frac{\iv_\theta(\theta)}{r}+\frac{\partial\stress_{fd}}{\partial\zj}\iv_\zj \\
 &=\lambda\left[\frac{\id}{\id r}\left(\frac{\id\uj_r}{\id r} + \frac{\uj_r}{r}\right)\iv_r(\theta)+ \frac{\id^2\uj_\zj}{\id\zj^2}\iv_\zj\right] \\
 &\quad +2\mu\left[\left(\frac{\id^2\uj_r}{\id r^2}+\frac{1}{r}\frac{\id\uj_r}{\id r}-\frac{\uj_r}{r^2}\right)\iv_r(\theta) + \frac{\id^2\uj_\zj}{\id\zj^2}\iv_\zj\right] \\
&=(\lambda+2\mu)\left[\frac{\id}{\id r}\left(\frac{\id\uj_r}{\id r} + \frac{\uj_r}{r}\right)\iv_r(\theta) + \frac{\id^2\uj_\zj}{\id\zj^2}\iv_\zj\right] \\
&=\bzero\,;
\end{split}
\end{displaymath}
\end{itemize}
\end{exampleblock}

\onslide<2|handout:2>
\vskip-20pt
\begin{exampleblock}{Question \#10: Differential equations for $\uj_r$ and $\uj_\zj$?}
\begin{itemize}
\item Consequently:
\begin{displaymath}
\begin{split}
\frac{\id}{\id r}\left(\frac{\id\uj_r}{\id r} + \frac{\uj_r}{r}\right) &=0\,, \\
\frac{\id^2\uj_\zj}{\id\zj^2} &=0\,.
\end{split}
\end{displaymath}
\end{itemize}
\end{exampleblock}

\end{overprint}

\end{frame}

\begin{frame}{Interference fit}{Solution}

\begin{overprint}

\onslide<1|handout:1>
\vskip-20pt
\begin{exampleblock}{Question \#11: $\uj_r$, $\uj_\zj$?}
\begin{itemize}
\item From question \#10:
\begin{displaymath}
\begin{split}
\frac{\id}{\id r}\left(\frac{1}{r}\frac{\id(r\uj_r)}{\id r}\right) &=0\,, \\
\frac{\id^2\uj_\zj}{\id\zj^2} &=0\,;
\end{split}
\end{displaymath}
\item Therefore $\uj_r(r)=Ar+\frac{B}{r}$, $\uj_\zj(\zj)=C\zj+D$, and:
\begin{displaymath}
\begin{split}
\mspace{-40mu}\stress_{fd} &=\lambda(2A+C)\Id +2\mu\left(A-\frac{B}{r^2}\right)\iv_r(\theta)\otimes\iv_r(\theta) \\
\mspace{-40mu} & \quad+ 2\mu \left(A+\frac{B}{r^2}\right)\iv_\theta(\theta)\otimes\iv_\theta(\theta)  + 2\mu C\iv_\zj\otimes\iv_\zj\,;
\end{split}
\end{displaymath}
\end{itemize}
\end{exampleblock}

\onslide<2|handout:2>
\vskip-20pt
\begin{exampleblock}{Question \#11: $\uj_r$, $\uj_\zj$?}
\begin{itemize}
\item Therefore:
\begin{displaymath}
\begin{split}
\mspace{-40mu}\stress_{fd} &=\lambda(2A+C)\Id +2\mu\left(A-\frac{B}{r^2}\right)\iv_r(\theta)\otimes\iv_r(\theta) \\
\mspace{-40mu} & \quad+ 2\mu \left(A+\frac{B}{r^2}\right)\iv_\theta(\theta)\otimes\iv_\theta(\theta)  + 2\mu C\iv_\zj\otimes\iv_\zj\,;
\end{split}
\end{displaymath}
\item From the boundary conditions:
\begin{displaymath}
\begin{split}
\mspace{-40mu}\stress(-\iv_r(\theta))|_{\{r=R_i\}}= P_0\iv_r(\theta) &= \scriptstyle\; -\left[\lambda(2A+C)+2\mu\left(A-\frac{B}{R_i^2}\right)\right]\iv_r(\theta)\,, \\
\mspace{-40mu}\stress\iv_r(\theta)|_{\{r=R_e\}} =\bzero &= \scriptstyle\; \left[\lambda(2A+C)+2\mu\left(A-\frac{B}{R_e^2}\right)\right]\iv_r(\theta)\,, \\
\mspace{-40mu}\stress\iv_\zj|_{\{\zj=0,H_\ihub\}} =\bzero &= \scriptstyle\; \left[\lambda(2A+C)+2\mu C\right]\iv_\zj\,; 
\end{split}
\end{displaymath}
\end{itemize}
\end{exampleblock}

\onslide<3|handout:3>
\vskip-20pt
\begin{exampleblock}{Question \#11: $\uj_r$, $\uj_\zj$?}
\begin{itemize}
\item We thus end up in the system:
\begin{displaymath}
\begin{split}
 2(\lambda+\mu)A-2\mu\frac{B}{R_i^2} +\lambda C &=-P_0\,, \\
2(\lambda+\mu)A-2\mu\frac{B}{R_e^2} +\lambda C &= 0\,, \\
2\lambda A + (\lambda+2\mu)C &=0\,; 
\end{split}
\end{displaymath}
\item This yields:
\begin{displaymath}
\begin{split}
A &=\frac{(\lambda+2\mu)P_0}{2\mu(3\lambda+2\mu)}\frac{R_i^2}{R_e^2-R_i^2}\,, \quad B = \frac{P_0}{2\mu}\frac{R_e^2R_i^2}{R_e^2-R_i^2}\,, \\
C &=-\frac{\lambda P_0}{\mu(3\lambda+2\mu)}\frac{R_i^2}{R_e^2-R_i^2}\,; 
\end{split}
\end{displaymath}
\end{itemize}
\end{exampleblock}

\onslide<4|handout:4>
\vskip-20pt
\begin{exampleblock}{Question \#11: $\uj_r$, $\uj_\zj$?}
\begin{itemize}
\item Since:
\begin{displaymath}
\lambda=\frac{\nu E}{(1+\nu)(1-2\nu)}\,,\quad\mu=\frac{E}{2(1+\nu)}\,,
\end{displaymath}
one also has:
\begin{displaymath}
\begin{split}
A &=(1-\nu)\frac{P_0}{E}\frac{R_i^2}{R_e^2-R_i^2}\,, \quad B = (1+\nu)\frac{P_0}{E}\frac{R_e^2R_i^2}{R_e^2-R_i^2}\,, \\
C &=-2\nu\frac{P_0}{E}\frac{R_i^2}{R_e^2-R_i^2}\,.
\end{split}
\end{displaymath}
\end{itemize}
\end{exampleblock}

\end{overprint}

\end{frame}

\begin{frame}{Interference fit}{Solution}

\begin{exampleblock}{Question \#12: $\stress_{fd}$?}
\begin{itemize}
\item From question \#11:
\begin{displaymath}
\begin{split}
\mspace{-40mu}\stress_{fd} &= \scriptstyle\; \left(2(\lambda+\mu)A-2\mu\frac{B}{r^2}+\lambda C\right)\iv_r(\theta)\otimes\iv_r(\theta) \\
\mspace{-40mu} & \scriptstyle\;\quad + \left(2(\lambda+\mu)A+2\mu\frac{B}{r^2}+\lambda C\right)\iv_\theta(\theta)\otimes\iv_\theta(\theta)  + \cancel{\left(2\lambda A+(\lambda+2\mu) C\right)}\iv_\zj\otimes\iv_\zj\,;
\end{split}
\end{displaymath}
\item Using the previous results for $A,B$ and $C$:
\begin{displaymath}
\begin{split}
\mspace{-40mu}\stress_{fd} &= \scriptstyle\; \frac{P_0 R_i^2}{R_e^2-R_i^2}\left[\left(1-\frac{R_e^2}{r^2}\right)\iv_r(\theta)\otimes\iv_r(\theta) + \left(1+\frac{R_e^2}{r^2}\right)\iv_\theta(\theta)\otimes\iv_\theta(\theta)\right]\,;
\end{split}
\end{displaymath}
\item The equilibrium equation and boundary conditions for $\medium_\ihub$ under internal pressure are for $\stress$ only and no kinematical unknown. Thus the constitutive equation has no impact. %On the other hand, $\strain_{fd}$ depends on $\lambda$ and $\mu$.
\end{itemize}
\end{exampleblock}

\end{frame}

\begin{frame}{Interference fit}{Solution}

\begin{center}
\textcolor{blue}{\LARGE {\bf Part 4: State after assembling}}
\end{center}
\begin{figure}[t]
\centering\includegraphics[scale=.4]{\figs/ch11-Assembling}
\end{figure}

\end{frame}

\begin{frame}{Interference fit}{Solution}

\begin{overprint}

\onslide<1|handout:1>
\vskip-20pt
\begin{exampleblock}{Question \#13: Equations for the assembly $\medium_\ihub\cup\medium_\ishaft$?}
\begin{itemize}
\item Equilibrium equation $\Div\stress^{\cany}+\fv_v=\roi\ddot{\uv}$ within the "infinitesimal deformation hypothesis" for $\iany=\ishaft,\ihub$;
\end{itemize}
\end{exampleblock}

\onslide<2|handout:2>
\vskip-20pt
\begin{exampleblock}{Question \#13: Equations for the assembly $\medium_\ihub\cup\medium_\ishaft$?}
\begin{itemize}
\item Equilibrium equation $\Div\stress^{\cany}+\fv_v=\roi\ddot{\uv}$ within the "infinitesimal deformation hypothesis" for $\iany=\ishaft,\ihub$;
\item "The effects of inertia and the action of gravity can be neglected," hence $\roi\ddot{\uv}=\bzero$, $\fv_v=\bzero$, and $\Div\stress^{\cany}=\bzero$ for $\iany=\ishaft,\ihub$;
\end{itemize}
\end{exampleblock}

\onslide<3|handout:3>
\vskip-20pt
\begin{exampleblock}{Question \#13: Equations for the assembly $\medium_\ihub\cup\medium_\ishaft$?}
\begin{itemize}
\item Equilibrium equation $\Div\stress^{\cany}=\bzero$ for $\iany=\ishaft,\ihub$;
\item Constitutive equation for a "linear elastic, isotropic, homogeneous" material for $\iany=\ishaft,\ihub$:
\begin{displaymath}
\begin{split}
\strain^{\cany} &=\frac{1+\nu}{E}\stress^{\cany}-\frac{\nu}{E}(\trace\stress^{\cany})\Id\,, \\
\stress^{\cany} &=\lambda(\trace\strain^{\cany})\Id+2\mu\strain^{\cany}\,;
\end{split}
\end{displaymath}
\end{itemize}
\end{exampleblock}

\onslide<4|handout:4>
\vskip-20pt
\begin{exampleblock}{Question \#13: Equations for the assembly $\medium_\ihub\cup\medium_\ishaft$?}
\begin{itemize}
\item Equilibrium equation $\Div\stress^{\cany}=\bzero$ for $\iany=\ishaft,\ihub$;
\item Constitutive equation for $\iany=\ishaft,\ihub$:
\begin{displaymath}
\begin{split}
\strain^{\cany} &=\frac{1+\nu}{E}\stress^{\cany}-\frac{\nu}{E}(\trace\stress^{\cany})\Id\,, \\
\stress^{\cany} &=\lambda(\trace\strain^{\cany})\Id+2\mu\strain^{\cany}\,;
\end{split}
\end{displaymath}
\item Initial conditions: useless for statics; 
\end{itemize}
\end{exampleblock}

\onslide<5|handout:5>
\vskip-20pt
\begin{exampleblock}{Question \#13: Equations for the assembly $\medium_\ihub\cup\medium_\ishaft$?}
\begin{itemize}
\item Equilibrium equation $\Div\stress^{\cany}=\bzero$ for $\iany=\ishaft,\ihub$;
\item Constitutive equation for $\iany=\ishaft,\ihub$:
\begin{displaymath}
\begin{split}
\strain^{\cany} &=\frac{1+\nu}{E}\stress^{\cany}-\frac{\nu}{E}(\trace\stress^{\cany})\Id\,, \\
\stress^{\cany} &=\lambda(\trace\strain^{\cany})\Id+2\mu\strain^{\cany},;
\end{split}
\end{displaymath}
\item Initial conditions: useless for statics; 
\item Boundary conditions: "the two parts are fastened,"
\begin{displaymath}
\begin{split}
\stress^{\cshaft}\iv_r(\theta)|_{\{r=R_e'\}} + \stress^{\chub}(-\iv_r(\theta))|_{\{r=R_i\}} &=\bzero\,, \\
\xv^{\cshaft}(R_e',\theta,\zj) -  \xv^{\chub}(R_i,\theta,\zj) &=\bzero\,; \\
\end{split}
\end{displaymath}
\end{itemize}
\end{exampleblock}

\onslide<6|handout:6>
\vskip-20pt
\begin{exampleblock}{Question \#13: Equations for the assembly $\medium_\ihub\cup\medium_\ishaft$?}
\begin{itemize}
\item Equilibrium equation $\Div\stress^{\cany}=\bzero$ for $\iany=\ishaft,\ihub$;
\item Constitutive equation for $\iany=\ishaft,\ihub$:
\begin{displaymath}
\begin{split}
\strain^{\cany} &=\frac{1+\nu}{E}\stress^{\cany}-\frac{\nu}{E}(\trace\stress^{\cany})\Id\,, \\
\stress^{\cany} &=\lambda(\trace\strain^{\cany})\Id+2\mu\strain^{\cany}\,;
\end{split}
\end{displaymath}
\item Initial conditions: useless for statics; 
\item Boundary conditions on the free surfaces for $\iany=\ishaft,\ihub$:
\begin{displaymath}
\begin{split}
\stress^{\cany}(-\iv_\zj)|_{\{\zj=0\}} &=\bzero\,, \\
\stress^{\cany}\iv_\zj|_{\{\zj=H_s\}} &=\bzero\,, \\
\stress^{\chub}\iv_r(\theta)|_{\{r=R_e\}} &=\bzero\,;
\end{split}
\end{displaymath}
\end{itemize}
\end{exampleblock}

\onslide<7|handout:7>
\vskip-20pt
\begin{exampleblock}{Question \#13: Equations for the assembly $\medium_\ihub\cup\medium_\ishaft$?}
\begin{itemize}
\item Recap for $\iany=\ishaft,\ihub$:
\begin{displaymath}
\Div\stress^{\cany}=\bzero \,,
\end{displaymath}
\begin{displaymath}
\begin{split}
\strain^{\cany} &=\frac{1+\nu}{E}\stress^{\cany}-\frac{\nu}{E}(\trace\stress^{\cany})\Id\,, \\
\stress^{\cany} &=\lambda(\trace\strain^{\cany})\Id+2\mu\strain^{\cany}\,, \\
\end{split}
\end{displaymath}
\begin{displaymath}
\begin{split}
\stress^{\cshaft}\iv_r(\theta)|_{\{r=R_e'\}} &= \stress^{\chub}\iv_r(\theta)|_{\{r=R_i\}}\,, \\
\xv^{\cshaft}(R_e',\theta,\zj) &=  \xv^{\chub}(R_i,\theta,\zj)\,, \\
\stress^{\cany}\iv_\zj|_{\{\zj=0\}} &=\bzero\,, \\
\stress^{\cany}\iv_\zj|_{\{\zj=H_s\}} &=\bzero\,, \\
\stress^{\chub}\iv_r(\theta)|_{\{r=R_e\}} &=\bzero\,.
\end{split}
\end{displaymath}
\end{itemize}
\end{exampleblock}

\end{overprint}

\end{frame}

\begin{frame}{Interference fit}{Solution}

\vskip-20pt
\begin{exampleblock}{Question \#14: $\strain^{\cshaft}$, $\strain^{\chub}$?}
\begin{itemize}
\item From question \#6 for the shaft:
\begin{displaymath}
\begin{split}
\mspace{-40mu} \strain^{\cshaft}=\strain_{fc} &=\scriptstyle\; \frac{P_0}{E}\left[(\nu-1)\left(\iv_r(\theta)\otimes\iv_r(\theta)+\iv_\theta(\theta)\otimes\iv_\theta(\theta)\right)+2\nu\iv_\zj\otimes\iv_\zj\right] \\
\mspace{-40mu} &= \scriptstyle\; \frac{P_0}{E}\left[(\nu-1)\Id+(\nu+1)\iv_\zj\otimes\iv_\zj\right]\,;
\end{split}
\end{displaymath}
\item From question \#9 for the hub:
\begin{displaymath}
\begin{split}
\mspace{-40mu} \strain^{\chub} =\strain_{fd} &= \scriptstyle\; \frac{\id\uj_r}{\id r}\iv_r(\theta)\otimes\iv_r(\theta) + \frac{\uj_r}{r}\iv_\theta(\theta)\otimes\iv_\theta(\theta) + \frac{\id\uj_\zj}{\id\zj}\iv_\zj\otimes\iv_\zj \\
\mspace{-40mu} &= \scriptstyle\; \left(A-\frac{B}{r^2}\right)\iv_r(\theta)\otimes\iv_r(\theta) + \left(A+\frac{B}{r^2}\right)\iv_\theta(\theta)\otimes\iv_\theta(\theta) + C\iv_\zj\otimes\iv_\zj \\
%\mspace{-40mu} &= \scriptstyle\; \frac{P_0}{E}\frac{R_i^2}{R_e^2-R_i^2}\left[\left(1-\nu-(1+\nu)\frac{R_e^2}{r^2}\right)\iv_r(\theta)\otimes\iv_r(\theta) \right. \\
%\mspace{-40mu} & \scriptstyle\; \quad\left.+ \left(1-\nu+(1+\nu)\frac{R_e^2}{r^2}\right)\iv_\theta(\theta)\otimes\iv_\theta(\theta) -2\nu\iv_\zj\otimes\iv_\zj\right] \\
\mspace{-40mu} &= \scriptstyle\; -\frac{P_0}{E}\frac{R_i^2}{R_e^2-R_i^2}\big[(\nu-1)\Id +(\nu+1)\iv_\zj\otimes\iv_\zj  \\
\mspace{-40mu} & \scriptstyle\;\quad\quad\quad +(1+\nu)\frac{R_e^2}{r^2}\left(\iv_r(\theta)\otimes\iv_r(\theta)-\iv_\theta(\theta)\otimes\iv_\theta(\theta)\right)\big]\,;
\end{split}
\end{displaymath}
\end{itemize}
\end{exampleblock}

\end{frame}

\begin{frame}{Interference fit}{Solution}

\begin{overprint}

\onslide<1|handout:1>
\vskip-20pt
\begin{exampleblock}{Question \#15: Inner radius of the deformed shaft?}
\begin{itemize}
\item From question \#13 and the displacement boundary condition:
\begin{displaymath}
\begin{split}
\scal{\xv^{\cshaft}(R_e',\theta,\zj),\iv_r(\theta)} &= \scal{\xv^{\chub}(R_i,\theta,\zj),\iv_r(\theta)} \\
\text{or}\quad R_e'+\uj_r^{\cshaft}(R_e') &= R_i + \uj_r^{\chub}(R_i)\,;
\end{split}
\end{displaymath}
\end{itemize}
\end{exampleblock}

\onslide<2|handout:2>
\vskip-20pt
\begin{exampleblock}{Question \#15: Inner radius of the deformed shaft?}
\begin{itemize}
\item From question \#13 and the displacement boundary condition:
\begin{displaymath}
R_e'+\uj_r^{\cshaft}(R_e') = R_i + \uj_r^{\chub}(R_i)\,;
\end{displaymath}
\item From question \#6 for the shaft $\uj_r^{\cshaft}(r)=\frac{P_0}{E}(\nu-1)r$;
\end{itemize}
\end{exampleblock}

\onslide<3|handout:3>
\vskip-20pt
\begin{exampleblock}{Question \#15: Inner radius of the deformed shaft?}
\begin{itemize}
\item From question \#13 and the displacement boundary condition:
\begin{displaymath}
R_e'+\uj_r^{\cshaft}(R_e') = R_i + \uj_r^{\chub}(R_i)\,;
\end{displaymath}
\item From question \#6 for the shaft $\uj_r^{\cshaft}(r)=\frac{P_0}{E}(\nu-1)r$;
\item From question \#11 for the hub $\uj_r^{\chub}(r)=Ar+\frac{B}{r}$;
\end{itemize}
\end{exampleblock}

\onslide<4|handout:4>
\vskip-20pt
\begin{exampleblock}{Question \#15: Inner radius of the deformed shaft?}
\begin{itemize}
\item From question \#13 and the displacement boundary condition:
\begin{displaymath}
R_e'+\uj_r^{\cshaft}(R_e') = R_i + \uj_r^{\chub}(R_i)\,;
\end{displaymath}
\item From question \#6 for the shaft $\uj_r^{\cshaft}(r)=\frac{P_0}{E}(\nu-1)r$;
\item From question \#11 for the hub $\uj_r^{\chub}(r)=Ar+\frac{B}{r}$;
\item Therefore:
\begin{displaymath}
\begin{split}
R_e'+ \frac{P_0}{E}(\nu-1)R_e' &= R_i +  A R_i +\frac{B}{R_i} \\
\scriptstyle{\left[1+(\nu-1)\frac{P_0}{E}\right]R_e'} \; &= \scriptstyle\;\left[1+\frac{P_0}{E}\frac{R_i^2}{R_e^2-R_i^2}\left(1-\nu+(1+\nu)\frac{R_e^2}{R_i^2}\right)\right]R_i\,.
\end{split}
\end{displaymath}
\end{itemize}
\end{exampleblock}

\end{overprint}

\end{frame}

\begin{frame}{Interference fit}{Solution}

\vskip-20pt
\begin{exampleblock}{Question \#16: $P_0$?}
\begin{itemize}
\item From question \#15:
\begin{displaymath}
\begin{split}
\scriptstyle{\left[1+(\nu-1)\frac{P_0}{E}\right]\left(R_i+\frac{s}{2}\right)} \; &= \scriptstyle\;\left[1+\frac{P_0}{E}\frac{R_i^2}{R_e^2-R_i^2}\left(1-\nu+(1+\nu)\frac{R_e^2}{R_i^2}\right)\right]R_i \\
\scriptstyle{\left[1+(\nu-1)\frac{P_0}{E}\right]\frac{s}{2}} \; &= \scriptstyle\;\frac{2P_0}{E}\frac{R_e^2}{R_e^2-R_i^2}R_i \\
\scriptstyle{\frac{s}{2}} &= \scriptstyle\;\frac{P_0}{E}\left[\frac{2R_e^2}{R_e^2-R_i^2}R_i+(1-\nu)\frac{s}{2}\right]\,,
\end{split}
\end{displaymath}
or: 
\begin{displaymath}
P_0\simeq\frac{Es}{4R_i}\left(1-\frac{R_i^2}{R_e^2}\right)\quad\text{QED}\,.
\end{displaymath}
\end{itemize}
\end{exampleblock}

\end{frame}

\begin{frame}{Interference fit}{Solution}

\begin{overprint}

\onslide<1|handout:1>
\vskip-20pt
\begin{exampleblock}{Question \#17: Tresca's criterion?}
\begin{itemize}
\item From question \#5 for the shaft:
\begin{displaymath}
\stress^{\cshaft}=\stress_{fc} =-P_0\left(\iv_r(\theta)\otimes\iv_r(\theta)+\iv_\theta(\theta)\otimes\iv_\theta(\theta)\right)\,;
\end{displaymath}
\item From question \#12 for the hub:
\begin{displaymath}
\stress^{\chub} =\stress_{fd} = \scriptstyle\; \frac{P_0 R_i^2}{R_e^2-R_i^2}\left[\left(1-\frac{R_e^2}{r^2}\right)\iv_r(\theta)\otimes\iv_r(\theta) + \left(1+\frac{R_e^2}{r^2}\right)\iv_\theta(\theta)\otimes\iv_\theta(\theta)\right]\,;
\end{displaymath}
\end{itemize}
\end{exampleblock}

\onslide<2|handout:2>
\vskip-20pt
\begin{exampleblock}{Question \#17: Tresca's criterion?}
\begin{itemize}
\item From question \#5 (shaft) and question \#12 (hub):
\begin{displaymath}
\begin{split}
\stress^{\cshaft}=\stress_{fc}  &=-P_0\left(\iv_r(\theta)\otimes\iv_r(\theta)+\iv_\theta(\theta)\otimes\iv_\theta(\theta)\right)\,, \\
\stress^{\chub} =\stress_{fd} &= \scriptstyle\; \frac{P_0 R_i^2}{R_e^2-R_i^2}\left[\left(1-\frac{R_e^2}{r^2}\right)\iv_r(\theta)\otimes\iv_r(\theta) + \left(1+\frac{R_e^2}{r^2}\right)\iv_\theta(\theta)\otimes\iv_\theta(\theta)\right]\,;
\end{split}
\end{displaymath}
\item Tresca' criterion: $\tressj_\text{eq}\leq\frac{\stressj_0}{2}$ where
\begin{displaymath}
\tressj_\text{eq}=\demi\max(\abs{\stressj_1-\stressj_2},\abs{\stressj_2-\stressj_3},\abs{\stressj_3-\stressj_1})\,;
\end{displaymath}
\end{itemize}
\end{exampleblock}

\onslide<3|handout:3>
\vskip-20pt
\begin{exampleblock}{Question \#17: Tresca's criterion?}
\begin{itemize}
\item From question \#5 (shaft) and question \#12 (hub):
\begin{displaymath}
\begin{split}
\stress^{\cshaft}=\stress_{fc}  &=-P_0\left(\iv_r(\theta)\otimes\iv_r(\theta)+\iv_\theta(\theta)\otimes\iv_\theta(\theta)\right)\,, \\
\stress^{\chub} =\stress_{fd} &= \scriptstyle\; \frac{P_0 R_i^2}{R_e^2-R_i^2}\left[\left(1-\frac{R_e^2}{r^2}\right)\iv_r(\theta)\otimes\iv_r(\theta) + \left(1+\frac{R_e^2}{r^2}\right)\iv_\theta(\theta)\otimes\iv_\theta(\theta)\right]\,;
\end{split}
\end{displaymath}
\item Tresca' criterion: $\tressj_\text{eq}\leq\frac{\stressj_0}{2}$ where
\begin{displaymath}
\tressj_\text{eq}=\demi\max(\abs{\stressj_1-\stressj_2},\abs{\stressj_2-\stressj_3},\abs{\stressj_3-\stressj_1})\,;
\end{displaymath}
\item Here $\tressj_\text{eq}=\demi\max\abs{\smash{\stressj_{rr}^{\chub}-\stressj_{\theta\theta}^{\chub}}}$ reached for $r=R_i$, or:
\begin{displaymath}
\tressj_\text{eq}= \frac{P_0 R_e^2}{R_e^2-R_i^2}=\frac{Es}{4R_i}=100\,\text{MPa}>\frac{\stressj_0}{2}\,!!!
\end{displaymath}
\end{itemize}
\end{exampleblock}

\end{overprint}

\end{frame}

\end{document}

