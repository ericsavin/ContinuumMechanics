\begin{frame}{Recap}

\begin{itemize}
\item Local equilibrium equation:
\begin{displaymath}
\Divx\stress+\fv_v=\roi\ddot{\uv}\,;
\end{displaymath}
\item Small strains assumption:
\begin{displaymath}
\strain=\demi(\Gradx\uv+\Gradx\uv^\itr)\,;
\end{displaymath}
\item Material constitutive equation:
\begin{displaymath}
\stress=\tenselas\strain
\end{displaymath}
($\stress=\lambda\trace(\strain)\Id+2\mu\strain$ for isotropic elasticity).
\end{itemize}

\end{frame}

\begin{frame}{Navier equation}{Elastic waves}

\begin{overprint}

\onslide<1|handout:1>
\begin{itemize}
\item $\stress_\xv(\uv)=\tenselas(\strain_\xv(\uv))$, $\strain_\xv(\uv)=\gradx\otimes_s\uv$, and:
\begin{displaymath}
\begin{array}{c}
\Divx\stress_\xv(\uv)+\fv_v=\roi\partial_t^2\uv \\
(\lambda+\mu)\gradx(\divx\uv)+\mu{\boldsymbol\Delta}_\xv\uv+\fv=\roi\partial_t^2\uv \\
(\lambda+\mu)\sum_{k=1}^3\frac{\partial^2\uj_k}{\partial\xj_j\partial\xj_k}+\mu\sum_{k=1}^3\frac{\partial^2\uj_j}{\partial\xj_k^2}+\fj_{vj}=\roi\frac{\partial^2\uj_j}{\partial t^2}
\end{array}
\end{displaymath}
%where ${\boldsymbol\Delta}_\xv=\Divx(\Gradx)$ (vector Laplacian).
\item Body waves {\small(Poisson 1828)}
\begin{figure}
\centering\includegraphics[scale=.3]{\figs/ch5-BodyWaves}
\end{figure}
\end{itemize}

\onslide<2|handout:2>
\begin{itemize}
\item $\stress_\xv(\uv)=\tenselas(\strain_\xv(\uv))$, $\strain_\xv(\uv)=\gradx\otimes_s\uv$, and:
\begin{displaymath}
\begin{array}{c}
\Divx\stress_\xv(\uv)+\fv_v=\roi\partial_t^2\uv \\
(\lambda+\mu)\gradx(\divx\uv)+\mu{\boldsymbol\Delta}_\xv\uv+\fv=\roi\partial_t^2\uv \\
(\lambda+\mu)\sum_{k=1}^3\frac{\partial^2\uj_k}{\partial\xj_j\partial\xj_k}+\mu\sum_{k=1}^3\frac{\partial^2\uj_j}{\partial\xj_k^2}+\fj_{vj}=\roi\frac{\partial^2\uj_j}{\partial t^2}
\end{array}
\end{displaymath}
%where ${\boldsymbol\Delta}_\xv=\Divx(\Gradx)$ (vector Laplacian).
\item Surface waves {\small(Rayleigh 1885, Love 1911, Stoneley 1924)}
\begin{figure}
\centering\includegraphics[scale=.3]{\figs/ch5-SurfaceWaves}
\end{figure}
\end{itemize}

\onslide<3|handout:3>
\begin{columns}[t]
\column{.4\textwidth}
\centering\includegraphics[scale=0.3]{\figs/ch5-Navier-portrait}
\column{.6\textwidth}
\vskip-130pt
Claude Louis Marie Henri Navier [1785-1836]
\begin{itemize}
\item X 1802
\item Ing\'enieur Ponts-et-Chauss\'ees
\item Acad\'emie des Sciences 1824
\item Prof. Analyse et M\'ecanique \`a l'X 1831-1836
\item \emph{Le Cur� de village} (1841)
\end{itemize}
\end{columns}
\href{https://mathshistory.st-andrews.ac.uk/Biographies/Navier/}{\tiny\texttt{https://mathshistory.st-andrews.ac.uk/Biographies/Navier/}}

\end{overprint}

\end{frame}

\begin{frame}{Navier equation}{Initial conditions}

\begin{itemize}
\item Initial displacement:
\begin{displaymath}
\uv(\xv,0)=\uv_0(\xv)\,,\quad\forall\xv\in\medium\,;
\end{displaymath}
\item Initial velocity:
\begin{displaymath}
\frac{\partial\uv}{\partial t}(\xv,0)=\vv_0(\xv)\,,\quad\forall\xv\in\medium\,.
\end{displaymath}
\end{itemize}

\end{frame}

\begin{frame}{Navier equation}{Boundary conditions}

%\begin{columns}[t]
%\column{.33\textwidth}
%\centering\includegraphics[scale=.25]{\figs/ch4-Isotropic}
%\column{.33\textwidth}
%\centering\includegraphics[scale=.25]{\figs/ch4-Anisotropic1}
%\column{.33\textwidth}
%\centering\includegraphics[scale=.25]{\figs/ch4-Anisotropic2}
%\end{columns}
\begin{itemize}
\item Rigid contact:
\begin{displaymath}
\uv(\xv,t)=\uv_S(\xv,t)\,,\quad\forall\xv\in\partial\medium_\uj\,,\forall t\,;
\end{displaymath}
\item Soft contact:
\begin{displaymath}
\stress(\xv,t)\nv(\xv)=\fv_S(\xv,t)\,,\quad\forall\xv\in\partial\medium_\stressj\,,\forall t\,;
\end{displaymath}
\item Internal full contact with \emph{e.g.} $\nv=\nv_1=-\nv_2$:
\begin{displaymath}
\begin{array}{c}
\uv_1(\xv,t)=\uv_2(\xv,t)\,, \\
\stress_1(\xv,t)\nv(\xv)=\stress_2(\xv,t)\nv(\xv)\,,
\end{array}
\end{displaymath}
$\forall\xv\in\partial\medium_1\cap\partial\medium_2$, $\forall t$;
\item Internal sliding contact without friction:
\begin{displaymath}
\begin{array}{c}
\scal{\uv_1(\xv,t),\nv(\xv)}=\scal{\uv_2(\xv,t),\nv(\xv)}\,, \\
\scal{\stress_1(\xv,t)\nv(\xv),{\boldsymbol\tau}}=\scal{\stress_2(\xv,t)\nv(\xv),{\boldsymbol\tau}}=\bzero\,,
\end{array}
\end{displaymath}
$\forall{\boldsymbol\tau}\perp\nv(\xv)$, $\forall\xv\in\partial\medium_1\cap\partial\medium_2$, $\forall t$.
\end{itemize}

\end{frame}

\begin{frame}{Navier equation}{Properties}

\begin{itemize}
\item \emphb{Existence} of a solution:
\begin{itemize}
\item Holds in statics provided that global equilibrium is satisfied;
\item Holds in dynamics.
\end{itemize}
\item \emphb{Uniqueness} of a solution:
\begin{itemize}
\item Holds in statics for strains, stresses, and displacements up to a rigid-body motion;
\item Holds in dynamics for strains, stresses, and displacements.
\end{itemize}
\item \emphb{Linearity} (superposition principle, symmetries...).
\end{itemize}

\end{frame}

\begin{frame}{Saint-Venant's principle}{...in homogeneous, linear elastic media}

\begin{overprint}

\onslide<1|handout:1>
\begin{figure}
\centering\includegraphics[scale=.2]{\figs/ch5-StVenant}
\end{figure}
\vskip-10pt
\begin{quote}
\item "...the difference between the effects of two different but statically equivalent loads becomes very small at sufficiently large distances from load."\footnote{\tiny A. J. C. B. Saint-Venant, M\'emoire sur la Torsion des Prismes, \emph{Mem. Divers Savants} {\bf 14}, 233-560 (1855).}
\end{quote}
\scriptsize{\par\hfill{Adh\'emar Barr\'e de Saint-Venant [1797-1886]}}

\onslide<2|handout:2>
\begin{columns}[t]
\column{.4\textwidth}
\centering\includegraphics[scale=.5]{\figs/ch5-StVenant-portrait}
\column{.6\textwidth}
\vskip-80pt
Adh\'emar Jean Claude Barr\'e de Saint-Venant [1797-1886]
\begin{itemize}
\item X 1813
\item Ing\'enieur Ponts-et-Chauss\'ees
\item Acad\'emie des Sciences 1868
\end{itemize}
\end{columns}
\href{https://mathshistory.st-andrews.ac.uk/Biographies/Saint-Venant/}{\tiny\texttt{https://mathshistory.st-andrews.ac.uk/Biographies/Saint-Venant/}}

\end{overprint}

\end{frame}

\begin{frame}{Thermoelasticity}{Isotropic case}

\begin{itemize}
\item Thermal strains in the isotropic case with temperature gradient $\Delta T$:
\begin{displaymath}
\strain_\text{th}=\alpha\Delta T\Id
\end{displaymath}
where $\alpha$ is the coefficient of linear thermal expansion.
\item Total strain tensor:
\begin{displaymath}
\begin{split}
\strain &=\strain_\text{elas}+\strain_\text{th} \\
&=\tenscomp\stress+\alpha\Delta T\Id\,.
\end{split}
\end{displaymath}
\item Linear thermoelastic constitutive equation:
\begin{displaymath}
\boxed{\stress=\tenselas(\strain-\alpha\Delta T\Id)}\,.
\end{displaymath}
\end{itemize}

\end{frame}
