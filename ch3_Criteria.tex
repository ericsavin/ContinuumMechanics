% MG3 TD #3: Design of a gravity dam
% V1.0 January 2021
% $Header: /cvsroot/latex-beamer/latex-beamer/solutions/generic-talks/generic-ornate-15min-45min.en.tex,v 1.5 2007/01/28 20:48:23 tantau Exp $
\def\webDOI{http://dx.doi.org}
\def\Folder{/Users/ericsavin/Documents/Cours/SG3-MMC/SLIDES_TD/}
\def\Year{\Folder/2020-2021}
\def\Sections{\Year/SECTIONS}
\def\figs{\Folder/FIGS}
%\def\figs{/Users/ericsavin/Documents//Cours/DynSto/Figs}
%\def\figs{/Users/ericsavin/Documents/Cours/SG3-MMC/SLIDES_TD/FIGS}
\def\figdynsto{/Users/ericsavin/Documents//Figures/DYNSTO}
\def\symb{/Users/ericsavin/Documents/Latex/SYMBOL}
\def\fonts{/Users/ericsavin/Documents/Latex/FONTS}
\def\logos{/Users/ericsavin/Documents/Latex/LOGOS}
\def\Onera{ONERA}
\def\ECP{CentraleSup\'elec}


\documentclass{beamer}

% This file is a solution template for:

% - Giving a talk on some subject.
% - The talk is between 15min and 45min long.
% - Style is ornate.



% Copyright 2004 by Till Tantau <tantau@users.sourceforge.net>.
%
% In principle, this file can be redistributed and/or modified under
% the terms of the GNU Public License, version 2.
%
% However, this file is supposed to be a template to be modified
% for your own needs. For this reason, if you use this file as a
% template and not specifically distribute it as part of a another
% package/program, I grant the extra permission to freely copy and
% modify this file as you see fit and even to delete this copyright
% notice. 


\mode<presentation>
{
  \usetheme{Berkeley}
  % or ...

  \setbeamercovered{transparent}
  % or whatever (possibly just delete it)
}


\usepackage[english]{babel}
% or whatever

\usepackage[latin1]{inputenc}
% or whatever

%\usepackage{mathtime}
\usefonttheme{serif}
%\usefonttheme{professionalfonts}
\usepackage{amsfonts}
\usepackage{amssymb}

\usepackage{amsmath}
\usepackage{multimedia}
\usepackage{mathrsfs}
\usepackage{mathabx}
\usepackage{color}
\usepackage{pstricks}
\usepackage{graphicx}
%\usepackage[pdftex, pdfborderstyle={/S/U/W 1}]{hyperref}
\usepackage{hyperref}
\usepackage{bbm}
\usepackage{cancel}
\usepackage[Symbol]{upgreek}
%\usepackage{mathbbol}
%\DeclareSymbolFontAlphabet{\amsmathbb}{AMSb}
%\usepackage[bbgreekl]{mathbbol}
%\usepackage[mtpbbi]{mtpro2}

%\usepackage[svgnames]{xcolor}

%\input{\fonts/math0}
%\input{\symb/structac} % Notations E. Savin
%\input{\symb/logos}

\newcommand{\ci}{\mathrm{i}}
\newcommand{\trace}{\operatorname{Tr}}
\newcommand{\Nset}{\mathbb{N}}
\newcommand{\Zset}{\mathbb{Z}}
\newcommand{\Rset}{\mathbb{R}}
\newcommand{\Cset}{\mathbb{C}}
\newcommand{\Sset}{\mathbb{S}}
\newcommand{\Mset}{\mathbb{M}}
\newcommand{\PhaseSpace}{\Omega}
\newcommand{\ContSet}{{\mathcal C}}
\newcommand{\id}{d}
\newcommand{\iD}{\mathrm{D}}
\newcommand{\iexp}{\mathrm{e}}
\newcommand{\demi}{\frac{1}{2}}
\newcommand{\imply}{\Rightarrow}

% Algebra
\newcommand{\itr}{{\sf T}}
\newcommand{\Id}{{\boldsymbol I}}
\newcommand{\IId}{\mathbb{I}}
\newcommand{\aj}{a}
\newcommand{\bj}{b}
\newcommand{\cj}{c}
\renewcommand{\dj}{d}
\newcommand{\av}{{\boldsymbol\aj}}
\newcommand{\bv}{{\boldsymbol\bj}}
\newcommand{\cv}{{\boldsymbol\cj}}
\newcommand{\dv}{{\boldsymbol\dj}}
\newcommand{\uj}{u}
\newcommand{\vj}{v}
\newcommand{\xj}{x}
\newcommand{\yj}{y}
\newcommand{\zj}{z}
\newcommand{\uv}{{\boldsymbol\uj}}
\newcommand{\vv}{{\boldsymbol\vj}}
\newcommand{\xv}{{\boldsymbol\xj}}
\newcommand{\yv}{{\boldsymbol\yj}}
\newcommand{\zv}{{\boldsymbol\zj}}
\newcommand{\Aj}{A}
\newcommand{\Bj}{B}
\newcommand{\Av}{{\boldsymbol\Aj}}
\newcommand{\Bv}{{\boldsymbol\Bj}}
\newcommand{\Zgv}{{\boldsymbol Z}}

% Analysis
\newcommand{\grad}{{\boldsymbol\nabla}}
\newcommand{\gradx}{{\grad_\xv}}
\newcommand{\Grad}{{\mathbb D}}
\newcommand{\Gradx}{{\Grad_\xv}}
\renewcommand{\div}{\mathrm{div}}
\newcommand{\divx}{{\div_\xv}}
\newcommand{\Div}{\mathbf{Div}}
\newcommand{\Divx}{{\Div_\xv}}

% Kinematics
\newcommand{\ej}{e}
\renewcommand{\ij}{i}
\newcommand{\pj}{p}
\newcommand{\ev}{{\boldsymbol\ej}}
\newcommand{\iv}{{\boldsymbol\ij}}
\newcommand{\pv}{{\boldsymbol\pj}}
\newcommand{\posij}{f}
\newcommand{\posiv}{{\boldsymbol\posij}}
\newcommand{\iposiv}{{\boldsymbol g}}
\newcommand{\Fp}{{\mathbb F}}
\newcommand{\GreenLj}{E}
\newcommand{\GreenL}{{\mathbb\GreenLj}}
\newcommand{\medium}{\Omega}
\newcommand{\strainj}{\varepsilon}
\newcommand*{\strain}{\mbox{$\hspace{0.2em}\rotatebox[x=0pt,y=0.2pt]{90}{\rule{0.02\linewidth}{0.4pt}}\hspace{-0.23em}\upvarepsilon$}} 

% Dynamics
\newcommand{\fj}{f}
\newcommand{\Fj}{F}
\newcommand{\mj}{m}
\newcommand{\nj}{n}
\newcommand{\Tj}{T}
\newcommand{\fv}{{\boldsymbol\fj}}
\newcommand{\Fv}{{\boldsymbol\Fj}}
\newcommand{\mv}{{\boldsymbol\mj}}
\newcommand{\nv}{{\boldsymbol\nj}}
\newcommand{\Tv}{{\boldsymbol\Tj}}
\newcommand{\roi}{\varrho}
\newcommand*{\stressj}{\sigma}
\newcommand*{\tressj}{\tau}
\newcommand{\stress}{\mbox{$\hspace{0.3em}\rotatebox[x=0pt,y=0.2pt]{90}{\rule{0.017\linewidth}{0.4pt}}\hspace{-0.25em}\upsigma$}}
\newcommand*{\tress}{{\boldsymbol\tressj}}
\newcommand{\acj}{a}
\newcommand{\acv}{{\boldsymbol\acj}}

%\newcommand{\Zg}{{\bf\zgj}}
%\newcommand{\xigj}{\xi}
%\newcommand{\xig}{{\boldsymbol\xigj}}
\newcommand{\kgj}{k}
%\newcommand{\kgh}{\kgj_\ygj}
%\newcommand{\kg}{{\bf\kgj}}
\newcommand{\Kg}{{\bf K}}
%\newcommand{\qg}{{\boldsymbol q}}
%\newcommand{\pg}{{\boldsymbol p}}
\newcommand{\hkg}{{\hat \kg}}
\newcommand{\hpg}{\hat{\pg}}
%\newcommand{\vg}{{\boldsymbol v}}
\newcommand{\sg}{{\boldsymbol s}}
%\newcommand{\stress}{\mathbb{\sigma}}
\newcommand{\tenselas}{{\rm {\large C}}}
\newcommand{\speci}{{\mathrm w}}
\newcommand{\specij}{{\mathrm W}}
\newcommand{\speciv}{{\bf \specij}}
%\newcommand{\cjg}[1]{\overline{#1}}
\newcommand{\eigv}{{\bf b}}
\newcommand{\eigw}{{\bf c}}
\newcommand{\eigl}{\lambda}
\newcommand{\jeig}{\alpha}
\newcommand{\keig}{\beta}
\newcommand{\cel}{c}
\newcommand{\bcel}{{\bf\cel}}
\newcommand{\deng}{{\mathcal E}}
\newcommand{\flowj}{\pi}
\newcommand{\flow}{\boldsymbol\flowj}
\newcommand{\Flowj}{\Pi}
\newcommand{\Flow}{\boldsymbol\Flowj}
\newcommand{\fluxinj}{g}
\newcommand{\fluxin}{{\bf\fluxinj}}
\newcommand{\dscat}{\sigma}
\newcommand{\tdscat}{\Sigma}
\newcommand{\collop}{{\mathcal Q}}
\newcommand{\epsd}{\delta}
\newcommand{\rscat}{\rho}
\newcommand{\tscat}{\tau}
\newcommand{\Rscat}{\mathcal{R}}
\newcommand{\Tscat}{\mathcal{T}}
\newcommand{\lscat}{\ell}
%\newcommand{\floss}{\eta}
\newcommand{\mdiff}{{\bf D}}
%\newcommand{\demi}{\frac{1}{2}}
\newcommand{\domain}{{\mathcal O}}
\newcommand{\bdomain}{{\mathcal D}}
\newcommand{\interface}{\Gamma}
\newcommand{\sinterface}{\gamma_D}
%\newcommand{\normal}{\hat{\bf n}}
\newcommand{\bnabla}{\boldsymbol\nabla}
%\newcommand{\esp}[1]{\mathbb{E}\{\smash{#1}\}}
\newcommand{\mean}[1]{\underline{#1}}
%\newcommand{\BB}{\mathbb{B}}
%\newcommand{\II}{{\boldsymbol I}}
%\newcommand{\TA}{\boldsymbol{\Gamma}}
\newcommand{\Mdisp}{{\mathbf H}}
\newcommand{\Hamil}{{\mathcal H}}
\newcommand{\bzero}{{\bf 0}}

\newcommand{\mass}{M}
\newcommand{\damp}{D}
\newcommand{\stif}{K}
\newcommand{\dsp}{S}
\newcommand{\dof}{q}
\newcommand{\pof}{p}
%\newcommand{\MM}{{\boldsymbol\mass}}
\newcommand{\MD}{{\boldsymbol\damp}}
\newcommand{\MK}{{\boldsymbol\stif}}
\newcommand{\MS}{{\boldsymbol\dsp}}
\newcommand{\Cov}{{\boldsymbol C}}
\newcommand{\dofg}{{\boldsymbol\dof}}
\newcommand{\pofg}{{\boldsymbol\pof}}
\newcommand{\driftj}{b}
\newcommand{\drifts}{{\boldsymbol \driftj}}
\newcommand{\drift}{{\underline\drifts}}
\newcommand{\scatj}{a}
\newcommand{\scat}{{\boldsymbol\scatj}}
\newcommand{\diff}{{\boldsymbol\sigma}}
\newcommand{\load}{F}
\newcommand{\loadg}{{\boldsymbol\load}}
\newcommand{\pdf}{\pi}
\newcommand{\tpdf}{\pdf_t}
\newcommand{\fg}{{\boldsymbol f}}
%\newcommand{\Ugj}{U}
\newcommand{\Vgj}{V}
\newcommand{\Xgj}{X}
\newcommand{\Ygj}{Y}
%\newcommand{\Ug}{{\boldsymbol\Ugj}}
\newcommand{\Vg}{{\boldsymbol\Vgj}}
%\newcommand{\Qg}{{\boldsymbol Q}}
\newcommand{\Pg}{{\boldsymbol P}}
\newcommand{\Xg}{{\boldsymbol\Xgj}}
\newcommand{\Yg}{{\boldsymbol\Ygj}}
\newcommand{\flux}{{\boldsymbol J}}
\newcommand{\wiener}{W}
\newcommand{\whitenoise}{B}
\newcommand{\Wiener}{{\boldsymbol\wiener}}
\newcommand{\White}{{\boldsymbol\white}}
\newcommand{\paraj}{\nu}
\newcommand{\parag}{{\boldsymbol\paraj}}
\newcommand{\parae}{\hat{\parag}}
\newcommand{\erroj}{\epsilon}
\newcommand{\error}{{\boldsymbol\erroj}}
\newcommand{\biaj}{b}
\newcommand{\bias}{{\boldsymbol\biaj}}
%\newcommand{\disp}{{\boldsymbol V}}
\newcommand{\Fisher}{{\mathcal I}}
\newcommand{\likelihood}{{\mathcal L}}

\newcommand{\heps}{\varepsilon}
%\newcommand{\roi}{\varrho}
\newcommand{\jump}[1]{\llbracket{#1}\rrbracket}
\newcommand{\scal}[1]{\left\langle{#1}\right\rangle}
\newcommand{\norm}[1]{\left\|#1\right\|}
\newcommand{\abs}[1]{\left|#1\right|}
%\newcommand{\po}{\operatorname{o}}
\newcommand{\FFT}[1]{\widehat{#1}}
\newcommand{\indic}[1]{{\mathbf 1}_{#1}}
\newcommand{\impulse}{{\mathbbm h}}
\newcommand{\frf}{\FFT{\impulse}}

%\renewcommand{\Moy}[1]{{\boldsymbol\mu}_{#1}}
%\renewcommand{\Rcor}[1]{{\boldsymbol R}_{#1}}
%\renewcommand{\Mw}[1]{{\boldsymbol M}_{#1}}
%\renewcommand{\Sw}[1]{{\boldsymbol S}_{#1}}
%\renewcommand{\esp}[1]{{\mathbb E}\{#1\}}

\newcommand{\emphb}[1]{\textcolor{blue}{#1}}
\newcommand{\mycite}[1]{\textcolor{red}{#1}}
\newcommand{\mycitb}[1]{\textcolor{red}{[{\it #1}]}}

\newcommand{\PDFU}{{\mathcal U}}
\newcommand{\PDFN}{{\mathcal N}}
\newcommand{\TK}{{\boldsymbol\Pi}}
\newcommand{\TKij}{\pi}
\newcommand{\TKi}{{\boldsymbol\pi}}
\newcommand{\SMi}{\TKij^*}
\newcommand{\SM}{\TKi^*}
\newcommand{\lagmuli}{\lambda}
\newcommand{\lagmul}{{\boldsymbol\lagmuli}}
\newcommand{\constraint}{{\boldsymbol C}}
\newcommand{\mconstraint}{\mean{\constraint}}

\newcommand{\mybox}[1]{\fbox{\begin{minipage}{0.93\textwidth}{#1}\end{minipage}}}
\newcommand{\defcolor}[1]{\textcolor{blue}{#1}}

%\definecolor{rose}{LightPink}%{rgb}{251,204,231}

\newtheorem{mydef}{Definition}
\newtheorem{mythe}{Theorem}
\newtheorem{myprop}{Proposition}

% Or whatever. Note that the encoding and the font should match. If T1
% does not look nice, try deleting the line with the fontenc.

\title[1EL5000/S3]
{Design of a gravity dam}

\subtitle{1EL5000--Continuum Mechanics -- Tutorial Class \#3} % (optional)

\author[\'E. Savin] % (optional, use only with lots of authors)
{\'E. Savin\inst{1,2}\\ \scriptsize{\texttt{eric.savin@\{centralesupelec,onera\}.fr}}}%\inst{1} }
% - Use the \inst{?} command only if the authors have different
%   affiliation.

\institute[Onera] % (optional, but mostly needed)
{\inst{1}{Information Processing and Systems Dept.\\\Onera, France}
\and
 \inst{2}{Mechanical and Civil Engineering Dept.\\\ECP, France}}%
%  Department of Theoretical Philosophy\\
%  University of Elsewhere}
% - Use the \inst command only if there are several affiliations.
% - Keep it simple, no one is interested in your street address.

%\date[Short Occasion] % (optional)
\date{\today}

\subject{Design of a gravity dam}
% This is only inserted into the PDF information catalog. Can be left
% out. 



% If you have a file called "university-logo-filename.xxx", where xxx
% is a graphic format that can be processed by latex or pdflatex,
% resp., then you can add a logo as follows:

% \pgfdeclareimage[height=0.5cm]{university-logo}{university-logo-filename}
% \logo{\pgfuseimage{university-logo}}



% Delete this, if you do not want the table of contents to pop up at
% the beginning of each subsection:
\AtBeginSection[]
%\AtBeginSubsection[]
{
  \begin{frame}<beamer>{Outline}
    \tableofcontents[currentsection]%,currentsubsection]
  \end{frame}
}


% If you wish to uncover everything in a step-wise fashion, uncomment
% the following command: 

%\beamerdefaultoverlayspecification{<+->}


\begin{document}

\begin{frame}
  \titlepage
\end{frame}

\begin{frame}{Outline}
  \tableofcontents
  % You might wish to add the option [pausesections]
\end{frame}


% Since this a solution template for a generic talk, very little can
% be said about how it should be structured. However, the talk length
% of between 15min and 45min and the theme suggest that you stick to
% the following rules:  

% - Exactly two or three sections (other than the summary).
% - At *most* three subsections per section.
% - Talk about 30s to 2min per frame. So there should be between about
%   15 and 30 frames, all told.

\section{Some algebra}
\subsection{Vector \& tensor products}

\begin{frame}{Some algebra}{Vector \& tensor products}

\begin{itemize}
\item Scalar product:
\begin{displaymath}
\av,\bv\in\Rset^a\,,\quad\scal{\av,\bv}=\sum_{j=1}^a\aj_j\bj_j=\aj_j\bj_j\,,
\end{displaymath}
The last equality is \emphb{Einstein's summation convention}.
\item Tensors and tensor product (or outer product):
\begin{displaymath}
\Av\in\Rset^a\to\Rset^b\,,\quad\Av=\av\otimes\bv\,,\quad\av\in\Rset^a\,,\bv\in\Rset^b\,.
\end{displaymath}
\item Tensor application to vectors:
\begin{displaymath}
\Av=\av\otimes\bv\in\Rset^a\to\Rset^b\,,\cv\in\Rset^b\,,\quad\Av\cv=\scal{\bv,\cv}\av\,.
\end{displaymath}
\item Product of tensors $\equiv$ composition of linear maps:
\begin{displaymath}
\Av=\av\otimes\bv\,,\Bv=\cv\otimes\dv\,,\quad\Av\Bv=\scal{\bv,\cv}\av\otimes\dv\,.
\end{displaymath}
\end{itemize}

\end{frame}

\begin{frame}{Some algebra}{Vector \& tensor products}

%\onslide<2|handout:2>

\begin{itemize}
\item Scalar product of tensors:
\begin{displaymath}
\scal{\Av,\Bv}=\trace(\Av\Bv^\itr):=\Av:\Bv=\Aj_{jk}\Bj_{jk}\,.
\end{displaymath}
\item Let $\{\ev_j\}_{j=1}^d$ be a Cartesian basis in $\Rset^d$. Then:
\begin{displaymath}
\begin{split}
\aj_j &=\scal{\av,\ev_j}\,, \\
\Aj_{jk} &=\scal{\Av,\ev_j\otimes\ev_k}=\Av:\ev_j\otimes\ev_k \\
&=\scal{\Av\ev_k,\ev_j}\,,
\end{split}
\end{displaymath}
such that:
\begin{displaymath}
\begin{split}
\av &=\aj_j\ev_j\,, \\
\Av &=\Aj_{jk}\ev_j\otimes\ev_k\,.
\end{split}
\end{displaymath}
\item Example: the identity matrix
\begin{displaymath}
\Id=\ev_j\otimes\ev_j\,.
\end{displaymath}
\end{itemize}

%\end{overprint}

\end{frame}

\subsection{Vector \& tensor analysis}

\begin{frame}{Some analysis}{Vector \& tensor analysis}

\begin{itemize}
\item Gradient of a vector function $\av(\xv)$, $\xv\in\Rset^d$:
\begin{displaymath}
\Gradx\av=\frac{\partial\av}{\partial\xj_j}\otimes\ev_j\,.
\end{displaymath}
\item Divergence of a vector function $\av(\xv)$, $\xv\in\Rset^d$:
\begin{displaymath}
\divx\av=\scal{\gradx,\av}=\trace(\Gradx\av)=\frac{\partial\aj_j}{\partial\xj_j}\,.
\end{displaymath}
\item Divergence of a tensor function $\Av(\xv)$, $\xv\in\Rset^d$:
\begin{displaymath}
\Divx\Av=\frac{\partial(\Av\ev_j)}{\partial\xj_j}\,.
\end{displaymath}
\end{itemize}

\end{frame}

\begin{frame}{Some analysis}{Vector \& tensor analysis in cylindrical coordinates}

\begin{itemize}
\item Gradient of a vector function $\av(r,\theta,\zj)$:
\begin{displaymath}
\Gradx\av=\frac{\partial\av}{\partial r}\otimes\ev_r+\frac{\partial\av}{\partial\theta}\otimes\frac{\ev_\theta}{r}+\frac{\partial\av}{\partial\zj}\otimes\ev_\zj\,.
\end{displaymath}
\item Divergence of a vector function $\av(r,\theta,\zj)$:
\begin{displaymath}
\divx\av=\scal{\frac{\partial\av}{\partial r},\ev_r}+\scal{\frac{\partial\av}{\partial\theta},\frac{\ev_\theta}{r}}+\scal{\frac{\partial\av}{\partial\zj},\ev_\zj}\,.
\end{displaymath}
\item Divergence of a tensor function $\Av(r,\theta,\zj)$:
\begin{displaymath}
\Divx\Av=\frac{\partial\Av}{\partial r}\ev_r+\frac{\partial\Av}{\partial\theta}\frac{\ev_\theta}{r}+\frac{\partial\Av}{\partial\zj}\ev_\zj\,.
\end{displaymath}
\end{itemize}

\end{frame}



\section{Strength criteria}
\begin{frame}{Recap}{}

\begin{itemize}
\item Displacement:
\begin{displaymath}
\uv=\xv-\pv=\fv(\pv,t)-\pv\,;
\end{displaymath}
\item Strains:
\begin{displaymath}
\GreenL=\demi\left(\Grad_\pv\uv+\Grad_\pv\uv^\itr+\Grad_\pv\uv^\itr\Grad_\pv\uv\right)\,;
\end{displaymath}
\item Small strains $\xv\sim\pv$, $\norm{\Grad_\pv\uv}\ll 1$:
\begin{displaymath}
\GreenL\simeq\strain=\demi\left(\Gradx\uv+\Gradx\uv^\itr\right)\,;
\end{displaymath}
\item Stresses:
\begin{displaymath}
\Divx\stress+\fv_v=\roi\av\,;
\end{displaymath}
\item What makes the difference between foams and steel???
\end{itemize}

\end{frame}

\begin{frame}{Stresses}{Tensile test}

\begin{overprint}

\onslide<1|handout:1>
\vskip-10pt
\begin{figure}
\centering\includegraphics[scale=.2]{\figs/ch3-car}
\end{figure}
\begin{figure}
\centering\includegraphics[scale=.2]{\figs/ch3-turbomachine}
\end{figure}

\onslide<2|handout:2>
\begin{figure}
\centering\includegraphics[scale=.4]{\figs/ch3-TractionTest}
\end{figure}
Stresses are harder to measure compared to strains or displacements which can be seen!

\onslide<3|handout:3>
\begin{columns}[t]
\column{.5\textwidth}
\centering\includegraphics[scale=.3]{\figs/ch3-TractionTest-start}
\column{.5\textwidth}
%\vskip-100pt
\centering\includegraphics[scale=.345]{\figs/ch3-TractionTest-end}
\end{columns}
\centering Plot $\strainj\mapsto\stressj$ up to fracture.

\end{overprint}

\end{frame}

\begin{frame}{Stress-strain curve}{Brittle materials}

\begin{columns}[t]
\column{.5\textwidth}
\hspace*{-0.8truecm}\centering\includegraphics[scale=.18]{\figs/ch3-StressStrain-brittle}
\column{.5\textwidth}
\hspace*{-1.0truecm}\centering\includegraphics[scale=.18]{\figs/ch3-fragile}
\end{columns}
\vskip10pt
\begin{itemize}
\item At microscopic level: splitting of two atomic planes driven by normal stress component $\scal{\stress\nv,\nv}$.
\item {\bf Examples}: cast iron, glass, stone, concrete, carbon fiber, ceramics, polymers (PMMA, polystyrene)...
\end{itemize}

\end{frame}

\begin{frame}{Stress-strain curve}{Ductile materials}

\begin{columns}[t]
\column{.5\textwidth}
\hspace*{-0.3truecm}\centering\includegraphics[scale=.17]{\figs/ch3-StressStrain-ductile}
\column{.5\textwidth}
\centering\includegraphics[scale=.2]{\figs/ch3-ductile}
\end{columns}
\vskip10pt
\begin{itemize}
\item At microscopic level:  linear crystallographic defects (dislocations) allowing atoms to slide over each other at low stress levels.
\item {\bf Examples}: structural steel and many alloys of other metals...
\end{itemize}

\end{frame}

\begin{frame}{Quantifying stresses}{Normal-shear stresses}

\begin{columns}[t]
\column{.5\textwidth}
\centering\includegraphics[scale=.25]{\figs/ch3-sigmann}
\column{.5\textwidth}
\centering\includegraphics[scale=.25]{\figs/ch3-tau_sigma}
\end{columns}
\begin{itemize}
\item Normal stress: $\stressj_{\nj\nj}=\scal{\stress\nv,\nv}$;
\item Shear (tangent) force: $\tress_\Sigma=\stress\nv-\stressj_{\nj\nj}\nv$;
\item Shear stress: $\tress_{\nj\mj}=\scal{\stress\nv,\mv}$.
\end{itemize}

\end{frame}

\begin{frame}{Quantifying stresses}{Principal stresses}

\begin{figure}
\centering\includegraphics[scale=.2]{\figs/ch3-PrincipalStresses}
\end{figure}
\begin{itemize}
\item Since $\stress$ is symmetric:
\begin{displaymath}
\stress{\boldsymbol\Phi}_\stressj=\lambda_\stressj{\boldsymbol\Phi}_\stressj\,.
\end{displaymath}
\item Characteristic polynomial ($I_1(\stress)=\trace\stress$, $I_2(\stress)=\demi((\trace\stress)^2-\trace(\stress^2))$, $I_3(\stress)=\det\stress$):
\begin{displaymath}
\det(\stress-\lambda_\stressj\Id)=-\lambda_\stressj^3+I_1(\stress)\lambda_\stressj^2-I_2(\stress)\lambda_\stressj+I_3(\stress)=0\,.
\end{displaymath}
\item Major principal stress:
\begin{displaymath}
\lambda_1(\stress)=\max_{\norm{\nv}=1}\scal{\stress\nv,\nv}<\stressj_r\,.
\end{displaymath}
\end{itemize}

\end{frame}

\begin{frame}{Quantifying stresses}{Deviatoric stress tensor}

\begin{itemize}
\item Deviatoric stress tensor:
\begin{displaymath}
\stress^D=\stress-\frac{\trace\stress}{3}\Id\,;
\end{displaymath}
\item Orthogonal expansion: $\stress=\underline{\stressj}\Id+\stress^D$, $\underline{\stressj}\Id:\stress^D=0$;
\item Deviatoric stress invariants $J_m=\frac{1}{m}\trace(\stress^{Dm})$ such that:
\begin{displaymath}
\!\!\!\!\det(\stress^D-\lambda_\stressj\Id)=-\lambda_\stressj^3+J_1(\stress^D)\lambda_\stressj^2-J_2(\stress^D)\lambda_\stressj+J_3(\stress^D)=0\,;
\end{displaymath}
\item The deviatoric stress tensor has the same principal directions and is a state of pure shear:
\begin{displaymath}
\begin{split}
J_2(\stress^D) &=0\imply\stress^D=\bzero\,, \\
\stress^D &=\bzero\imply\tress_\Sigma=\stress\nv-\stressj_{\nj\nj}\nv=\bzero\,,\forall\nv\,.
\end{split}
\end{displaymath}
\end{itemize}

\end{frame}

\begin{frame}{Quantifying stresses}{von Mises criterion}

\begin{itemize}
\item Equivalent or von Mises stress:
\begin{displaymath}
\begin{split}
\stressj_\text{eq} &=\sqrt{3J_2(\stress^D)} \\
&=\sqrt{\frac{(\stressj_1-\stressj_2)^2+(\stressj_2-\stressj_3)^2+(\stressj_3-\stressj_1)^2}{2}}\,,
\end{split}
\end{displaymath}
where $\stressj_1\geq\stressj_2\geq\stressj_3$ are the principal stresses of $\stress$.
\item {\bf Examples}:
\begin{displaymath}
\begin{split}
\stress =\stressj\ev\otimes\ev &\imply\stressj_\text{eq}=\stressj \\
\stress =\tressj\mv\otimes_s\nv &\imply\stressj_\text{eq}=\sqrt{3}\tressj\,.
\end{split}
\end{displaymath}
\item von Mises yield criterion: 
\begin{displaymath}
\stressj_\text{eq}\leq\stressj_0\,.
\end{displaymath}
\end{itemize}

\end{frame}

\begin{frame}{Quantifying stresses}{Tresca's criterion}

\begin{itemize}
\item Tresca's stress for $\tress_\Sigma=\stress\nv-\stressj_{\nj\nj}\nv$: 
\begin{displaymath}
\begin{split}
\tressj_\text{eq} &=\max_{\norm{\nv}=1}\norm{\tress_\Sigma} \\
&=\demi\max(\abs{\stressj_1-\stressj_2},\abs{\stressj_2-\stressj_3},\abs{\stressj_3-\stressj_1})\,.
\end{split}
\end{displaymath}
\item Maximum shear stress for $\stressj_1\geq\stressj_2\geq\stressj_3$:
\begin{displaymath}
\tressj_\text{eq}=\frac{\stressj_1-\stressj_3}{2}\,,
\end{displaymath}
which acts on the plane with unit normal $\nv=\frac{{\boldsymbol\Phi}_1\pm{\boldsymbol\Phi}_3}{\sqrt{2}}$ for which:
\begin{displaymath}
\stressj_{\nj\nj}=\frac{\stressj_1+\stressj_3}{2}\,.
\end{displaymath}
\item Tresca's criterion: 
\begin{displaymath}
\tressj_\text{eq}\leq\tressj_0=\frac{\stressj_0}{2}\,.
\end{displaymath}
\end{itemize}

\end{frame}


\begin{frame}{To go further...}{}

\begin{itemize}
\item 2EL1830 Non-linear behavior of materials
\item ST7-81-CS Conception en fabrication additive
\end{itemize}

\end{frame}

\section{3.1 Design of a gravity dam}

\begin{frame}{Design of a gravity dam}{Setup}

\begin{columns}[t]
\column{.5\textwidth}
\centering\includegraphics[scale=.35]{\figs/ch3-Dam}
\column{.5\textwidth}
\vskip-100pt
\centering\includegraphics[scale=.35]{\figs/ch3-WillowCreek}
\vskip-20pt{\hspace{5.2truecm}\mbox{\tiny{Willow Creek dam (Oregon, USA)}}}
\end{columns}
\begin{displaymath}
\stress=b_1\xj_2\iv_1\otimes\iv_1 + (a_2\xj_1+b_2\xj_2)\iv_2\otimes\iv_2-2(\roi_b g+b_2)\xj_1\iv_1\otimes_s\iv_2\,,
\end{displaymath}
with:
\begin{displaymath}
b_1=-\roi_e g\,,\quad a_2=\frac{\roi_b g}{\alpha}-2\frac{\roi_e g}{\alpha^3}\,,\quad b_2=-\roi_b g+\frac{\roi_e g}{\alpha^2}\,.
\end{displaymath}

\end{frame}

\begin{frame}{Design of a gravity dam}{Solution}

\begin{overprint}

\onslide<1|handout:1>
\vskip-20pt
\begin{exampleblock}{Question \#1: Local equilibrium equation?}
\begin{itemize}
\item The local equilibrium equation reads $\Divx\stress+\fv_v=\roi\acv$.
\item In the present case $\fv_v=+\roi_b g\iv_2$ and $\acv=\bzero$ in static analysis.
\end{itemize}
\end{exampleblock}

\onslide<2|handout:2>
\vskip-20pt
\begin{exampleblock}{Question \#1: Local equilibrium equation?}
\begin{itemize}
\item The local equilibrium equation reads $\Divx\stress+\fv_v=\roi\acv$.
\item In the present case $\fv_v=+\roi_b g\iv_2$ and $\acv=\bzero$.
\item Besides $\Divx\stress=\frac{\partial}{\partial\xj_j}(\stress\iv_j)$ and:
\begin{displaymath}
\begin{split}
\frac{\partial\stress}{\partial\xj_1}&=a_2\iv_2\otimes\iv_2-2(\roi_b g+b_2)\iv_1\otimes_s\iv_2\,, \\
\frac{\partial\stress}{\partial\xj_2} &=b_1\iv_1\otimes\iv_1+b_2\iv_2\otimes\iv_2\,, \\
\frac{\partial\stress}{\partial\xj_3} &=\bzero\,.
\end{split}
\end{displaymath}
\end{itemize}
\end{exampleblock}

\onslide<3|handout:3>
\vskip-20pt
\begin{exampleblock}{Question \#1: Local equilibrium equation?}
\begin{itemize}
\item The local equilibrium equation reads $\Divx\stress+\fv_v=\roi\acv$.
\item In the present case $\fv_v=+\roi_b g\iv_2$ and $\acv=\bzero$.
\item Besides $\Divx\stress=\frac{\partial}{\partial\xj_j}(\stress\iv_j)$ and:
\begin{displaymath}
\begin{split}
\frac{\partial\stress}{\partial\xj_1}&=a_2\iv_2\otimes\iv_2-2(\roi_b g+b_2)\iv_1\otimes_s\iv_2\,, \\
\frac{\partial\stress}{\partial\xj_2} &=b_1\iv_1\otimes\iv_1+b_2\iv_2\otimes\iv_2\,. \\
\end{split}
\end{displaymath}
\item Hence (remind that $(\av\otimes\bv)\cv=\scal{\bv,\cv}\av$):
\begin{displaymath}
\Divx\stress=-(\roi_b g+b_2)\iv_2+b_2\iv_2=-\roi_b g\iv_2=-\fv_v\quad\text{QED}\,.
\end{displaymath}
\end{itemize}
\end{exampleblock}

\end{overprint}

\end{frame}

\begin{frame}{Design of a gravity dam}{Solution}

\begin{overprint}

\onslide<1|handout:1>
\vskip-20pt
\begin{exampleblock}{Question \#2: Free edge BC on downstream wall?}
\begin{itemize}
\item Downstream wall $\Gamma_\text{d}=\{0\leq\xj_2\leq H,\,\xj_1-\alpha\xj_2=0\}$.
\end{itemize}
\end{exampleblock}

\onslide<2|handout:2>
\vskip-20pt
\begin{exampleblock}{Question \#2: Free edge BC on downstream wall?}
\begin{itemize}
\item Downstream wall $\Gamma_d=\{0\leq\xj_2\leq H,\,\xj_1-\alpha\xj_2=0\}$.
\item Outward unit normal $\nv=\frac{1}{\sqrt{1+\alpha^2}}(\iv_1-\alpha\iv_2)$.
\end{itemize}
\end{exampleblock}

\onslide<3|handout:3>
\vskip-20pt
\begin{exampleblock}{Question \#2: Free edge BC on downstream wall?}
\begin{itemize}
\item Downstream wall $\Gamma_d=\{0\leq\xj_2\leq H,\,\xj_1-\alpha\xj_2=0\}$.
\item Outward unit normal $\nv=\frac{1}{\sqrt{1+\alpha^2}}(\iv_1-\alpha\iv_2)$.
\item Therefore:
\begin{displaymath}
\begin{split}
\stress\nv\mid_{\Gamma_d}= &\;\frac{1}{\sqrt{1+\alpha^2}}\left(b_1\xj_2\iv_1-(\roi_b g+b_2)\xj_1\iv_2\right. \\
&\;\left. -\alpha(a_2\xj_1+b_2\xj_2)\iv_2-\alpha(\roi_b g+b_2)\xj_1\iv_1\right) \\
= &\;\frac{\xj_2}{\sqrt{1+\alpha^2}}\left[\left(b_1-\alpha^2(\roi_b g+b_2)\right)\iv_1 \right. \\
&\;\left. -\left(\alpha(\roi_b g+2b_2)+\alpha^2a_2\right)\iv_2\right] \\
= &\;\frac{\xj_2}{\sqrt{1+\alpha^2}}\left[\left(-\roi_e g+\roi_e g\right)\iv_1 \right. \\
&\;\left. -\left(\alpha\left(-\roi_b g+2\frac{\roi_e g}{\alpha^2}\right) +\alpha\roi_b g-2 \frac{\roi_e g}{\alpha}\right)\iv_2\right]\;\text{QED}\,.
\end{split}
\end{displaymath}
\end{itemize}
\end{exampleblock}

\end{overprint}

\end{frame}

\begin{frame}{Design of a gravity dam}{Solution}

\begin{overprint}

\onslide<1|handout:1>
\vskip-20pt
\begin{exampleblock}{Question \#3: Hydrostatic pressure on upstream wall?}
\begin{itemize}
\item Upstream wall $\Gamma_u=\{0\leq\xj_2\leq H,\,\xj_1=0\}$.
\end{itemize}
\end{exampleblock}

\onslide<2|handout:2>
\vskip-20pt
\begin{exampleblock}{Question \#3: Hydrostatic pressure on upstream wall?}
\begin{itemize}
\item Upstream wall $\Gamma_u=\{0\leq\xj_2\leq H,\,\xj_1=0\}$.
\item Outward unit normal $\nv=-\iv_1$.
\end{itemize}
\end{exampleblock}

\onslide<3|handout:3>
\vskip-20pt
\begin{exampleblock}{Question \#3: Hydrostatic pressure on upstream wall?}
\begin{itemize}
\item Upstream wall $\Gamma_u=\{0\leq\xj_2\leq H,\,\xj_1=0\}$.
\item Outward unit normal $\nv=-\iv_1$.
\item Hydrostatic pressure $\fv_s=+\roi_e g\xj_2\iv_1$.
\end{itemize}
\end{exampleblock}

\onslide<4|handout:4>
\vskip-20pt
\begin{exampleblock}{Question \#3: Hydrostatic pressure on upstream wall?}
\begin{itemize}
\item Upstream wall $\Gamma_u=\{0\leq\xj_2\leq H,\,\xj_1=0\}$.
\item Outward unit normal $\nv=-\iv_1$.
\item Hydrostatic pressure $\fv_s=+\roi_e g\xj_2\iv_1$.
\item But:
\begin{displaymath}
\begin{split}
\stress\nv\mid_{\Gamma_u} =&\;-b_1\xj_2\iv_1+(\roi_b g+b_2)\cancel{\xj_1}\iv_2 \\
=&\;+\roi_e g\xj_2\iv_1 \\
=&\;\fv_s\quad\text{QED}\,.
\end{split}
\end{displaymath}
\end{itemize}
\end{exampleblock}

\end{overprint}

\end{frame}

\begin{frame}{Design of a gravity dam}{Solution}

\begin{overprint}

\onslide<1|handout:1>
\vskip-20pt
\begin{exampleblock}{Question \#4: $H$ to withstand maximum normal stress $\stressj_r$?}
\begin{itemize}
\item $\max_{\norm{\nv}=1}\stressj_{\nj\nj}=\lambda_1(\stress)$.
\end{itemize}
\end{exampleblock}

\onslide<2|handout:2>
\vskip-20pt
\begin{exampleblock}{Question \#4: $H$ to withstand maximum normal stress $\stressj_r$?}
\begin{itemize}
\item $\max_{\norm{\nv}=1}\stressj_{\nj\nj}=\lambda_1(\stress)$.
\item But on $\Gamma_u$: $\stress=b_1\xj_2\iv_1\otimes\iv_1 + b_2\xj_2\iv_2\otimes\iv_2$.
\end{itemize}
\end{exampleblock}

\onslide<3|handout:3>
\vskip-20pt
\begin{exampleblock}{Question \#4: $H$ to withstand maximum normal stress $\stressj_r$?}
\begin{itemize}
\item $\max_{\norm{\nv}=1}\stressj_{\nj\nj}=\lambda_1(\stress)$.
\item But on $\Gamma_u$: $\stress=b_1\xj_2\iv_1\otimes\iv_1 + b_2\xj_2\iv_2\otimes\iv_2$.
\item Therefore:
\begin{displaymath}
\begin{split}
\lambda_1(\stress) &=\xj_2\times\max(b_1,b_2) \\
&=g\xj_2\times\max\left(-\roi_e, -\roi_b+\frac{\roi_e}{\alpha^2}\right)\,.
\end{split}
\end{displaymath}
\end{itemize}
\end{exampleblock}

\onslide<4|handout:4>
\vskip-20pt
\begin{exampleblock}{Question \#4: $H$ to withstand maximum normal stress $\stressj_r$?}
\begin{itemize}
\item $\max_{\norm{\nv}=1}\stressj_{\nj\nj}=\lambda_1(\stress)$.
\item But on $\Gamma_u$: $\stress=b_1\xj_2\iv_1\otimes\iv_1 + b_2\xj_2\iv_2\otimes\iv_2$.
\item Therefore:
\begin{displaymath}
\begin{split}
\lambda_1(\stress) &=\xj_2\times\max(b_1,b_2) \\
&=g\xj_2\times\max\left(-\roi_e, -\roi_b+\frac{\roi_e}{\alpha^2}\right)\,.
\end{split}
\end{displaymath}
\item If $\alpha>\smash{\sqrt{\roi_e/\roi_b}}$, then $\lambda_1(\stress)<0$ and the dam is in compression; this condition is always fulfilled when $\roi_e=0$ (no water).
\end{itemize}
\end{exampleblock}

\onslide<5|handout:5>
\vskip-20pt
\begin{exampleblock}{Question \#4: $H$ to withstand maximum normal stress $\stressj_r$?}
\begin{itemize}
\item $\max_{\norm{\nv}=1}\stressj_{\nj\nj}=\lambda_1(\stress)$.
\item But on $\Gamma_u$: $\stress=b_1\xj_2\iv_1\otimes\iv_1 + b_2\xj_2\iv_2\otimes\iv_2$.
\item Therefore:
\begin{displaymath}
\begin{split}
\lambda_1(\stress) &=\xj_2\times\max(b_1,b_2) \\
&=g\xj_2\times\max\left(-\roi_e, -\roi_b+\frac{\roi_e}{\alpha^2}\right)\,.
\end{split}
\end{displaymath}
\item If $\alpha\leq\smash{\sqrt{\roi_e/\roi_b}}$, then $\lambda_1(\stress)=(\frac{\roi_e}{\alpha^2}-\roi_b)g\xj_2$ and $\lambda_1(\stress)=\stressj_r$ whenever $\xj_2=\frac{\alpha^2\stressj_r}{g(\roi_e-\alpha^2\roi_b)}$; therefore:
\begin{displaymath}
H<\frac{\alpha^2\stressj_r}{g(\roi_e-\alpha^2\roi_b)}\,.
\end{displaymath}
\end{itemize}
\end{exampleblock}

\end{overprint}

\end{frame}

\end{document}

\section{3.3 Torsion of a cylindrical transmission shaft}

\begin{frame}{Torsion of a cylindrical transmission shaft}{Setup}

\centering\includegraphics[scale=.3]{\figs/ch3-Shaft}

\begin{displaymath}
\medium_t=\{\xv=(r,\theta,\zj);\,0\leq r< a, 0\leq\theta\leq 2\pi,0<\zj<L\}\,;
\end{displaymath}
\begin{displaymath}
\stress(\xv)=2kr\iv_\theta(\theta)\otimes_s\iv_\zj\,,\quad\xv\in\medium_t\,.
\end{displaymath}

\end{frame}

\begin{frame}{Torsion of a cylindrical transmission shaft}{Solution}

\begin{overprint}

\onslide<1|handout:1>
\vskip-20pt
\begin{exampleblock}{Question \#1: Local equilibrium and boundary conditions?}
\begin{itemize}
\item The local equilibrium equation reads $\Divx\stress+\fv_v=\roi\acv$.
\item In the present case $\fv_v=\bzero$ ("gravity can be neglected") and $\acv=\bzero$ ("effects of inertia can be neglected").
\end{itemize}
\end{exampleblock}

\end{overprint}

\end{frame}

\end{document}

