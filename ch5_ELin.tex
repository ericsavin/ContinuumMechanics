% MG3 TD #5: Linkage
% V1.0 February 2021
% $Header: /cvsroot/latex-beamer/latex-beamer/solutions/generic-talks/generic-ornate-15min-45min.en.tex,v 1.5 2007/01/28 20:48:23 tantau Exp $
\def\webDOI{http://dx.doi.org}
\def\Folder{/Users/ericsavin/Documents/Cours/SG3-MMC/SLIDES_TD/}
\def\Year{\Folder/2020-2021}
\def\Sections{\Year/SECTIONS}
\def\figs{\Folder/FIGS}
%\def\figs{/Users/ericsavin/Documents//Cours/DynSto/Figs}
%\def\figs{/Users/ericsavin/Documents/Cours/SG3-MMC/SLIDES_TD/FIGS}
\def\figdynsto{/Users/ericsavin/Documents//Figures/DYNSTO}
\def\symb{/Users/ericsavin/Documents/Latex/SYMBOL}
\def\fonts{/Users/ericsavin/Documents/Latex/FONTS}
\def\logos{/Users/ericsavin/Documents/Latex/LOGOS}
\def\Onera{ONERA}
\def\ECP{CentraleSup\'elec}


\documentclass{beamer}

% This file is a solution template for:

% - Giving a talk on some subject.
% - The talk is between 15min and 45min long.
% - Style is ornate.



% Copyright 2004 by Till Tantau <tantau@users.sourceforge.net>.
%
% In principle, this file can be redistributed and/or modified under
% the terms of the GNU Public License, version 2.
%
% However, this file is supposed to be a template to be modified
% for your own needs. For this reason, if you use this file as a
% template and not specifically distribute it as part of a another
% package/program, I grant the extra permission to freely copy and
% modify this file as you see fit and even to delete this copyright
% notice. 


\mode<presentation>
{
  \usetheme{Berkeley}
  % or ...

  \setbeamercovered{transparent}
  % or whatever (possibly just delete it)
}


\usepackage[english]{babel}
% or whatever

\usepackage[latin1]{inputenc}
% or whatever

%\usepackage{mathtime}
\usefonttheme{serif}
%\usefonttheme{professionalfonts}
\usepackage{amsfonts}
\usepackage{amssymb}

\usepackage{amsmath}
\usepackage{multimedia}
\usepackage{mathrsfs}
\usepackage{mathabx}
\usepackage{color}
\usepackage{pstricks}
\usepackage{graphicx}
%\usepackage[pdftex, pdfborderstyle={/S/U/W 1}]{hyperref}
\usepackage{hyperref}
\usepackage{bbm}
\usepackage{cancel}
\usepackage[Symbol]{upgreek}
%\usepackage{mathbbol}
%\DeclareSymbolFontAlphabet{\amsmathbb}{AMSb}
%\usepackage[bbgreekl]{mathbbol}
%\usepackage[mtpbbi]{mtpro2}

%\usepackage[svgnames]{xcolor}

%\input{\fonts/math0}
%\input{\symb/structac} % Notations E. Savin
%\input{\symb/logos}

\newcommand{\ci}{\mathrm{i}}
\newcommand{\trace}{\operatorname{Tr}}
\newcommand{\Nset}{\mathbb{N}}
\newcommand{\Zset}{\mathbb{Z}}
\newcommand{\Rset}{\mathbb{R}}
\newcommand{\Cset}{\mathbb{C}}
\newcommand{\Sset}{\mathbb{S}}
\newcommand{\Mset}{\mathbb{M}}
\newcommand{\PhaseSpace}{\Omega}
\newcommand{\ContSet}{{\mathcal C}}
\newcommand{\id}{d}
\newcommand{\iD}{\mathrm{D}}
\newcommand{\iexp}{\mathrm{e}}
\newcommand{\demi}{\frac{1}{2}}
\newcommand{\imply}{\Rightarrow}

% Algebra
\newcommand{\itr}{{\sf T}}
\newcommand{\Id}{{\boldsymbol I}}
\newcommand{\IId}{\mathbb{I}}
\newcommand{\aj}{a}
\newcommand{\bj}{b}
\newcommand{\cj}{c}
\renewcommand{\dj}{d}
\newcommand{\av}{{\boldsymbol\aj}}
\newcommand{\bv}{{\boldsymbol\bj}}
\newcommand{\cv}{{\boldsymbol\cj}}
\newcommand{\dv}{{\boldsymbol\dj}}
\newcommand{\uj}{u}
\newcommand{\vj}{v}
\newcommand{\xj}{x}
\newcommand{\yj}{y}
\newcommand{\zj}{z}
\newcommand{\uv}{{\boldsymbol\uj}}
\newcommand{\vv}{{\boldsymbol\vj}}
\newcommand{\xv}{{\boldsymbol\xj}}
\newcommand{\yv}{{\boldsymbol\yj}}
\newcommand{\zv}{{\boldsymbol\zj}}
\newcommand{\Aj}{A}
\newcommand{\Bj}{B}
\newcommand{\Av}{{\boldsymbol\Aj}}
\newcommand{\Bv}{{\boldsymbol\Bj}}
\newcommand{\Zgv}{{\boldsymbol Z}}

% Analysis
\newcommand{\grad}{{\boldsymbol\nabla}}
\newcommand{\gradx}{{\grad_\xv}}
\newcommand{\Grad}{{\mathbb D}}
\newcommand{\Gradx}{{\Grad_\xv}}
\renewcommand{\div}{\mathrm{div}}
\newcommand{\divx}{{\div_\xv}}
\newcommand{\Div}{\mathbf{Div}}
\newcommand{\Divx}{{\Div_\xv}}

% Kinematics
\newcommand{\ej}{e}
\renewcommand{\ij}{i}
\newcommand{\pj}{p}
\newcommand{\ev}{{\boldsymbol\ej}}
\newcommand{\iv}{{\boldsymbol\ij}}
\newcommand{\pv}{{\boldsymbol\pj}}
\newcommand{\posij}{f}
\newcommand{\posiv}{{\boldsymbol\posij}}
\newcommand{\iposiv}{{\boldsymbol g}}
\newcommand{\Fp}{{\mathbb F}}
\newcommand{\GreenLj}{E}
\newcommand{\GreenL}{{\mathbb\GreenLj}}
\newcommand{\medium}{\Omega}
\newcommand{\strainj}{\varepsilon}
\newcommand*{\strain}{\mbox{$\hspace{0.2em}\rotatebox[x=0pt,y=0.2pt]{90}{\rule{0.02\linewidth}{0.4pt}}\hspace{-0.23em}\upvarepsilon$}}
\newcommand*{\rotation}{{\boldsymbol R}}

% Dynamics
\newcommand{\fj}{f}
\newcommand{\Fj}{F}
\newcommand{\mj}{m}
\newcommand{\nj}{n}
\newcommand{\Tj}{T}
\newcommand{\fv}{{\boldsymbol\fj}}
\newcommand{\Fv}{{\boldsymbol\Fj}}
\newcommand{\mv}{{\boldsymbol\mj}}
\newcommand{\nv}{{\boldsymbol\nj}}
\newcommand{\Tv}{{\boldsymbol\Tj}}
\newcommand{\roi}{\varrho}
\newcommand*{\stressj}{\sigma}
\newcommand*{\tressj}{\tau}
\newcommand{\stress}{\mbox{$\hspace{0.3em}\rotatebox[x=0pt,y=0.2pt]{90}{\rule{0.017\linewidth}{0.4pt}}\hspace{-0.25em}\upsigma$}}
\newcommand*{\tress}{{\boldsymbol\tressj}}
\newcommand{\acj}{a}
\newcommand{\acv}{{\boldsymbol\acj}}

%\newcommand{\Zg}{{\bf\zgj}}
%\newcommand{\xigj}{\xi}
%\newcommand{\xig}{{\boldsymbol\xigj}}
\newcommand{\kgj}{k}
%\newcommand{\kgh}{\kgj_\ygj}
%\newcommand{\kg}{{\bf\kgj}}
\newcommand{\Kg}{{\bf K}}
%\newcommand{\qg}{{\boldsymbol q}}
%\newcommand{\pg}{{\boldsymbol p}}
\newcommand{\hkg}{{\hat \kg}}
\newcommand{\hpg}{\hat{\pg}}
%\newcommand{\vg}{{\boldsymbol v}}
\newcommand{\sg}{{\boldsymbol s}}
%\newcommand{\stress}{\mathbb{\sigma}}
\newcommand{\tenselasj}{{\Large C}}
\newcommand{\tenselas}{\boldsymbol{\mathsf{\tenselasj}}}
\newcommand{\tenscomp}{\boldsymbol{\mathsf{\Large S}}}
\newcommand{\speci}{{\mathrm w}}
\newcommand{\specij}{{\mathrm W}}
\newcommand{\speciv}{{\bf \specij}}
%\newcommand{\cjg}[1]{\overline{#1}}
\newcommand{\eigv}{{\bf b}}
\newcommand{\eigw}{{\bf c}}
\newcommand{\eigl}{\lambda}
\newcommand{\jeig}{\alpha}
\newcommand{\keig}{\beta}
\newcommand{\cel}{c}
\newcommand{\bcel}{{\bf\cel}}
\newcommand{\deng}{{\mathcal E}}
\newcommand{\flowj}{\pi}
\newcommand{\flow}{\boldsymbol\flowj}
\newcommand{\Flowj}{\Pi}
\newcommand{\Flow}{\boldsymbol\Flowj}
\newcommand{\fluxinj}{g}
\newcommand{\fluxin}{{\bf\fluxinj}}
\newcommand{\dscat}{\sigma}
\newcommand{\tdscat}{\Sigma}
\newcommand{\collop}{{\mathcal Q}}
\newcommand{\epsd}{\delta}
\newcommand{\rscat}{\rho}
\newcommand{\tscat}{\tau}
\newcommand{\Rscat}{\mathcal{R}}
\newcommand{\Tscat}{\mathcal{T}}
\newcommand{\lscat}{\ell}
%\newcommand{\floss}{\eta}
\newcommand{\mdiff}{{\bf D}}
%\newcommand{\demi}{\frac{1}{2}}
\newcommand{\domain}{{\mathcal O}}
\newcommand{\bdomain}{{\mathcal D}}
\newcommand{\interface}{\Gamma}
\newcommand{\sinterface}{\gamma_D}
%\newcommand{\normal}{\hat{\bf n}}
\newcommand{\bnabla}{\boldsymbol\nabla}
%\newcommand{\esp}[1]{\mathbb{E}\{\smash{#1}\}}
\newcommand{\mean}[1]{\underline{#1}}
%\newcommand{\BB}{\mathbb{B}}
%\newcommand{\II}{{\boldsymbol I}}
%\newcommand{\TA}{\boldsymbol{\Gamma}}
\newcommand{\Mdisp}{{\mathbf H}}
\newcommand{\Hamil}{{\mathcal H}}
\newcommand{\bzero}{{\bf 0}}

\newcommand{\mass}{M}
\newcommand{\damp}{D}
\newcommand{\stif}{K}
\newcommand{\dsp}{S}
\newcommand{\dof}{q}
\newcommand{\pof}{p}
%\newcommand{\MM}{{\boldsymbol\mass}}
\newcommand{\MD}{{\boldsymbol\damp}}
\newcommand{\MK}{{\boldsymbol\stif}}
\newcommand{\MS}{{\boldsymbol\dsp}}
\newcommand{\Cov}{{\boldsymbol C}}
\newcommand{\dofg}{{\boldsymbol\dof}}
\newcommand{\pofg}{{\boldsymbol\pof}}
\newcommand{\driftj}{b}
\newcommand{\drifts}{{\boldsymbol \driftj}}
\newcommand{\drift}{{\underline\drifts}}
\newcommand{\scatj}{a}
\newcommand{\scat}{{\boldsymbol\scatj}}
\newcommand{\diff}{{\boldsymbol\sigma}}
\newcommand{\load}{F}
\newcommand{\loadg}{{\boldsymbol\load}}
\newcommand{\pdf}{\pi}
\newcommand{\tpdf}{\pdf_t}
\newcommand{\fg}{{\boldsymbol f}}
%\newcommand{\Ugj}{U}
\newcommand{\Vgj}{V}
\newcommand{\Xgj}{X}
\newcommand{\Ygj}{Y}
%\newcommand{\Ug}{{\boldsymbol\Ugj}}
\newcommand{\Vg}{{\boldsymbol\Vgj}}
%\newcommand{\Qg}{{\boldsymbol Q}}
\newcommand{\Pg}{{\boldsymbol P}}
\newcommand{\Xg}{{\boldsymbol\Xgj}}
\newcommand{\Yg}{{\boldsymbol\Ygj}}
\newcommand{\flux}{{\boldsymbol J}}
\newcommand{\wiener}{W}
\newcommand{\whitenoise}{B}
\newcommand{\Wiener}{{\boldsymbol\wiener}}
\newcommand{\White}{{\boldsymbol\white}}
\newcommand{\paraj}{\nu}
\newcommand{\parag}{{\boldsymbol\paraj}}
\newcommand{\parae}{\hat{\parag}}
\newcommand{\erroj}{\epsilon}
\newcommand{\error}{{\boldsymbol\erroj}}
\newcommand{\biaj}{b}
\newcommand{\bias}{{\boldsymbol\biaj}}
%\newcommand{\disp}{{\boldsymbol V}}
\newcommand{\Fisher}{{\mathcal I}}
\newcommand{\likelihood}{{\mathcal L}}

\newcommand{\heps}{\varepsilon}
%\newcommand{\roi}{\varrho}
\newcommand{\jump}[1]{\llbracket{#1}\rrbracket}
\newcommand{\scal}[1]{\left\langle{#1}\right\rangle}
\newcommand{\norm}[1]{\left\|#1\right\|}
\newcommand{\abs}[1]{\left|#1\right|}
%\newcommand{\po}{\operatorname{o}}
\newcommand{\FFT}[1]{\widehat{#1}}
\newcommand{\indic}[1]{{\mathbf 1}_{#1}}
\newcommand{\impulse}{{\mathbbm h}}
\newcommand{\frf}{\FFT{\impulse}}

%\renewcommand{\Moy}[1]{{\boldsymbol\mu}_{#1}}
%\renewcommand{\Rcor}[1]{{\boldsymbol R}_{#1}}
%\renewcommand{\Mw}[1]{{\boldsymbol M}_{#1}}
%\renewcommand{\Sw}[1]{{\boldsymbol S}_{#1}}
%\renewcommand{\esp}[1]{{\mathbb E}\{#1\}}

\newcommand{\emphb}[1]{\textcolor{blue}{#1}}
\newcommand{\mycite}[1]{\textcolor{red}{#1}}
\newcommand{\mycitb}[1]{\textcolor{red}{[{\it #1}]}}

\newcommand{\PDFU}{{\mathcal U}}
\newcommand{\PDFN}{{\mathcal N}}
\newcommand{\TK}{{\boldsymbol\Pi}}
\newcommand{\TKij}{\pi}
\newcommand{\TKi}{{\boldsymbol\pi}}
\newcommand{\SMi}{\TKij^*}
\newcommand{\SM}{\TKi^*}
\newcommand{\lagmuli}{\lambda}
\newcommand{\lagmul}{{\boldsymbol\lagmuli}}
\newcommand{\constraint}{{\boldsymbol C}}
\newcommand{\mconstraint}{\mean{\constraint}}

\newcommand{\mybox}[1]{\fbox{\begin{minipage}{0.93\textwidth}{#1}\end{minipage}}}
\newcommand{\defcolor}[1]{\textcolor{blue}{#1}}

%\definecolor{rose}{LightPink}%{rgb}{251,204,231}

\newtheorem{mydef}{Definition}
\newtheorem{mythe}{Theorem}
\newtheorem{myprop}{Proposition}

% Or whatever. Note that the encoding and the font should match. If T1
% does not look nice, try deleting the line with the fontenc.

\title[1EL5000/S5]
{Stiffness of an elastic linkage}

\subtitle{1EL5000--Continuum Mechanics -- Tutorial Class \#5} % (optional)

\author[\'E. Savin] % (optional, use only with lots of authors)
{\'E. Savin\inst{1,2}\\ \scriptsize{\texttt{eric.savin@\{centralesupelec,onera\}.fr}}}%\inst{1} }
% - Use the \inst{?} command only if the authors have different
%   affiliation.

\institute[Onera] % (optional, but mostly needed)
{\inst{1}{Information Processing and Systems Dept.\\\Onera, France}
\and
 \inst{2}{Mechanical and Civil Engineering Dept.\\\ECP, France}}%
%  Department of Theoretical Philosophy\\
%  University of Elsewhere}
% - Use the \inst command only if there are several affiliations.
% - Keep it simple, no one is interested in your street address.

%\date[Short Occasion] % (optional)
\date{\today}

\subject{Stiffness of an elastic linkage}
% This is only inserted into the PDF information catalog. Can be left
% out. 



% If you have a file called "university-logo-filename.xxx", where xxx
% is a graphic format that can be processed by latex or pdflatex,
% resp., then you can add a logo as follows:

% \pgfdeclareimage[height=0.5cm]{university-logo}{university-logo-filename}
% \logo{\pgfuseimage{university-logo}}



% Delete this, if you do not want the table of contents to pop up at
% the beginning of each subsection:
\AtBeginSection[]
%\AtBeginSubsection[]
{
  \begin{frame}<beamer>{Outline}
    \tableofcontents[currentsection]%,currentsubsection]
  \end{frame}
}


% If you wish to uncover everything in a step-wise fashion, uncomment
% the following command: 

%\beamerdefaultoverlayspecification{<+->}


\begin{document}

\begin{frame}
  \titlepage
\end{frame}

\begin{frame}{Outline}
  \tableofcontents
  % You might wish to add the option [pausesections]
\end{frame}


% Since this a solution template for a generic talk, very little can
% be said about how it should be structured. However, the talk length
% of between 15min and 45min and the theme suggest that you stick to
% the following rules:  

% - Exactly two or three sections (other than the summary).
% - At *most* three subsections per section.
% - Talk about 30s to 2min per frame. So there should be between about
%   15 and 30 frames, all told.

\section{Some algebra}
\subsection{Vector \& tensor products}

\begin{frame}{Some algebra}{Vector \& tensor products}

\begin{itemize}
\item Scalar product:
\begin{displaymath}
\av,\bv\in\Rset^a\,,\quad\scal{\av,\bv}=\sum_{j=1}^a\aj_j\bj_j=\aj_j\bj_j\,,
\end{displaymath}
The last equality is \emphb{Einstein's summation convention}.
\item Tensors and tensor product (or outer product):
\begin{displaymath}
\Av\in\Rset^a\to\Rset^b\,,\quad\Av=\av\otimes\bv\,,\quad\av\in\Rset^a\,,\bv\in\Rset^b\,.
\end{displaymath}
\item Tensor application to vectors:
\begin{displaymath}
\Av=\av\otimes\bv\in\Rset^a\to\Rset^b\,,\cv\in\Rset^b\,,\quad\Av\cv=\scal{\bv,\cv}\av\,.
\end{displaymath}
\item Product of tensors $\equiv$ composition of linear maps:
\begin{displaymath}
\Av=\av\otimes\bv\,,\Bv=\cv\otimes\dv\,,\quad\Av\Bv=\scal{\bv,\cv}\av\otimes\dv\,.
\end{displaymath}
\end{itemize}

\end{frame}

\begin{frame}{Some algebra}{Vector \& tensor products}

%\onslide<2|handout:2>

\begin{itemize}
\item Scalar product of tensors:
\begin{displaymath}
\scal{\Av,\Bv}=\trace(\Av\Bv^\itr):=\Av:\Bv=\Aj_{jk}\Bj_{jk}\,.
\end{displaymath}
\item Let $\{\ev_j\}_{j=1}^d$ be a Cartesian basis in $\Rset^d$. Then:
\begin{displaymath}
\begin{split}
\aj_j &=\scal{\av,\ev_j}\,, \\
\Aj_{jk} &=\scal{\Av,\ev_j\otimes\ev_k}=\Av:\ev_j\otimes\ev_k \\
&=\scal{\Av\ev_k,\ev_j}\,,
\end{split}
\end{displaymath}
such that:
\begin{displaymath}
\begin{split}
\av &=\aj_j\ev_j\,, \\
\Av &=\Aj_{jk}\ev_j\otimes\ev_k\,.
\end{split}
\end{displaymath}
\item Example: the identity matrix
\begin{displaymath}
\Id=\ev_j\otimes\ev_j\,.
\end{displaymath}
\end{itemize}

%\end{overprint}

\end{frame}

\subsection{Vector \& tensor analysis}

\begin{frame}{Some analysis}{Vector \& tensor analysis}

\begin{itemize}
\item Gradient of a vector function $\av(\xv)$, $\xv\in\Rset^d$:
\begin{displaymath}
\Gradx\av=\frac{\partial\av}{\partial\xj_j}\otimes\ev_j\,.
\end{displaymath}
\item Divergence of a vector function $\av(\xv)$, $\xv\in\Rset^d$:
\begin{displaymath}
\divx\av=\scal{\gradx,\av}=\trace(\Gradx\av)=\frac{\partial\aj_j}{\partial\xj_j}\,.
\end{displaymath}
\item Divergence of a tensor function $\Av(\xv)$, $\xv\in\Rset^d$:
\begin{displaymath}
\Divx\Av=\frac{\partial(\Av\ev_j)}{\partial\xj_j}\,.
\end{displaymath}
\end{itemize}

\end{frame}

\begin{frame}{Some analysis}{Vector \& tensor analysis in cylindrical coordinates}

\begin{itemize}
\item Gradient of a vector function $\av(r,\theta,\zj)$:
\begin{displaymath}
\Gradx\av=\frac{\partial\av}{\partial r}\otimes\ev_r+\frac{\partial\av}{\partial\theta}\otimes\frac{\ev_\theta}{r}+\frac{\partial\av}{\partial\zj}\otimes\ev_\zj\,.
\end{displaymath}
\item Divergence of a vector function $\av(r,\theta,\zj)$:
\begin{displaymath}
\divx\av=\scal{\frac{\partial\av}{\partial r},\ev_r}+\scal{\frac{\partial\av}{\partial\theta},\frac{\ev_\theta}{r}}+\scal{\frac{\partial\av}{\partial\zj},\ev_\zj}\,.
\end{displaymath}
\item Divergence of a tensor function $\Av(r,\theta,\zj)$:
\begin{displaymath}
\Divx\Av=\frac{\partial\Av}{\partial r}\ev_r+\frac{\partial\Av}{\partial\theta}\frac{\ev_\theta}{r}+\frac{\partial\Av}{\partial\zj}\ev_\zj\,.
\end{displaymath}
\end{itemize}

\end{frame}



\section{Linear elasticity}
\begin{frame}{Recap}

\begin{itemize}
\item Local equilibrium equation:
\begin{displaymath}
\Divx\stress+\fv_v=\roi\ddot{\uv}\,;
\end{displaymath}
\item Small strains assumption:
\begin{displaymath}
\strain=\demi(\Gradx\uv+\Gradx\uv^\itr)\,;
\end{displaymath}
\item Material constitutive equation:
\begin{displaymath}
\stress=\tenselas\strain
\end{displaymath}
($\stress=\lambda\trace(\strain)\Id+2\mu\strain$ for isotropic elasticity).
\end{itemize}

\end{frame}

\begin{frame}{Navier equation}{Elastic waves}

\begin{overprint}

\onslide<1|handout:1>
\begin{itemize}
\item $\stress_\xv(\uv)=\tenselas(\strain_\xv(\uv))$, $\strain_\xv(\uv)=\gradx\otimes_s\uv$, and:
\begin{displaymath}
\begin{array}{c}
\Divx\stress_\xv(\uv)+\fv_v=\roi\partial_t^2\uv \\
(\lambda+\mu)\gradx(\divx\uv)+\mu{\boldsymbol\Delta}_\xv\uv+\fv=\roi\partial_t^2\uv \\
(\lambda+\mu)\sum_{k=1}^3\frac{\partial^2\uj_k}{\partial\xj_j\partial\xj_k}+\mu\sum_{k=1}^3\frac{\partial^2\uj_j}{\partial\xj_k^2}+\fj_{vj}=\roi\frac{\partial^2\uj_j}{\partial t^2}
\end{array}
\end{displaymath}
%where ${\boldsymbol\Delta}_\xv=\Divx(\Gradx)$ (vector Laplacian).
\item Body waves {\small(Poisson 1828)}
\begin{figure}
\centering\includegraphics[scale=.3]{\figs/ch5-BodyWaves}
\end{figure}
\end{itemize}

\onslide<2|handout:2>
\begin{itemize}
\item $\stress_\xv(\uv)=\tenselas(\strain_\xv(\uv))$, $\strain_\xv(\uv)=\gradx\otimes_s\uv$, and:
\begin{displaymath}
\begin{array}{c}
\Divx\stress_\xv(\uv)+\fv_v=\roi\partial_t^2\uv \\
(\lambda+\mu)\gradx(\divx\uv)+\mu{\boldsymbol\Delta}_\xv\uv+\fv=\roi\partial_t^2\uv \\
(\lambda+\mu)\sum_{k=1}^3\frac{\partial^2\uj_k}{\partial\xj_j\partial\xj_k}+\mu\sum_{k=1}^3\frac{\partial^2\uj_j}{\partial\xj_k^2}+\fj_{vj}=\roi\frac{\partial^2\uj_j}{\partial t^2}
\end{array}
\end{displaymath}
%where ${\boldsymbol\Delta}_\xv=\Divx(\Gradx)$ (vector Laplacian).
\item Surface waves {\small(Rayleigh 1885, Love 1911, Stoneley 1924)}
\begin{figure}
\centering\includegraphics[scale=.3]{\figs/ch5-SurfaceWaves}
\end{figure}
\end{itemize}

\onslide<3|handout:3>
\begin{columns}[t]
\column{.4\textwidth}
\centering\includegraphics[scale=0.3]{\figs/ch5-Navier-portrait}
\column{.6\textwidth}
\vskip-130pt
Claude Louis Marie Henri Navier [1785-1836]
\begin{itemize}
\item X 1802
\item Ing\'enieur Ponts-et-Chauss\'ees
\item Acad\'emie des Sciences 1824
\item Prof. Analyse et M\'ecanique \`a l'X 1831-1836
\item \emph{Le Cur� de village} (1841)
\end{itemize}
\end{columns}
\href{https://mathshistory.st-andrews.ac.uk/Biographies/Navier/}{\tiny\texttt{https://mathshistory.st-andrews.ac.uk/Biographies/Navier/}}

\end{overprint}

\end{frame}

\begin{frame}{Navier equation}{Initial conditions}

\begin{itemize}
\item Initial displacement:
\begin{displaymath}
\uv(\xv,0)=\uv_0(\xv)\,,\quad\forall\xv\in\medium\,;
\end{displaymath}
\item Initial velocity:
\begin{displaymath}
\frac{\partial\uv}{\partial t}(\xv,0)=\vv_0(\xv)\,,\quad\forall\xv\in\medium\,.
\end{displaymath}
\end{itemize}

\end{frame}

\begin{frame}{Navier equation}{Boundary conditions}

%\begin{columns}[t]
%\column{.33\textwidth}
%\centering\includegraphics[scale=.25]{\figs/ch4-Isotropic}
%\column{.33\textwidth}
%\centering\includegraphics[scale=.25]{\figs/ch4-Anisotropic1}
%\column{.33\textwidth}
%\centering\includegraphics[scale=.25]{\figs/ch4-Anisotropic2}
%\end{columns}
\begin{itemize}
\item Rigid contact:
\begin{displaymath}
\uv(\xv,t)=\uv_S(\xv,t)\,,\quad\forall\xv\in\partial\medium_\uj\,,\forall t\,;
\end{displaymath}
\item Soft contact:
\begin{displaymath}
\stress(\xv,t)\nv(\xv)=\fv_S(\xv,t)\,,\quad\forall\xv\in\partial\medium_\stressj\,,\forall t\,;
\end{displaymath}
\item Internal full contact with \emph{e.g.} $\nv=\nv_1=-\nv_2$:
\begin{displaymath}
\begin{array}{c}
\uv_1(\xv,t)=\uv_2(\xv,t)\,, \\
\stress_1(\xv,t)\nv(\xv)=\stress_2(\xv,t)\nv(\xv)\,,
\end{array}
\end{displaymath}
$\forall\xv\in\partial\medium_1\cap\partial\medium_2$, $\forall t$;
\item Internal sliding contact without friction:
\begin{displaymath}
\begin{array}{c}
\scal{\uv_1(\xv,t),\nv(\xv)}=\scal{\uv_2(\xv,t),\nv(\xv)}\,, \\
\scal{\stress_1(\xv,t)\nv(\xv),{\boldsymbol\tau}}=\scal{\stress_2(\xv,t)\nv(\xv),{\boldsymbol\tau}}=\bzero\,,
\end{array}
\end{displaymath}
$\forall{\boldsymbol\tau}\perp\nv(\xv)$, $\forall\xv\in\partial\medium_1\cap\partial\medium_2$, $\forall t$.
\end{itemize}

\end{frame}

\begin{frame}{Navier equation}{Properties}

\begin{itemize}
\item \emphb{Existence} of a solution:
\begin{itemize}
\item Holds in statics provided that global equilibrium is satisfied;
\item Holds in dynamics.
\end{itemize}
\item \emphb{Uniqueness} of a solution:
\begin{itemize}
\item Holds in statics for strains, stresses, and displacements up to a rigid-body motion;
\item Holds in dynamics for strains, stresses, and displacements.
\end{itemize}
\item \emphb{Linearity} (superposition principle, symmetries...).
\end{itemize}

\end{frame}

\begin{frame}{Saint-Venant's principle}{...in homogeneous, linear elastic media}

\begin{overprint}

\onslide<1|handout:1>
\begin{figure}
\centering\includegraphics[scale=.2]{\figs/ch5-StVenant}
\end{figure}
\vskip-10pt
\begin{quote}
\item "...the difference between the effects of two different but statically equivalent loads becomes very small at sufficiently large distances from load."\footnote{\tiny A. J. C. B. Saint-Venant, M\'emoire sur la Torsion des Prismes, \emph{Mem. Divers Savants} {\bf 14}, 233-560 (1855).}
\end{quote}
\scriptsize{\par\hfill{Adh\'emar Barr\'e de Saint-Venant [1797-1886]}}

\onslide<2|handout:2>
\begin{columns}[t]
\column{.4\textwidth}
\centering\includegraphics[scale=.5]{\figs/ch5-StVenant-portrait}
\column{.6\textwidth}
\vskip-80pt
Adh\'emar Jean Claude Barr\'e de Saint-Venant [1797-1886]
\begin{itemize}
\item X 1813
\item Ing\'enieur Ponts-et-Chauss\'ees
\item Acad\'emie des Sciences 1868
\end{itemize}
\end{columns}
\href{https://mathshistory.st-andrews.ac.uk/Biographies/Saint-Venant/}{\tiny\texttt{https://mathshistory.st-andrews.ac.uk/Biographies/Saint-Venant/}}

\end{overprint}

\end{frame}

\begin{frame}{Thermoelasticity}{Isotropic case}

\begin{itemize}
\item Thermal strains in the isotropic case with temperature gradient $\Delta T$:
\begin{displaymath}
\strain_\text{th}=\alpha\Delta T\Id
\end{displaymath}
where $\alpha$ is the coefficient of linear thermal expansion.
\item Total strain tensor:
\begin{displaymath}
\begin{split}
\strain &=\strain_\text{elas}+\strain_\text{th} \\
&=\tenscomp\stress+\alpha\Delta T\Id\,.
\end{split}
\end{displaymath}
\item Linear thermoelastic constitutive equation:
\begin{displaymath}
\boxed{\stress=\tenselas(\strain-\alpha\Delta T\Id)}\,.
\end{displaymath}
\end{itemize}

\end{frame}



\section{5.1 Stiffness of an elastic linkage}

\begin{frame}{Stiffness of an elastic linkage}{Setup}

\begin{columns}[t]
\column{.5\textwidth}
\centering\includegraphics[scale=.4]{\figs/ch5-Linkage}
\vskip-10pt{\hspace{2.5truecm}\mbox{\tiny{\copyright\ G. Puel}}}
\column{.5\textwidth}
\vskip-90pt
\centering\includegraphics[scale=.5]{\figs/ch5-Tieback}
\end{columns}
\vskip20pt
\begin{displaymath}
\uv_{fd}(\xv)=\uj_\zj(r)\iv_z\,,
\end{displaymath}
where:
\begin{displaymath}
\xv\in\medium=\{\xv=(r,\theta,\zj);\,r_i< r <r_e,\,0\leq\theta\leq 2\pi,\,0<z<L\}\,.
\end{displaymath}
\end{frame}

\begin{frame}{Stiffness of an elastic linkage}{Solution}

\begin{exampleblock}{Question \#0: Why $\uv_{fd}(\xv)=\uj_\zj(r)\iv_z$?}
\begin{itemize}
\item $\uv(\xv)=\uj_r(r,\theta,\zj)\iv_r+\uj_\theta(r,\theta,\zj)\iv_\theta+\uj_\zj(r,\theta,\zj)\iv_\zj$ but the problem is axisymmetric, hence $\uj_\theta=0$ and $\uj_r,\uj_\zj$ do not depend on $\theta$;
\item The problem is also invariant vs. $\zj$ hence $\uj_r,\uj_\zj$ do not depend on $\zj$ either;
\item Lastly the problem is radially constrained such that $\uj_r=0$.
\end{itemize}
\end{exampleblock}

\end{frame}

\begin{frame}{Stiffness of an elastic linkage}{Solution}

\begin{overprint}

\onslide<1|handout:1>
\vskip-20pt
\begin{exampleblock}{Question \#1: Equations satisfied by $\uv$ and $\stress$?}
\begin{enumerate}
\item Local equilibrium equation $\Divx\stress+\fv_v=\smash{\roi\frac{\partial^2\uv}{\partial t^2}}$, where "the effects of inertia and the action of gravity can be neglected:"
\begin{displaymath}
\Divx\stress=\bzero\,,\quad\xv\in\medium\,.
\end{displaymath}
\end{enumerate}
\end{exampleblock}

\onslide<2|handout:2>
\vskip-20pt
\begin{exampleblock}{Question \#1: Equations satisfied by $\uv$ and $\stress$?}
\begin{enumerate}
\item Local equilibrium equation $\Divx\stress=\bzero$;
\item Material constitutive equation:
\begin{displaymath}
\begin{split}
\stress &=\lambda\trace(\strain)\Id+2\mu\strain\,, \\
\text{or}\;\strain &=\frac{1+\nu}{E}\stress-\frac{\nu}{E}\trace(\stress)\Id\,,
\end{split}
\end{displaymath}
since "the constitutive material is isotropic homogeneous, and linear elastic, of Lam� parameters $(\lambda,\mu)$."
\end{enumerate}
\end{exampleblock}

\onslide<3|handout:3>
\vskip-20pt
\begin{exampleblock}{Question \#1: Equations satisfied by $\uv$ and $\stress$?}
\begin{enumerate}
\item Local equilibrium equation $\Divx\stress=\bzero$;
\item Material constitutive equation $\stress=\lambda\trace(\strain)\Id+2\mu\strain$;
\item Initial conditions: here they are not needed since the problem is time-independent (static).
\end{enumerate}
\end{exampleblock}

\onslide<4|handout:4>
\vskip-20pt
\begin{exampleblock}{Question \#1: Equations satisfied by $\uv$ and $\stress$?}
\begin{enumerate}
\item Local equilibrium equation $\Divx\stress=\bzero$;
\item Material constitutive equation $\stress=\lambda\trace(\strain)\Id+2\mu\strain$;
\item Initial conditions (if needed);
\item Boundary conditions on $\partial\medium$, where:
\begin{displaymath}
\partial\medium=\{r=r_e\}\cup\{r=r_i\}\cup\{\zj=0\}\cup\{\zj=L\}\,.
\end{displaymath}
\end{enumerate}
\end{exampleblock}

\onslide<5|handout:5>
\vskip-20pt
\begin{exampleblock}{Question \#1: Equations satisfied by $\uv$ and $\stress$?}
\begin{enumerate}
\item Local equilibrium equation $\Divx\stress=\bzero$;
\item Material constitutive equation $\stress=\lambda\trace(\strain)\Id+2\mu\strain$;
\item Initial conditions (if needed);
\item Boundary conditions on $\partial\medium$, where:
\begin{displaymath}
\partial\medium=\{r=r_e\}\cup\{r=r_i\}\cup\{\zj=0\}\cup\{\zj=L\}\,;
\end{displaymath}
\begin{itemize}
\item $\{r=r_e\}$: $\uv=\bzero$ since the "outer lateral surface is glued on a cylindrical support of the same radius, assumed fixed and perfectly rigid;"
\end{itemize}
\end{enumerate}
\end{exampleblock}

\onslide<6|handout:6>
\vskip-20pt
\begin{exampleblock}{Question \#1: Equations satisfied by $\uv$ and $\stress$?}
\begin{enumerate}
\item Local equilibrium equation $\Divx\stress=\bzero$;
\item Material constitutive equation $\stress=\lambda\trace(\strain)\Id+2\mu\strain$;
\item Initial conditions (if needed);
\item Boundary conditions on $\partial\medium$, where:
\begin{displaymath}
\partial\medium=\{r=r_e\}\cup\{r=r_i\}\cup\{\zj=0\}\cup\{\zj=L\}\,;
\end{displaymath}
\begin{itemize}
\item $\{r=r_e\}$: $\uv=\bzero$;
\item $\{r=r_i\}$: $\uv=\uj_{\zj d}\iv_\zj$ since the "inner lateral surface is glued to a perfectly rigid cylinder of the same radius, whose overall movement is constrained as $\uv_d= \uj_{\zj d}\iv_\zj$;"
\end{itemize}
\end{enumerate}
\end{exampleblock}

\onslide<7|handout:7>
\vskip-20pt
\begin{exampleblock}{Question \#1: Equations satisfied by $\uv$ and $\stress$?}
\begin{enumerate}
\item Local equilibrium equation $\Divx\stress=\bzero$;
\item Material constitutive equation $\stress=\lambda\trace(\strain)\Id+2\mu\strain$;
\item Initial conditions (if needed);
\item Boundary conditions on $\partial\medium$, where:
\begin{displaymath}
\partial\medium=\{r=r_e\}\cup\{r=r_i\}\cup\{\zj=0\}\cup\{\zj=L\}\,;
\end{displaymath}
\begin{itemize}
\item $\{r=r_e\}$: $\uv=\bzero$;
\item $\{r=r_i\}$: $\uv=\uj_{\zj d}\iv_\zj$;
\item $\{\zj=0\}$: $\stress\nv=\bzero$, $\nv=-\iv_\zj$, since "the end faces are assumed to be free of forces;"
\end{itemize}
\end{enumerate}
\end{exampleblock}

\onslide<8|handout:8>
\vskip-20pt
\begin{exampleblock}{Question \#1: Equations satisfied by $\uv$ and $\stress$?}
\begin{enumerate}
\item Local equilibrium equation $\Divx\stress=\bzero$;
\item Material constitutive equation $\stress=\lambda\trace(\strain)\Id+2\mu\strain$;
\item Initial conditions (if needed);
\item Boundary conditions on $\partial\medium$, where:
\begin{displaymath}
\partial\medium=\{r=r_e\}\cup\{r=r_i\}\cup\{\zj=0\}\cup\{\zj=L\}\,;
\end{displaymath}
\begin{itemize}
\item $\{r=r_e\}$: $\uv=\bzero$;
\item $\{r=r_i\}$: $\uv=\uj_{\zj d}\iv_\zj$;
\item $\{\zj=0\}$: $\stress\iv_\zj=\bzero$ ($\nv=-\iv_\zj$);
\item $\{\zj=L\}$: $\stress\iv_\zj=\bzero$ ($\nv=+\iv_\zj$).
\end{itemize}
\end{enumerate}
\end{exampleblock}

\end{overprint}

\end{frame}

\begin{frame}{Stiffness of an elastic linkage}{Solution}

\begin{overprint}

\onslide<1|handout:1>
\vskip-20pt
\begin{exampleblock}{Question \#2: $\strain_{fd}$?}
\begin{itemize}
\item In cylindrical coordinates:
\begin{displaymath}
\strain=\frac{\partial\uv}{\partial r}\otimes_s\iv_r+\frac{\partial\uv}{\partial\theta}\otimes_s\frac{\iv_\theta}{r}+\frac{\partial\uv}{\partial\zj}\otimes_s\iv_\zj\,.
\end{displaymath}
\end{itemize}
\end{exampleblock}

\onslide<2|handout:2>
\vskip-20pt
\begin{exampleblock}{Question \#2: $\strain_{fd}$?}
\begin{itemize}
\item In cylindrical coordinates:
\begin{displaymath}
\strain=\frac{\partial\uv}{\partial r}\otimes_s\iv_r+\frac{\partial\uv}{\partial\theta}\otimes_s\frac{\iv_\theta}{r}+\frac{\partial\uv}{\partial\zj}\otimes_s\iv_\zj\,.
\end{displaymath}
\item Here $\uv=\uj_\zj(r)\iv_\zj$ thus:
\begin{displaymath}
\begin{split}
\frac{\partial\uv}{\partial r} &=\uj_\zj'(r)\iv_\zj\,,
\end{split}
\end{displaymath}
where $\uj_\zj'(r)=\frac{\id\uj_\zj}{\id r}$.
\end{itemize}
\end{exampleblock}

\onslide<3|handout:3>
\vskip-20pt
\begin{exampleblock}{Question \#2: $\strain_{fd}$?}
\begin{itemize}
\item In cylindrical coordinates:
\begin{displaymath}
\strain=\frac{\partial\uv}{\partial r}\otimes_s\iv_r+\frac{\partial\uv}{\partial\theta}\otimes_s\frac{\iv_\theta}{r}+\frac{\partial\uv}{\partial\zj}\otimes_s\iv_\zj\,.
\end{displaymath}
\item Here $\uv=\uj_\zj(r)\iv_\zj$ thus:
\begin{displaymath}
\begin{split}
\frac{\partial\uv}{\partial r} &=\uj_\zj'(r)\iv_\zj\,; \\
\frac{\partial\uv}{\partial\theta} &=\frac{\partial\uv}{\partial\zj}=\bzero\,.
\end{split}
\end{displaymath}
\item Therefore:
\begin{displaymath}
\strain_{fd}=\uj_\zj'(r)\iv_r(\theta)\otimes_s\iv_\zj\,.
\end{displaymath}
\end{itemize}
\end{exampleblock}

\onslide<4|handout:4>
\vskip-20pt
\begin{exampleblock}{Question \#2: $\stress_{fd}$?}
\begin{itemize}
\item $\strain_{fd}=\uj_\zj'(r)\iv_r(\theta)\otimes_s\iv_\zj$;
\item Constitutive relation $\stress_{fd}=\lambda(\trace\strain_{fd})\Id+2\mu\strain_{fd}$; therefore $\stress_{fd}=2\mu\uj_\zj'(r)\iv_r(\theta)\otimes_s\iv_\zj$ since $\trace(\strain_{fd})=0$.
\end{itemize}
\end{exampleblock}

\end{overprint}

\end{frame}

\begin{frame}{Stiffness of an elastic linkage}{Solution}

\begin{overprint}

\onslide<1|handout:1>
\vskip-20pt
\begin{exampleblock}{Question \#3: Differential equation satisfied by $\uj_\zj$?}
\begin{itemize}
\item Local equilibrium equation $\Divx\stress_{fd}=\bzero$, where $\stress_{fd}=2\mu\uj_\zj'(r)\iv_r(\theta)\otimes_s\iv_\zj$;
\item In cylindrical coordinates:
\begin{displaymath}
\begin{split}
\Divx\stress_{fd} &=\frac{\partial\stress_{fd}}{\partial r}\iv_r+\frac{\partial\stress_{fd}}{\partial \theta}\frac{\iv_\theta}{r}+\frac{\partial\stress_{fd}}{\partial\zj}\iv_\zj \\
&=\mu\uj_\zj''(r)\iv_\zj+\mu\frac{\uj_\zj'(r)}{r}\iv_\zj\,.
\end{split}
\end{displaymath}
\end{itemize}
\end{exampleblock}

\onslide<2|handout:2>
\vskip-20pt
\begin{exampleblock}{Question \#3: Differential equation satisfied by $\uj_\zj$?}
\begin{itemize}
\item Local equilibrium equation $\Divx\stress_{fd}=\bzero$, where $\stress_{fd}=2\mu\uj_\zj'(r)\iv_r(\theta)\otimes_s\iv_\zj$;
\item In cylindrical coordinates:
\begin{displaymath}
\begin{split}
\Divx\stress_{fd} &=\frac{\partial\stress_{fd}}{\partial r}\iv_r+\frac{\partial\stress_{fd}}{\partial \theta}\frac{\iv_\theta}{r}+\frac{\partial\stress_{fd}}{\partial\zj}\iv_\zj \\
&=\mu\uj_\zj''(r)\iv_\zj+\mu\frac{\uj_\zj'(r)}{r}\iv_\zj\,;
\end{split}
\end{displaymath}
\item Hence $r\mapsto\uj_\zj(r)$ satisfies:
\begin{displaymath}
\uj_\zj''(r)+\frac{\uj_\zj'(r)}{r}=\frac{1}{r}(r\uj_\zj'(r))'=0\,.
\end{displaymath}
\end{itemize}
\end{exampleblock}

\end{overprint}

\end{frame}

\begin{frame}{Stiffness of an elastic linkage}{Solution}

\begin{exampleblock}{Question \#4: $r\mapsto\uj_\zj(r)$?}
\begin{itemize}
\item From question \#3, $\uj_\zj(r)=A\ln r + B$ where $A$ and $B$ are given by the boundary conditions.
\item On $\{r=r_e\}$, $\uj_\zj(r_e)=A\ln r_e+B=0$;
\item On $\{r=r_i\}$, $\uj_\zj(r_i)=A\ln r_i+B=\uj_{\zj d}$;
\item Hence:
\begin{displaymath}
\boxed{\uj_\zj(r)=\uj_{\zj d}\frac{\ln r_e-\ln r}{\ln r_e-\ln r_i}}\,.
\end{displaymath}
\end{itemize}
\end{exampleblock}

\end{frame}

\begin{frame}{Stiffness of an elastic linkage}{Solution}

\begin{overprint}

\onslide<1|handout:1>
\vskip-20pt
\begin{exampleblock}{Question \#5: Stiffness $K$?}
\begin{itemize}
\item $K=\frac{\norm{{\boldsymbol R}^e}}{\abs{\uj_{\zj d}}}$ where ${\boldsymbol R}^e$ is "the resultant force of the action of the moving cylinder on the cylinder $\medium$, on its inner lateral surface."
\end{itemize}
\end{exampleblock}

\onslide<2|handout:2>
\vskip-20pt
\begin{exampleblock}{Question \#5: Stiffness $K$?}
\begin{itemize}
\item $K=\frac{\norm{{\boldsymbol R}^i}}{\abs{\uj_{\zj d}}}$;
\item Resultant force of the action of the inner cylinder on $\medium$:
\begin{displaymath}
\begin{split}
{\boldsymbol R}^i &\overset{\text{def}}{=}\int_{\{r=r_i\}}\stress_{fd}\nv \id S \\
 &=\int_0^L\int_0^{2\pi} \stress_{fd}(-\iv_r(\theta)) r_i \id\theta \id\zj \\
 &=-\int_0^L\int_0^{2\pi}\mu\uj_\zj'(r_i)\iv_\zj r_i\id\theta \id\zj  \\
 &=\frac{2\pi\mu L\uj_{\zj d}}{\ln r_e-\ln r_i}\iv_\zj\,.
\end{split}
\end{displaymath}
\end{itemize}
\end{exampleblock}

\onslide<3|handout:3>
\vskip-20pt
\begin{exampleblock}{Question \#5: Stiffness $K$?}
\begin{itemize}
\item $K=\frac{\norm{{\boldsymbol R}^i}}{\abs{\uj_{\zj d}}}$;
\item Resultant force of the action of the inner cylinder on $\medium$:
\begin{displaymath}
{\boldsymbol R}^i =\frac{2\pi\mu L\uj_{\zj d}}{\ln\frac{r_e}{r_i}}\iv_\zj\,;
\end{displaymath}
\item Therefore the stiffness is:
\begin{displaymath}
\boxed{K=\frac{2\pi\mu L}{\ln\frac{r_e}{r_i}}}\,.
\end{displaymath}
\end{itemize}
\end{exampleblock}

\end{overprint}

\end{frame}

\begin{frame}{Stiffness of an elastic linkage}{Solution}

\begin{overprint}

\onslide<1|handout:1>
\vskip-20pt
\begin{exampleblock}{Question \#6: Stress vector at the end faces?}
\begin{itemize}
\item Stress vector at the end faces $\{\zj=0\}\cup\{\zj=L\}$: $\stress\nv=\stress(\mp\iv_\zj)$ for either faces;
\item But $\stress_{fd}=2\mu\uj_\zj'(r)\iv_r(\theta)\otimes_s\iv_\zj$, hence:
\begin{displaymath}
\begin{split}
\stress_{fd}(-\iv_\zj)\mid_{\zj=0} &=-\stress_{fd}\iv_\zj\mid_{\zj=L} \\
&=-\mu\uj_\zj'(r)\iv_r(\theta) \\
& \neq\bzero \;\text{!!!}
\end{split}
\end{displaymath}
\end{itemize}
\end{exampleblock}

\onslide<2|handout:2>
\vskip-20pt
\begin{exampleblock}{Question \#6: Stress vector at the end faces?}
\begin{itemize}
\item Stress vector at the end faces $\{\zj=0\}\cup\{\zj=L\}$: $\stress\nv=\stress(\mp\iv_\zj)$ for either faces;
\item But $\stress_{fd}=2\mu\uj_\zj'(r)\iv_r(\theta)\otimes_s\iv_\zj$, hence:
\begin{displaymath}
\stress_{fd}(-\iv_\zj)\mid_{\zj=0}=-\stress_{fd}\iv_\zj\mid_{\zj=L}\neq\bzero \;\text{!!!}
\end{displaymath}
\item The boundary conditions (free surfaces) at the end faces are not satisfied, thus the ansatz "$\uv(\xv)=\uj_\zj(r)\iv_z$" is incorrect.
\end{itemize}
\end{exampleblock}

\onslide<3|handout:3>
\vskip-20pt
\begin{exampleblock}{Question \#6: Stress vector at the end faces?}
\begin{itemize}
\item Stress vector at the end faces $\{\zj=0\}\cup\{\zj=L\}$: $\stress\nv=\stress(\mp\iv_\zj)$ for either faces;
\item But $\stress_{fd}=2\mu\uj_\zj'(r)\iv_r(\theta)\otimes_s\iv_\zj$, hence:
\begin{displaymath}
\stress_{fd}(-\iv_\zj)\mid_{\zj=0}=-\stress_{fd}\iv_\zj\mid_{\zj=L}\neq\bzero \;\text{!!!}
\end{displaymath}
\item The boundary conditions (free surfaces) at the end faces are not satisfied, thus the ansatz "$\uv(\xv)=\uj_\zj(r)\iv_z$" is incorrect;
\item However one can show that the resultant forces vanish, and invoke the \emphb{Saint-Venant principle} to claim that this ansatz gives a good approximation of the solution away from the end faces.
\end{itemize}
\end{exampleblock}

\onslide<4|handout:4>
\vskip-20pt
\begin{exampleblock}{Question \#6: Stress vector at the end faces?}
\begin{itemize}
\item Resultant force at $\{z=L\}$:
\begin{displaymath}
\begin{split}
{\boldsymbol R}^L &\overset{\text{def}}{=}\int_{\{z=L\}}\stress_{fd}\iv_\zj \id S \\
&=\int_{r_i}^{r_e}\int_0^{2\pi}\mu\uj_\zj'(r)\iv_r(\theta) r\id r\id\theta \\
&=\frac{-\mu \uj_{\zj d}}{\ln r_e -\ln r_i}\int_{r_i}^{r_e}\id r\cancel{\int_0^{2\pi}\iv_r(\theta)\id\theta} \\
&=\bzero\,;
\end{split}
\end{displaymath}
\item ${\boldsymbol R}^0=-{\boldsymbol R}^L=\bzero$.
\end{itemize}
\end{exampleblock}

\onslide<5|handout:5>
\vskip-20pt
\begin{exampleblock}{Question \#6: Stress vector at the end faces?}
\begin{itemize}
\item Resultant moment at $\{z=L\}$:
\begin{displaymath}
\begin{split}
{\boldsymbol M}^L &\overset{\text{def}}{=}\int_{\{z=L\}}r\iv_r(\theta)\times\stress_{fd}\iv_\zj \id S \\
&=\int_{r_i}^{r_e}\int_0^{2\pi}r\iv_r(\theta)\times\mu\uj_\zj'(r)\iv_r(\theta) r\id r\id\theta \\
&=\frac{-\mu \uj_{\zj d}}{\ln r_e -\ln r_i}\int_{r_i}^{r_e}r\id r\cancel{\int_0^{2\pi}\iv_r(\theta)\times\iv_r(\theta)\id\theta} \\
&=\bzero\,;
\end{split}
\end{displaymath}
\item ${\boldsymbol M}^0=-{\boldsymbol M}^L=\bzero$.
\end{itemize}
\end{exampleblock}

\end{overprint}

\end{frame}

\end{document}

