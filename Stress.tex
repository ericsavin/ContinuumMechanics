\begin{frame}{Cauchy's stress tensor}{Modeling}

\begin{figure}
\centering\includegraphics[scale=.25]{\figs/ch2-Domains}
\end{figure}
\begin{itemize}
\item Modeling the actions of $V_t^*=\medium_t\setminus V_t$ on $V_t$:
\begin{itemize}
\item Local contact actions $\Tv$ exerted by proximal subdomains with short ranges;
\item Virtual cutting surface $\partial V_t$ with outward unit normal $\nv(\xv)$ defining the tangent plane at $\xv$.
\end{itemize}
\end{itemize}

\end{frame}

\begin{frame}{Cauchy's stress tensor}{Cauchy's postulates}

\begin{figure}
\centering\includegraphics[scale=.25]{\figs/ch2-Domains}
\end{figure}
\begin{itemize}
\item\emphb{Cauchy's postulate \#1}: surface traction (in N/m$^2$)
\begin{displaymath}
\Tv(\xv,t;\partial V_t)=\fv_{\partial V_t}(\xv,t)\,,\quad\forall\xv\in\partial V_t\,,\forall V_t\,.
\end{displaymath}
\item\emphb{Cauchy's postulate \#2}: the surface curvature has no influence on the traction
\begin{displaymath}
\Tv(\xv,t;\partial V_t)=\Tv(\xv,t;\nv(\xv))\,,\quad\forall\xv\in\partial V_t\,,\forall V_t\,.
\end{displaymath}
\end{itemize}

\end{frame}

\begin{frame}{Cauchy's stress tensor}{Cauchy's theorem}

\begin{figure}
\centering\includegraphics[scale=.25]{\figs/ch2-Cauchy2}
\end{figure}
\begin{itemize}
\item\emphb{Cauchy's theorem}: consider elementary $V_t$ (thin cylinder, tetrahedron), then $\forall\xv\in\partial V_t$
\begin{enumerate}
\item Action/reaction $\Tv(\xv,t;\nv(\xv))+\Tv(\xv,t;-\nv(\xv))=\bzero$;
\item Tetrahedron's lemma $\Tv(\xv,t;\nv(\xv))=\stress(\xv,t)\nv(\xv)$.
\end{enumerate}
\item Boundary conditions:
\begin{displaymath}
\stress(\xv,t)\nv(\xv)=\fv_S(\xv,t)\,,\quad\forall\xv\in\partial\medium_t\,.
\end{displaymath}
\end{itemize}

\end{frame}

\begin{frame}{Cauchy's stress tensor}{Cauchy!}

\begin{columns}[t]
\column{.4\textwidth}
\centering\includegraphics[scale=.15]{\figs/ch2-Cauchy-portrait}
\column{.6\textwidth}
\vskip-100pt
Augustin Louis Cauchy [1789-1857]
\begin{itemize}
\item X 1805
\item Ing\'enieur Ponts-et-Chauss\'ees
\item Acad\'emie des Sciences 1816
\item Prof. Analyse et M\'ecanique \`a l'X 1815-1830
\end{itemize}
\end{columns}
\href{https://mathshistory.st-andrews.ac.uk/Biographies/Cauchy/}{\tiny\texttt{https://mathshistory.st-andrews.ac.uk/Biographies/Cauchy/}}

\end{frame}

\begin{frame}{Equations of motion}{}

\begin{overprint}

\onslide<1|handout:1>
\begin{figure}
\centering\includegraphics[scale=.25]{\figs/ch2-Cauchy2}
\end{figure}
\begin{itemize}
\item $\forall V_t\subseteq\medium_t$:
\begin{displaymath}
\begin{split}
\int_{V_t}\roi\acv\id V &=\int_{V_t}\fv_v\id V + \int_{\partial V_t}\Tv\id S\,, \\
\int_{V_t}\xv\times\roi\acv\id V &=\int_{V_t}\xv\times\fv_v\id V + \int_{\partial V_t}\xv\times\Tv\id S\,.
\end{split}
\end{displaymath}
\end{itemize}

\onslide<2|handout:2>
\begin{itemize}
\item $\forall V_t\subseteq\medium_t$, $\forall\cv\in\Rset^3$:
\begin{displaymath}
\begin{split}
\int_{V_t}\scal{\roi\acv-\fv_v,\cv}\id V &=\int_{\partial V_t}\scal{\stress\nv,\cv}\id S \\
&=\int_{\partial V_t}\scal{\nv,\stress^\itr\cv}\id S \\
&=\int_{V_t}\div(\stress^\itr\cv)\id V\quad\;\;(\text{Stokes formula}) \\
&\overset{\text{def}}{=}\int_{V_t}\scal{\Div\stress,\cv}\id V \\
\int_{V_t}(\roi\acv-\fv_v)\id V &=\int_{V_t}\Div\stress\,\id V \\
\roi\acv-\fv_v &=\Div\stress\quad\quad\quad\quad(\text{localization lemma})
\end{split}
\end{displaymath}
\end{itemize}

\onslide<3|handout:3>
\begin{itemize}
\item The equation of motion for a continuous medium $\xv\in\medium_t$:
\begin{displaymath}
\boxed{\roi\acv=\Divx\stress+\fv_v}\,,
\end{displaymath}
where $\Divx\stress\approx\frac{1}{V}\int_{\partial V}\stress\nv\id S$ as $\abs{V}\to 0$.
\item The balance of momentum yields:
\begin{displaymath}
\stress^\itr=\stress\,.
\end{displaymath}
This is no longer the case if Cauchy's postulate \#1 does not hold (surface torques N.m/m$^2$).
\item Boundary conditions $\xv\in\partial \medium_t$:
\begin{displaymath}
\stress\nv=\fv_S\,.
\end{displaymath}

\end{itemize}

\end{overprint}

\end{frame}