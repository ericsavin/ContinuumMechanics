\begin{frame}{Recap}

\begin{itemize}
\item Local equilibrium equation:
\begin{displaymath}
\Divx\stress+\fv_v=\roi\ddot{\uv}\,.
\end{displaymath}
\item Small strains assumption:
\begin{displaymath}
\strain=\demi(\Gradx\uv+\Gradx\uv^\itr)\,.
\end{displaymath}
\item Closure is missing: $\stress=f(\strain)$ or $\strain=g(\stress)$, the \emphb{material constitutive equation}.
\end{itemize}

\end{frame}

\begin{frame}{Recap}{Traction test}

\begin{overprint}

\onslide<1|handout:1>
\begin{columns}[t]
\column{.6\textwidth}
\centering\includegraphics[scale=.3]{\figs/ch4-TractionTest}
\column{.4\textwidth}
\vskip-120pt
%\begin{center}
Measurements along $\ev$:
\begin{itemize}
\item $\stressj_{ee}=\frac{F}{S}$;
\item $\strainj_{ee}=\frac{\Delta L}{L}$.
\end{itemize}
%\end{center}
\end{columns}

\onslide<2|handout:2>
\begin{figure}
\centering\includegraphics[scale=.2]{\figs/ch4-StressStrain}
\end{figure}
\begin{itemize}
\item Ductile vs. fragile materials;
\item At small strains $F=K\Delta L$ or $\stressj_{ee}=k\strainj_{ee}$;
\item Generalization: $\stress=\tenselas\strain$ where $\tenselas$ is the fourth-order ($\tenselasj_{jklm}$) \emphb{elasticity tensor}.
\end{itemize}

\end{overprint}

\end{frame}

\begin{frame}{Elasticity tensor}{Symmetries}

\begin{itemize}
\item Fourth-order tensor: linear application between second-order tensors,
\begin{displaymath}
\begin{split}
\stress(\xv,t) &=\tenselas(\xv)\strain(\xv,t)\,,\\
\stressj_{jk}(\xv,t) &=\tenselasj_{jklm}(\xv)\strainj_{lm}(\xv,t)\,,\quad\forall\xv\in\medium\,,\forall t\in\Rset\,.
\end{split}
\end{displaymath}
\item Minor symmetries from the symmetry of $\strain$ and $\stress$:
\begin{displaymath}
\begin{split}
\tenselasj_{jklm}(\xv) &= \tenselasj_{kjlm}(\xv)\,, \\
\tenselasj_{jklm}(\xv) &= \tenselasj_{kjml}(\xv)\,.
\end{split}
\end{displaymath}
\item Major symmetry from thermodynamics first principle:
\begin{displaymath}
\tenselasj_{jklm}(\xv) = \tenselasj_{lmjk}(\xv)\,.
\end{displaymath}
\item \# coefficients: $81\xrightarrow[]{\text{minor symmetries}}36\xrightarrow[]{\text{major symmetry}}21$.
\end{itemize}

\end{frame}

\begin{frame}{Elasticity tensor}{Isotropy}

\begin{columns}[t]
\column{.33\textwidth}
\centering\includegraphics[scale=.25]{\figs/ch4-Isotropic}
\column{.33\textwidth}
\centering\includegraphics[scale=.25]{\figs/ch4-Anisotropic1}
\column{.33\textwidth}
\centering\includegraphics[scale=.25]{\figs/ch4-Anisotropic2}
\end{columns}
\begin{itemize}
\item Isotropy: the constitutive equation is the same whatever the orientation of the sample is,
\begin{displaymath}
\stress=\tenselas\strain\;(\text{sample \#1})\,,\quad\stress^*=\tenselas\strain^*\;(\text{sample \#2})\,.
\end{displaymath}
\item Letting $\rotation$ be an arbitrary rotation ($\rotation\rotation^\itr=\Id$):
\begin{displaymath}
\strain^*=\rotation\strain\rotation^\itr\,,\quad\stress^*=\rotation\stress\rotation^\itr\,,
\end{displaymath}
then:
\begin{displaymath}
\stress^*=\rotation\stress\rotation^\itr=\tenselas(\rotation\strain\rotation^\itr)\imply \rotation(\tenselas\strain)\rotation^\itr=\tenselas(\rotation\strain\rotation^\itr)\,.
\end{displaymath}
\end{itemize}

\end{frame}

\begin{frame}{Elasticity tensor}{Isotropy}

\begin{itemize}
\item \emphb{Rivlin-Ericksen theorem}:
\begin{displaymath}
\rotation(\tenselas\strain)\rotation^\itr=\tenselas(\rotation\strain\rotation^\itr)\imply\tenselas\strain=\alpha_0\Id+\alpha_1\strain+\alpha_2\strain^2\,,
\end{displaymath}
where $\alpha_m(\trace(\strain),\trace(\strain^2),\trace(\strain^3))$, $m=1,2,3$.
%\item To the leading order in $\text{O}(\strain)$ ($\lambda,\mu$ are Lam\'e's parameters):
%\begin{displaymath}
%\tenselas\strain=\lambda(\trace\strain)\Id+2\mu\strain\,.
%\end{displaymath}
\vskip20pt
\begin{columns}[t]
\column{.5\textwidth}
\centering\includegraphics[scale=.3]{\figs/ch4-RivlinR}\\
{\scriptsize Ronald Rivlin [1915--2005]}
\column{.5\textwidth}
%\centering\includegraphics[scale=.38]{\figs/ch4-Ericksen}\\
%\centering\includegraphics[scale=.3]{\figs/ch4-EricksenJ}\\
\centering\includegraphics[scale=.09]{\figs/ch4-SavinE}\\
%\centering\includegraphics[scale=.19]{\figs/ch4-AustinP}\\
{\scriptsize Jerald Ericksen [1925--]}
\end{columns}
\end{itemize}

\end{frame}

\begin{frame}{Elasticity tensor}{Isotropy}

\begin{overprint}

\onslide<1|handout:1>
\begin{itemize}
\item To the leading order in $\text{O}(\strain)$:
\begin{displaymath}
\boxed{\tenselas\strain=\lambda\trace(\strain)\Id+2\mu\strain}
\end{displaymath}
where $\lambda,\mu$ are \emphb{Lam\'e's modulii}.
\item This relationship can be inverted:
\begin{displaymath}
\begin{split}
\strain &=-\frac{\lambda}{2\mu(3\lambda+2\mu)}\trace(\stress)\Id+\frac{1}{2\mu}\stress \\
&=\alpha\trace(\stress)\Id+\beta\stress \\
\end{split}
\end{displaymath}
where $\alpha,\beta$ are obtained from measurements.
\item More generally $\strain=\tenscomp\stress$, where $\tenscomp$ is the compliance (fourth-order) tensor.
\end{itemize}

\onslide<2|handout:2>
\begin{itemize}
\item Traction test $\stress=\frac{F}{S}\ev\otimes\ev$:
\begin{columns}[t]
\column{.5\textwidth}
\hspace*{-0.3truecm}\centering\includegraphics[scale=.25]{\figs/ch4-TestEnu}
\column{.5\textwidth}
\vskip-140pt
\emphb{Young's modulus} $E$: 
\begin{displaymath}
E=\frac{\stressj_{ee}}{\strainj_{ee}}=\frac{\frac{F}{S}}{\frac{\Delta L}{L_0}}\,;
\end{displaymath}
\emphb{Poisson's coefficient} $\nu$:
\begin{displaymath}
\nu=-\frac{\strainj_{hh}}{\strainj_{ee}}=-\frac{\frac{\Delta L_H}{L_{H0}}}{\frac{\Delta L}{L_0}}\,.
\end{displaymath}
\end{columns}
\item $E$ is in Pa (GPa), and $\nu$ is dimensionless.
\end{itemize}

\onslide<3|handout:3>
\begin{itemize}
\item $\stress$ and $\strain$ have the same principal directions:
\begin{figure}
\centering\includegraphics[scale=.2]{\figs/ch4-Hooke}
\end{figure}
\item Along \emph{e.g.} the principal direction \#1:
\begin{displaymath}
\strainj_1=\frac{\stressj_1}{E}-\nu\frac{\stressj_2}{E}-\nu\frac{\stressj_3}{E}\,.
\end{displaymath}
\item In the basis $(\iv_1,\iv_2,\iv_3)$ of the principal directions:
\begin{displaymath}
\begin{split}
\strain &=\, \scriptstyle (\frac{\stressj_1}{E}-\nu\frac{\stressj_2+\stressj_3}{E}) \iv_1\otimes\iv_1 + (\frac{\stressj_2}{E}-\nu\frac{\stressj_1+\stressj_3}{E}) \iv_2\otimes\iv_2 + (\frac{\stressj_3}{E}-\nu\frac{\stressj_1+\stressj_2}{E}) \iv_3\otimes\iv_3 \\
&=\, \scriptstyle \frac{1+\nu}{E}\stress-\frac{\nu}{E}\trace(\stress)\Id\,.
\end{split}
\end{displaymath}
\end{itemize}

\onslide<4|handout:4>
\begin{itemize}
\item Recap: linear elastic isotropy has 2 coefficients,
\begin{displaymath}
\begin{split}
\stress &=\tenselas\strain=\lambda\trace(\strain)\Id+2\mu\strain\,, \\
\strain &=\tenscomp\stress=\frac{1+\nu}{E}\stress-\frac{\nu}{E}\trace(\stress)\Id\,.
\end{split}
\end{displaymath}
\item $\lambda(\xv),\mu(\xv)$ are Lam\'e's modulii, $E(\xv)$ is Young's modulus, $\nu(\xv)$ is Poisson's ratio:
\begin{displaymath}
\lambda=\frac{\nu E}{(1+\nu)(1-2\nu)}\,,\quad\mu=\frac{E}{2(1+\nu)}\,.
\end{displaymath}
\item $E>0$, $\mu>0$, $3\lambda+2\mu>0$, and $-1<\nu<0.5$.
\end{itemize}

\end{overprint}

\end{frame}

\begin{frame}{Thermoelasticity}{Isotropic case}

\begin{itemize}
\item Thermal strains in the isotropic case with temperature gradient $\Delta T$:
\begin{displaymath}
\strain_\text{th}=\alpha\Delta T\Id
\end{displaymath}
where $\alpha$ is the coefficient of linear thermal expansion.
\item Total strain tensor:
\begin{displaymath}
\begin{split}
\strain &=\strain_\text{elas}+\strain_\text{th} \\
&=\tenscomp\stress+\alpha\Delta T\Id\,.
\end{split}
\end{displaymath}
\item Linear thermoelastic constitutive equation:
\begin{displaymath}
\boxed{\stress=\tenselas(\strain-\alpha\Delta T\Id)}\,.
\end{displaymath}
\end{itemize}

\end{frame}